\documentclass[11pt]{article}

    \usepackage[breakable]{tcolorbox}
    \usepackage{parskip} % Stop auto-indenting (to mimic markdown behaviour)
    
    \usepackage{iftex}
    \ifPDFTeX
    	\usepackage[T1]{fontenc}
    	\usepackage{mathpazo}
    \else
    	\usepackage{fontspec}
    \fi

    % Basic figure setup, for now with no caption control since it's done
    % automatically by Pandoc (which extracts ![](path) syntax from Markdown).
    \usepackage{graphicx}
    % Maintain compatibility with old templates. Remove in nbconvert 6.0
    \let\Oldincludegraphics\includegraphics
    % Ensure that by default, figures have no caption (until we provide a
    % proper Figure object with a Caption API and a way to capture that
    % in the conversion process - todo).
    \usepackage{caption}
    \DeclareCaptionFormat{nocaption}{}
    \captionsetup{format=nocaption,aboveskip=0pt,belowskip=0pt}

    \usepackage{float}
    \floatplacement{figure}{H} % forces figures to be placed at the correct location
    \usepackage{xcolor} % Allow colors to be defined
    \usepackage{enumerate} % Needed for markdown enumerations to work
    \usepackage{geometry} % Used to adjust the document margins
    \usepackage{amsmath} % Equations
    \usepackage{amssymb} % Equations
    \usepackage{textcomp} % defines textquotesingle
    % Hack from http://tex.stackexchange.com/a/47451/13684:
    \AtBeginDocument{%
        \def\PYZsq{\textquotesingle}% Upright quotes in Pygmentized code
    }
    \usepackage{upquote} % Upright quotes for verbatim code
    \usepackage{eurosym} % defines \euro
    \usepackage[mathletters]{ucs} % Extended unicode (utf-8) support
    \usepackage{fancyvrb} % verbatim replacement that allows latex
    \usepackage{grffile} % extends the file name processing of package graphics 
                         % to support a larger range
    \makeatletter % fix for old versions of grffile with XeLaTeX
    \@ifpackagelater{grffile}{2019/11/01}
    {
      % Do nothing on new versions
    }
    {
      \def\Gread@@xetex#1{%
        \IfFileExists{"\Gin@base".bb}%
        {\Gread@eps{\Gin@base.bb}}%
        {\Gread@@xetex@aux#1}%
      }
    }
    \makeatother
    \usepackage[Export]{adjustbox} % Used to constrain images to a maximum size
    \adjustboxset{max size={0.9\linewidth}{0.9\paperheight}}

    % The hyperref package gives us a pdf with properly built
    % internal navigation ('pdf bookmarks' for the table of contents,
    % internal cross-reference links, web links for URLs, etc.)
    \usepackage{hyperref}
    % The default LaTeX title has an obnoxious amount of whitespace. By default,
    % titling removes some of it. It also provides customization options.
    \usepackage{titling}
    \usepackage{longtable} % longtable support required by pandoc >1.10
    \usepackage{booktabs}  % table support for pandoc > 1.12.2
    \usepackage[inline]{enumitem} % IRkernel/repr support (it uses the enumerate* environment)
    \usepackage[normalem]{ulem} % ulem is needed to support strikethroughs (\sout)
                                % normalem makes italics be italics, not underlines
    \usepackage{mathrsfs}
    

    
    % Colors for the hyperref package
    \definecolor{urlcolor}{rgb}{0,.145,.698}
    \definecolor{linkcolor}{rgb}{.71,0.21,0.01}
    \definecolor{citecolor}{rgb}{.12,.54,.11}

    % ANSI colors
    \definecolor{ansi-black}{HTML}{3E424D}
    \definecolor{ansi-black-intense}{HTML}{282C36}
    \definecolor{ansi-red}{HTML}{E75C58}
    \definecolor{ansi-red-intense}{HTML}{B22B31}
    \definecolor{ansi-green}{HTML}{00A250}
    \definecolor{ansi-green-intense}{HTML}{007427}
    \definecolor{ansi-yellow}{HTML}{DDB62B}
    \definecolor{ansi-yellow-intense}{HTML}{B27D12}
    \definecolor{ansi-blue}{HTML}{208FFB}
    \definecolor{ansi-blue-intense}{HTML}{0065CA}
    \definecolor{ansi-magenta}{HTML}{D160C4}
    \definecolor{ansi-magenta-intense}{HTML}{A03196}
    \definecolor{ansi-cyan}{HTML}{60C6C8}
    \definecolor{ansi-cyan-intense}{HTML}{258F8F}
    \definecolor{ansi-white}{HTML}{C5C1B4}
    \definecolor{ansi-white-intense}{HTML}{A1A6B2}
    \definecolor{ansi-default-inverse-fg}{HTML}{FFFFFF}
    \definecolor{ansi-default-inverse-bg}{HTML}{000000}

    % common color for the border for error outputs.
    \definecolor{outerrorbackground}{HTML}{FFDFDF}

    % commands and environments needed by pandoc snippets
    % extracted from the output of `pandoc -s`
    \providecommand{\tightlist}{%
      \setlength{\itemsep}{0pt}\setlength{\parskip}{0pt}}
    \DefineVerbatimEnvironment{Highlighting}{Verbatim}{commandchars=\\\{\}}
    % Add ',fontsize=\small' for more characters per line
    \newenvironment{Shaded}{}{}
    \newcommand{\KeywordTok}[1]{\textcolor[rgb]{0.00,0.44,0.13}{\textbf{{#1}}}}
    \newcommand{\DataTypeTok}[1]{\textcolor[rgb]{0.56,0.13,0.00}{{#1}}}
    \newcommand{\DecValTok}[1]{\textcolor[rgb]{0.25,0.63,0.44}{{#1}}}
    \newcommand{\BaseNTok}[1]{\textcolor[rgb]{0.25,0.63,0.44}{{#1}}}
    \newcommand{\FloatTok}[1]{\textcolor[rgb]{0.25,0.63,0.44}{{#1}}}
    \newcommand{\CharTok}[1]{\textcolor[rgb]{0.25,0.44,0.63}{{#1}}}
    \newcommand{\StringTok}[1]{\textcolor[rgb]{0.25,0.44,0.63}{{#1}}}
    \newcommand{\CommentTok}[1]{\textcolor[rgb]{0.38,0.63,0.69}{\textit{{#1}}}}
    \newcommand{\OtherTok}[1]{\textcolor[rgb]{0.00,0.44,0.13}{{#1}}}
    \newcommand{\AlertTok}[1]{\textcolor[rgb]{1.00,0.00,0.00}{\textbf{{#1}}}}
    \newcommand{\FunctionTok}[1]{\textcolor[rgb]{0.02,0.16,0.49}{{#1}}}
    \newcommand{\RegionMarkerTok}[1]{{#1}}
    \newcommand{\ErrorTok}[1]{\textcolor[rgb]{1.00,0.00,0.00}{\textbf{{#1}}}}
    \newcommand{\NormalTok}[1]{{#1}}
    
    % Additional commands for more recent versions of Pandoc
    \newcommand{\ConstantTok}[1]{\textcolor[rgb]{0.53,0.00,0.00}{{#1}}}
    \newcommand{\SpecialCharTok}[1]{\textcolor[rgb]{0.25,0.44,0.63}{{#1}}}
    \newcommand{\VerbatimStringTok}[1]{\textcolor[rgb]{0.25,0.44,0.63}{{#1}}}
    \newcommand{\SpecialStringTok}[1]{\textcolor[rgb]{0.73,0.40,0.53}{{#1}}}
    \newcommand{\ImportTok}[1]{{#1}}
    \newcommand{\DocumentationTok}[1]{\textcolor[rgb]{0.73,0.13,0.13}{\textit{{#1}}}}
    \newcommand{\AnnotationTok}[1]{\textcolor[rgb]{0.38,0.63,0.69}{\textbf{\textit{{#1}}}}}
    \newcommand{\CommentVarTok}[1]{\textcolor[rgb]{0.38,0.63,0.69}{\textbf{\textit{{#1}}}}}
    \newcommand{\VariableTok}[1]{\textcolor[rgb]{0.10,0.09,0.49}{{#1}}}
    \newcommand{\ControlFlowTok}[1]{\textcolor[rgb]{0.00,0.44,0.13}{\textbf{{#1}}}}
    \newcommand{\OperatorTok}[1]{\textcolor[rgb]{0.40,0.40,0.40}{{#1}}}
    \newcommand{\BuiltInTok}[1]{{#1}}
    \newcommand{\ExtensionTok}[1]{{#1}}
    \newcommand{\PreprocessorTok}[1]{\textcolor[rgb]{0.74,0.48,0.00}{{#1}}}
    \newcommand{\AttributeTok}[1]{\textcolor[rgb]{0.49,0.56,0.16}{{#1}}}
    \newcommand{\InformationTok}[1]{\textcolor[rgb]{0.38,0.63,0.69}{\textbf{\textit{{#1}}}}}
    \newcommand{\WarningTok}[1]{\textcolor[rgb]{0.38,0.63,0.69}{\textbf{\textit{{#1}}}}}
    
    
    % Define a nice break command that doesn't care if a line doesn't already
    % exist.
    \def\br{\hspace*{\fill} \\* }
    % Math Jax compatibility definitions
    \def\gt{>}
    \def\lt{<}
    \let\Oldtex\TeX
    \let\Oldlatex\LaTeX
    \renewcommand{\TeX}{\textrm{\Oldtex}}
    \renewcommand{\LaTeX}{\textrm{\Oldlatex}}
    % Document parameters
    % Document title
    \title{taruma\_LI01\_ann\_ka}
    
    
    
    
    
% Pygments definitions
\makeatletter
\def\PY@reset{\let\PY@it=\relax \let\PY@bf=\relax%
    \let\PY@ul=\relax \let\PY@tc=\relax%
    \let\PY@bc=\relax \let\PY@ff=\relax}
\def\PY@tok#1{\csname PY@tok@#1\endcsname}
\def\PY@toks#1+{\ifx\relax#1\empty\else%
    \PY@tok{#1}\expandafter\PY@toks\fi}
\def\PY@do#1{\PY@bc{\PY@tc{\PY@ul{%
    \PY@it{\PY@bf{\PY@ff{#1}}}}}}}
\def\PY#1#2{\PY@reset\PY@toks#1+\relax+\PY@do{#2}}

\@namedef{PY@tok@w}{\def\PY@tc##1{\textcolor[rgb]{0.73,0.73,0.73}{##1}}}
\@namedef{PY@tok@c}{\let\PY@it=\textit\def\PY@tc##1{\textcolor[rgb]{0.24,0.48,0.48}{##1}}}
\@namedef{PY@tok@cp}{\def\PY@tc##1{\textcolor[rgb]{0.61,0.40,0.00}{##1}}}
\@namedef{PY@tok@k}{\let\PY@bf=\textbf\def\PY@tc##1{\textcolor[rgb]{0.00,0.50,0.00}{##1}}}
\@namedef{PY@tok@kp}{\def\PY@tc##1{\textcolor[rgb]{0.00,0.50,0.00}{##1}}}
\@namedef{PY@tok@kt}{\def\PY@tc##1{\textcolor[rgb]{0.69,0.00,0.25}{##1}}}
\@namedef{PY@tok@o}{\def\PY@tc##1{\textcolor[rgb]{0.40,0.40,0.40}{##1}}}
\@namedef{PY@tok@ow}{\let\PY@bf=\textbf\def\PY@tc##1{\textcolor[rgb]{0.67,0.13,1.00}{##1}}}
\@namedef{PY@tok@nb}{\def\PY@tc##1{\textcolor[rgb]{0.00,0.50,0.00}{##1}}}
\@namedef{PY@tok@nf}{\def\PY@tc##1{\textcolor[rgb]{0.00,0.00,1.00}{##1}}}
\@namedef{PY@tok@nc}{\let\PY@bf=\textbf\def\PY@tc##1{\textcolor[rgb]{0.00,0.00,1.00}{##1}}}
\@namedef{PY@tok@nn}{\let\PY@bf=\textbf\def\PY@tc##1{\textcolor[rgb]{0.00,0.00,1.00}{##1}}}
\@namedef{PY@tok@ne}{\let\PY@bf=\textbf\def\PY@tc##1{\textcolor[rgb]{0.80,0.25,0.22}{##1}}}
\@namedef{PY@tok@nv}{\def\PY@tc##1{\textcolor[rgb]{0.10,0.09,0.49}{##1}}}
\@namedef{PY@tok@no}{\def\PY@tc##1{\textcolor[rgb]{0.53,0.00,0.00}{##1}}}
\@namedef{PY@tok@nl}{\def\PY@tc##1{\textcolor[rgb]{0.46,0.46,0.00}{##1}}}
\@namedef{PY@tok@ni}{\let\PY@bf=\textbf\def\PY@tc##1{\textcolor[rgb]{0.44,0.44,0.44}{##1}}}
\@namedef{PY@tok@na}{\def\PY@tc##1{\textcolor[rgb]{0.41,0.47,0.13}{##1}}}
\@namedef{PY@tok@nt}{\let\PY@bf=\textbf\def\PY@tc##1{\textcolor[rgb]{0.00,0.50,0.00}{##1}}}
\@namedef{PY@tok@nd}{\def\PY@tc##1{\textcolor[rgb]{0.67,0.13,1.00}{##1}}}
\@namedef{PY@tok@s}{\def\PY@tc##1{\textcolor[rgb]{0.73,0.13,0.13}{##1}}}
\@namedef{PY@tok@sd}{\let\PY@it=\textit\def\PY@tc##1{\textcolor[rgb]{0.73,0.13,0.13}{##1}}}
\@namedef{PY@tok@si}{\let\PY@bf=\textbf\def\PY@tc##1{\textcolor[rgb]{0.64,0.35,0.47}{##1}}}
\@namedef{PY@tok@se}{\let\PY@bf=\textbf\def\PY@tc##1{\textcolor[rgb]{0.67,0.36,0.12}{##1}}}
\@namedef{PY@tok@sr}{\def\PY@tc##1{\textcolor[rgb]{0.64,0.35,0.47}{##1}}}
\@namedef{PY@tok@ss}{\def\PY@tc##1{\textcolor[rgb]{0.10,0.09,0.49}{##1}}}
\@namedef{PY@tok@sx}{\def\PY@tc##1{\textcolor[rgb]{0.00,0.50,0.00}{##1}}}
\@namedef{PY@tok@m}{\def\PY@tc##1{\textcolor[rgb]{0.40,0.40,0.40}{##1}}}
\@namedef{PY@tok@gh}{\let\PY@bf=\textbf\def\PY@tc##1{\textcolor[rgb]{0.00,0.00,0.50}{##1}}}
\@namedef{PY@tok@gu}{\let\PY@bf=\textbf\def\PY@tc##1{\textcolor[rgb]{0.50,0.00,0.50}{##1}}}
\@namedef{PY@tok@gd}{\def\PY@tc##1{\textcolor[rgb]{0.63,0.00,0.00}{##1}}}
\@namedef{PY@tok@gi}{\def\PY@tc##1{\textcolor[rgb]{0.00,0.52,0.00}{##1}}}
\@namedef{PY@tok@gr}{\def\PY@tc##1{\textcolor[rgb]{0.89,0.00,0.00}{##1}}}
\@namedef{PY@tok@ge}{\let\PY@it=\textit}
\@namedef{PY@tok@gs}{\let\PY@bf=\textbf}
\@namedef{PY@tok@gp}{\let\PY@bf=\textbf\def\PY@tc##1{\textcolor[rgb]{0.00,0.00,0.50}{##1}}}
\@namedef{PY@tok@go}{\def\PY@tc##1{\textcolor[rgb]{0.44,0.44,0.44}{##1}}}
\@namedef{PY@tok@gt}{\def\PY@tc##1{\textcolor[rgb]{0.00,0.27,0.87}{##1}}}
\@namedef{PY@tok@err}{\def\PY@bc##1{{\setlength{\fboxsep}{\string -\fboxrule}\fcolorbox[rgb]{1.00,0.00,0.00}{1,1,1}{\strut ##1}}}}
\@namedef{PY@tok@kc}{\let\PY@bf=\textbf\def\PY@tc##1{\textcolor[rgb]{0.00,0.50,0.00}{##1}}}
\@namedef{PY@tok@kd}{\let\PY@bf=\textbf\def\PY@tc##1{\textcolor[rgb]{0.00,0.50,0.00}{##1}}}
\@namedef{PY@tok@kn}{\let\PY@bf=\textbf\def\PY@tc##1{\textcolor[rgb]{0.00,0.50,0.00}{##1}}}
\@namedef{PY@tok@kr}{\let\PY@bf=\textbf\def\PY@tc##1{\textcolor[rgb]{0.00,0.50,0.00}{##1}}}
\@namedef{PY@tok@bp}{\def\PY@tc##1{\textcolor[rgb]{0.00,0.50,0.00}{##1}}}
\@namedef{PY@tok@fm}{\def\PY@tc##1{\textcolor[rgb]{0.00,0.00,1.00}{##1}}}
\@namedef{PY@tok@vc}{\def\PY@tc##1{\textcolor[rgb]{0.10,0.09,0.49}{##1}}}
\@namedef{PY@tok@vg}{\def\PY@tc##1{\textcolor[rgb]{0.10,0.09,0.49}{##1}}}
\@namedef{PY@tok@vi}{\def\PY@tc##1{\textcolor[rgb]{0.10,0.09,0.49}{##1}}}
\@namedef{PY@tok@vm}{\def\PY@tc##1{\textcolor[rgb]{0.10,0.09,0.49}{##1}}}
\@namedef{PY@tok@sa}{\def\PY@tc##1{\textcolor[rgb]{0.73,0.13,0.13}{##1}}}
\@namedef{PY@tok@sb}{\def\PY@tc##1{\textcolor[rgb]{0.73,0.13,0.13}{##1}}}
\@namedef{PY@tok@sc}{\def\PY@tc##1{\textcolor[rgb]{0.73,0.13,0.13}{##1}}}
\@namedef{PY@tok@dl}{\def\PY@tc##1{\textcolor[rgb]{0.73,0.13,0.13}{##1}}}
\@namedef{PY@tok@s2}{\def\PY@tc##1{\textcolor[rgb]{0.73,0.13,0.13}{##1}}}
\@namedef{PY@tok@sh}{\def\PY@tc##1{\textcolor[rgb]{0.73,0.13,0.13}{##1}}}
\@namedef{PY@tok@s1}{\def\PY@tc##1{\textcolor[rgb]{0.73,0.13,0.13}{##1}}}
\@namedef{PY@tok@mb}{\def\PY@tc##1{\textcolor[rgb]{0.40,0.40,0.40}{##1}}}
\@namedef{PY@tok@mf}{\def\PY@tc##1{\textcolor[rgb]{0.40,0.40,0.40}{##1}}}
\@namedef{PY@tok@mh}{\def\PY@tc##1{\textcolor[rgb]{0.40,0.40,0.40}{##1}}}
\@namedef{PY@tok@mi}{\def\PY@tc##1{\textcolor[rgb]{0.40,0.40,0.40}{##1}}}
\@namedef{PY@tok@il}{\def\PY@tc##1{\textcolor[rgb]{0.40,0.40,0.40}{##1}}}
\@namedef{PY@tok@mo}{\def\PY@tc##1{\textcolor[rgb]{0.40,0.40,0.40}{##1}}}
\@namedef{PY@tok@ch}{\let\PY@it=\textit\def\PY@tc##1{\textcolor[rgb]{0.24,0.48,0.48}{##1}}}
\@namedef{PY@tok@cm}{\let\PY@it=\textit\def\PY@tc##1{\textcolor[rgb]{0.24,0.48,0.48}{##1}}}
\@namedef{PY@tok@cpf}{\let\PY@it=\textit\def\PY@tc##1{\textcolor[rgb]{0.24,0.48,0.48}{##1}}}
\@namedef{PY@tok@c1}{\let\PY@it=\textit\def\PY@tc##1{\textcolor[rgb]{0.24,0.48,0.48}{##1}}}
\@namedef{PY@tok@cs}{\let\PY@it=\textit\def\PY@tc##1{\textcolor[rgb]{0.24,0.48,0.48}{##1}}}

\def\PYZbs{\char`\\}
\def\PYZus{\char`\_}
\def\PYZob{\char`\{}
\def\PYZcb{\char`\}}
\def\PYZca{\char`\^}
\def\PYZam{\char`\&}
\def\PYZlt{\char`\<}
\def\PYZgt{\char`\>}
\def\PYZsh{\char`\#}
\def\PYZpc{\char`\%}
\def\PYZdl{\char`\$}
\def\PYZhy{\char`\-}
\def\PYZsq{\char`\'}
\def\PYZdq{\char`\"}
\def\PYZti{\char`\~}
% for compatibility with earlier versions
\def\PYZat{@}
\def\PYZlb{[}
\def\PYZrb{]}
\makeatother


    % For linebreaks inside Verbatim environment from package fancyvrb. 
    \makeatletter
        \newbox\Wrappedcontinuationbox 
        \newbox\Wrappedvisiblespacebox 
        \newcommand*\Wrappedvisiblespace {\textcolor{red}{\textvisiblespace}} 
        \newcommand*\Wrappedcontinuationsymbol {\textcolor{red}{\llap{\tiny$\m@th\hookrightarrow$}}} 
        \newcommand*\Wrappedcontinuationindent {3ex } 
        \newcommand*\Wrappedafterbreak {\kern\Wrappedcontinuationindent\copy\Wrappedcontinuationbox} 
        % Take advantage of the already applied Pygments mark-up to insert 
        % potential linebreaks for TeX processing. 
        %        {, <, #, %, $, ' and ": go to next line. 
        %        _, }, ^, &, >, - and ~: stay at end of broken line. 
        % Use of \textquotesingle for straight quote. 
        \newcommand*\Wrappedbreaksatspecials {% 
            \def\PYGZus{\discretionary{\char`\_}{\Wrappedafterbreak}{\char`\_}}% 
            \def\PYGZob{\discretionary{}{\Wrappedafterbreak\char`\{}{\char`\{}}% 
            \def\PYGZcb{\discretionary{\char`\}}{\Wrappedafterbreak}{\char`\}}}% 
            \def\PYGZca{\discretionary{\char`\^}{\Wrappedafterbreak}{\char`\^}}% 
            \def\PYGZam{\discretionary{\char`\&}{\Wrappedafterbreak}{\char`\&}}% 
            \def\PYGZlt{\discretionary{}{\Wrappedafterbreak\char`\<}{\char`\<}}% 
            \def\PYGZgt{\discretionary{\char`\>}{\Wrappedafterbreak}{\char`\>}}% 
            \def\PYGZsh{\discretionary{}{\Wrappedafterbreak\char`\#}{\char`\#}}% 
            \def\PYGZpc{\discretionary{}{\Wrappedafterbreak\char`\%}{\char`\%}}% 
            \def\PYGZdl{\discretionary{}{\Wrappedafterbreak\char`\$}{\char`\$}}% 
            \def\PYGZhy{\discretionary{\char`\-}{\Wrappedafterbreak}{\char`\-}}% 
            \def\PYGZsq{\discretionary{}{\Wrappedafterbreak\textquotesingle}{\textquotesingle}}% 
            \def\PYGZdq{\discretionary{}{\Wrappedafterbreak\char`\"}{\char`\"}}% 
            \def\PYGZti{\discretionary{\char`\~}{\Wrappedafterbreak}{\char`\~}}% 
        } 
        % Some characters . , ; ? ! / are not pygmentized. 
        % This macro makes them "active" and they will insert potential linebreaks 
        \newcommand*\Wrappedbreaksatpunct {% 
            \lccode`\~`\.\lowercase{\def~}{\discretionary{\hbox{\char`\.}}{\Wrappedafterbreak}{\hbox{\char`\.}}}% 
            \lccode`\~`\,\lowercase{\def~}{\discretionary{\hbox{\char`\,}}{\Wrappedafterbreak}{\hbox{\char`\,}}}% 
            \lccode`\~`\;\lowercase{\def~}{\discretionary{\hbox{\char`\;}}{\Wrappedafterbreak}{\hbox{\char`\;}}}% 
            \lccode`\~`\:\lowercase{\def~}{\discretionary{\hbox{\char`\:}}{\Wrappedafterbreak}{\hbox{\char`\:}}}% 
            \lccode`\~`\?\lowercase{\def~}{\discretionary{\hbox{\char`\?}}{\Wrappedafterbreak}{\hbox{\char`\?}}}% 
            \lccode`\~`\!\lowercase{\def~}{\discretionary{\hbox{\char`\!}}{\Wrappedafterbreak}{\hbox{\char`\!}}}% 
            \lccode`\~`\/\lowercase{\def~}{\discretionary{\hbox{\char`\/}}{\Wrappedafterbreak}{\hbox{\char`\/}}}% 
            \catcode`\.\active
            \catcode`\,\active 
            \catcode`\;\active
            \catcode`\:\active
            \catcode`\?\active
            \catcode`\!\active
            \catcode`\/\active 
            \lccode`\~`\~ 	
        }
    \makeatother

    \let\OriginalVerbatim=\Verbatim
    \makeatletter
    \renewcommand{\Verbatim}[1][1]{%
        %\parskip\z@skip
        \sbox\Wrappedcontinuationbox {\Wrappedcontinuationsymbol}%
        \sbox\Wrappedvisiblespacebox {\FV@SetupFont\Wrappedvisiblespace}%
        \def\FancyVerbFormatLine ##1{\hsize\linewidth
            \vtop{\raggedright\hyphenpenalty\z@\exhyphenpenalty\z@
                \doublehyphendemerits\z@\finalhyphendemerits\z@
                \strut ##1\strut}%
        }%
        % If the linebreak is at a space, the latter will be displayed as visible
        % space at end of first line, and a continuation symbol starts next line.
        % Stretch/shrink are however usually zero for typewriter font.
        \def\FV@Space {%
            \nobreak\hskip\z@ plus\fontdimen3\font minus\fontdimen4\font
            \discretionary{\copy\Wrappedvisiblespacebox}{\Wrappedafterbreak}
            {\kern\fontdimen2\font}%
        }%
        
        % Allow breaks at special characters using \PYG... macros.
        \Wrappedbreaksatspecials
        % Breaks at punctuation characters . , ; ? ! and / need catcode=\active 	
        \OriginalVerbatim[#1,codes*=\Wrappedbreaksatpunct]%
    }
    \makeatother

    % Exact colors from NB
    \definecolor{incolor}{HTML}{303F9F}
    \definecolor{outcolor}{HTML}{D84315}
    \definecolor{cellborder}{HTML}{CFCFCF}
    \definecolor{cellbackground}{HTML}{F7F7F7}
    
    % prompt
    \makeatletter
    \newcommand{\boxspacing}{\kern\kvtcb@left@rule\kern\kvtcb@boxsep}
    \makeatother
    \newcommand{\prompt}[4]{
        {\ttfamily\llap{{\color{#2}[#3]:\hspace{3pt}#4}}\vspace{-\baselineskip}}
    }
    

    
    % Prevent overflowing lines due to hard-to-break entities
    \sloppy 
    % Setup hyperref package
    \hypersetup{
      breaklinks=true,  % so long urls are correctly broken across lines
      colorlinks=true,
      urlcolor=urlcolor,
      linkcolor=linkcolor,
      citecolor=citecolor,
      }
    % Slightly bigger margins than the latex defaults
    
    \geometry{verbose,tmargin=1in,bmargin=1in,lmargin=1in,rmargin=1in}
    
    

\begin{document}
    
    % \maketitle

	\begin{titlepage}
		\vspace*{\fill}
		\begin{center}
 		\normalsize Laporan Implementasi (LI-01)\\
		\huge Prediksi Kualitas Air menggunakan \emph{Artificial Neural Networks} (ANN) \\ 
		\normalsize Versi 2.0.0 \\[0.2cm]
      	\small Berdasarkan \emph{Jupyter Notebook}: \texttt{taruma\_LI01\_ann\_ka.ipynb} \\[0.5cm]
      	\normalsize Buku ini menyajikan implementasi \emph{Deep Learning} pada kasus memprediksikan kualitas air. \\[0.5cm]
		\normalsize oleh Taruma Sakti Megariansyah\\[0.5cm]
      	\normalsize 13 Juli 2019\\[1cm]
    	\adjustimage{max size={0.9\linewidth}{1cm}}{hidrokit-400x100.jpg}\\
      	\normalsize \href{https://hidrokit.github.io/hidrokit}{hidrokit.github.io/hidrokit}  \\[0.5cm]
      	\small disponsori oleh \\[0.2cm]
    	\href{https://www.fiako.co.id}{\adjustimage{max size={0.9\linewidth}{2.5cm}}{logo_fiako.png}}\\
		\end{center}
    	\vspace*{\fill}
	\end{titlepage}

    

    
    \hypertarget{single-output-regression-neural-network-pada-kasus-prediksi-kualitas-air}{%
\section{Single-Output Regression Neural Network pada kasus Prediksi
Kualitas
Air}\label{single-output-regression-neural-network-pada-kasus-prediksi-kualitas-air}}

Notebook ini hanya \textbf{contoh} dan dibuat untuk
\textbf{pembelajaran} mengenai \emph{Deep Learning/Neural Networks} dan
penggunaan praktis bidang sumberdaya air menggunakan \emph{Python}.
Notebook ini masih perlu dievaluasi kembali jika digunakan untuk
kepenting riset/penelitian ataupun proyek.

\hypertarget{informasi-notebook}{%
\subsection{Informasi Notebook}\label{informasi-notebook}}

\begin{itemize}
\tightlist
\item
  \textbf{notebook name}: \texttt{taruma\_demo\_ann\_ka\_2.0.0}
\item
  \textbf{notebook version/date}: \texttt{2.0.0}/\texttt{20190713}
\item
  \textbf{notebook server}: Google Colab
\item
  \textbf{hidrokit version}: \texttt{0.2.0}
\item
  \textbf{python version}: \texttt{3.7}
\end{itemize}

    \hypertarget{deskripsi-kasus}{%
\section{Deskripsi Kasus}\label{deskripsi-kasus}}

Bagian ini akan menjelaskan gambaran umum mengenai dataset,
permasalahan/tujuan, dan langkah penyelesaiannya

    \hypertarget{dataset}{%
\subsection{Dataset}\label{dataset}}

Dataset merupakan data kualitas air \textbf{bulanan} dari \textbf{Juni
2000 hingga Desember 2017} (211 data bulanan). Dataset memiliki 15 kolom
yaitu (berurutan): - \textbf{1 kolom berupa tanggal} - \textbf{11 kolom
\emph{independent variables} yang diperoleh di stasiun A}: - temperatur
udara (\texttt{temp\_udara}), lama penyinaran (\texttt{lama\_sinar}),
kecepatan angin (\texttt{kec\_angin}), \(Q_{in}\)
(\texttt{debit\_masuk}), \(Q_{out}\) (\texttt{debit\_keluar}), volume
(\texttt{volume}), temperatur air (\texttt{temp\_air}), \(O_2\), Oksigen
(\texttt{oksigen}), \(NO_2\), Nitrogen (\texttt{nitrogen}), \(NO_3\),
Nitrat (\texttt{nitrat}), \(NH_3\), Amonia (\texttt{amonia}). -
\textbf{3 kolom \emph{dependent variables} pada stasiun B}: - \(NO_2\),
Nitrogen (\texttt{out\_nitrogen}), \(NO_3\), Nitrat
(\texttt{out\_nitrat}), \(NH_3\), Amonia (\texttt{out\_amonia}).

    \hypertarget{permasalahan-dan-tujuan}{%
\subsection{Permasalahan dan Tujuan}\label{permasalahan-dan-tujuan}}

\textbf{Permasalahan}: - Peneliti ingin mengetahui nilai tiga kualitas
air berupa \(NO_2, NO_3, NH_3\) pada stasiun B berdasarkan informasi
yang diperoleh di stasiun A.

\textbf{Batasan Masalah}: - Dalam \emph{notebook} ini \emph{target} yang
digunakan hanya \(NH_3\) (\emph{single-output}). - Arsitektur
\emph{Neural Networks} yang digunakan adalah \emph{Multi Layer
Perceptron} (MLP), dengan setidaknya \emph{hidden layers} lebih dari
satu. - Kasus disini merupakan contoh permasalahan \emph{supervised
learning}.

\textbf{Pertanyaan}: - Berapa nilai NH3 pada stasiun B pada waktu \(t\)
jika \textbf{telah diketahui} hasil pengukuran di stasiun A pada waktu
\(t\) \textbf{dan} hasil observasi stasiun B pada waktu sebelumnya
(\(t-1, t-2, ..., t-n\), dengan \(n\) adalah jumlah \emph{timesteps})? -
Apa arsitektur \emph{NN} yang optimal untuk melakukan prediksi kualitas
air?

\emph{Catatan}: Dalam kasus ini akan digunakan \emph{\(2\) timesteps}
yang berarti data \textbf{dua bulan yang lalu} akan digunakan sebagai
\emph{input} untuk memprediksikan \emph{target} pada waktu \(t\).

    \hypertarget{strategi-penyelesaian-masalah}{%
\subsection{Strategi Penyelesaian
Masalah}\label{strategi-penyelesaian-masalah}}

\hypertarget{tahap-1-data-exploration-and-data-munging}{%
\subsubsection{\texorpdfstring{Tahap 1: \emph{Data exploration and Data
munging}}{Tahap 1: Data exploration and Data munging}}\label{tahap-1-data-exploration-and-data-munging}}

\begin{enumerate}
\def\labelenumi{\arabic{enumi}.}
\tightlist
\item
  \emph{Import} dataset dari berkas excel ke dalam
  \texttt{pandas.DataFrame}.
\item
  Eksplorasi dataset berupa memeriksa kehilangan data disertai validasi
  dan verifikasi data. \emph{Data exploration} dapat berupa visualisasi,
  deskripsi statistik, dan memeriksa jika terdapat nilai \emph{outlier}
  atau tidak.
\item
  Jika terdapat data yang hilang, maka \textbf{diasumsikan} data
  memiliki sifat \emph{linear} sehingga data yang hilang diisi dengan
  menginterpolasikan dengan metode linear. (Langkah ini dapat diganti
  dengan kesesuaian keahlian bidangnya)
\item
  Diperiksa kembali data yang telah diisi dan melakukan ekplorasi data
  lagi untuk memastikan dataset sudah siap untuk diolah di tahap
  berikutnya.
\end{enumerate}

\hypertarget{tahap-2-data-preprocessing}{%
\subsubsection{\texorpdfstring{Tahap 2: \emph{Data
Preprocessing}}{Tahap 2: Data Preprocessing}}\label{tahap-2-data-preprocessing}}

\begin{enumerate}
\def\labelenumi{\arabic{enumi}.}
\tightlist
\item
  Membagi dataset menjadi dua bagian yaitu \textbf{training set dan test
  set}. Pemotongan data ini tidak acak. \textbf{Training set} merupakan
  potongan data dari \textbf{Juni 2000 hingga Desember 2014}.
  \textbf{Test set} merupakan potongan data dari \textbf{Januari 2015
  hingga Desember 2017}. Untuk selanjutnya hanya \textbf{training set}
  yang akan diproses untuk training dan \textbf{test set} disimpan untuk
  tahap evaluasi model.
\item
  Dilakukan tahap \textbf{\emph{Scaling}} yaitu menormalisasikan nilai
  pada dataset ke dalam skala yang sama. Digunakan metode
  \texttt{MinMaxScaler} karena diketahui data tidak terdistribusi
  normal.
\item
  Membuat kolom yang menunjukkan \emph{\(2\) timesteps}, sehingga
  dataset yang sebelumnya 14 kolom menjadi
  \(feature_{original}\times(timesteps+1)=14\times(2+1)=42\).
\item
  Menampilkan hasil penambahan \emph{timesteps} dalam bentuk
  \texttt{pandas.DataFrame}.
\item
  \textbf{training set} dibagi menjadi \texttt{X\_train} dan
  \texttt{y\_train}. Dengan \texttt{X\_train} merupakan seluruh kolom
  \textbf{kecuali} nilai observasi dari stasiun B pada waktu \(t\)
  (kolom:
  \texttt{out\_nitrogen\_tmin0,\ out\_nitrat\_tmin0,\ out\_amonia\_tmin0})
  sebagai \emph{feature}. Sedangkan \texttt{y\_train} merupakan kolom
  \texttt{out\_amonia\_tmin0} sebagai \emph{target}.
\end{enumerate}

\hypertarget{tahap-3-building-neural-networks}{%
\subsubsection{\texorpdfstring{Tahap 3: \emph{Building Neural
Networks}}{Tahap 3: Building Neural Networks}}\label{tahap-3-building-neural-networks}}

\begin{enumerate}
\def\labelenumi{\arabic{enumi}.}
\tightlist
\item
  Untuk Neural Networks digunakan \emph{Multi Layer Perceptron (MLP)}
  dengan target \emph{single-output}.
\item
  Membuat fungsi \texttt{build\_model} yang memberikan fleksibilitas
  arsitektur sehingga memiliki hidden layer lebih dari satu dengan
  parameter \texttt{hidden\_layers}.
\item
  Menggunakan \texttt{GridSearchCV} untuk mengetahui parameter terbaik.
\item
  Melalukan proses \texttt{fit} terhadap \texttt{X\_train,\ y\_train}.
  Besarnya \emph{validation split} yang digunakan sebesar \(0.2\).
\end{enumerate}

\hypertarget{tahap-4-evaluation-models}{%
\subsubsection{\texorpdfstring{Tahap 4: \emph{Evaluation
Models}}{Tahap 4: Evaluation Models}}\label{tahap-4-evaluation-models}}

\begin{enumerate}
\def\labelenumi{\arabic{enumi}.}
\tightlist
\item
  Pada tahap evaluasi, dataset yang digunakan adalah \textbf{test set}.
  \emph{test set} yang diperoleh dari pembagian set perlu diproses lebih
  lanjut sebelum digunakan untuk memprediksi.
\item
  Dilakukan proses \emph{data preprocessing} seperti tahap 2 langkah 2-5
  pada \textbf{test set}.
\item
  Memprediksi nilai dengan model yang terbaik hasil \emph{Grid Search}.
\item
  Mengembalikan nilai prediksi dan nilai \texttt{y\_test} ke skala
  original dengan menggunakan \emph{method} \texttt{inverse\_transform}.
  Hal ini dilakukan dengan mentransfer atribut \texttt{MinMaxScaler}
  sebelumnya ke \emph{object} baru, agar menyederhanakan proses
  pengembalian ke skala asli.
\item
  Evaluasi menggunakan metrik RMSE/MAE.
\item
  Evaluasi menggunakan grafik.
\item
  Menghitung nilai beda prediksi dan observasi, dan dilakukan statistik
  deskriptif disertai membuat histogram nilai residu tersebut.
\end{enumerate}

\hypertarget{tahap-5-conclusion-and-interpretation}{%
\subsubsection{\texorpdfstring{Tahap 5: \emph{Conclusion and
Interpretation}}{Tahap 5: Conclusion and Interpretation}}\label{tahap-5-conclusion-and-interpretation}}

\begin{enumerate}
\def\labelenumi{\arabic{enumi}.}
\tightlist
\item
  Tahap ini menyimpulkan hasil model dan menginterpretasikan model NN.
\item
  Langkah yang bisa meningkatkan hasil model dengan mengurangi
  kemungkinan \emph{overfitting} ataupun \emph{underfitting}.
\end{enumerate}

    \hypertarget{tahap-0-pengaturan-awal}{%
\section{Tahap 0: Pengaturan Awal}\label{tahap-0-pengaturan-awal}}

Bagian ini merupakan pengaturan awal pribadi, dapat diabaikan.

    \begin{tcolorbox}[breakable, size=fbox, boxrule=1pt, pad at break*=1mm,colback=cellbackground, colframe=cellborder]
\prompt{In}{incolor}{ }{\boxspacing}
\begin{Verbatim}[commandchars=\\\{\}]
\PY{k+kn}{from} \PY{n+nn}{datetime} \PY{k+kn}{import} \PY{n}{datetime}

\PY{c+c1}{\PYZsh{}\PYZsh{}\PYZsh{}\PYZsh{} PROJECT DESCRIPTION}
\PY{n}{notebook\PYZus{}version} \PY{o}{=} \PY{l+s+s1}{\PYZsq{}}\PY{l+s+s1}{2.0.0}\PY{l+s+s1}{\PYZsq{}}
\PY{n}{notebook\PYZus{}title} \PY{o}{=} \PY{l+s+s1}{\PYZsq{}}\PY{l+s+s1}{kualitas\PYZus{}air\PYZus{}ann\PYZus{}so}\PY{l+s+s1}{\PYZsq{}} \PY{o}{+} \PY{l+s+s1}{\PYZsq{}}\PY{l+s+s1}{\PYZus{}}\PY{l+s+s1}{\PYZsq{}} \PY{o}{+} \PY{n}{notebook\PYZus{}version}
\PY{n}{prefix} \PY{o}{=} \PY{n}{datetime}\PY{o}{.}\PY{n}{utcnow}\PY{p}{(}\PY{p}{)}\PY{o}{.}\PY{n}{strftime}\PY{p}{(}\PY{l+s+s2}{\PYZdq{}}\PY{l+s+s2}{\PYZpc{}}\PY{l+s+s2}{Y}\PY{l+s+s2}{\PYZpc{}}\PY{l+s+s2}{m}\PY{l+s+si}{\PYZpc{}d}\PY{l+s+s2}{\PYZus{}}\PY{l+s+s2}{\PYZpc{}}\PY{l+s+s2}{H}\PY{l+s+s2}{\PYZpc{}}\PY{l+s+s2}{M}\PY{l+s+s2}{\PYZdq{}}\PY{p}{)}
\PY{n}{project\PYZus{}title} \PY{o}{=} \PY{n}{prefix} \PY{o}{+} \PY{l+s+s1}{\PYZsq{}}\PY{l+s+s1}{\PYZus{}}\PY{l+s+s1}{\PYZsq{}} \PY{o}{+} \PY{n}{notebook\PYZus{}title}

\PY{n+nb}{print}\PY{p}{(}\PY{l+s+sa}{f}\PY{l+s+s1}{\PYZsq{}}\PY{l+s+s1}{Judul Notebook: }\PY{l+s+si}{\PYZob{}}\PY{n}{notebook\PYZus{}title}\PY{l+s+si}{\PYZcb{}}\PY{l+s+s1}{\PYZsq{}}\PY{p}{)}
\PY{n+nb}{print}\PY{p}{(}\PY{l+s+sa}{f}\PY{l+s+s1}{\PYZsq{}}\PY{l+s+s1}{Judul Proyek: }\PY{l+s+si}{\PYZob{}}\PY{n}{project\PYZus{}title}\PY{l+s+si}{\PYZcb{}}\PY{l+s+s1}{\PYZsq{}}\PY{p}{)}
\end{Verbatim}
\end{tcolorbox}

    \begin{Verbatim}[commandchars=\\\{\}]
Judul Notebook: kualitas\_air\_ann\_so\_2.0.0
Judul Proyek: 20190713\_0507\_kualitas\_air\_ann\_so\_2.0.0
    \end{Verbatim}

    \begin{tcolorbox}[breakable, size=fbox, boxrule=1pt, pad at break*=1mm,colback=cellbackground, colframe=cellborder]
\prompt{In}{incolor}{ }{\boxspacing}
\begin{Verbatim}[commandchars=\\\{\}]
\PY{c+c1}{\PYZsh{}\PYZsh{}\PYZsh{}\PYZsh{} Memasang Akses Google Drive (untuk tempat menyimpan hasil training)}
\PY{k+kn}{from} \PY{n+nn}{google}\PY{n+nn}{.}\PY{n+nn}{colab} \PY{k+kn}{import} \PY{n}{drive}
\PY{n}{drive}\PY{o}{.}\PY{n}{mount}\PY{p}{(}\PY{l+s+s1}{\PYZsq{}}\PY{l+s+s1}{/content/gdrive}\PY{l+s+s1}{\PYZsq{}}\PY{p}{)}
\PY{n}{drop\PYZus{}path} \PY{o}{=} \PY{l+s+s1}{\PYZsq{}}\PY{l+s+s1}{/content/gdrive/My Drive/Colab Notebooks/\PYZus{}dropbox}\PY{l+s+s1}{\PYZsq{}}
\end{Verbatim}
\end{tcolorbox}

    \begin{tcolorbox}[breakable, size=fbox, boxrule=1pt, pad at break*=1mm,colback=cellbackground, colframe=cellborder]
\prompt{In}{incolor}{ }{\boxspacing}
\begin{Verbatim}[commandchars=\\\{\}]
\PY{c+c1}{\PYZsh{}\PYZsh{}\PYZsh{}\PYZsh{} Instalasi Paket Python Pribadi (untuk Logging)}
\PY{c+c1}{\PYZsh{}\PYZsh{}\PYZsh{}\PYZsh{} https://github.com/taruma/umakit}
\PY{o}{!}pip install umakit
\PY{k+kn}{from} \PY{n+nn}{umakit}\PY{n+nn}{.}\PY{n+nn}{logtool} \PY{k+kn}{import} \PY{n}{LogTool}
\PY{n}{mylog} \PY{o}{=} \PY{n}{LogTool}\PY{p}{(}\PY{p}{)}
\PY{n}{mylog}\PY{o}{.}\PY{n}{\PYZus{}reset}\PY{p}{(}\PY{p}{)}

\PY{c+c1}{\PYZsh{}\PYZsh{}\PYZsh{}\PYZsh{} Instalasi Paket Hidrokit (untuk plotting, dan transformasi kolom saat data preprocessing)}
\PY{c+c1}{\PYZsh{}\PYZsh{}\PYZsh{}\PYZsh{} https://github.com/taruma/hidrokit}
\PY{o}{!}pip install hidrokit
\end{Verbatim}
\end{tcolorbox}

    \begin{Verbatim}[commandchars=\\\{\}]
Collecting umakit
  Downloading https://files.pythonhosted.org/packages/76/64/d972d8fd9936a7a498c7
5e593d1986ec4d19204c84a636e5b5783c34806e/umakit-0.1.1-py3-none-any.whl
Installing collected packages: umakit
Successfully installed umakit-0.1.1
Collecting hidrokit
  Downloading https://files.pythonhosted.org/packages/43/9d/343d2a413a07463a21dd
13369e31d664d6733bbfd46276abef5d804c83d1/hidrokit-0.2.0-py2.py3-none-any.whl
Requirement already satisfied: matplotlib in /usr/local/lib/python3.6/dist-
packages (from hidrokit) (3.0.3)
Requirement already satisfied: pandas in /usr/local/lib/python3.6/dist-packages
(from hidrokit) (0.24.2)
Requirement already satisfied: numpy in /usr/local/lib/python3.6/dist-packages
(from hidrokit) (1.16.4)
Requirement already satisfied: python-dateutil>=2.1 in
/usr/local/lib/python3.6/dist-packages (from matplotlib->hidrokit) (2.5.3)
Requirement already satisfied: cycler>=0.10 in /usr/local/lib/python3.6/dist-
packages (from matplotlib->hidrokit) (0.10.0)
Requirement already satisfied: kiwisolver>=1.0.1 in
/usr/local/lib/python3.6/dist-packages (from matplotlib->hidrokit) (1.1.0)
Requirement already satisfied: pyparsing!=2.0.4,!=2.1.2,!=2.1.6,>=2.0.1 in
/usr/local/lib/python3.6/dist-packages (from matplotlib->hidrokit) (2.4.0)
Requirement already satisfied: pytz>=2011k in /usr/local/lib/python3.6/dist-
packages (from pandas->hidrokit) (2018.9)
Requirement already satisfied: six>=1.5 in /usr/local/lib/python3.6/dist-
packages (from python-dateutil>=2.1->matplotlib->hidrokit) (1.12.0)
Requirement already satisfied: setuptools in /usr/local/lib/python3.6/dist-
packages (from kiwisolver>=1.0.1->matplotlib->hidrokit) (41.0.1)
Installing collected packages: hidrokit
Successfully installed hidrokit-0.2.0
    \end{Verbatim}

    \hypertarget{tahap-1-data-exploration-and-data-munging}{%
\section{\texorpdfstring{Tahap 1: \emph{Data Exploration and Data
Munging}}{Tahap 1: Data Exploration and Data Munging}}\label{tahap-1-data-exploration-and-data-munging}}

Ringkasan pada tahap ini:

\begin{enumerate}
\def\labelenumi{\arabic{enumi}.}
\tightlist
\item
  \emph{Import} dataset.
\item
  Eksplorasi dataset.
\item
  Mengisi data yang hilang.
\item
  Eksplorasi dataset kembali.
\end{enumerate}

    \hypertarget{import-dataset}{%
\subsection{1.1 Import dataset}\label{import-dataset}}

    \begin{tcolorbox}[breakable, size=fbox, boxrule=1pt, pad at break*=1mm,colback=cellbackground, colframe=cellborder]
\prompt{In}{incolor}{ }{\boxspacing}
\begin{Verbatim}[commandchars=\\\{\}]
\PY{c+c1}{\PYZsh{}\PYZsh{}\PYZsh{}\PYZsh{} Mengunggah dataset}
\PY{k+kn}{from} \PY{n+nn}{google}\PY{n+nn}{.}\PY{n+nn}{colab} \PY{k+kn}{import} \PY{n}{files}
\PY{n}{uploaded} \PY{o}{=} \PY{n}{files}\PY{o}{.}\PY{n}{upload}\PY{p}{(}\PY{p}{)}
\end{Verbatim}
\end{tcolorbox}

    
    \begin{Verbatim}[commandchars=\\\{\}]
<IPython.core.display.HTML object>
    \end{Verbatim}

    
    \begin{Verbatim}[commandchars=\\\{\}]
Saving data\_ka.xlsx to data\_ka.xlsx
    \end{Verbatim}

    \begin{tcolorbox}[breakable, size=fbox, boxrule=1pt, pad at break*=1mm,colback=cellbackground, colframe=cellborder]
\prompt{In}{incolor}{ }{\boxspacing}
\begin{Verbatim}[commandchars=\\\{\}]
\PY{c+c1}{\PYZsh{}\PYZsh{}\PYZsh{}\PYZsh{} 1. Import dataset ke pandas.DataFrame}
\PY{k+kn}{import} \PY{n+nn}{pandas} \PY{k}{as} \PY{n+nn}{pd}

\PY{c+c1}{\PYZsh{}\PYZsh{} import dataset}
\PY{n}{dataset} \PY{o}{=} \PY{n}{pd}\PY{o}{.}\PY{n}{read\PYZus{}excel}\PY{p}{(}\PY{l+s+s1}{\PYZsq{}}\PY{l+s+s1}{data\PYZus{}ka.xlsx}\PY{l+s+s1}{\PYZsq{}}\PY{p}{,} \PY{n}{skiprows}\PY{o}{=}\PY{p}{[}\PY{l+m+mi}{0}\PY{p}{]}\PY{p}{)}

\PY{c+c1}{\PYZsh{}\PYZsh{} Menamai kolom}
\PY{n}{dataset}\PY{o}{.}\PY{n}{columns} \PY{o}{=} \PY{p}{[}\PY{l+s+s2}{\PYZdq{}}\PY{l+s+s2}{date}\PY{l+s+s2}{\PYZdq{}}\PY{p}{,} \PY{l+s+s2}{\PYZdq{}}\PY{l+s+s2}{temp\PYZus{}udara}\PY{l+s+s2}{\PYZdq{}}\PY{p}{,} \PY{l+s+s2}{\PYZdq{}}\PY{l+s+s2}{lama\PYZus{}sinar}\PY{l+s+s2}{\PYZdq{}}\PY{p}{,} \PY{l+s+s2}{\PYZdq{}}\PY{l+s+s2}{kec\PYZus{}angin}\PY{l+s+s2}{\PYZdq{}}\PY{p}{,} 
                   \PY{l+s+s2}{\PYZdq{}}\PY{l+s+s2}{debit\PYZus{}masuk}\PY{l+s+s2}{\PYZdq{}}\PY{p}{,} \PY{l+s+s2}{\PYZdq{}}\PY{l+s+s2}{debit\PYZus{}keluar}\PY{l+s+s2}{\PYZdq{}}\PY{p}{,} \PY{l+s+s2}{\PYZdq{}}\PY{l+s+s2}{volume}\PY{l+s+s2}{\PYZdq{}}\PY{p}{,} \PY{l+s+s2}{\PYZdq{}}\PY{l+s+s2}{temp\PYZus{}air}\PY{l+s+s2}{\PYZdq{}}\PY{p}{,} 
                   \PY{l+s+s2}{\PYZdq{}}\PY{l+s+s2}{oksigen}\PY{l+s+s2}{\PYZdq{}}\PY{p}{,} \PY{l+s+s2}{\PYZdq{}}\PY{l+s+s2}{nitrogen}\PY{l+s+s2}{\PYZdq{}}\PY{p}{,} \PY{l+s+s2}{\PYZdq{}}\PY{l+s+s2}{nitrat}\PY{l+s+s2}{\PYZdq{}}\PY{p}{,} \PY{l+s+s2}{\PYZdq{}}\PY{l+s+s2}{amonia}\PY{l+s+s2}{\PYZdq{}}\PY{p}{,} 
                   \PY{l+s+s2}{\PYZdq{}}\PY{l+s+s2}{out\PYZus{}nitrogen}\PY{l+s+s2}{\PYZdq{}}\PY{p}{,} \PY{l+s+s2}{\PYZdq{}}\PY{l+s+s2}{out\PYZus{}nitrat}\PY{l+s+s2}{\PYZdq{}}\PY{p}{,} \PY{l+s+s2}{\PYZdq{}}\PY{l+s+s2}{out\PYZus{}amonia}\PY{l+s+s2}{\PYZdq{}}\PY{p}{]}

\PY{c+c1}{\PYZsh{}\PYZsh{} Mengatur index dataframe ke tanggal}
\PY{n}{dataset} \PY{o}{=} \PY{n}{dataset}\PY{o}{.}\PY{n}{set\PYZus{}index}\PY{p}{(}\PY{l+s+s1}{\PYZsq{}}\PY{l+s+s1}{date}\PY{l+s+s1}{\PYZsq{}}\PY{p}{)}

\PY{n}{dataset}\PY{o}{.}\PY{n}{head}\PY{p}{(}\PY{l+m+mi}{5}\PY{p}{)}
\end{Verbatim}
\end{tcolorbox}

            \begin{tcolorbox}[breakable, size=fbox, boxrule=.5pt, pad at break*=1mm, opacityfill=0]
\prompt{Out}{outcolor}{ }{\boxspacing}
\begin{Verbatim}[commandchars=\\\{\}]
            temp\_udara  lama\_sinar  {\ldots}  out\_nitrat  out\_amonia
date                                {\ldots}
2000-06-20        26.5    6.050000  {\ldots}    0.524643    0.589523
2000-07-20        25.2    4.366667  {\ldots}    0.727771    0.602750
2000-08-20        26.9    7.616667  {\ldots}    0.742150    0.627844
2000-09-20        26.4    5.100000  {\ldots}    0.457229    0.430150
2000-10-20        25.2    6.233333  {\ldots}    0.461479    0.559891

[5 rows x 14 columns]
\end{Verbatim}
\end{tcolorbox}
        
    \hypertarget{eksplorasi-dataset}{%
\subsection{1.2 Eksplorasi dataset}\label{eksplorasi-dataset}}

    \begin{tcolorbox}[breakable, size=fbox, boxrule=1pt, pad at break*=1mm,colback=cellbackground, colframe=cellborder]
\prompt{In}{incolor}{ }{\boxspacing}
\begin{Verbatim}[commandchars=\\\{\}]
\PY{c+c1}{\PYZsh{}\PYZsh{} Gamabaran umum dataset}
\PY{n}{dataset}\PY{o}{.}\PY{n}{info}\PY{p}{(}\PY{p}{)}
\end{Verbatim}
\end{tcolorbox}

    \begin{Verbatim}[commandchars=\\\{\}]
<class 'pandas.core.frame.DataFrame'>
DatetimeIndex: 211 entries, 2000-06-20 to 2017-12-13
Data columns (total 14 columns):
temp\_udara      210 non-null float64
lama\_sinar      210 non-null float64
kec\_angin       210 non-null float64
debit\_masuk     210 non-null float64
debit\_keluar    210 non-null float64
volume          210 non-null float64
temp\_air        209 non-null float64
oksigen         210 non-null float64
nitrogen        189 non-null float64
nitrat          208 non-null float64
amonia          210 non-null float64
out\_nitrogen    210 non-null float64
out\_nitrat      208 non-null float64
out\_amonia      210 non-null float64
dtypes: float64(14)
memory usage: 24.7 KB
    \end{Verbatim}

    Dari dataset dapat disimpulkan: 1. Kolom \texttt{date} digunakan sebagai
acuan total jumlah data yang tersedia. Diketahui bahwa ada \textbf{211
data}. 2. Kolom \texttt{temp\_udara}, \texttt{lama\_sinar},
\texttt{kec\_angin}, \texttt{debit\_masuk}, \texttt{debit\_keluar},
\texttt{volume}, \texttt{temp\_air}, \texttt{amonia},
\texttt{out\_nitrogen}, \texttt{out\_amonia} \textbf{kehilangan 1 data,
sehingga jumlah datanya hanya 210}. 3. Kolom berikut memiliki kehilangan
data lebih dari 1 data: - \texttt{temp\_air}: total: 209 data,
\textbf{kehilangan 2 data}. - \texttt{nitrogen}: total: 189 data,
\textbf{kehilangan 22 data}. - \texttt{nitrat}: total: 208 data,
\textbf{kehilangan 3 data}. - \texttt{out\_nitrat}: total: 208 data,
\textbf{kehilangan 3 data}.

    \hypertarget{visualisasi}{%
\subsubsection{Visualisasi}\label{visualisasi}}

    \begin{tcolorbox}[breakable, size=fbox, boxrule=1pt, pad at break*=1mm,colback=cellbackground, colframe=cellborder]
\prompt{In}{incolor}{ }{\boxspacing}
\begin{Verbatim}[commandchars=\\\{\}]
\PY{k+kn}{from} \PY{n+nn}{hidrokit}\PY{n+nn}{.}\PY{n+nn}{viz} \PY{k+kn}{import} \PY{n}{graph}
\PY{n}{graph}\PY{o}{.}\PY{n}{subplots}\PY{p}{(}\PY{n}{dataset}\PY{p}{,} \PY{n}{ncols}\PY{o}{=}\PY{l+m+mi}{1}\PY{p}{,} \PY{n}{nrows}\PY{o}{=}\PY{l+m+mi}{14}\PY{p}{,} \PY{n}{figsize}\PY{o}{=}\PY{p}{(}\PY{l+m+mi}{15}\PY{p}{,} \PY{l+m+mi}{15}\PY{p}{)}\PY{p}{)}\PY{p}{;}
\end{Verbatim}
\end{tcolorbox}

    \begin{center}
    \adjustimage{max size={0.9\linewidth}{0.9\paperheight}}{taruma_LI01_ann_ka_files/taruma_LI01_ann_ka_17_0.png}
    \end{center}
    { \hspace*{\fill} \\}
    
    Dari grafik diatas terlihat ada data yang hilang.

    \hypertarget{kehilangan-data}{%
\subsection{1.3 Kehilangan Data}\label{kehilangan-data}}

    \begin{tcolorbox}[breakable, size=fbox, boxrule=1pt, pad at break*=1mm,colback=cellbackground, colframe=cellborder]
\prompt{In}{incolor}{ }{\boxspacing}
\begin{Verbatim}[commandchars=\\\{\}]
\PY{c+c1}{\PYZsh{}\PYZsh{} Memperoleh informasi kehilangan data}
\PY{k+kn}{from} \PY{n+nn}{hidrokit}\PY{n+nn}{.}\PY{n+nn}{prep} \PY{k+kn}{import} \PY{n}{read}

\PY{n}{data\PYZus{}hilang} \PY{o}{=} \PY{n}{read}\PY{o}{.}\PY{n}{missing\PYZus{}row}\PY{p}{(}\PY{n}{dataset}\PY{p}{,} \PY{n}{date\PYZus{}format}\PY{o}{=}\PY{l+s+s1}{\PYZsq{}}\PY{l+s+s1}{\PYZpc{}}\PY{l+s+s1}{Y/}\PY{l+s+s1}{\PYZpc{}}\PY{l+s+s1}{m}\PY{l+s+s1}{\PYZsq{}}\PY{p}{)}

\PY{n+nb}{print}\PY{p}{(}\PY{l+s+s2}{\PYZdq{}}\PY{l+s+s2}{Daftar kehilangan data:}\PY{l+s+s2}{\PYZdq{}}\PY{p}{)}
\PY{k}{for} \PY{n}{column}\PY{p}{,} \PY{n}{value} \PY{o+ow}{in} \PY{n}{data\PYZus{}hilang}\PY{o}{.}\PY{n}{items}\PY{p}{(}\PY{p}{)}\PY{p}{:}
    \PY{n+nb}{print}\PY{p}{(}
        \PY{l+s+s2}{\PYZdq{}}\PY{l+s+s2}{Kolom }\PY{l+s+si}{\PYZob{}\PYZcb{}}\PY{l+s+s2}{: }\PY{l+s+si}{\PYZob{}\PYZcb{}}\PY{l+s+s2}{\PYZdq{}}\PY{o}{.}\PY{n}{format}\PY{p}{(}\PY{n}{column}\PY{p}{,} \PY{n}{value}\PY{p}{)}
    \PY{p}{)}
\end{Verbatim}
\end{tcolorbox}

    \begin{Verbatim}[commandchars=\\\{\}]
Daftar kehilangan data:
Kolom temp\_udara: ['2005/12']
Kolom lama\_sinar: ['2005/12']
Kolom kec\_angin: ['2005/12']
Kolom debit\_masuk: ['2005/12']
Kolom debit\_keluar: ['2005/12']
Kolom volume: ['2005/12']
Kolom temp\_air: ['2005/11', '2005/12']
Kolom oksigen: ['2005/12']
Kolom nitrogen: ['2005/12', '2007/04', '2007/05', '2007/06', '2007/07',
'2007/08', '2007/09', '2007/10', '2007/11', '2007/12', '2008/01', '2008/02',
'2008/03', '2008/04', '2008/05', '2008/06', '2008/07', '2008/08', '2008/09',
'2008/10', '2008/11', '2008/12']
Kolom nitrat: ['2005/06', '2005/07', '2005/12']
Kolom amonia: ['2005/12']
Kolom out\_nitrogen: ['2005/12']
Kolom out\_nitrat: ['2005/06', '2005/07', '2005/12']
Kolom out\_amonia: ['2005/12']
    \end{Verbatim}

    Dari proses diatas diperoleh informasi: 1. Pada bulan \texttt{2005/12}
(Desember 2005), data tidak tersedia pada \textbf{seluruh kolom}. 2.
Kolom \texttt{temp\_air}, kehilangan data pada \texttt{2005/11}
(November 2005). 3. Kolom \texttt{nitrogen}, kehilangan data pada dari
\texttt{2007/04} (April 2007) sampai \texttt{2008/12} (Desember 2008).
4. Kolom \texttt{nitrat} dan \texttt{out\_amonia}, kehilangan data pada
\texttt{2005/06} (Juni 2005) dan \texttt{2005/07} (Juli 2005).`

    \hypertarget{mengisi-kehilangan-data}{%
\subsubsection{Mengisi Kehilangan data}\label{mengisi-kehilangan-data}}

\textbf{Diasumsikan} bahwa data yang hilang dapat diisi dengan
\textbf{metode interpolasi linear}. Dengan catatan metode tersebut
digunakan dalam notebook ini untuk \textbf{pembelajaran}. Dimungkinkan
untuk mengisi data hilang dengan teknik yang tersedia dalam bidang
keahliannya.

    \begin{tcolorbox}[breakable, size=fbox, boxrule=1pt, pad at break*=1mm,colback=cellbackground, colframe=cellborder]
\prompt{In}{incolor}{ }{\boxspacing}
\begin{Verbatim}[commandchars=\\\{\}]
\PY{c+c1}{\PYZsh{}\PYZsh{} Mengisi Data yang hilang dengan metode linear}
\PY{n}{new\PYZus{}dataset} \PY{o}{=} \PY{n}{dataset}\PY{o}{.}\PY{n}{interpolate}\PY{p}{(}\PY{n}{method}\PY{o}{=}\PY{l+s+s1}{\PYZsq{}}\PY{l+s+s1}{linear}\PY{l+s+s1}{\PYZsq{}}\PY{p}{)}
\end{Verbatim}
\end{tcolorbox}

    \hypertarget{eksplorasi-dataset}{%
\subsection{1.4 Eksplorasi dataset}\label{eksplorasi-dataset}}

    \begin{tcolorbox}[breakable, size=fbox, boxrule=1pt, pad at break*=1mm,colback=cellbackground, colframe=cellborder]
\prompt{In}{incolor}{ }{\boxspacing}
\begin{Verbatim}[commandchars=\\\{\}]
\PY{c+c1}{\PYZsh{} statistik deskriptif}
\PY{n}{new\PYZus{}dataset}\PY{o}{.}\PY{n}{describe}\PY{p}{(}\PY{p}{)}
\end{Verbatim}
\end{tcolorbox}

            \begin{tcolorbox}[breakable, size=fbox, boxrule=.5pt, pad at break*=1mm, opacityfill=0]
\prompt{Out}{outcolor}{ }{\boxspacing}
\begin{Verbatim}[commandchars=\\\{\}]
       temp\_udara  lama\_sinar   kec\_angin  {\ldots}  out\_nitrogen  out\_nitrat
out\_amonia
count  211.000000  211.000000  211.000000  {\ldots}    211.000000  211.000000
211.000000
mean    26.297867    5.429739   89.090929  {\ldots}      0.124222    0.369451
0.258836
std      1.230595    2.653354   56.941944  {\ldots}      0.216860    0.591848
0.360054
min     22.400000    0.000000    0.400000  {\ldots}      0.000000    0.004667
0.000000
25\%     25.750000    3.591667   47.350000  {\ldots}      0.006673    0.052589
0.002378
50\%     26.200000    5.750000   83.300000  {\ldots}      0.019494    0.143512
0.066165
75\%     27.000000    7.683333  125.400000  {\ldots}      0.085891    0.452636
0.448939
max     29.150000   10.083333  365.900000  {\ldots}      1.164635    5.460060
2.252332

[8 rows x 14 columns]
\end{Verbatim}
\end{tcolorbox}
        
    \hypertarget{visualisasi}{%
\subsubsection{Visualisasi}\label{visualisasi}}

    \begin{tcolorbox}[breakable, size=fbox, boxrule=1pt, pad at break*=1mm,colback=cellbackground, colframe=cellborder]
\prompt{In}{incolor}{ }{\boxspacing}
\begin{Verbatim}[commandchars=\\\{\}]
\PY{n}{graph}\PY{o}{.}\PY{n}{subplots}\PY{p}{(}\PY{n}{new\PYZus{}dataset}\PY{p}{,} \PY{n}{ncols}\PY{o}{=}\PY{l+m+mi}{1}\PY{p}{,} \PY{n}{nrows}\PY{o}{=}\PY{l+m+mi}{14}\PY{p}{,} \PY{n}{figsize}\PY{o}{=}\PY{p}{(}\PY{l+m+mi}{15}\PY{p}{,} \PY{l+m+mi}{15}\PY{p}{)}\PY{p}{)}\PY{p}{;}
\end{Verbatim}
\end{tcolorbox}

    \begin{center}
    \adjustimage{max size={0.9\linewidth}{0.9\paperheight}}{taruma_LI01_ann_ka_files/taruma_LI01_ann_ka_27_0.png}
    \end{center}
    { \hspace*{\fill} \\}
    
    \begin{tcolorbox}[breakable, size=fbox, boxrule=1pt, pad at break*=1mm,colback=cellbackground, colframe=cellborder]
\prompt{In}{incolor}{ }{\boxspacing}
\begin{Verbatim}[commandchars=\\\{\}]
\PY{c+c1}{\PYZsh{}\PYZsh{} Pairplot untuk melihat hubungan masing\PYZhy{}masing kolom}
\PY{k+kn}{import} \PY{n+nn}{seaborn} \PY{k}{as} \PY{n+nn}{sns}

\PY{n}{g} \PY{o}{=} \PY{n}{sns}\PY{o}{.}\PY{n}{pairplot}\PY{p}{(}\PY{n}{new\PYZus{}dataset}\PY{p}{)}
\PY{n}{g}\PY{o}{.}\PY{n}{fig}\PY{o}{.}\PY{n}{set\PYZus{}size\PYZus{}inches}\PY{p}{(}\PY{l+m+mi}{15}\PY{p}{,}\PY{l+m+mi}{15}\PY{p}{)}
\end{Verbatim}
\end{tcolorbox}

    \begin{center}
    \adjustimage{max size={0.9\linewidth}{0.9\paperheight}}{taruma_LI01_ann_ka_files/taruma_LI01_ann_ka_28_0.png}
    \end{center}
    { \hspace*{\fill} \\}
    
    Dari visualisasi diatas diketahui bahwa terdapat kolom yang memiliki
distribusi tidak normal, sehingga dalam \emph{scaling} digunakan
\texttt{MinMaxScaler}.

    \hypertarget{tahap-2-data-preprocessing}{%
\section{\texorpdfstring{Tahap 2: \emph{Data
Preprocessing}}{Tahap 2: Data Preprocessing}}\label{tahap-2-data-preprocessing}}

\begin{enumerate}
\def\labelenumi{\arabic{enumi}.}
\tightlist
\item
  Membagi dataset menjadi \emph{training set} dan \emph{test set}.
\item
  \emph{Scaling} dengan \texttt{MinMaxScaler}.
\item
  Menambahkan kolom \emph{timesteps}.
\item
  Menampilkan hasil penambahan \emph{timesteps}.
\item
  Mempersiapkan data \emph{training} dari \emph{training set} menjadi
  \texttt{X\_train} dan \texttt{y\_train}.
\end{enumerate}

    \begin{tcolorbox}[breakable, size=fbox, boxrule=1pt, pad at break*=1mm,colback=cellbackground, colframe=cellborder]
\prompt{In}{incolor}{ }{\boxspacing}
\begin{Verbatim}[commandchars=\\\{\}]
\PY{c+c1}{\PYZsh{}\PYZsh{}\PYZsh{}\PYZsh{} Import Libraries}
\PY{k+kn}{import} \PY{n+nn}{numpy} \PY{k}{as} \PY{n+nn}{np}
\PY{k+kn}{import} \PY{n+nn}{matplotlib}\PY{n+nn}{.}\PY{n+nn}{pyplot} \PY{k}{as} \PY{n+nn}{plt}
\PY{k+kn}{import} \PY{n+nn}{pandas} \PY{k}{as} \PY{n+nn}{pd}
\end{Verbatim}
\end{tcolorbox}

    \hypertarget{training-set-dan-test-set}{%
\subsection{\texorpdfstring{2.1 \emph{Training set} dan \emph{Test
set}}{2.1 Training set dan Test set}}\label{training-set-dan-test-set}}

Dari \texttt{new\_dataset} yang telah diinterpolasi, akan dibagi menjadi
dua bagian data yaitu training dan testing.

\begin{itemize}
\tightlist
\item
  Training data merupakan data dari \texttt{2000/06} hingga
  \texttt{2014/12}
\item
  Testing data merupakan data dari \texttt{2015/01} hingga
  \texttt{2017/12}
\end{itemize}

    \begin{tcolorbox}[breakable, size=fbox, boxrule=1pt, pad at break*=1mm,colback=cellbackground, colframe=cellborder]
\prompt{In}{incolor}{ }{\boxspacing}
\begin{Verbatim}[commandchars=\\\{\}]
\PY{c+c1}{\PYZsh{}\PYZsh{} Memisahkan dataset training dan test}

\PY{n}{training\PYZus{}dataset} \PY{o}{=} \PY{n}{new\PYZus{}dataset}\PY{o}{.}\PY{n}{loc}\PY{p}{[}\PY{p}{:}\PY{l+s+s2}{\PYZdq{}}\PY{l+s+s2}{20141231}\PY{l+s+s2}{\PYZdq{}}\PY{p}{,} \PY{p}{:}\PY{p}{]}
\PY{n}{test\PYZus{}dataset} \PY{o}{=} \PY{n}{new\PYZus{}dataset}\PY{o}{.}\PY{n}{loc}\PY{p}{[}\PY{l+s+s2}{\PYZdq{}}\PY{l+s+s2}{20150101}\PY{l+s+s2}{\PYZdq{}}\PY{p}{:}\PY{p}{,} \PY{p}{:}\PY{p}{]}

\PY{c+c1}{\PYZsh{}\PYZsh{} Informasi training set dan testing set}
\PY{n+nb}{print}\PY{p}{(}\PY{l+s+s2}{\PYZdq{}}\PY{l+s+s2}{Informasi training set: }\PY{l+s+si}{\PYZob{}\PYZcb{}}\PY{l+s+s2}{ baris, }\PY{l+s+si}{\PYZob{}\PYZcb{}}\PY{l+s+s2}{ kolom}\PY{l+s+s2}{\PYZdq{}}\PY{o}{.}\PY{n}{format}\PY{p}{(}
    \PY{n}{training\PYZus{}dataset}\PY{o}{.}\PY{n}{shape}\PY{p}{[}\PY{l+m+mi}{0}\PY{p}{]}\PY{p}{,} \PY{n}{training\PYZus{}dataset}\PY{o}{.}\PY{n}{shape}\PY{p}{[}\PY{l+m+mi}{1}\PY{p}{]}\PY{p}{)}
     \PY{p}{)}
\PY{n+nb}{print}\PY{p}{(}\PY{l+s+s2}{\PYZdq{}}\PY{l+s+s2}{Informasi testing set: }\PY{l+s+si}{\PYZob{}\PYZcb{}}\PY{l+s+s2}{ baris, }\PY{l+s+si}{\PYZob{}\PYZcb{}}\PY{l+s+s2}{ kolom}\PY{l+s+s2}{\PYZdq{}}\PY{o}{.}\PY{n}{format}\PY{p}{(}
    \PY{n}{test\PYZus{}dataset}\PY{o}{.}\PY{n}{shape}\PY{p}{[}\PY{l+m+mi}{0}\PY{p}{]}\PY{p}{,} \PY{n}{test\PYZus{}dataset}\PY{o}{.}\PY{n}{shape}\PY{p}{[}\PY{l+m+mi}{1}\PY{p}{]}\PY{p}{)}
     \PY{p}{)}

\PY{c+c1}{\PYZsh{}\PYZsh{} Menampilkan training set}
\PY{n}{training\PYZus{}dataset}\PY{o}{.}\PY{n}{head}\PY{p}{(}\PY{p}{)}
\end{Verbatim}
\end{tcolorbox}

    \begin{Verbatim}[commandchars=\\\{\}]
Informasi training set: 175 baris, 14 kolom
Informasi testing set: 36 baris, 14 kolom
    \end{Verbatim}

            \begin{tcolorbox}[breakable, size=fbox, boxrule=.5pt, pad at break*=1mm, opacityfill=0]
\prompt{Out}{outcolor}{ }{\boxspacing}
\begin{Verbatim}[commandchars=\\\{\}]
            temp\_udara  lama\_sinar  {\ldots}  out\_nitrat  out\_amonia
date                                {\ldots}
2000-06-20        26.5    6.050000  {\ldots}    0.524643    0.589523
2000-07-20        25.2    4.366667  {\ldots}    0.727771    0.602750
2000-08-20        26.9    7.616667  {\ldots}    0.742150    0.627844
2000-09-20        26.4    5.100000  {\ldots}    0.457229    0.430150
2000-10-20        25.2    6.233333  {\ldots}    0.461479    0.559891

[5 rows x 14 columns]
\end{Verbatim}
\end{tcolorbox}
        
    Menyimpan nilai dalam \texttt{dataframe} dalam bentuk
\texttt{numpy.array}.

    \begin{tcolorbox}[breakable, size=fbox, boxrule=1pt, pad at break*=1mm,colback=cellbackground, colframe=cellborder]
\prompt{In}{incolor}{ }{\boxspacing}
\begin{Verbatim}[commandchars=\\\{\}]
\PY{n}{array\PYZus{}train} \PY{o}{=} \PY{n}{training\PYZus{}dataset}\PY{o}{.}\PY{n}{values}
\PY{n}{array\PYZus{}train}\PY{p}{[}\PY{p}{:}\PY{l+m+mi}{1}\PY{p}{,} \PY{p}{:}\PY{p}{]}
\end{Verbatim}
\end{tcolorbox}

            \begin{tcolorbox}[breakable, size=fbox, boxrule=.5pt, pad at break*=1mm, opacityfill=0]
\prompt{Out}{outcolor}{ }{\boxspacing}
\begin{Verbatim}[commandchars=\\\{\}]
array([[2.65000000e+01, 6.05000000e+00, 1.40000000e+00, 9.13500000e+01,
        1.68320000e+02, 2.33428460e+09, 2.39375000e+01, 5.06250000e+00,
        2.35750000e-01, 4.81875000e-01, 8.25375000e-01, 3.26914583e-01,
        5.24642857e-01, 5.89522917e-01]])
\end{Verbatim}
\end{tcolorbox}
        
    \hypertarget{scaling}{%
\subsection{\texorpdfstring{2.2
\emph{Scaling}}{2.2 Scaling}}\label{scaling}}

Skala yang digunakan adalah \texttt{MinMaxScaler}.

    \begin{tcolorbox}[breakable, size=fbox, boxrule=1pt, pad at break*=1mm,colback=cellbackground, colframe=cellborder]
\prompt{In}{incolor}{ }{\boxspacing}
\begin{Verbatim}[commandchars=\\\{\}]
\PY{c+c1}{\PYZsh{}\PYZsh{}\PYZsh{}\PYZsh{} Scaling dataset}
\PY{k+kn}{from} \PY{n+nn}{sklearn}\PY{n+nn}{.}\PY{n+nn}{preprocessing} \PY{k+kn}{import} \PY{n}{MinMaxScaler}
\PY{n}{sc} \PY{o}{=} \PY{n}{MinMaxScaler}\PY{p}{(}\PY{n}{feature\PYZus{}range}\PY{o}{=}\PY{p}{(}\PY{l+m+mi}{0}\PY{p}{,}\PY{l+m+mi}{1}\PY{p}{)}\PY{p}{)}
\PY{n}{array\PYZus{}train} \PY{o}{=} \PY{n}{sc}\PY{o}{.}\PY{n}{fit\PYZus{}transform}\PY{p}{(}\PY{n}{array\PYZus{}train}\PY{p}{)}
\PY{n}{array\PYZus{}train}\PY{p}{[}\PY{p}{:}\PY{l+m+mi}{1}\PY{p}{,} \PY{p}{:}\PY{p}{]}
\end{Verbatim}
\end{tcolorbox}

            \begin{tcolorbox}[breakable, size=fbox, boxrule=.5pt, pad at break*=1mm, opacityfill=0]
\prompt{Out}{outcolor}{ }{\boxspacing}
\begin{Verbatim}[commandchars=\\\{\}]
array([[0.67213115, 0.6       , 0.00273598, 0.09964254, 0.17944199,
        0.738755  , 0.31118314, 0.50451442, 0.19395311, 0.08149884,
        0.19043453, 0.28070132, 0.09466023, 0.26173895]])
\end{Verbatim}
\end{tcolorbox}
        
    \hypertarget{kolom-timesteps}{%
\subsection{\texorpdfstring{2.3 Kolom
\emph{Timesteps}}{2.3 Kolom Timesteps}}\label{kolom-timesteps}}

Setelah \emph{scaling}, dataset dibuat kolom tambahan yang
merepresentasikan data \emph{timesteps} sebelumnya. Untuk kasus ini
dipilih timesteps sebanyak 2 bulan sebelumnya. Sehingga persamaan
umumnya berupa:

\(Output_t = f(Input_t, Input_{t-1}, Input_{t-2}, Output_{t-1}, Output_{t-2})\)

Diketahui terdapat 14 kolom (termasuk output) dengan \emph{timesteps} 2,
sehingga dimensi \texttt{array} menjadi \(14 \times (2+1)=42\).

\textbf{Catatan}: - Karena hidrokit menggunakan input dalam bentuk
DataFrame, hasil \emph{scaling} yang masih berbentuk array dibuat
DataFramenya. - Dapat digunakan langsung dengan input array dengan
menggunakan \emph{private function}
\texttt{\_multi\_column\_timesteps()}

    \begin{tcolorbox}[breakable, size=fbox, boxrule=1pt, pad at break*=1mm,colback=cellbackground, colframe=cellborder]
\prompt{In}{incolor}{ }{\boxspacing}
\begin{Verbatim}[commandchars=\\\{\}]
\PY{c+c1}{\PYZsh{} Membuat dataframe baru setelah proses scaling}
\PY{n}{training\PYZus{}dataset\PYZus{}scale} \PY{o}{=} \PY{n}{pd}\PY{o}{.}\PY{n}{DataFrame}\PY{p}{(}
    \PY{n}{data}\PY{o}{=}\PY{n}{array\PYZus{}train}\PY{p}{,}
    \PY{n}{columns}\PY{o}{=}\PY{n}{training\PYZus{}dataset}\PY{o}{.}\PY{n}{columns}\PY{p}{,}
    \PY{n}{index}\PY{o}{=}\PY{n}{training\PYZus{}dataset}\PY{o}{.}\PY{n}{index}
\PY{p}{)}

\PY{c+c1}{\PYZsh{} Membuat tabel timesteps}
\PY{k+kn}{from} \PY{n+nn}{hidrokit}\PY{n+nn}{.}\PY{n+nn}{prep} \PY{k+kn}{import} \PY{n}{timeseries}

\PY{n}{n\PYZus{}timesteps} \PY{o}{=} \PY{l+m+mi}{2}
\PY{n}{df\PYZus{}train\PYZus{}ts} \PY{o}{=} \PY{n}{timeseries}\PY{o}{.}\PY{n}{timestep\PYZus{}table}\PY{p}{(}\PY{n}{training\PYZus{}dataset\PYZus{}scale}\PY{p}{,} \PY{n}{timesteps}\PY{o}{=}\PY{n}{n\PYZus{}timesteps}\PY{p}{)}
\PY{n}{array\PYZus{}train\PYZus{}ts} \PY{o}{=} \PY{n}{df\PYZus{}train\PYZus{}ts}\PY{o}{.}\PY{n}{values}

\PY{n+nb}{print}\PY{p}{(}\PY{l+s+s2}{\PYZdq{}}\PY{l+s+s2}{Dimensi array setelah diberi kolom timesteps: }\PY{l+s+si}{\PYZob{}\PYZcb{}}\PY{l+s+s2}{\PYZdq{}}\PY{o}{.}\PY{n}{format}\PY{p}{(}\PY{n}{array\PYZus{}train\PYZus{}ts}\PY{o}{.}\PY{n}{shape}\PY{p}{)}\PY{p}{)}
\end{Verbatim}
\end{tcolorbox}

    \begin{Verbatim}[commandchars=\\\{\}]
Dimensi array setelah diberi kolom timesteps: (173, 42)
    \end{Verbatim}

    \hypertarget{tabel-training-set}{%
\subsection{\texorpdfstring{2.4 Tabel \emph{Training
set}}{2.4 Tabel Training set}}\label{tabel-training-set}}

Tabel \texttt{pandas.DataFrame} untuk \emph{training set} yang telah
diberi kolom \emph{timesteps}.

    \begin{tcolorbox}[breakable, size=fbox, boxrule=1pt, pad at break*=1mm,colback=cellbackground, colframe=cellborder]
\prompt{In}{incolor}{ }{\boxspacing}
\begin{Verbatim}[commandchars=\\\{\}]
\PY{c+c1}{\PYZsh{}\PYZsh{} Menampilkan hasil pemberian kolom timestep dengan pandas.DataFrame}
\PY{n}{df\PYZus{}train\PYZus{}ts}\PY{o}{.}\PY{n}{head}\PY{p}{(}\PY{p}{)}
\end{Verbatim}
\end{tcolorbox}

            \begin{tcolorbox}[breakable, size=fbox, boxrule=.5pt, pad at break*=1mm, opacityfill=0]
\prompt{Out}{outcolor}{ }{\boxspacing}
\begin{Verbatim}[commandchars=\\\{\}]
            temp\_udara\_tmin0  {\ldots}  out\_amonia\_tmin2
date                          {\ldots}
2000-08-20          0.737705  {\ldots}          0.261739
2000-09-20          0.655738  {\ldots}          0.267612
2000-10-20          0.459016  {\ldots}          0.278753
2000-11-20          0.475410  {\ldots}          0.190980
2000-12-20          0.803279  {\ldots}          0.248583

[5 rows x 42 columns]
\end{Verbatim}
\end{tcolorbox}
        
    \hypertarget{x_train-dan-y_train}{%
\subsection{\texorpdfstring{2.5 \texttt{X\_train} dan
\texttt{y\_train}}{2.5 X\_train dan y\_train}}\label{x_train-dan-y_train}}

Berdasarkan persamaan sebelumnya, kolom \emph{output} pada waktu \(t\)
harus dipisahkan sebagai label \emph{training}-nya. Dan karena dalam
kasus ini fokus pada \emph{single-output}, maka kolom \emph{output} yang
digunakan sebagai \emph{target} hanya \texttt{out\_amonia\_tmin0}.

    \begin{tcolorbox}[breakable, size=fbox, boxrule=1pt, pad at break*=1mm,colback=cellbackground, colframe=cellborder]
\prompt{In}{incolor}{ }{\boxspacing}
\begin{Verbatim}[commandchars=\\\{\}]
\PY{c+c1}{\PYZsh{}\PYZsh{} Pembagian X\PYZus{}train dan y\PYZus{}train untuk }
\PY{c+c1}{\PYZsh{}\PYZsh{} Kasus Single\PYZhy{}Output Regression Neural Network}

\PY{n}{target\PYZus{}col} \PY{o}{=} \PY{p}{[}\PY{l+s+s2}{\PYZdq{}}\PY{l+s+s2}{out\PYZus{}amonia\PYZus{}tmin0}\PY{l+s+s2}{\PYZdq{}}\PY{p}{]}
\PY{n}{drop\PYZus{}col} \PY{o}{=} \PY{p}{[}\PY{l+s+s2}{\PYZdq{}}\PY{l+s+s2}{out\PYZus{}nitrogen\PYZus{}tmin0}\PY{l+s+s2}{\PYZdq{}}\PY{p}{,} \PY{l+s+s2}{\PYZdq{}}\PY{l+s+s2}{out\PYZus{}nitrat\PYZus{}tmin0}\PY{l+s+s2}{\PYZdq{}}\PY{p}{,} \PY{l+s+s2}{\PYZdq{}}\PY{l+s+s2}{out\PYZus{}amonia\PYZus{}tmin0}\PY{l+s+s2}{\PYZdq{}}\PY{p}{]}
\PY{n}{df\PYZus{}X\PYZus{}train} \PY{o}{=} \PY{n}{df\PYZus{}train\PYZus{}ts}\PY{o}{.}\PY{n}{drop}\PY{p}{(}\PY{n}{drop\PYZus{}col}\PY{p}{,} \PY{n}{axis}\PY{o}{=}\PY{l+m+mi}{1}\PY{p}{)}
\PY{n}{df\PYZus{}y\PYZus{}train} \PY{o}{=} \PY{n}{df\PYZus{}train\PYZus{}ts}\PY{p}{[}\PY{n}{target\PYZus{}col}\PY{p}{]}
\PY{n}{X\PYZus{}train} \PY{o}{=} \PY{n}{df\PYZus{}X\PYZus{}train}\PY{o}{.}\PY{n}{values}
\PY{n}{y\PYZus{}train} \PY{o}{=} \PY{n}{df\PYZus{}y\PYZus{}train}\PY{o}{.}\PY{n}{values}\PY{o}{.}\PY{n}{flatten}\PY{p}{(}\PY{p}{)}
\PY{n+nb}{print}\PY{p}{(}\PY{l+s+sa}{f}\PY{l+s+s2}{\PYZdq{}}\PY{l+s+s2}{Dimensi X\PYZus{}train = }\PY{l+s+si}{\PYZob{}}\PY{n}{X\PYZus{}train}\PY{o}{.}\PY{n}{shape}\PY{l+s+si}{\PYZcb{}}\PY{l+s+s2}{\PYZdq{}}\PY{p}{)}
\PY{n+nb}{print}\PY{p}{(}\PY{l+s+sa}{f}\PY{l+s+s2}{\PYZdq{}}\PY{l+s+s2}{Dimensi y\PYZus{}train = }\PY{l+s+si}{\PYZob{}}\PY{n}{y\PYZus{}train}\PY{o}{.}\PY{n}{shape}\PY{l+s+si}{\PYZcb{}}\PY{l+s+s2}{\PYZdq{}}\PY{p}{)}
\end{Verbatim}
\end{tcolorbox}

    \begin{Verbatim}[commandchars=\\\{\}]
Dimensi X\_train = (173, 39)
Dimensi y\_train = (173,)
    \end{Verbatim}

    Catatan: Terdapat \textbf{39 \emph{features}} dengan \textbf{1
\emph{target}}.

    \hypertarget{tahap-3-building-neural-networks}{%
\section{\texorpdfstring{Tahap 3: \emph{Building Neural
Networks}}{Tahap 3: Building Neural Networks}}\label{tahap-3-building-neural-networks}}

\begin{enumerate}
\def\labelenumi{\arabic{enumi}.}
\tightlist
\item
  Fungsi \texttt{build\_model}.
\item
  Penggunaan \texttt{GridSearchCV}.
\item
  Melalukan proses \texttt{fit} terhadap \texttt{X\_train,\ y\_train}
\end{enumerate}

    \hypertarget{fungsi-build_model}{%
\subsection{\texorpdfstring{3.1 Fungsi
\texttt{build\_model}}{3.1 Fungsi build\_model}}\label{fungsi-build_model}}

    \begin{tcolorbox}[breakable, size=fbox, boxrule=1pt, pad at break*=1mm,colback=cellbackground, colframe=cellborder]
\prompt{In}{incolor}{ }{\boxspacing}
\begin{Verbatim}[commandchars=\\\{\}]
\PY{k+kn}{from} \PY{n+nn}{keras}\PY{n+nn}{.}\PY{n+nn}{models} \PY{k+kn}{import} \PY{n}{Sequential}
\PY{k+kn}{from} \PY{n+nn}{keras}\PY{n+nn}{.}\PY{n+nn}{layers} \PY{k+kn}{import} \PY{n}{Dense}\PY{p}{,} \PY{n}{Dropout}
\PY{k+kn}{from} \PY{n+nn}{keras}\PY{n+nn}{.}\PY{n+nn}{wrappers}\PY{n+nn}{.}\PY{n+nn}{scikit\PYZus{}learn} \PY{k+kn}{import} \PY{n}{KerasRegressor}

\PY{k}{def} \PY{n+nf}{build\PYZus{}model}\PY{p}{(}\PY{n}{optimizer}\PY{o}{=}\PY{l+s+s1}{\PYZsq{}}\PY{l+s+s1}{adam}\PY{l+s+s1}{\PYZsq{}}\PY{p}{,} \PY{n}{activation}\PY{o}{=}\PY{l+s+s1}{\PYZsq{}}\PY{l+s+s1}{sigmoid}\PY{l+s+s1}{\PYZsq{}}\PY{p}{,} \PY{n}{first\PYZus{}layer}\PY{o}{=}\PY{l+m+mi}{10}\PY{p}{,} 
                \PY{n}{hidden\PYZus{}layers}\PY{o}{=}\PY{p}{[}\PY{l+m+mi}{30}\PY{p}{]}\PY{p}{,} \PY{n}{p}\PY{o}{=}\PY{l+m+mi}{0}\PY{p}{,} \PY{n}{message}\PY{o}{=}\PY{k+kc}{True}\PY{p}{)}\PY{p}{:}
    \PY{k}{global} \PY{n}{idx}
    \PY{n}{model} \PY{o}{=} \PY{n}{Sequential}\PY{p}{(}\PY{p}{)}
    \PY{n}{model}\PY{o}{.}\PY{n}{add}\PY{p}{(}\PY{n}{Dense}\PY{p}{(}\PY{n}{first\PYZus{}layer}\PY{p}{,} \PY{n}{activation}\PY{o}{=}\PY{n}{activation}\PY{p}{,} \PY{n}{input\PYZus{}dim}\PY{o}{=}\PY{l+m+mi}{39}\PY{p}{)}\PY{p}{)}
    \PY{n}{model}\PY{o}{.}\PY{n}{add}\PY{p}{(}\PY{n}{Dropout}\PY{p}{(}\PY{n}{p}\PY{p}{)}\PY{p}{)}
    
    \PY{k}{if} \PY{n}{hidden\PYZus{}layers}\PY{p}{:}
        \PY{k}{for} \PY{n}{x} \PY{o+ow}{in} \PY{n}{hidden\PYZus{}layers}\PY{p}{:}
            \PY{n}{model}\PY{o}{.}\PY{n}{add}\PY{p}{(}\PY{n}{Dense}\PY{p}{(}\PY{n}{x}\PY{p}{,} \PY{n}{activation}\PY{o}{=}\PY{n}{activation}\PY{p}{)}\PY{p}{)}
            \PY{k}{if} \PY{n}{x} \PY{o}{==} \PY{n}{hidden\PYZus{}layers}\PY{p}{[}\PY{o}{\PYZhy{}}\PY{l+m+mi}{1}\PY{p}{]}\PY{p}{:}
                \PY{n}{model}\PY{o}{.}\PY{n}{add}\PY{p}{(}\PY{n}{Dropout}\PY{p}{(}\PY{n}{p}\PY{o}{/}\PY{l+m+mi}{2}\PY{p}{)}\PY{p}{)}
            \PY{k}{else}\PY{p}{:}
                \PY{n}{model}\PY{o}{.}\PY{n}{add}\PY{p}{(}\PY{n}{Dropout}\PY{p}{(}\PY{n}{p}\PY{p}{)}\PY{p}{)}
    
    \PY{n}{model}\PY{o}{.}\PY{n}{add}\PY{p}{(}\PY{n}{Dense}\PY{p}{(}\PY{l+m+mi}{1}\PY{p}{)}\PY{p}{)}
    \PY{n}{model}\PY{o}{.}\PY{n}{compile}\PY{p}{(}\PY{n}{optimizer}\PY{o}{=}\PY{n}{optimizer}\PY{p}{,} \PY{n}{loss}\PY{o}{=}\PY{l+s+s1}{\PYZsq{}}\PY{l+s+s1}{mean\PYZus{}squared\PYZus{}error}\PY{l+s+s1}{\PYZsq{}}\PY{p}{,} 
                  \PY{n}{metrics}\PY{o}{=}\PY{p}{[}\PY{l+s+s1}{\PYZsq{}}\PY{l+s+s1}{mse}\PY{l+s+s1}{\PYZsq{}}\PY{p}{,} \PY{l+s+s1}{\PYZsq{}}\PY{l+s+s1}{mae}\PY{l+s+s1}{\PYZsq{}}\PY{p}{]}\PY{p}{)}
    
    \PY{k}{if} \PY{n}{message} \PY{o+ow}{and} \PY{p}{(}\PY{l+s+s1}{\PYZsq{}}\PY{l+s+s1}{idx}\PY{l+s+s1}{\PYZsq{}} \PY{o+ow}{in} \PY{n+nb}{globals}\PY{p}{(}\PY{p}{)}\PY{p}{)}\PY{p}{:}
        \PY{n+nb}{print}\PY{p}{(}\PY{l+s+sa}{f}\PY{l+s+s2}{\PYZdq{}}\PY{l+s+si}{\PYZob{}}\PY{n}{idx}\PY{l+s+si}{\PYZcb{}}\PY{l+s+s2}{\PYZgt{}}\PY{l+s+s2}{\PYZdq{}}\PY{p}{,} \PY{n}{end}\PY{o}{=}\PY{l+s+s2}{\PYZdq{}}\PY{l+s+s2}{\PYZdq{}}\PY{p}{)}
        \PY{n}{idx} \PY{o}{\PYZhy{}}\PY{o}{=} \PY{l+m+mi}{1}
        \PY{k}{if} \PY{p}{(}\PY{n}{idx} \PY{o}{\PYZpc{}} \PY{l+m+mi}{10}\PY{p}{)} \PY{o}{==} \PY{l+m+mi}{0}\PY{p}{:}
            \PY{n+nb}{print}\PY{p}{(}\PY{p}{)}
    
    \PY{k}{return} \PY{n}{model}

\PY{n}{model} \PY{o}{=} \PY{n}{KerasRegressor}\PY{p}{(}\PY{n}{build\PYZus{}fn}\PY{o}{=}\PY{n}{build\PYZus{}model}\PY{p}{,} \PY{n}{verbose}\PY{o}{=}\PY{l+m+mi}{0}\PY{p}{)}
\end{Verbatim}
\end{tcolorbox}

    \begin{Verbatim}[commandchars=\\\{\}]
Using TensorFlow backend.
    \end{Verbatim}

    \hypertarget{parameter-gridsearchcv}{%
\subsection{\texorpdfstring{3.2 Parameter
\texttt{GridSearchCV}}{3.2 Parameter GridSearchCV}}\label{parameter-gridsearchcv}}

    \begin{tcolorbox}[breakable, size=fbox, boxrule=1pt, pad at break*=1mm,colback=cellbackground, colframe=cellborder]
\prompt{In}{incolor}{ }{\boxspacing}
\begin{Verbatim}[commandchars=\\\{\}]
\PY{k+kn}{from} \PY{n+nn}{sklearn}\PY{n+nn}{.}\PY{n+nn}{model\PYZus{}selection} \PY{k+kn}{import} \PY{n}{GridSearchCV}

\PY{n}{param\PYZus{}grid} \PY{o}{=} \PY{n+nb}{dict}\PY{p}{(}\PY{n}{epochs}\PY{o}{=}\PY{p}{[}\PY{l+m+mi}{100}\PY{p}{,} \PY{l+m+mi}{150}\PY{p}{,} \PY{l+m+mi}{200}\PY{p}{]}\PY{p}{,}
                  \PY{n}{batch\PYZus{}size}\PY{o}{=}\PY{p}{[}\PY{l+m+mi}{5}\PY{p}{,} \PY{l+m+mi}{10}\PY{p}{,} \PY{l+m+mi}{20}\PY{p}{]}\PY{p}{,}
                  \PY{n}{first\PYZus{}layer}\PY{o}{=}\PY{p}{[}\PY{l+m+mi}{10}\PY{p}{,} \PY{l+m+mi}{20}\PY{p}{,} \PY{l+m+mi}{30}\PY{p}{]}\PY{p}{,}
                  \PY{n}{hidden\PYZus{}layers}\PY{o}{=}\PY{p}{[}\PY{p}{[}\PY{l+m+mi}{20}\PY{p}{]}\PY{p}{,} \PY{p}{[}\PY{l+m+mi}{30}\PY{p}{]}\PY{p}{]}\PY{p}{,}
                  \PY{n}{activation}\PY{o}{=}\PY{p}{[}\PY{l+s+s1}{\PYZsq{}}\PY{l+s+s1}{sigmoid}\PY{l+s+s1}{\PYZsq{}}\PY{p}{]}\PY{p}{,}
                  \PY{n}{optimizer}\PY{o}{=}\PY{p}{[}\PY{l+s+s1}{\PYZsq{}}\PY{l+s+s1}{adam}\PY{l+s+s1}{\PYZsq{}}\PY{p}{]}\PY{p}{,}
                 \PY{p}{)}

\PY{c+c1}{\PYZsh{} Ignore K\PYZhy{}Fold Cross Validation}
\PY{n}{cv} \PY{o}{=} \PY{p}{[}\PY{p}{(}\PY{n+nb}{slice}\PY{p}{(}\PY{k+kc}{None}\PY{p}{)}\PY{p}{,} \PY{n+nb}{slice}\PY{p}{(}\PY{k+kc}{None}\PY{p}{)}\PY{p}{)}\PY{p}{]}
\PY{c+c1}{\PYZsh{} cv = 3}

\PY{n}{grid\PYZus{}search} \PY{o}{=} \PY{n}{GridSearchCV}\PY{p}{(}\PY{n}{estimator}\PY{o}{=}\PY{n}{model}\PY{p}{,} 
                           \PY{n}{param\PYZus{}grid}\PY{o}{=}\PY{n}{param\PYZus{}grid}\PY{p}{,}
                           \PY{n}{cv}\PY{o}{=}\PY{n}{cv}\PY{p}{,}
                           \PY{n}{return\PYZus{}train\PYZus{}score}\PY{o}{=}\PY{k+kc}{True}\PY{p}{,}
                           \PY{n}{verbose}\PY{o}{=}\PY{l+m+mi}{1}\PY{p}{,}
                           \PY{n}{scoring}\PY{o}{=}\PY{l+s+s1}{\PYZsq{}}\PY{l+s+s1}{neg\PYZus{}mean\PYZus{}squared\PYZus{}error}\PY{l+s+s1}{\PYZsq{}}\PY{p}{,}
                          \PY{p}{)}
\end{Verbatim}
\end{tcolorbox}

    \hypertarget{fit}{%
\subsection{\texorpdfstring{3.3 \texttt{Fit}}{3.3 Fit}}\label{fit}}

    \hypertarget{fitting}{%
\subsubsection{\texorpdfstring{3.3.1
\texttt{Fitting}}{3.3.1 Fitting}}\label{fitting}}

    \begin{tcolorbox}[breakable, size=fbox, boxrule=1pt, pad at break*=1mm,colback=cellbackground, colframe=cellborder]
\prompt{In}{incolor}{ }{\boxspacing}
\begin{Verbatim}[commandchars=\\\{\}]
\PY{c+c1}{\PYZsh{} idx}
\PY{n}{search\PYZus{}steps} \PY{o}{=} \PY{l+m+mi}{1}
\PY{k}{for} \PY{n}{key}\PY{p}{,} \PY{n}{val} \PY{o+ow}{in} \PY{n}{param\PYZus{}grid}\PY{o}{.}\PY{n}{items}\PY{p}{(}\PY{p}{)}\PY{p}{:}
    \PY{n}{search\PYZus{}steps} \PY{o}{*}\PY{o}{=} \PY{n+nb}{len}\PY{p}{(}\PY{n}{val}\PY{p}{)}
\PY{n}{idx} \PY{o}{=} \PY{n}{search\PYZus{}steps}\PY{o}{*}\PY{n}{cv} \PY{k}{if} \PY{p}{(}\PY{n+nb}{type}\PY{p}{(}\PY{n}{cv}\PY{p}{)} \PY{o+ow}{is} \PY{n+nb}{int}\PY{p}{)} \PY{k}{else} \PY{n}{search\PYZus{}steps}

\PY{c+c1}{\PYZsh{} Fitting}
\PY{n+nb}{print}\PY{p}{(}\PY{n}{mylog}\PY{o}{.}\PY{n}{add\PYZus{}savepoint}\PY{p}{(}\PY{l+s+s2}{\PYZdq{}}\PY{l+s+s2}{START FITTING}\PY{l+s+s2}{\PYZdq{}}\PY{p}{,} \PY{l+s+s1}{\PYZsq{}}\PY{l+s+s1}{fit}\PY{l+s+s1}{\PYZsq{}}\PY{p}{)}\PY{p}{)}

\PY{n}{grid\PYZus{}search} \PY{o}{=} \PY{n}{grid\PYZus{}search}\PY{o}{.}\PY{n}{fit}\PY{p}{(}\PY{n}{X\PYZus{}train}\PY{p}{,} \PY{n}{y\PYZus{}train}\PY{p}{,} \PY{n}{verbose}\PY{o}{=}\PY{l+m+mi}{0}\PY{p}{,} \PY{n}{validation\PYZus{}split}\PY{o}{=}\PY{l+m+mf}{0.2}\PY{p}{)}

\PY{n+nb}{print}\PY{p}{(}\PY{n}{mylog}\PY{o}{.}\PY{n}{add\PYZus{}savepoint}\PY{p}{(}\PY{l+s+s2}{\PYZdq{}}\PY{l+s+s2}{END FITTING}\PY{l+s+s2}{\PYZdq{}}\PY{p}{,} \PY{l+s+s1}{\PYZsq{}}\PY{l+s+s1}{fit}\PY{l+s+s1}{\PYZsq{}}\PY{p}{)}\PY{p}{)}
\PY{n+nb}{print}\PY{p}{(}\PY{n}{mylog}\PY{o}{.}\PY{n}{add\PYZus{}duration}\PY{p}{(}\PY{l+s+s1}{\PYZsq{}}\PY{l+s+s1}{fit}\PY{l+s+s1}{\PYZsq{}}\PY{p}{)}\PY{p}{)}
\end{Verbatim}
\end{tcolorbox}

    \begin{Verbatim}[commandchars=\\\{\}]
[Parallel(n\_jobs=1)]: Using backend SequentialBackend with 1 concurrent workers.
WARNING: Logging before flag parsing goes to stderr.
W0713 05:12:41.348032 140582811764608 deprecation\_wrapper.py:119] From
/usr/local/lib/python3.6/dist-packages/keras/backend/tensorflow\_backend.py:74:
The name tf.get\_default\_graph is deprecated. Please use
tf.compat.v1.get\_default\_graph instead.

W0713 05:12:41.400704 140582811764608 deprecation\_wrapper.py:119] From
/usr/local/lib/python3.6/dist-packages/keras/backend/tensorflow\_backend.py:517:
The name tf.placeholder is deprecated. Please use tf.compat.v1.placeholder
instead.

W0713 05:12:41.409335 140582811764608 deprecation\_wrapper.py:119] From
/usr/local/lib/python3.6/dist-packages/keras/backend/tensorflow\_backend.py:4138:
The name tf.random\_uniform is deprecated. Please use tf.random.uniform instead.

W0713 05:12:41.464263 140582811764608 deprecation\_wrapper.py:119] From
/usr/local/lib/python3.6/dist-packages/keras/optimizers.py:790: The name
tf.train.Optimizer is deprecated. Please use tf.compat.v1.train.Optimizer
instead.

    \end{Verbatim}

    \begin{Verbatim}[commandchars=\\\{\}]
[2019-07-13 05:12:41] START FITTING
Fitting 1 folds for each of 54 candidates, totalling 54 fits
54>
    \end{Verbatim}

    \begin{Verbatim}[commandchars=\\\{\}]
W0713 05:12:42.024242 140582811764608 deprecation\_wrapper.py:119] From
/usr/local/lib/python3.6/dist-packages/keras/backend/tensorflow\_backend.py:986:
The name tf.assign\_add is deprecated. Please use tf.compat.v1.assign\_add
instead.

W0713 05:12:42.117183 140582811764608 deprecation\_wrapper.py:119] From
/usr/local/lib/python3.6/dist-packages/keras/backend/tensorflow\_backend.py:973:
The name tf.assign is deprecated. Please use tf.compat.v1.assign instead.

    \end{Verbatim}

    \begin{Verbatim}[commandchars=\\\{\}]
53>52>51>
50>49>48>47>46>45>44>43>42>41>
40>39>38>37>36>35>34>33>32>31>
30>29>28>27>26>25>24>23>22>21>
20>19>18>17>16>15>14>13>12>11>
10>9>8>7>6>5>4>3>2>1>
    \end{Verbatim}

    \begin{Verbatim}[commandchars=\\\{\}]
[Parallel(n\_jobs=1)]: Done  54 out of  54 | elapsed:  5.5min finished
    \end{Verbatim}

    \begin{Verbatim}[commandchars=\\\{\}]
0>[2019-07-13 05:18:28] END FITTING
0:5:46
    \end{Verbatim}

    \begin{tcolorbox}[breakable, size=fbox, boxrule=1pt, pad at break*=1mm,colback=cellbackground, colframe=cellborder]
\prompt{In}{incolor}{ }{\boxspacing}
\begin{Verbatim}[commandchars=\\\{\}]
\PY{c+c1}{\PYZsh{} Menyimpan object keras di final\PYZus{}model}
\PY{n}{final\PYZus{}model} \PY{o}{=} \PY{n}{grid\PYZus{}search}\PY{o}{.}\PY{n}{best\PYZus{}estimator\PYZus{}}\PY{o}{.}\PY{n}{model}

\PY{c+c1}{\PYZsh{} Simpan hasil grid search pada dataframe}
\PY{n}{df\PYZus{}cv} \PY{o}{=} \PY{n}{pd}\PY{o}{.}\PY{n}{DataFrame}\PY{p}{(}\PY{n}{grid\PYZus{}search}\PY{o}{.}\PY{n}{cv\PYZus{}results\PYZus{}}\PY{p}{)}
\end{Verbatim}
\end{tcolorbox}

    \hypertarget{saving}{%
\subsubsection{\texorpdfstring{3.3.2
\emph{Saving}}{3.3.2 Saving}}\label{saving}}

    \begin{tcolorbox}[breakable, size=fbox, boxrule=1pt, pad at break*=1mm,colback=cellbackground, colframe=cellborder]
\prompt{In}{incolor}{ }{\boxspacing}
\begin{Verbatim}[commandchars=\\\{\}]
\PY{c+c1}{\PYZsh{} Save Model in JSON}
\PY{n}{fmodel\PYZus{}json} \PY{o}{=} \PY{n}{final\PYZus{}model}\PY{o}{.}\PY{n}{to\PYZus{}json}\PY{p}{(}\PY{p}{)}
\PY{n}{fmodel\PYZus{}j\PYZus{}path} \PY{o}{=} \PY{n}{drop\PYZus{}path} \PY{o}{+} \PY{l+s+s1}{\PYZsq{}}\PY{l+s+s1}{/}\PY{l+s+si}{\PYZob{}\PYZcb{}}\PY{l+s+s1}{.json}\PY{l+s+s1}{\PYZsq{}}\PY{o}{.}\PY{n}{format}\PY{p}{(}\PY{n}{project\PYZus{}title}\PY{p}{)}
\PY{k}{with} \PY{n+nb}{open}\PY{p}{(}\PY{n}{fmodel\PYZus{}j\PYZus{}path}\PY{p}{,} \PY{l+s+s1}{\PYZsq{}}\PY{l+s+s1}{w}\PY{l+s+s1}{\PYZsq{}}\PY{p}{)} \PY{k}{as} \PY{n}{json\PYZus{}file}\PY{p}{:}
    \PY{n}{json\PYZus{}file}\PY{o}{.}\PY{n}{write}\PY{p}{(}\PY{n}{fmodel\PYZus{}json}\PY{p}{)}
\PY{n}{mylog}\PY{o}{.}\PY{n}{add}\PY{p}{(}\PY{l+s+sa}{f}\PY{l+s+s1}{\PYZsq{}}\PY{l+s+s1}{Model JSON disimpan di }\PY{l+s+si}{\PYZob{}}\PY{n}{fmodel\PYZus{}j\PYZus{}path}\PY{l+s+si}{\PYZcb{}}\PY{l+s+s1}{\PYZsq{}}\PY{p}{)}
\PY{n+nb}{print}\PY{p}{(}\PY{l+s+s1}{\PYZsq{}}\PY{l+s+s1}{save: }\PY{l+s+si}{\PYZob{}\PYZcb{}}\PY{l+s+s1}{\PYZsq{}}\PY{o}{.}\PY{n}{format}\PY{p}{(}\PY{n}{fmodel\PYZus{}j\PYZus{}path}\PY{p}{)}\PY{p}{)}

\PY{c+c1}{\PYZsh{} Save Weights of model}
\PY{n}{fmodel\PYZus{}w\PYZus{}path} \PY{o}{=} \PY{n}{drop\PYZus{}path} \PY{o}{+} \PY{l+s+s1}{\PYZsq{}}\PY{l+s+s1}{/}\PY{l+s+si}{\PYZob{}\PYZcb{}}\PY{l+s+s1}{\PYZus{}weights.h5}\PY{l+s+s1}{\PYZsq{}}\PY{o}{.}\PY{n}{format}\PY{p}{(}\PY{n}{project\PYZus{}title}\PY{p}{)}
\PY{n}{final\PYZus{}model}\PY{o}{.}\PY{n}{save\PYZus{}weights}\PY{p}{(}\PY{n}{fmodel\PYZus{}w\PYZus{}path}\PY{p}{)}
\PY{n}{mylog}\PY{o}{.}\PY{n}{add}\PY{p}{(}\PY{l+s+sa}{f}\PY{l+s+s1}{\PYZsq{}}\PY{l+s+s1}{Model Weights disimpan di }\PY{l+s+si}{\PYZob{}}\PY{n}{fmodel\PYZus{}w\PYZus{}path}\PY{l+s+si}{\PYZcb{}}\PY{l+s+s1}{\PYZsq{}}\PY{p}{)}
\PY{n+nb}{print}\PY{p}{(}\PY{l+s+s1}{\PYZsq{}}\PY{l+s+s1}{save: }\PY{l+s+si}{\PYZob{}\PYZcb{}}\PY{l+s+s1}{\PYZsq{}}\PY{o}{.}\PY{n}{format}\PY{p}{(}\PY{n}{fmodel\PYZus{}w\PYZus{}path}\PY{p}{)}\PY{p}{)}

\PY{c+c1}{\PYZsh{} Simpan model dan grid\PYZus{}search object}
\PY{n}{save\PYZus{}model\PYZus{}path} \PY{o}{=} \PY{n}{drop\PYZus{}path} \PY{o}{+} \PY{l+s+s1}{\PYZsq{}}\PY{l+s+s1}{/}\PY{l+s+s1}{\PYZsq{}} \PY{o}{+} \PY{n}{project\PYZus{}title} \PY{o}{+} \PY{l+s+s1}{\PYZsq{}}\PY{l+s+s1}{.h5}\PY{l+s+s1}{\PYZsq{}}
\PY{n}{final\PYZus{}model}\PY{o}{.}\PY{n}{save}\PY{p}{(}\PY{n}{save\PYZus{}model\PYZus{}path}\PY{p}{)}
\PY{n}{mylog}\PY{o}{.}\PY{n}{add}\PY{p}{(}\PY{l+s+sa}{f}\PY{l+s+s1}{\PYZsq{}}\PY{l+s+s1}{Model disimpan di }\PY{l+s+si}{\PYZob{}}\PY{n}{save\PYZus{}model\PYZus{}path}\PY{l+s+si}{\PYZcb{}}\PY{l+s+s1}{\PYZsq{}}\PY{p}{)}
\PY{n+nb}{print}\PY{p}{(}\PY{l+s+s1}{\PYZsq{}}\PY{l+s+s1}{save: }\PY{l+s+si}{\PYZob{}\PYZcb{}}\PY{l+s+s1}{\PYZsq{}}\PY{o}{.}\PY{n}{format}\PY{p}{(}\PY{n}{save\PYZus{}model\PYZus{}path}\PY{p}{)}\PY{p}{)}

\PY{c+c1}{\PYZsh{} Simpan hasil GridSearch}
\PY{n}{save\PYZus{}grid\PYZus{}path} \PY{o}{=} \PY{n}{drop\PYZus{}path} \PY{o}{+} \PY{l+s+s1}{\PYZsq{}}\PY{l+s+s1}{/}\PY{l+s+si}{\PYZob{}\PYZcb{}}\PY{l+s+s1}{.csv}\PY{l+s+s1}{\PYZsq{}}\PY{o}{.}\PY{n}{format}\PY{p}{(}\PY{n}{project\PYZus{}title}\PY{p}{)}
\PY{n}{df\PYZus{}cv}\PY{o}{.}\PY{n}{to\PYZus{}csv}\PY{p}{(}\PY{n}{save\PYZus{}grid\PYZus{}path}\PY{p}{)}
\PY{n}{mylog}\PY{o}{.}\PY{n}{add}\PY{p}{(}\PY{l+s+sa}{f}\PY{l+s+s1}{\PYZsq{}}\PY{l+s+s1}{Tabel GridSearch disimpan di }\PY{l+s+si}{\PYZob{}}\PY{n}{save\PYZus{}grid\PYZus{}path}\PY{l+s+si}{\PYZcb{}}\PY{l+s+s1}{\PYZsq{}}\PY{p}{)}
\PY{n+nb}{print}\PY{p}{(}\PY{l+s+s1}{\PYZsq{}}\PY{l+s+s1}{save: }\PY{l+s+si}{\PYZob{}\PYZcb{}}\PY{l+s+s1}{\PYZsq{}}\PY{o}{.}\PY{n}{format}\PY{p}{(}\PY{n}{save\PYZus{}grid\PYZus{}path}\PY{p}{)}\PY{p}{)}
\end{Verbatim}
\end{tcolorbox}

    \begin{Verbatim}[commandchars=\\\{\}]
save: /content/gdrive/My Drive/Colab
Notebooks/\_dropbox/20190713\_0507\_kualitas\_air\_ann\_so\_2.0.0.json
save: /content/gdrive/My Drive/Colab
Notebooks/\_dropbox/20190713\_0507\_kualitas\_air\_ann\_so\_2.0.0\_weights.h5
save: /content/gdrive/My Drive/Colab
Notebooks/\_dropbox/20190713\_0507\_kualitas\_air\_ann\_so\_2.0.0.h5
save: /content/gdrive/My Drive/Colab
Notebooks/\_dropbox/20190713\_0507\_kualitas\_air\_ann\_so\_2.0.0.csv
    \end{Verbatim}

    \hypertarget{loading-jika-sudah-disimpan-model-dan-hasil-grid-search}{%
\subsubsection{\texorpdfstring{3.3.3 Loading (Jika sudah disimpan Model
dan Hasil \emph{grid
search})}{3.3.3 Loading (Jika sudah disimpan Model dan Hasil grid search)}}\label{loading-jika-sudah-disimpan-model-dan-hasil-grid-search}}

    \begin{tcolorbox}[breakable, size=fbox, boxrule=1pt, pad at break*=1mm,colback=cellbackground, colframe=cellborder]
\prompt{In}{incolor}{ }{\boxspacing}
\begin{Verbatim}[commandchars=\\\{\}]
\PY{c+c1}{\PYZsh{} load\PYZus{}model\PYZus{}path = drop\PYZus{}path + \PYZsq{}/20190512\PYZus{}2037\PYZus{}kualitas\PYZus{}air\PYZus{}ann\PYZus{}so.h5\PYZsq{}}
\PY{c+c1}{\PYZsh{} load\PYZus{}cvgrid\PYZus{}path = drop\PYZus{}path + \PYZsq{}/20190512\PYZus{}2037\PYZus{}kualitas\PYZus{}air\PYZus{}ann\PYZus{}so.csv\PYZsq{}}

\PY{c+c1}{\PYZsh{} from keras.models import load\PYZus{}model}
\PY{c+c1}{\PYZsh{} final\PYZus{}model = load\PYZus{}model(load\PYZus{}model\PYZus{}path)}

\PY{c+c1}{\PYZsh{} df\PYZus{}cv = pd.read\PYZus{}csv(load\PYZus{}cvgrid\PYZus{}path, index\PYZus{}col=[0])}
\PY{c+c1}{\PYZsh{} df\PYZus{}cv.head()}
\end{Verbatim}
\end{tcolorbox}

    \hypertarget{tahap-4-evaluate-model}{%
\section{\texorpdfstring{Tahap 4: \emph{Evaluate
Model}}{Tahap 4: Evaluate Model}}\label{tahap-4-evaluate-model}}

\begin{enumerate}
\def\labelenumi{\arabic{enumi}.}
\tightlist
\item
  \emph{Data preprocessing} pada \emph{test set}.
\item
  Memprediksi nilai dengan model yang terbaik hasil \emph{Grid Search}.
\item
  Mengembalikan ke skala aslinya.
\item
  Evaluasi Model pada \emph{test set}.
\item
  Nilai beda prediksi dan observasi.
\end{enumerate}

    \hypertarget{data-preprocessing-pada-test-set}{%
\subsection{\texorpdfstring{4.1 \emph{Data preprocessing} pada
\emph{test
set}}{4.1 Data preprocessing pada test set}}\label{data-preprocessing-pada-test-set}}

    \begin{tcolorbox}[breakable, size=fbox, boxrule=1pt, pad at break*=1mm,colback=cellbackground, colframe=cellborder]
\prompt{In}{incolor}{ }{\boxspacing}
\begin{Verbatim}[commandchars=\\\{\}]
\PY{c+c1}{\PYZsh{}\PYZsh{} Menampilkan test dataset}
\PY{n}{test\PYZus{}dataset}\PY{o}{.}\PY{n}{head}\PY{p}{(}\PY{p}{)}
\end{Verbatim}
\end{tcolorbox}

            \begin{tcolorbox}[breakable, size=fbox, boxrule=.5pt, pad at break*=1mm, opacityfill=0]
\prompt{Out}{outcolor}{ }{\boxspacing}
\begin{Verbatim}[commandchars=\\\{\}]
            temp\_udara  lama\_sinar  {\ldots}  out\_nitrat  out\_amonia
date                                {\ldots}
2015-01-14        25.5    4.500000  {\ldots}    0.261190    0.088036
2015-02-16        25.6    3.833333  {\ldots}    0.152798    0.578571
2015-03-16        25.4    5.500000  {\ldots}    0.109405    0.153452
2015-04-14        25.4    3.500000  {\ldots}    0.308036    0.102560
2015-05-12        23.2    9.666667  {\ldots}    0.069940    0.165476

[5 rows x 14 columns]
\end{Verbatim}
\end{tcolorbox}
        
    \begin{tcolorbox}[breakable, size=fbox, boxrule=1pt, pad at break*=1mm,colback=cellbackground, colframe=cellborder]
\prompt{In}{incolor}{ }{\boxspacing}
\begin{Verbatim}[commandchars=\\\{\}]
\PY{c+c1}{\PYZsh{}\PYZsh{} pandas.DataFrame ke numpy.array}
\PY{n}{array\PYZus{}test} \PY{o}{=} \PY{n}{test\PYZus{}dataset}\PY{o}{.}\PY{n}{values}
\PY{n}{array\PYZus{}test} \PY{o}{=} \PY{n}{sc}\PY{o}{.}\PY{n}{transform}\PY{p}{(}\PY{n}{array\PYZus{}test}\PY{p}{)}

\PY{n}{test\PYZus{}dataset\PYZus{}scale} \PY{o}{=} \PY{n}{pd}\PY{o}{.}\PY{n}{DataFrame}\PY{p}{(}
    \PY{n}{data}\PY{o}{=}\PY{n}{array\PYZus{}test}\PY{p}{,}
    \PY{n}{columns}\PY{o}{=}\PY{n}{test\PYZus{}dataset}\PY{o}{.}\PY{n}{columns}\PY{p}{,}
    \PY{n}{index}\PY{o}{=}\PY{n}{test\PYZus{}dataset}\PY{o}{.}\PY{n}{index}
\PY{p}{)}

\PY{c+c1}{\PYZsh{}\PYZsh{} timestep table}
\PY{n}{df\PYZus{}test} \PY{o}{=} \PY{n}{timeseries}\PY{o}{.}\PY{n}{timestep\PYZus{}table}\PY{p}{(}\PY{n}{test\PYZus{}dataset\PYZus{}scale}\PY{p}{,} \PY{n}{timesteps}\PY{o}{=}\PY{n}{n\PYZus{}timesteps}\PY{p}{)}
\PY{n}{array\PYZus{}test\PYZus{}ts} \PY{o}{=} \PY{n}{df\PYZus{}test}\PY{o}{.}\PY{n}{values}
\PY{n}{df\PYZus{}test}\PY{o}{.}\PY{n}{head}\PY{p}{(}\PY{p}{)}
\end{Verbatim}
\end{tcolorbox}

            \begin{tcolorbox}[breakable, size=fbox, boxrule=.5pt, pad at break*=1mm, opacityfill=0]
\prompt{Out}{outcolor}{ }{\boxspacing}
\begin{Verbatim}[commandchars=\\\{\}]
            temp\_udara\_tmin0  {\ldots}  out\_amonia\_tmin2
date                          {\ldots}
2015-03-16          0.491803  {\ldots}          0.039086
2015-04-14          0.491803  {\ldots}          0.256877
2015-05-12          0.131148  {\ldots}          0.068130
2015-06-15          0.131148  {\ldots}          0.045535
2015-07-08          0.147541  {\ldots}          0.073469

[5 rows x 42 columns]
\end{Verbatim}
\end{tcolorbox}
        
    \begin{tcolorbox}[breakable, size=fbox, boxrule=1pt, pad at break*=1mm,colback=cellbackground, colframe=cellborder]
\prompt{In}{incolor}{ }{\boxspacing}
\begin{Verbatim}[commandchars=\\\{\}]
\PY{c+c1}{\PYZsh{}\PYZsh{} Pembagian X\PYZus{}test dan y\PYZus{}test untuk }
\PY{c+c1}{\PYZsh{}\PYZsh{} Kasus Single\PYZhy{}Output Regression Neural Network}
\PY{c+c1}{\PYZsh{}\PYZsh{} Meninjau output\PYZus{}amonia }

\PY{n}{df\PYZus{}X\PYZus{}test} \PY{o}{=} \PY{n}{df\PYZus{}test}\PY{o}{.}\PY{n}{drop}\PY{p}{(}\PY{n}{drop\PYZus{}col}\PY{p}{,} \PY{n}{axis}\PY{o}{=}\PY{l+m+mi}{1}\PY{p}{)}
\PY{n}{df\PYZus{}y\PYZus{}test} \PY{o}{=} \PY{n}{df\PYZus{}test}\PY{p}{[}\PY{n}{target\PYZus{}col}\PY{p}{]}
\PY{n}{X\PYZus{}test} \PY{o}{=} \PY{n}{df\PYZus{}X\PYZus{}test}\PY{o}{.}\PY{n}{values}
\PY{n}{y\PYZus{}test} \PY{o}{=} \PY{n}{df\PYZus{}y\PYZus{}test}\PY{o}{.}\PY{n}{values}\PY{o}{.}\PY{n}{flatten}\PY{p}{(}\PY{p}{)}
\PY{n+nb}{print}\PY{p}{(}\PY{l+s+sa}{f}\PY{l+s+s2}{\PYZdq{}}\PY{l+s+s2}{Dimensi X\PYZus{}test = }\PY{l+s+si}{\PYZob{}}\PY{n}{X\PYZus{}test}\PY{o}{.}\PY{n}{shape}\PY{l+s+si}{\PYZcb{}}\PY{l+s+s2}{\PYZdq{}}\PY{p}{)}
\PY{n+nb}{print}\PY{p}{(}\PY{l+s+sa}{f}\PY{l+s+s2}{\PYZdq{}}\PY{l+s+s2}{Dimensi y\PYZus{}test = }\PY{l+s+si}{\PYZob{}}\PY{n}{y\PYZus{}test}\PY{o}{.}\PY{n}{shape}\PY{l+s+si}{\PYZcb{}}\PY{l+s+s2}{\PYZdq{}}\PY{p}{)}
\end{Verbatim}
\end{tcolorbox}

    \begin{Verbatim}[commandchars=\\\{\}]
Dimensi X\_test = (34, 39)
Dimensi y\_test = (34,)
    \end{Verbatim}

    \hypertarget{prediksi-model-dari-parameter-terbaik-hasil-grid-search}{%
\subsection{\texorpdfstring{4.2 Prediksi model dari parameter terbaik
hasil \emph{Grid
Search}}{4.2 Prediksi model dari parameter terbaik hasil Grid Search}}\label{prediksi-model-dari-parameter-terbaik-hasil-grid-search}}

    \begin{tcolorbox}[breakable, size=fbox, boxrule=1pt, pad at break*=1mm,colback=cellbackground, colframe=cellborder]
\prompt{In}{incolor}{ }{\boxspacing}
\begin{Verbatim}[commandchars=\\\{\}]
\PY{c+c1}{\PYZsh{} Prediksi}
\PY{n}{predict} \PY{o}{=} \PY{n}{final\PYZus{}model}\PY{o}{.}\PY{n}{predict}\PY{p}{(}\PY{n}{X\PYZus{}test}\PY{p}{)}
\PY{n}{truth} \PY{o}{=} \PY{n}{y\PYZus{}test}
\end{Verbatim}
\end{tcolorbox}

    digunakan istilah \emph{predict} sebagai nilai prediksi dari
\texttt{final\_model}, dan \emph{truth} sebagai nilai observasi di
stasiun B.

    \hypertarget{mengembalikan-ke-skala-aslinya}{%
\subsection{4.3 Mengembalikan ke skala
aslinya}\label{mengembalikan-ke-skala-aslinya}}

    \begin{tcolorbox}[breakable, size=fbox, boxrule=1pt, pad at break*=1mm,colback=cellbackground, colframe=cellborder]
\prompt{In}{incolor}{ }{\boxspacing}
\begin{Verbatim}[commandchars=\\\{\}]
\PY{c+c1}{\PYZsh{} Transfer attribute from MinMax Scaler (specific for last column (output) only)}

\PY{n}{sc\PYZus{}test} \PY{o}{=} \PY{n}{MinMaxScaler}\PY{p}{(}\PY{p}{)}
\PY{n}{sc\PYZus{}test}\PY{o}{.}\PY{n}{min\PYZus{}}\PY{p}{,} \PY{n}{sc\PYZus{}test}\PY{o}{.}\PY{n}{scale\PYZus{}}\PY{p}{,} \PY{n}{sc\PYZus{}test}\PY{o}{.}\PY{n}{data\PYZus{}min\PYZus{}}\PY{p}{,} \PY{n}{sc\PYZus{}test}\PY{o}{.}\PY{n}{data\PYZus{}max\PYZus{}} \PY{o}{=} \PY{n}{sc}\PY{o}{.}\PY{n}{min\PYZus{}}\PY{p}{[}\PY{o}{\PYZhy{}}\PY{l+m+mi}{1}\PY{p}{]}\PY{p}{,} \PY{n}{sc}\PY{o}{.}\PY{n}{scale\PYZus{}}\PY{p}{[}\PY{o}{\PYZhy{}}\PY{l+m+mi}{1}\PY{p}{]}\PY{p}{,} \PY{n}{sc}\PY{o}{.}\PY{n}{data\PYZus{}min\PYZus{}}\PY{p}{[}\PY{o}{\PYZhy{}}\PY{l+m+mi}{1}\PY{p}{]}\PY{p}{,} \PY{n}{sc}\PY{o}{.}\PY{n}{data\PYZus{}max\PYZus{}}\PY{p}{[}\PY{o}{\PYZhy{}}\PY{l+m+mi}{1}\PY{p}{]}

\PY{c+c1}{\PYZsh{} Mengembalikan ke skala original}
\PY{n}{predict\PYZus{}real} \PY{o}{=} \PY{n}{sc\PYZus{}test}\PY{o}{.}\PY{n}{inverse\PYZus{}transform}\PY{p}{(}\PY{n}{predict}\PY{o}{.}\PY{n}{reshape}\PY{p}{(}\PY{o}{\PYZhy{}}\PY{l+m+mi}{1}\PY{p}{,}\PY{l+m+mi}{1}\PY{p}{)}\PY{p}{)}
\PY{n}{truth\PYZus{}real} \PY{o}{=} \PY{n}{sc\PYZus{}test}\PY{o}{.}\PY{n}{inverse\PYZus{}transform}\PY{p}{(}\PY{n}{truth}\PY{o}{.}\PY{n}{reshape}\PY{p}{(}\PY{o}{\PYZhy{}}\PY{l+m+mi}{1}\PY{p}{,}\PY{l+m+mi}{1}\PY{p}{)}\PY{p}{)}
\end{Verbatim}
\end{tcolorbox}

    Nilai \texttt{-1} menunjukkan posisi kolom \texttt{out\_amonia} saat
melakukan proses \texttt{fit} pada \emph{object} \texttt{sc}
(\texttt{MinMaxScaler}).

    \begin{tcolorbox}[breakable, size=fbox, boxrule=1pt, pad at break*=1mm,colback=cellbackground, colframe=cellborder]
\prompt{In}{incolor}{ }{\boxspacing}
\begin{Verbatim}[commandchars=\\\{\}]
\PY{c+c1}{\PYZsh{} Dalam bentuk pandas.DataFrame}
\PY{n}{diff\PYZus{}table} \PY{o}{=} \PY{n}{pd}\PY{o}{.}\PY{n}{DataFrame}\PY{p}{(}\PY{n+nb}{dict}\PY{p}{(}\PY{n}{predict}\PY{o}{=}\PY{n}{predict\PYZus{}real}\PY{o}{.}\PY{n}{flatten}\PY{p}{(}\PY{p}{)}\PY{p}{,}
                               \PY{n}{truth}\PY{o}{=}\PY{n}{truth\PYZus{}real}\PY{o}{.}\PY{n}{flatten}\PY{p}{(}\PY{p}{)}\PY{p}{,}
                              \PY{p}{)}\PY{p}{)}
\PY{n}{diff\PYZus{}table}\PY{p}{[}\PY{l+s+s1}{\PYZsq{}}\PY{l+s+s1}{diff}\PY{l+s+s1}{\PYZsq{}}\PY{p}{]} \PY{o}{=} \PY{p}{(}\PY{n}{diff\PYZus{}table}\PY{o}{.}\PY{n}{predict} \PY{o}{\PYZhy{}} \PY{n}{diff\PYZus{}table}\PY{o}{.}\PY{n}{truth}\PY{p}{)}\PY{o}{.}\PY{n}{abs}\PY{p}{(}\PY{p}{)}
\PY{n}{diff\PYZus{}table}\PY{o}{.}\PY{n}{T}
\end{Verbatim}
\end{tcolorbox}

            \begin{tcolorbox}[breakable, size=fbox, boxrule=.5pt, pad at break*=1mm, opacityfill=0]
\prompt{Out}{outcolor}{ }{\boxspacing}
\begin{Verbatim}[commandchars=\\\{\}]
               0         1         2   {\ldots}        31        32        33
predict  0.134772  0.007490  0.126282  {\ldots}  0.125497  0.073648 -0.202659
truth    0.153452  0.102560  0.165476  {\ldots}  0.052371  0.025548  0.013155
diff     0.018681  0.095069  0.039195  {\ldots}  0.073126  0.048101  0.215814

[3 rows x 34 columns]
\end{Verbatim}
\end{tcolorbox}
        
    \hypertarget{evaluasi-model-test-set}{%
\subsection{\texorpdfstring{4.4 Evaluasi Model \emph{test
set}}{4.4 Evaluasi Model test set}}\label{evaluasi-model-test-set}}

    \hypertarget{metrik}{%
\subsubsection{4.4.1 Metrik}\label{metrik}}

    \begin{tcolorbox}[breakable, size=fbox, boxrule=1pt, pad at break*=1mm,colback=cellbackground, colframe=cellborder]
\prompt{In}{incolor}{ }{\boxspacing}
\begin{Verbatim}[commandchars=\\\{\}]
\PY{n}{metrics\PYZus{}train} \PY{o}{=} \PY{n}{final\PYZus{}model}\PY{o}{.}\PY{n}{evaluate}\PY{p}{(}\PY{n}{X\PYZus{}train}\PY{p}{,} \PY{n}{y\PYZus{}train}\PY{p}{,} \PY{n}{verbose}\PY{o}{=}\PY{l+m+mi}{0}\PY{p}{)}
\PY{n}{metrics\PYZus{}test} \PY{o}{=} \PY{n}{final\PYZus{}model}\PY{o}{.}\PY{n}{evaluate}\PY{p}{(}\PY{n}{X\PYZus{}test}\PY{p}{,} \PY{n}{y\PYZus{}test}\PY{p}{,} \PY{n}{verbose}\PY{o}{=}\PY{l+m+mi}{0}\PY{p}{)}

\PY{k}{for} \PY{n}{i}\PY{p}{,} \PY{n}{metrics} \PY{o+ow}{in} \PY{n+nb}{enumerate}\PY{p}{(}\PY{n}{final\PYZus{}model}\PY{o}{.}\PY{n}{metrics\PYZus{}names}\PY{p}{)}\PY{p}{:}
    \PY{n+nb}{print}\PY{p}{(}\PY{l+s+sa}{f}\PY{l+s+s2}{\PYZdq{}}\PY{l+s+s2}{Metrics: }\PY{l+s+si}{\PYZob{}}\PY{n}{metrics}\PY{l+s+si}{\PYZcb{}}\PY{l+s+s2}{\PYZdq{}}\PY{p}{)}
    \PY{n+nb}{print}\PY{p}{(}\PY{l+s+sa}{f}\PY{l+s+s2}{\PYZdq{}}\PY{l+s+s2}{Train: }\PY{l+s+si}{\PYZob{}}\PY{n}{metrics\PYZus{}train}\PY{p}{[}\PY{n}{i}\PY{p}{]}\PY{l+s+si}{:}\PY{l+s+s2}{.5f}\PY{l+s+si}{\PYZcb{}}\PY{l+s+s2}{\PYZdq{}}\PY{p}{)}
    \PY{n+nb}{print}\PY{p}{(}\PY{l+s+sa}{f}\PY{l+s+s2}{\PYZdq{}}\PY{l+s+s2}{Test: }\PY{l+s+si}{\PYZob{}}\PY{n}{metrics\PYZus{}test}\PY{p}{[}\PY{n}{i}\PY{p}{]}\PY{l+s+si}{:}\PY{l+s+s2}{.5f}\PY{l+s+si}{\PYZcb{}}\PY{l+s+s2}{\PYZdq{}}\PY{p}{)}
    \PY{n+nb}{print}\PY{p}{(}\PY{p}{)}
\end{Verbatim}
\end{tcolorbox}

    \begin{Verbatim}[commandchars=\\\{\}]
Metrics: loss
Train: 0.00403
Test: 0.02630

Metrics: mean\_squared\_error
Train: 0.00403
Test: 0.02630

Metrics: mean\_absolute\_error
Train: 0.04247
Test: 0.10324

    \end{Verbatim}

    Fungsi \texttt{loss} yang digunakan saat training adalah
\texttt{mean\_squared\ error}. Dari informasi diatas diketahui bahwa
nilai \emph{MSE} dan \emph{MAE} pada saat \emph{training} hampir
mendekati 0. Perlu dicatat bahwa nilai tersebut berdasarkan nilai yang
telah ditransformasi dengan metode \texttt{MinMaxScaler}.

    \begin{tcolorbox}[breakable, size=fbox, boxrule=1pt, pad at break*=1mm,colback=cellbackground, colframe=cellborder]
\prompt{In}{incolor}{ }{\boxspacing}
\begin{Verbatim}[commandchars=\\\{\}]
\PY{c+c1}{\PYZsh{}\PYZsh{} menghitung MSE dan MAE test set dengan skala original}
\PY{k+kn}{from} \PY{n+nn}{sklearn}\PY{n+nn}{.}\PY{n+nn}{metrics} \PY{k+kn}{import} \PY{n}{mean\PYZus{}squared\PYZus{}error}\PY{p}{,} \PY{n}{mean\PYZus{}absolute\PYZus{}error}

\PY{n}{mse\PYZus{}real} \PY{o}{=} \PY{n}{mean\PYZus{}squared\PYZus{}error}\PY{p}{(}\PY{n}{truth\PYZus{}real}\PY{p}{,} \PY{n}{predict\PYZus{}real}\PY{p}{)}
\PY{n}{mae\PYZus{}real} \PY{o}{=} \PY{n}{mean\PYZus{}absolute\PYZus{}error}\PY{p}{(}\PY{n}{truth\PYZus{}real}\PY{p}{,} \PY{n}{predict\PYZus{}real}\PY{p}{)}

\PY{n+nb}{print}\PY{p}{(}\PY{l+s+sa}{f}\PY{l+s+s2}{\PYZdq{}}\PY{l+s+s2}{MSE (Original Scale): }\PY{l+s+si}{\PYZob{}}\PY{n}{mse\PYZus{}real}\PY{l+s+si}{:}\PY{l+s+s2}{.4f}\PY{l+s+si}{\PYZcb{}}\PY{l+s+s2}{\PYZdq{}}\PY{p}{)}
\PY{n+nb}{print}\PY{p}{(}\PY{l+s+sa}{f}\PY{l+s+s2}{\PYZdq{}}\PY{l+s+s2}{MAE (Original Scale): }\PY{l+s+si}{\PYZob{}}\PY{n}{mae\PYZus{}real}\PY{l+s+si}{:}\PY{l+s+s2}{.4f}\PY{l+s+si}{\PYZcb{}}\PY{l+s+s2}{\PYZdq{}}\PY{p}{)}
\end{Verbatim}
\end{tcolorbox}

    \begin{Verbatim}[commandchars=\\\{\}]
MSE (Original Scale): 0.1334
MAE (Original Scale): 0.2325
    \end{Verbatim}

    \hypertarget{visualiasi}{%
\subsubsection{4.4.2 Visualiasi}\label{visualiasi}}

    \begin{tcolorbox}[breakable, size=fbox, boxrule=1pt, pad at break*=1mm,colback=cellbackground, colframe=cellborder]
\prompt{In}{incolor}{ }{\boxspacing}
\begin{Verbatim}[commandchars=\\\{\}]
\PY{c+c1}{\PYZsh{}\PYZsh{}\PYZsh{}\PYZsh{} PLOT OUT\PYZus{}AMONIA PREDICTION AND TRUTH (OBSERVED VALUE)}
\PY{n}{plt}\PY{o}{.}\PY{n}{plot}\PY{p}{(}\PY{n}{truth\PYZus{}real}\PY{p}{,} \PY{l+s+s1}{\PYZsq{}}\PY{l+s+s1}{r\PYZhy{}\PYZhy{}}\PY{l+s+s1}{\PYZsq{}}\PY{p}{,} \PY{n}{label}\PY{o}{=}\PY{l+s+s1}{\PYZsq{}}\PY{l+s+s1}{truth}\PY{l+s+s1}{\PYZsq{}}\PY{p}{)}
\PY{n}{plt}\PY{o}{.}\PY{n}{plot}\PY{p}{(}\PY{n}{predict\PYZus{}real}\PY{p}{,} \PY{l+s+s1}{\PYZsq{}}\PY{l+s+s1}{b}\PY{l+s+s1}{\PYZsq{}}\PY{p}{,} \PY{n}{label}\PY{o}{=}\PY{l+s+s1}{\PYZsq{}}\PY{l+s+s1}{predict}\PY{l+s+s1}{\PYZsq{}}\PY{p}{)}
\PY{n}{plt}\PY{o}{.}\PY{n}{title}\PY{p}{(}\PY{l+s+s1}{\PYZsq{}}\PY{l+s+s1}{Grafik Nilai Prediksi dan Observasi}\PY{l+s+s1}{\PYZsq{}}\PY{p}{)}
\PY{n}{plt}\PY{o}{.}\PY{n}{legend}\PY{p}{(}\PY{p}{)}
\PY{n}{plt}\PY{o}{.}\PY{n}{show}\PY{p}{(}\PY{p}{)}
\end{Verbatim}
\end{tcolorbox}

    \begin{center}
    \adjustimage{max size={0.9\linewidth}{0.9\paperheight}}{taruma_LI01_ann_ka_files/taruma_LI01_ann_ka_76_0.png}
    \end{center}
    { \hspace*{\fill} \\}
    
    Dari grafik diatas diketahui bahwa \texttt{final\_model} mampu mengikuti
fluktuasi data observasi.

    \begin{tcolorbox}[breakable, size=fbox, boxrule=1pt, pad at break*=1mm,colback=cellbackground, colframe=cellborder]
\prompt{In}{incolor}{ }{\boxspacing}
\begin{Verbatim}[commandchars=\\\{\}]
\PY{c+c1}{\PYZsh{} PLOT TRUTH vs. PREDICT}

\PY{n}{plt}\PY{o}{.}\PY{n}{scatter}\PY{p}{(}\PY{n}{y}\PY{o}{=}\PY{n}{predict\PYZus{}real}\PY{p}{,} \PY{n}{x}\PY{o}{=}\PY{n}{truth\PYZus{}real}\PY{p}{)}
\PY{n}{plt}\PY{o}{.}\PY{n}{xlabel}\PY{p}{(}\PY{l+s+s1}{\PYZsq{}}\PY{l+s+s1}{Truth Value}\PY{l+s+s1}{\PYZsq{}}\PY{p}{)}
\PY{n}{plt}\PY{o}{.}\PY{n}{ylabel}\PY{p}{(}\PY{l+s+s1}{\PYZsq{}}\PY{l+s+s1}{Prediction Value}\PY{l+s+s1}{\PYZsq{}}\PY{p}{)}
\PY{n}{plt}\PY{o}{.}\PY{n}{title}\PY{p}{(}\PY{l+s+s1}{\PYZsq{}}\PY{l+s+s1}{Plot Titik Prediksi dan Observasi}\PY{l+s+s1}{\PYZsq{}}\PY{p}{)}
\PY{n}{plt}\PY{o}{.}\PY{n}{show}\PY{p}{(}\PY{p}{)}
\end{Verbatim}
\end{tcolorbox}

    \begin{center}
    \adjustimage{max size={0.9\linewidth}{0.9\paperheight}}{taruma_LI01_ann_ka_files/taruma_LI01_ann_ka_78_0.png}
    \end{center}
    { \hspace*{\fill} \\}
    
    \begin{tcolorbox}[breakable, size=fbox, boxrule=1pt, pad at break*=1mm,colback=cellbackground, colframe=cellborder]
\prompt{In}{incolor}{ }{\boxspacing}
\begin{Verbatim}[commandchars=\\\{\}]
\PY{c+c1}{\PYZsh{} Menggunakan seaborn}

\PY{n}{sns}\PY{o}{.}\PY{n}{jointplot}\PY{p}{(}\PY{n}{x}\PY{o}{=}\PY{l+s+s1}{\PYZsq{}}\PY{l+s+s1}{truth}\PY{l+s+s1}{\PYZsq{}}\PY{p}{,} \PY{n}{y}\PY{o}{=}\PY{l+s+s1}{\PYZsq{}}\PY{l+s+s1}{predict}\PY{l+s+s1}{\PYZsq{}}\PY{p}{,} \PY{n}{kind}\PY{o}{=}\PY{l+s+s1}{\PYZsq{}}\PY{l+s+s1}{reg}\PY{l+s+s1}{\PYZsq{}}\PY{p}{,} \PY{n}{data}\PY{o}{=}\PY{n}{diff\PYZus{}table}\PY{p}{)}\PY{p}{;}
\end{Verbatim}
\end{tcolorbox}

    \begin{center}
    \adjustimage{max size={0.9\linewidth}{0.9\paperheight}}{taruma_LI01_ann_ka_files/taruma_LI01_ann_ka_79_0.png}
    \end{center}
    { \hspace*{\fill} \\}
    
    Dari dua grafik diatas terlihat bahwa terdapat hasil prediksi yang
negatif meski pada kejadian nyata bahwa nilai negatif tidak mungkin
untuk kualitas air.

    \hypertarget{nilai-beda-prediksi-dan-observasi}{%
\subsection{4.5 Nilai beda prediksi dan
observasi}\label{nilai-beda-prediksi-dan-observasi}}

    \hypertarget{metrik}{%
\subsubsection{4.5.1 Metrik}\label{metrik}}

    \begin{tcolorbox}[breakable, size=fbox, boxrule=1pt, pad at break*=1mm,colback=cellbackground, colframe=cellborder]
\prompt{In}{incolor}{ }{\boxspacing}
\begin{Verbatim}[commandchars=\\\{\}]
\PY{c+c1}{\PYZsh{} statistik deskriptif nilai Beda (Residu)}
\PY{n}{diff\PYZus{}table}\PY{p}{[}\PY{l+s+s1}{\PYZsq{}}\PY{l+s+s1}{diff}\PY{l+s+s1}{\PYZsq{}}\PY{p}{]}\PY{o}{.}\PY{n}{describe}\PY{p}{(}\PY{p}{)}
\end{Verbatim}
\end{tcolorbox}

            \begin{tcolorbox}[breakable, size=fbox, boxrule=.5pt, pad at break*=1mm, opacityfill=0]
\prompt{Out}{outcolor}{ }{\boxspacing}
\begin{Verbatim}[commandchars=\\\{\}]
count    34.000000
mean      0.232529
std       0.285942
min       0.014104
25\%       0.048180
50\%       0.123307
75\%       0.326472
max       1.175157
Name: diff, dtype: float64
\end{Verbatim}
\end{tcolorbox}
        
    \hypertarget{visualisasi}{%
\subsubsection{4.5.2 Visualisasi}\label{visualisasi}}

    \begin{tcolorbox}[breakable, size=fbox, boxrule=1pt, pad at break*=1mm,colback=cellbackground, colframe=cellborder]
\prompt{In}{incolor}{ }{\boxspacing}
\begin{Verbatim}[commandchars=\\\{\}]
\PY{c+c1}{\PYZsh{} plot histogram}
\PY{n}{diff} \PY{o}{=} \PY{n}{diff\PYZus{}table}\PY{p}{[}\PY{l+s+s1}{\PYZsq{}}\PY{l+s+s1}{diff}\PY{l+s+s1}{\PYZsq{}}\PY{p}{]}\PY{o}{.}\PY{n}{values}
\PY{n}{plt}\PY{o}{.}\PY{n}{hist}\PY{p}{(}\PY{n}{diff}\PY{p}{)}
\PY{n}{plt}\PY{o}{.}\PY{n}{xlabel}\PY{p}{(}\PY{l+s+s1}{\PYZsq{}}\PY{l+s+s1}{Nilai Beda/Residu}\PY{l+s+s1}{\PYZsq{}}\PY{p}{)}
\PY{n}{plt}\PY{o}{.}\PY{n}{ylabel}\PY{p}{(}\PY{l+s+s1}{\PYZsq{}}\PY{l+s+s1}{Frekuensi}\PY{l+s+s1}{\PYZsq{}}\PY{p}{)}
\PY{n}{plt}\PY{o}{.}\PY{n}{title}\PY{p}{(}\PY{l+s+s1}{\PYZsq{}}\PY{l+s+s1}{Histogram Residu Nilai Prediksi dan Asli}\PY{l+s+s1}{\PYZsq{}}\PY{p}{)}
\PY{n}{plt}\PY{o}{.}\PY{n}{show}\PY{p}{(}\PY{p}{)}
\end{Verbatim}
\end{tcolorbox}

    \begin{center}
    \adjustimage{max size={0.9\linewidth}{0.9\paperheight}}{taruma_LI01_ann_ka_files/taruma_LI01_ann_ka_85_0.png}
    \end{center}
    { \hspace*{\fill} \\}
    
    \hypertarget{tahap-5-conclusion-and-interpretation}{%
\section{\texorpdfstring{Tahap 5: \emph{Conclusion and
Interpretation}}{Tahap 5: Conclusion and Interpretation}}\label{tahap-5-conclusion-and-interpretation}}

\begin{enumerate}
\def\labelenumi{\arabic{enumi}.}
\tightlist
\item
  Interpretasi model ANN
\item
  Evaluasi hasil GridSearch
\item
  Ringkasan
\item
  Saran dan langkah selanjutnya
\end{enumerate}

    \hypertarget{interpretasi-model-neural-networks}{%
\subsection{\texorpdfstring{5.1 Interpretasi Model \emph{Neural
Networks}}{5.1 Interpretasi Model Neural Networks}}\label{interpretasi-model-neural-networks}}

    \begin{tcolorbox}[breakable, size=fbox, boxrule=1pt, pad at break*=1mm,colback=cellbackground, colframe=cellborder]
\prompt{In}{incolor}{ }{\boxspacing}
\begin{Verbatim}[commandchars=\\\{\}]
\PY{c+c1}{\PYZsh{} menghilangkan [] pada hidden layers di df\PYZus{}cv}
\PY{n}{df\PYZus{}cv}\PY{p}{[}\PY{l+s+s1}{\PYZsq{}}\PY{l+s+s1}{param\PYZus{}hidden\PYZus{}layers}\PY{l+s+s1}{\PYZsq{}}\PY{p}{]} \PY{o}{=} \PY{n}{df\PYZus{}cv}\PY{p}{[}\PY{l+s+s1}{\PYZsq{}}\PY{l+s+s1}{param\PYZus{}hidden\PYZus{}layers}\PY{l+s+s1}{\PYZsq{}}\PY{p}{]}\PY{o}{.}\PY{n}{apply}\PY{p}{(}\PY{k}{lambda} \PY{n}{x}\PY{p}{:} \PY{p}{(}\PY{n+nb}{str}\PY{p}{(}\PY{n}{x}\PY{p}{)}\PY{p}{[}\PY{l+m+mi}{1}\PY{p}{:}\PY{o}{\PYZhy{}}\PY{l+m+mi}{1}\PY{p}{]}\PY{p}{)} \PY{k}{if} \PY{n+nb}{str}\PY{p}{(}\PY{n}{x}\PY{p}{)}\PY{p}{[}\PY{l+m+mi}{0}\PY{p}{]} \PY{o}{==} \PY{l+s+s1}{\PYZsq{}}\PY{l+s+s1}{[}\PY{l+s+s1}{\PYZsq{}} \PY{k}{else} \PY{n}{x}\PY{p}{)}

\PY{c+c1}{\PYZsh{} memilih kolom yang akan digunakan untuk interpretasi}
\PY{n}{col\PYZus{}grid} \PY{o}{=} \PY{p}{[}\PY{l+s+s1}{\PYZsq{}}\PY{l+s+s1}{param\PYZus{}activation}\PY{l+s+s1}{\PYZsq{}}\PY{p}{,} \PY{l+s+s1}{\PYZsq{}}\PY{l+s+s1}{param\PYZus{}batch\PYZus{}size}\PY{l+s+s1}{\PYZsq{}}\PY{p}{,} \PY{l+s+s1}{\PYZsq{}}\PY{l+s+s1}{param\PYZus{}epochs}\PY{l+s+s1}{\PYZsq{}}\PY{p}{,} 
            \PY{l+s+s1}{\PYZsq{}}\PY{l+s+s1}{param\PYZus{}first\PYZus{}layer}\PY{l+s+s1}{\PYZsq{}}\PY{p}{,} \PY{l+s+s1}{\PYZsq{}}\PY{l+s+s1}{param\PYZus{}hidden\PYZus{}layers}\PY{l+s+s1}{\PYZsq{}}\PY{p}{,} \PY{l+s+s1}{\PYZsq{}}\PY{l+s+s1}{param\PYZus{}optimizer}\PY{l+s+s1}{\PYZsq{}}\PY{p}{,}
            \PY{l+s+s1}{\PYZsq{}}\PY{l+s+s1}{mean\PYZus{}test\PYZus{}score}\PY{l+s+s1}{\PYZsq{}}\PY{p}{,} \PY{l+s+s1}{\PYZsq{}}\PY{l+s+s1}{rank\PYZus{}test\PYZus{}score}\PY{l+s+s1}{\PYZsq{}}
           \PY{p}{]}
\PY{n}{df\PYZus{}grid} \PY{o}{=} \PY{n}{df\PYZus{}cv}\PY{p}{[}\PY{n}{col\PYZus{}grid}\PY{p}{]}
\PY{n+nb}{print}\PY{p}{(}\PY{n}{df\PYZus{}grid}\PY{o}{.}\PY{n}{shape}\PY{p}{)}
\PY{n}{df\PYZus{}grid}\PY{o}{.}\PY{n}{head}\PY{p}{(}\PY{p}{)}
\end{Verbatim}
\end{tcolorbox}

    \begin{Verbatim}[commandchars=\\\{\}]
(54, 8)
    \end{Verbatim}

            \begin{tcolorbox}[breakable, size=fbox, boxrule=.5pt, pad at break*=1mm, opacityfill=0]
\prompt{Out}{outcolor}{ }{\boxspacing}
\begin{Verbatim}[commandchars=\\\{\}]
  param\_activation param\_batch\_size  {\ldots} mean\_test\_score rank\_test\_score
0          sigmoid                5  {\ldots}       -0.004687              22
1          sigmoid                5  {\ldots}       -0.005299              31
2          sigmoid                5  {\ldots}       -0.004517              17
3          sigmoid                5  {\ldots}       -0.004670              21
4          sigmoid                5  {\ldots}       -0.006117              37

[5 rows x 8 columns]
\end{Verbatim}
\end{tcolorbox}
        
    \begin{tcolorbox}[breakable, size=fbox, boxrule=1pt, pad at break*=1mm,colback=cellbackground, colframe=cellborder]
\prompt{In}{incolor}{ }{\boxspacing}
\begin{Verbatim}[commandchars=\\\{\}]
\PY{c+c1}{\PYZsh{} mengurutkan berdasarkan mean\PYZus{}test\PYZus{}score / rank\PYZus{}test\PYZus{}score}
\PY{n}{df\PYZus{}grid\PYZus{}sorted} \PY{o}{=} \PY{n}{df\PYZus{}grid}\PY{o}{.}\PY{n}{sort\PYZus{}values}\PY{p}{(}\PY{l+s+s1}{\PYZsq{}}\PY{l+s+s1}{rank\PYZus{}test\PYZus{}score}\PY{l+s+s1}{\PYZsq{}}\PY{p}{)}
\PY{n}{df\PYZus{}grid\PYZus{}sorted}\PY{o}{.}\PY{n}{head}\PY{p}{(}\PY{p}{)}
\end{Verbatim}
\end{tcolorbox}

            \begin{tcolorbox}[breakable, size=fbox, boxrule=.5pt, pad at break*=1mm, opacityfill=0]
\prompt{Out}{outcolor}{ }{\boxspacing}
\begin{Verbatim}[commandchars=\\\{\}]
   param\_activation param\_batch\_size  {\ldots} mean\_test\_score rank\_test\_score
14          sigmoid                5  {\ldots}       -0.003704               1
16          sigmoid                5  {\ldots}       -0.003746               2
17          sigmoid                5  {\ldots}       -0.003886               3
13          sigmoid                5  {\ldots}       -0.003978               4
29          sigmoid               10  {\ldots}       -0.004032               5

[5 rows x 8 columns]
\end{Verbatim}
\end{tcolorbox}
        
    \begin{tcolorbox}[breakable, size=fbox, boxrule=1pt, pad at break*=1mm,colback=cellbackground, colframe=cellborder]
\prompt{In}{incolor}{ }{\boxspacing}
\begin{Verbatim}[commandchars=\\\{\}]
\PY{c+c1}{\PYZsh{} hasil grid terburuk}
\PY{n}{df\PYZus{}grid\PYZus{}sorted}\PY{o}{.}\PY{n}{tail}\PY{p}{(}\PY{p}{)}
\end{Verbatim}
\end{tcolorbox}

            \begin{tcolorbox}[breakable, size=fbox, boxrule=.5pt, pad at break*=1mm, opacityfill=0]
\prompt{Out}{outcolor}{ }{\boxspacing}
\begin{Verbatim}[commandchars=\\\{\}]
   param\_activation param\_batch\_size  {\ldots} mean\_test\_score rank\_test\_score
40          sigmoid               20  {\ldots}       -0.008445              50
42          sigmoid               20  {\ldots}       -0.009949              51
18          sigmoid               10  {\ldots}       -0.009951              52
36          sigmoid               20  {\ldots}       -0.010798              53
37          sigmoid               20  {\ldots}       -0.012189              54

[5 rows x 8 columns]
\end{Verbatim}
\end{tcolorbox}
        
    \hypertarget{evaluasi-hasil-gridsearchcv}{%
\subsection{\texorpdfstring{5.2 Evaluasi hasil
\texttt{GridSearchCV}}{5.2 Evaluasi hasil GridSearchCV}}\label{evaluasi-hasil-gridsearchcv}}

    \begin{tcolorbox}[breakable, size=fbox, boxrule=1pt, pad at break*=1mm,colback=cellbackground, colframe=cellborder]
\prompt{In}{incolor}{ }{\boxspacing}
\begin{Verbatim}[commandchars=\\\{\}]
\PY{c+c1}{\PYZsh{} evaluasi 20 hasil terbaik}
\PY{n}{df\PYZus{}grid\PYZus{}top} \PY{o}{=} \PY{n}{df\PYZus{}grid\PYZus{}sorted}\PY{o}{.}\PY{n}{iloc}\PY{p}{[}\PY{p}{:}\PY{l+m+mi}{20}\PY{p}{,}\PY{p}{:}\PY{p}{]}\PY{o}{.}\PY{n}{copy}\PY{p}{(}\PY{p}{)}

\PY{c+c1}{\PYZsh{} menghitung nilai unik tiap kolom}
\PY{k}{for} \PY{n}{col} \PY{o+ow}{in} \PY{n}{df\PYZus{}grid\PYZus{}top}\PY{o}{.}\PY{n}{columns}\PY{p}{[}\PY{p}{:}\PY{o}{\PYZhy{}}\PY{l+m+mi}{2}\PY{p}{]}\PY{p}{:}
    \PY{n+nb}{print}\PY{p}{(}\PY{n}{df\PYZus{}grid\PYZus{}top}\PY{p}{[}\PY{n}{col}\PY{p}{]}\PY{o}{.}\PY{n}{value\PYZus{}counts}\PY{p}{(}\PY{p}{)}\PY{p}{)}
\end{Verbatim}
\end{tcolorbox}

    \begin{Verbatim}[commandchars=\\\{\}]
sigmoid    20
Name: param\_activation, dtype: int64
5     10
10     8
20     2
Name: param\_batch\_size, dtype: int64
200    10
150     7
100     3
Name: param\_epochs, dtype: int64
30    9
10    6
20    5
Name: param\_first\_layer, dtype: int64
20    13
30     7
Name: param\_hidden\_layers, dtype: int64
adam    20
Name: param\_optimizer, dtype: int64
    \end{Verbatim}

    \begin{tcolorbox}[breakable, size=fbox, boxrule=1pt, pad at break*=1mm,colback=cellbackground, colframe=cellborder]
\prompt{In}{incolor}{ }{\boxspacing}
\begin{Verbatim}[commandchars=\\\{\}]
\PY{c+c1}{\PYZsh{} Grafik antara rank\PYZus{}test\PYZus{}score dan mean\PYZus{}test\PYZus{}score}
\PY{n}{sns}\PY{o}{.}\PY{n}{set}\PY{p}{(}\PY{n}{style}\PY{o}{=}\PY{l+s+s1}{\PYZsq{}}\PY{l+s+s1}{ticks}\PY{l+s+s1}{\PYZsq{}}\PY{p}{)}
\PY{n}{relplot} \PY{o}{=} \PY{n}{sns}\PY{o}{.}\PY{n}{relplot}\PY{p}{(}\PY{n}{x}\PY{o}{=}\PY{l+s+s1}{\PYZsq{}}\PY{l+s+s1}{rank\PYZus{}test\PYZus{}score}\PY{l+s+s1}{\PYZsq{}}\PY{p}{,} \PY{n}{y}\PY{o}{=}\PY{l+s+s1}{\PYZsq{}}\PY{l+s+s1}{mean\PYZus{}test\PYZus{}score}\PY{l+s+s1}{\PYZsq{}}\PY{p}{,} \PY{n}{data}\PY{o}{=}\PY{n}{df\PYZus{}grid}\PY{p}{,} \PY{n}{kind}\PY{o}{=}\PY{l+s+s1}{\PYZsq{}}\PY{l+s+s1}{line}\PY{l+s+s1}{\PYZsq{}}\PY{p}{)}
\PY{n}{plt}\PY{o}{.}\PY{n}{gca}\PY{p}{(}\PY{p}{)}\PY{o}{.}\PY{n}{set\PYZus{}title}\PY{p}{(}\PY{l+s+s2}{\PYZdq{}}\PY{l+s+s2}{Kurva antara rank\PYZus{}test\PYZus{}score dengan mean\PYZus{}test\PYZus{}score}\PY{l+s+s2}{\PYZdq{}}\PY{p}{)}
\PY{n}{plt}\PY{o}{.}\PY{n}{gca}\PY{p}{(}\PY{p}{)}\PY{o}{.}\PY{n}{invert\PYZus{}xaxis}\PY{p}{(}\PY{p}{)}
\end{Verbatim}
\end{tcolorbox}

    \begin{center}
    \adjustimage{max size={0.9\linewidth}{0.9\paperheight}}{taruma_LI01_ann_ka_files/taruma_LI01_ann_ka_93_0.png}
    \end{center}
    { \hspace*{\fill} \\}
    
    \hypertarget{ringkasan}{%
\subsection{5.3 Ringkasan}\label{ringkasan}}

\begin{enumerate}
\def\labelenumi{\arabic{enumi}.}
\tightlist
\item
  Model terbaik diperoleh dengan parameter:

  \begin{itemize}
  \tightlist
  \item
    activation: \texttt{sigmoid}
  \item
    batch\_size: \texttt{5}
  \item
    epochs: \texttt{200}
  \item
    first\_layer: \texttt{20}
  \item
    hidden\_layer: \texttt{20}
  \item
    optimizer: \texttt{adam}
  \end{itemize}
\item
  Dalam evaluasi model dengan \emph{test set} diperoleh informasi:

  \begin{itemize}
  \tightlist
  \item
    MSE dan MAE sebesar \(0.1334\) dan \(0.2325\) (dalam skala
    original).
  \item
    Nilai rata-rata nilai beda (\texttt{diff}) sebesar \(0.232529\)
    (MAE), dengan standar deviasi sebesar \(0.285942\).
  \end{itemize}
\end{enumerate}

    \hypertarget{saran-dan-langkah-selanjutnya}{%
\subsection{5.4 Saran dan Langkah
Selanjutnya}\label{saran-dan-langkah-selanjutnya}}

\begin{itemize}
\tightlist
\item
  Diketahui bahwa terdapat nilai negatif pada prediksi, sehingga untuk
  model selanjutnya lebih baik bahwa pada \emph{output layer}
  menggunakan aktivasi \texttt{relu} sehingga tidak mengeluarkan hasil
  negatif.
\item
  Mengurangi vektor \emph{input}, dengan menentukan kolom yang dirasa
  penting saja untuk pemberian kolom tambahan \emph{timesteps}.
\item
  Mengevaluasi pengisian data kosong, karena pendekatan \emph{linear}
  belum tentu tepat.
\item
  Memeriksa kondisi \emph{outlier}. Dipilihnya \emph{output} amonia
  karena pada nitrogen dan nitrat terdapat kondisi \emph{outlier}.
\end{itemize}

    \begin{tcolorbox}[breakable, size=fbox, boxrule=1pt, pad at break*=1mm,colback=cellbackground, colframe=cellborder]
\prompt{In}{incolor}{ }{\boxspacing}
\begin{Verbatim}[commandchars=\\\{\}]
\PY{c+c1}{\PYZsh{} Log summary}
\PY{n+nb}{print}\PY{p}{(}\PY{n}{mylog}\PY{o}{.}\PY{n}{summary}\PY{p}{(}\PY{p}{)}\PY{p}{)}
\end{Verbatim}
\end{tcolorbox}

    \begin{Verbatim}[commandchars=\\\{\}]
[2019-07-13 05:12:41] START FITTING
[2019-07-13 05:18:28] END FITTING
[2019-07-13 05:18:28] Duration: 0:5:46
[2019-07-13 05:18:39] Model JSON disimpan di /content/gdrive/My Drive/Colab
Notebooks/\_dropbox/20190713\_0507\_kualitas\_air\_ann\_so\_2.0.0.json
[2019-07-13 05:18:40] Model Weights disimpan di /content/gdrive/My Drive/Colab
Notebooks/\_dropbox/20190713\_0507\_kualitas\_air\_ann\_so\_2.0.0\_weights.h5
[2019-07-13 05:18:42] Model disimpan di /content/gdrive/My Drive/Colab
Notebooks/\_dropbox/20190713\_0507\_kualitas\_air\_ann\_so\_2.0.0.h5
[2019-07-13 05:18:42] Tabel GridSearch disimpan di /content/gdrive/My
Drive/Colab Notebooks/\_dropbox/20190713\_0507\_kualitas\_air\_ann\_so\_2.0.0.csv
    \end{Verbatim}

    \hypertarget{changelog}{%
\section{Changelog}\label{changelog}}

\begin{verbatim}
- 20190713 - 2.0.0 - Use hidrokit 0.2.0. refactoring document/code.
- 20190622 - 1.1.0 - Copyright/License notice
- 20190414 - 1.0.0 - Initial
\end{verbatim}

    \hypertarget{copyright-2019-taruma-sakti-megariansyah}{%
\paragraph{\texorpdfstring{Copyright © 2019
\href{https://taruma.github.io}{Taruma Sakti
Megariansyah}}{Copyright © 2019 Taruma Sakti Megariansyah}}\label{copyright-2019-taruma-sakti-megariansyah}}

Source code in this notebook is licensed under a
\href{https://opensource.org/licenses/MIT}{MIT License}. Data in this
notebook is licensed under a
\href{https://choosealicense.com/licenses/cc-by-4.0/}{Creative Common
Attribution 4.0 International}.


    % Add a bibliography block to the postdoc
    
    
    
\end{document}
