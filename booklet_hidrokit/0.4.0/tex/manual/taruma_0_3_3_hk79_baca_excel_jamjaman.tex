\documentclass[11pt]{article}

    \usepackage[breakable]{tcolorbox}
    \usepackage{parskip} % Stop auto-indenting (to mimic markdown behaviour)
    
    \usepackage{iftex}
    \ifPDFTeX
    	\usepackage[T1]{fontenc}
    	\usepackage{mathpazo}
    \else
    	\usepackage{fontspec}
    \fi

    % Basic figure setup, for now with no caption control since it's done
    % automatically by Pandoc (which extracts ![](path) syntax from Markdown).
    \usepackage{graphicx}
    % Maintain compatibility with old templates. Remove in nbconvert 6.0
    \let\Oldincludegraphics\includegraphics
    % Ensure that by default, figures have no caption (until we provide a
    % proper Figure object with a Caption API and a way to capture that
    % in the conversion process - todo).
    \usepackage{caption}
    \DeclareCaptionFormat{nocaption}{}
    \captionsetup{format=nocaption,aboveskip=0pt,belowskip=0pt}

    \usepackage{float}
    \floatplacement{figure}{H} % forces figures to be placed at the correct location
    \usepackage{xcolor} % Allow colors to be defined
    \usepackage{enumerate} % Needed for markdown enumerations to work
    \usepackage{geometry} % Used to adjust the document margins
    \usepackage{amsmath} % Equations
    \usepackage{amssymb} % Equations
    \usepackage{textcomp} % defines textquotesingle
    % Hack from http://tex.stackexchange.com/a/47451/13684:
    \AtBeginDocument{%
        \def\PYZsq{\textquotesingle}% Upright quotes in Pygmentized code
    }
    \usepackage{upquote} % Upright quotes for verbatim code
    \usepackage{eurosym} % defines \euro
    \usepackage[mathletters]{ucs} % Extended unicode (utf-8) support
    \usepackage{fancyvrb} % verbatim replacement that allows latex
    \usepackage{grffile} % extends the file name processing of package graphics 
                         % to support a larger range
    \makeatletter % fix for old versions of grffile with XeLaTeX
    \@ifpackagelater{grffile}{2019/11/01}
    {
      % Do nothing on new versions
    }
    {
      \def\Gread@@xetex#1{%
        \IfFileExists{"\Gin@base".bb}%
        {\Gread@eps{\Gin@base.bb}}%
        {\Gread@@xetex@aux#1}%
      }
    }
    \makeatother
    \usepackage[Export]{adjustbox} % Used to constrain images to a maximum size
    \adjustboxset{max size={0.9\linewidth}{0.9\paperheight}}

    % The hyperref package gives us a pdf with properly built
    % internal navigation ('pdf bookmarks' for the table of contents,
    % internal cross-reference links, web links for URLs, etc.)
    \usepackage{hyperref}
    % The default LaTeX title has an obnoxious amount of whitespace. By default,
    % titling removes some of it. It also provides customization options.
    \usepackage{titling}
    \usepackage{longtable} % longtable support required by pandoc >1.10
    \usepackage{booktabs}  % table support for pandoc > 1.12.2
    \usepackage[inline]{enumitem} % IRkernel/repr support (it uses the enumerate* environment)
    \usepackage[normalem]{ulem} % ulem is needed to support strikethroughs (\sout)
                                % normalem makes italics be italics, not underlines
    \usepackage{mathrsfs}
    

    
    % Colors for the hyperref package
    \definecolor{urlcolor}{rgb}{0,.145,.698}
    \definecolor{linkcolor}{rgb}{.71,0.21,0.01}
    \definecolor{citecolor}{rgb}{.12,.54,.11}

    % ANSI colors
    \definecolor{ansi-black}{HTML}{3E424D}
    \definecolor{ansi-black-intense}{HTML}{282C36}
    \definecolor{ansi-red}{HTML}{E75C58}
    \definecolor{ansi-red-intense}{HTML}{B22B31}
    \definecolor{ansi-green}{HTML}{00A250}
    \definecolor{ansi-green-intense}{HTML}{007427}
    \definecolor{ansi-yellow}{HTML}{DDB62B}
    \definecolor{ansi-yellow-intense}{HTML}{B27D12}
    \definecolor{ansi-blue}{HTML}{208FFB}
    \definecolor{ansi-blue-intense}{HTML}{0065CA}
    \definecolor{ansi-magenta}{HTML}{D160C4}
    \definecolor{ansi-magenta-intense}{HTML}{A03196}
    \definecolor{ansi-cyan}{HTML}{60C6C8}
    \definecolor{ansi-cyan-intense}{HTML}{258F8F}
    \definecolor{ansi-white}{HTML}{C5C1B4}
    \definecolor{ansi-white-intense}{HTML}{A1A6B2}
    \definecolor{ansi-default-inverse-fg}{HTML}{FFFFFF}
    \definecolor{ansi-default-inverse-bg}{HTML}{000000}

    % common color for the border for error outputs.
    \definecolor{outerrorbackground}{HTML}{FFDFDF}

    % commands and environments needed by pandoc snippets
    % extracted from the output of `pandoc -s`
    \providecommand{\tightlist}{%
      \setlength{\itemsep}{0pt}\setlength{\parskip}{0pt}}
    \DefineVerbatimEnvironment{Highlighting}{Verbatim}{commandchars=\\\{\}}
    % Add ',fontsize=\small' for more characters per line
    \newenvironment{Shaded}{}{}
    \newcommand{\KeywordTok}[1]{\textcolor[rgb]{0.00,0.44,0.13}{\textbf{{#1}}}}
    \newcommand{\DataTypeTok}[1]{\textcolor[rgb]{0.56,0.13,0.00}{{#1}}}
    \newcommand{\DecValTok}[1]{\textcolor[rgb]{0.25,0.63,0.44}{{#1}}}
    \newcommand{\BaseNTok}[1]{\textcolor[rgb]{0.25,0.63,0.44}{{#1}}}
    \newcommand{\FloatTok}[1]{\textcolor[rgb]{0.25,0.63,0.44}{{#1}}}
    \newcommand{\CharTok}[1]{\textcolor[rgb]{0.25,0.44,0.63}{{#1}}}
    \newcommand{\StringTok}[1]{\textcolor[rgb]{0.25,0.44,0.63}{{#1}}}
    \newcommand{\CommentTok}[1]{\textcolor[rgb]{0.38,0.63,0.69}{\textit{{#1}}}}
    \newcommand{\OtherTok}[1]{\textcolor[rgb]{0.00,0.44,0.13}{{#1}}}
    \newcommand{\AlertTok}[1]{\textcolor[rgb]{1.00,0.00,0.00}{\textbf{{#1}}}}
    \newcommand{\FunctionTok}[1]{\textcolor[rgb]{0.02,0.16,0.49}{{#1}}}
    \newcommand{\RegionMarkerTok}[1]{{#1}}
    \newcommand{\ErrorTok}[1]{\textcolor[rgb]{1.00,0.00,0.00}{\textbf{{#1}}}}
    \newcommand{\NormalTok}[1]{{#1}}
    
    % Additional commands for more recent versions of Pandoc
    \newcommand{\ConstantTok}[1]{\textcolor[rgb]{0.53,0.00,0.00}{{#1}}}
    \newcommand{\SpecialCharTok}[1]{\textcolor[rgb]{0.25,0.44,0.63}{{#1}}}
    \newcommand{\VerbatimStringTok}[1]{\textcolor[rgb]{0.25,0.44,0.63}{{#1}}}
    \newcommand{\SpecialStringTok}[1]{\textcolor[rgb]{0.73,0.40,0.53}{{#1}}}
    \newcommand{\ImportTok}[1]{{#1}}
    \newcommand{\DocumentationTok}[1]{\textcolor[rgb]{0.73,0.13,0.13}{\textit{{#1}}}}
    \newcommand{\AnnotationTok}[1]{\textcolor[rgb]{0.38,0.63,0.69}{\textbf{\textit{{#1}}}}}
    \newcommand{\CommentVarTok}[1]{\textcolor[rgb]{0.38,0.63,0.69}{\textbf{\textit{{#1}}}}}
    \newcommand{\VariableTok}[1]{\textcolor[rgb]{0.10,0.09,0.49}{{#1}}}
    \newcommand{\ControlFlowTok}[1]{\textcolor[rgb]{0.00,0.44,0.13}{\textbf{{#1}}}}
    \newcommand{\OperatorTok}[1]{\textcolor[rgb]{0.40,0.40,0.40}{{#1}}}
    \newcommand{\BuiltInTok}[1]{{#1}}
    \newcommand{\ExtensionTok}[1]{{#1}}
    \newcommand{\PreprocessorTok}[1]{\textcolor[rgb]{0.74,0.48,0.00}{{#1}}}
    \newcommand{\AttributeTok}[1]{\textcolor[rgb]{0.49,0.56,0.16}{{#1}}}
    \newcommand{\InformationTok}[1]{\textcolor[rgb]{0.38,0.63,0.69}{\textbf{\textit{{#1}}}}}
    \newcommand{\WarningTok}[1]{\textcolor[rgb]{0.38,0.63,0.69}{\textbf{\textit{{#1}}}}}
    
    
    % Define a nice break command that doesn't care if a line doesn't already
    % exist.
    \def\br{\hspace*{\fill} \\* }
    % Math Jax compatibility definitions
    \def\gt{>}
    \def\lt{<}
    \let\Oldtex\TeX
    \let\Oldlatex\LaTeX
    \renewcommand{\TeX}{\textrm{\Oldtex}}
    \renewcommand{\LaTeX}{\textrm{\Oldlatex}}
    % Document parameters
    % Document title
    \title{taruma\_0\_3\_3\_hk79\_baca\_excel\_jamjaman}
    
    
    
    
    
% Pygments definitions
\makeatletter
\def\PY@reset{\let\PY@it=\relax \let\PY@bf=\relax%
    \let\PY@ul=\relax \let\PY@tc=\relax%
    \let\PY@bc=\relax \let\PY@ff=\relax}
\def\PY@tok#1{\csname PY@tok@#1\endcsname}
\def\PY@toks#1+{\ifx\relax#1\empty\else%
    \PY@tok{#1}\expandafter\PY@toks\fi}
\def\PY@do#1{\PY@bc{\PY@tc{\PY@ul{%
    \PY@it{\PY@bf{\PY@ff{#1}}}}}}}
\def\PY#1#2{\PY@reset\PY@toks#1+\relax+\PY@do{#2}}

\@namedef{PY@tok@w}{\def\PY@tc##1{\textcolor[rgb]{0.73,0.73,0.73}{##1}}}
\@namedef{PY@tok@c}{\let\PY@it=\textit\def\PY@tc##1{\textcolor[rgb]{0.24,0.48,0.48}{##1}}}
\@namedef{PY@tok@cp}{\def\PY@tc##1{\textcolor[rgb]{0.61,0.40,0.00}{##1}}}
\@namedef{PY@tok@k}{\let\PY@bf=\textbf\def\PY@tc##1{\textcolor[rgb]{0.00,0.50,0.00}{##1}}}
\@namedef{PY@tok@kp}{\def\PY@tc##1{\textcolor[rgb]{0.00,0.50,0.00}{##1}}}
\@namedef{PY@tok@kt}{\def\PY@tc##1{\textcolor[rgb]{0.69,0.00,0.25}{##1}}}
\@namedef{PY@tok@o}{\def\PY@tc##1{\textcolor[rgb]{0.40,0.40,0.40}{##1}}}
\@namedef{PY@tok@ow}{\let\PY@bf=\textbf\def\PY@tc##1{\textcolor[rgb]{0.67,0.13,1.00}{##1}}}
\@namedef{PY@tok@nb}{\def\PY@tc##1{\textcolor[rgb]{0.00,0.50,0.00}{##1}}}
\@namedef{PY@tok@nf}{\def\PY@tc##1{\textcolor[rgb]{0.00,0.00,1.00}{##1}}}
\@namedef{PY@tok@nc}{\let\PY@bf=\textbf\def\PY@tc##1{\textcolor[rgb]{0.00,0.00,1.00}{##1}}}
\@namedef{PY@tok@nn}{\let\PY@bf=\textbf\def\PY@tc##1{\textcolor[rgb]{0.00,0.00,1.00}{##1}}}
\@namedef{PY@tok@ne}{\let\PY@bf=\textbf\def\PY@tc##1{\textcolor[rgb]{0.80,0.25,0.22}{##1}}}
\@namedef{PY@tok@nv}{\def\PY@tc##1{\textcolor[rgb]{0.10,0.09,0.49}{##1}}}
\@namedef{PY@tok@no}{\def\PY@tc##1{\textcolor[rgb]{0.53,0.00,0.00}{##1}}}
\@namedef{PY@tok@nl}{\def\PY@tc##1{\textcolor[rgb]{0.46,0.46,0.00}{##1}}}
\@namedef{PY@tok@ni}{\let\PY@bf=\textbf\def\PY@tc##1{\textcolor[rgb]{0.44,0.44,0.44}{##1}}}
\@namedef{PY@tok@na}{\def\PY@tc##1{\textcolor[rgb]{0.41,0.47,0.13}{##1}}}
\@namedef{PY@tok@nt}{\let\PY@bf=\textbf\def\PY@tc##1{\textcolor[rgb]{0.00,0.50,0.00}{##1}}}
\@namedef{PY@tok@nd}{\def\PY@tc##1{\textcolor[rgb]{0.67,0.13,1.00}{##1}}}
\@namedef{PY@tok@s}{\def\PY@tc##1{\textcolor[rgb]{0.73,0.13,0.13}{##1}}}
\@namedef{PY@tok@sd}{\let\PY@it=\textit\def\PY@tc##1{\textcolor[rgb]{0.73,0.13,0.13}{##1}}}
\@namedef{PY@tok@si}{\let\PY@bf=\textbf\def\PY@tc##1{\textcolor[rgb]{0.64,0.35,0.47}{##1}}}
\@namedef{PY@tok@se}{\let\PY@bf=\textbf\def\PY@tc##1{\textcolor[rgb]{0.67,0.36,0.12}{##1}}}
\@namedef{PY@tok@sr}{\def\PY@tc##1{\textcolor[rgb]{0.64,0.35,0.47}{##1}}}
\@namedef{PY@tok@ss}{\def\PY@tc##1{\textcolor[rgb]{0.10,0.09,0.49}{##1}}}
\@namedef{PY@tok@sx}{\def\PY@tc##1{\textcolor[rgb]{0.00,0.50,0.00}{##1}}}
\@namedef{PY@tok@m}{\def\PY@tc##1{\textcolor[rgb]{0.40,0.40,0.40}{##1}}}
\@namedef{PY@tok@gh}{\let\PY@bf=\textbf\def\PY@tc##1{\textcolor[rgb]{0.00,0.00,0.50}{##1}}}
\@namedef{PY@tok@gu}{\let\PY@bf=\textbf\def\PY@tc##1{\textcolor[rgb]{0.50,0.00,0.50}{##1}}}
\@namedef{PY@tok@gd}{\def\PY@tc##1{\textcolor[rgb]{0.63,0.00,0.00}{##1}}}
\@namedef{PY@tok@gi}{\def\PY@tc##1{\textcolor[rgb]{0.00,0.52,0.00}{##1}}}
\@namedef{PY@tok@gr}{\def\PY@tc##1{\textcolor[rgb]{0.89,0.00,0.00}{##1}}}
\@namedef{PY@tok@ge}{\let\PY@it=\textit}
\@namedef{PY@tok@gs}{\let\PY@bf=\textbf}
\@namedef{PY@tok@gp}{\let\PY@bf=\textbf\def\PY@tc##1{\textcolor[rgb]{0.00,0.00,0.50}{##1}}}
\@namedef{PY@tok@go}{\def\PY@tc##1{\textcolor[rgb]{0.44,0.44,0.44}{##1}}}
\@namedef{PY@tok@gt}{\def\PY@tc##1{\textcolor[rgb]{0.00,0.27,0.87}{##1}}}
\@namedef{PY@tok@err}{\def\PY@bc##1{{\setlength{\fboxsep}{\string -\fboxrule}\fcolorbox[rgb]{1.00,0.00,0.00}{1,1,1}{\strut ##1}}}}
\@namedef{PY@tok@kc}{\let\PY@bf=\textbf\def\PY@tc##1{\textcolor[rgb]{0.00,0.50,0.00}{##1}}}
\@namedef{PY@tok@kd}{\let\PY@bf=\textbf\def\PY@tc##1{\textcolor[rgb]{0.00,0.50,0.00}{##1}}}
\@namedef{PY@tok@kn}{\let\PY@bf=\textbf\def\PY@tc##1{\textcolor[rgb]{0.00,0.50,0.00}{##1}}}
\@namedef{PY@tok@kr}{\let\PY@bf=\textbf\def\PY@tc##1{\textcolor[rgb]{0.00,0.50,0.00}{##1}}}
\@namedef{PY@tok@bp}{\def\PY@tc##1{\textcolor[rgb]{0.00,0.50,0.00}{##1}}}
\@namedef{PY@tok@fm}{\def\PY@tc##1{\textcolor[rgb]{0.00,0.00,1.00}{##1}}}
\@namedef{PY@tok@vc}{\def\PY@tc##1{\textcolor[rgb]{0.10,0.09,0.49}{##1}}}
\@namedef{PY@tok@vg}{\def\PY@tc##1{\textcolor[rgb]{0.10,0.09,0.49}{##1}}}
\@namedef{PY@tok@vi}{\def\PY@tc##1{\textcolor[rgb]{0.10,0.09,0.49}{##1}}}
\@namedef{PY@tok@vm}{\def\PY@tc##1{\textcolor[rgb]{0.10,0.09,0.49}{##1}}}
\@namedef{PY@tok@sa}{\def\PY@tc##1{\textcolor[rgb]{0.73,0.13,0.13}{##1}}}
\@namedef{PY@tok@sb}{\def\PY@tc##1{\textcolor[rgb]{0.73,0.13,0.13}{##1}}}
\@namedef{PY@tok@sc}{\def\PY@tc##1{\textcolor[rgb]{0.73,0.13,0.13}{##1}}}
\@namedef{PY@tok@dl}{\def\PY@tc##1{\textcolor[rgb]{0.73,0.13,0.13}{##1}}}
\@namedef{PY@tok@s2}{\def\PY@tc##1{\textcolor[rgb]{0.73,0.13,0.13}{##1}}}
\@namedef{PY@tok@sh}{\def\PY@tc##1{\textcolor[rgb]{0.73,0.13,0.13}{##1}}}
\@namedef{PY@tok@s1}{\def\PY@tc##1{\textcolor[rgb]{0.73,0.13,0.13}{##1}}}
\@namedef{PY@tok@mb}{\def\PY@tc##1{\textcolor[rgb]{0.40,0.40,0.40}{##1}}}
\@namedef{PY@tok@mf}{\def\PY@tc##1{\textcolor[rgb]{0.40,0.40,0.40}{##1}}}
\@namedef{PY@tok@mh}{\def\PY@tc##1{\textcolor[rgb]{0.40,0.40,0.40}{##1}}}
\@namedef{PY@tok@mi}{\def\PY@tc##1{\textcolor[rgb]{0.40,0.40,0.40}{##1}}}
\@namedef{PY@tok@il}{\def\PY@tc##1{\textcolor[rgb]{0.40,0.40,0.40}{##1}}}
\@namedef{PY@tok@mo}{\def\PY@tc##1{\textcolor[rgb]{0.40,0.40,0.40}{##1}}}
\@namedef{PY@tok@ch}{\let\PY@it=\textit\def\PY@tc##1{\textcolor[rgb]{0.24,0.48,0.48}{##1}}}
\@namedef{PY@tok@cm}{\let\PY@it=\textit\def\PY@tc##1{\textcolor[rgb]{0.24,0.48,0.48}{##1}}}
\@namedef{PY@tok@cpf}{\let\PY@it=\textit\def\PY@tc##1{\textcolor[rgb]{0.24,0.48,0.48}{##1}}}
\@namedef{PY@tok@c1}{\let\PY@it=\textit\def\PY@tc##1{\textcolor[rgb]{0.24,0.48,0.48}{##1}}}
\@namedef{PY@tok@cs}{\let\PY@it=\textit\def\PY@tc##1{\textcolor[rgb]{0.24,0.48,0.48}{##1}}}

\def\PYZbs{\char`\\}
\def\PYZus{\char`\_}
\def\PYZob{\char`\{}
\def\PYZcb{\char`\}}
\def\PYZca{\char`\^}
\def\PYZam{\char`\&}
\def\PYZlt{\char`\<}
\def\PYZgt{\char`\>}
\def\PYZsh{\char`\#}
\def\PYZpc{\char`\%}
\def\PYZdl{\char`\$}
\def\PYZhy{\char`\-}
\def\PYZsq{\char`\'}
\def\PYZdq{\char`\"}
\def\PYZti{\char`\~}
% for compatibility with earlier versions
\def\PYZat{@}
\def\PYZlb{[}
\def\PYZrb{]}
\makeatother


    % For linebreaks inside Verbatim environment from package fancyvrb. 
    \makeatletter
        \newbox\Wrappedcontinuationbox 
        \newbox\Wrappedvisiblespacebox 
        \newcommand*\Wrappedvisiblespace {\textcolor{red}{\textvisiblespace}} 
        \newcommand*\Wrappedcontinuationsymbol {\textcolor{red}{\llap{\tiny$\m@th\hookrightarrow$}}} 
        \newcommand*\Wrappedcontinuationindent {3ex } 
        \newcommand*\Wrappedafterbreak {\kern\Wrappedcontinuationindent\copy\Wrappedcontinuationbox} 
        % Take advantage of the already applied Pygments mark-up to insert 
        % potential linebreaks for TeX processing. 
        %        {, <, #, %, $, ' and ": go to next line. 
        %        _, }, ^, &, >, - and ~: stay at end of broken line. 
        % Use of \textquotesingle for straight quote. 
        \newcommand*\Wrappedbreaksatspecials {% 
            \def\PYGZus{\discretionary{\char`\_}{\Wrappedafterbreak}{\char`\_}}% 
            \def\PYGZob{\discretionary{}{\Wrappedafterbreak\char`\{}{\char`\{}}% 
            \def\PYGZcb{\discretionary{\char`\}}{\Wrappedafterbreak}{\char`\}}}% 
            \def\PYGZca{\discretionary{\char`\^}{\Wrappedafterbreak}{\char`\^}}% 
            \def\PYGZam{\discretionary{\char`\&}{\Wrappedafterbreak}{\char`\&}}% 
            \def\PYGZlt{\discretionary{}{\Wrappedafterbreak\char`\<}{\char`\<}}% 
            \def\PYGZgt{\discretionary{\char`\>}{\Wrappedafterbreak}{\char`\>}}% 
            \def\PYGZsh{\discretionary{}{\Wrappedafterbreak\char`\#}{\char`\#}}% 
            \def\PYGZpc{\discretionary{}{\Wrappedafterbreak\char`\%}{\char`\%}}% 
            \def\PYGZdl{\discretionary{}{\Wrappedafterbreak\char`\$}{\char`\$}}% 
            \def\PYGZhy{\discretionary{\char`\-}{\Wrappedafterbreak}{\char`\-}}% 
            \def\PYGZsq{\discretionary{}{\Wrappedafterbreak\textquotesingle}{\textquotesingle}}% 
            \def\PYGZdq{\discretionary{}{\Wrappedafterbreak\char`\"}{\char`\"}}% 
            \def\PYGZti{\discretionary{\char`\~}{\Wrappedafterbreak}{\char`\~}}% 
        } 
        % Some characters . , ; ? ! / are not pygmentized. 
        % This macro makes them "active" and they will insert potential linebreaks 
        \newcommand*\Wrappedbreaksatpunct {% 
            \lccode`\~`\.\lowercase{\def~}{\discretionary{\hbox{\char`\.}}{\Wrappedafterbreak}{\hbox{\char`\.}}}% 
            \lccode`\~`\,\lowercase{\def~}{\discretionary{\hbox{\char`\,}}{\Wrappedafterbreak}{\hbox{\char`\,}}}% 
            \lccode`\~`\;\lowercase{\def~}{\discretionary{\hbox{\char`\;}}{\Wrappedafterbreak}{\hbox{\char`\;}}}% 
            \lccode`\~`\:\lowercase{\def~}{\discretionary{\hbox{\char`\:}}{\Wrappedafterbreak}{\hbox{\char`\:}}}% 
            \lccode`\~`\?\lowercase{\def~}{\discretionary{\hbox{\char`\?}}{\Wrappedafterbreak}{\hbox{\char`\?}}}% 
            \lccode`\~`\!\lowercase{\def~}{\discretionary{\hbox{\char`\!}}{\Wrappedafterbreak}{\hbox{\char`\!}}}% 
            \lccode`\~`\/\lowercase{\def~}{\discretionary{\hbox{\char`\/}}{\Wrappedafterbreak}{\hbox{\char`\/}}}% 
            \catcode`\.\active
            \catcode`\,\active 
            \catcode`\;\active
            \catcode`\:\active
            \catcode`\?\active
            \catcode`\!\active
            \catcode`\/\active 
            \lccode`\~`\~ 	
        }
    \makeatother

    \let\OriginalVerbatim=\Verbatim
    \makeatletter
    \renewcommand{\Verbatim}[1][1]{%
        %\parskip\z@skip
        \sbox\Wrappedcontinuationbox {\Wrappedcontinuationsymbol}%
        \sbox\Wrappedvisiblespacebox {\FV@SetupFont\Wrappedvisiblespace}%
        \def\FancyVerbFormatLine ##1{\hsize\linewidth
            \vtop{\raggedright\hyphenpenalty\z@\exhyphenpenalty\z@
                \doublehyphendemerits\z@\finalhyphendemerits\z@
                \strut ##1\strut}%
        }%
        % If the linebreak is at a space, the latter will be displayed as visible
        % space at end of first line, and a continuation symbol starts next line.
        % Stretch/shrink are however usually zero for typewriter font.
        \def\FV@Space {%
            \nobreak\hskip\z@ plus\fontdimen3\font minus\fontdimen4\font
            \discretionary{\copy\Wrappedvisiblespacebox}{\Wrappedafterbreak}
            {\kern\fontdimen2\font}%
        }%
        
        % Allow breaks at special characters using \PYG... macros.
        \Wrappedbreaksatspecials
        % Breaks at punctuation characters . , ; ? ! and / need catcode=\active 	
        \OriginalVerbatim[#1,codes*=\Wrappedbreaksatpunct]%
    }
    \makeatother

    % Exact colors from NB
    \definecolor{incolor}{HTML}{303F9F}
    \definecolor{outcolor}{HTML}{D84315}
    \definecolor{cellborder}{HTML}{CFCFCF}
    \definecolor{cellbackground}{HTML}{F7F7F7}
    
    % prompt
    \makeatletter
    \newcommand{\boxspacing}{\kern\kvtcb@left@rule\kern\kvtcb@boxsep}
    \makeatother
    \newcommand{\prompt}[4]{
        {\ttfamily\llap{{\color{#2}[#3]:\hspace{3pt}#4}}\vspace{-\baselineskip}}
    }
    

    
    % Prevent overflowing lines due to hard-to-break entities
    \sloppy 
    % Setup hyperref package
    \hypersetup{
      breaklinks=true,  % so long urls are correctly broken across lines
      colorlinks=true,
      urlcolor=urlcolor,
      linkcolor=linkcolor,
      citecolor=citecolor,
      }
    % Slightly bigger margins than the latex defaults
    
    \geometry{verbose,tmargin=1in,bmargin=1in,lmargin=1in,rmargin=1in}
    
    

\begin{document}
    
    \maketitle
    
    

    
    Berdasarkan isu
\href{https://github.com/taruma/hidrokit/issues/79}{\#79}:
\textbf{request: ambil dataset hujan jam-jaman dari excel}

Referensi isu: - \texttt{hidrokit.contrib.taruma.hk43}
\href{https://github.com/taruma/hidrokit/issues/43}{\#43}.
(\href{https://nbviewer.jupyter.org/gist/taruma/a9dd4ea61db2526853b99600909e9c50}{view
notebook / manual}). (Menggunakan fungsi \texttt{\_get\_years()} untuk
memperoleh \texttt{list} tahun pada berkas excel.

Deskripsi permasalahan: - Membaca data jam-jaman dari excel - Data dalam
excel berupa \emph{pivot table} - Ubah \emph{pivot table} ke
\emph{regular table} - Mengubah tabel tersebut menjadi
\texttt{pandas.DataFrame}

Strategi penyelesaian masalah: - Periksa \emph{sheet} didalam berkas
excel - Baca metadata/konfigurasi excel (nama stasiun) - Untuk setiap
\emph{sheet} dengan digit tahun, baca setiap \emph{sheet} (menggunakan
\texttt{hk43.\_get\_years()}) - Dalam \emph{sheet} tunggal: - Baca
informasi tahun di lembar aktif - Membaca dan mempraproses dataset
setiap bulannya - Menggabungkan hasil pengubahan pivot ke tabel reguler
- Menggabungkan tabel reguler dalam satu tahun - Menggabungkan tabel
setiap tahun menjadi satu \texttt{pandas.DataFrame}

Catatan: Untuk prapemrosesan tabel seperti (cek data yang hilang,
dlsbnya) akan dikembangkan dengan modul yang terpisah karena beberapa
fungsi sudah tersedia di modul \texttt{hidrokit.contrib.taruma.hk73}
(untuk mengolah berkas dari bmkg).

    \hypertarget{persiapan-dan-dataset}{%
\section{PERSIAPAN DAN DATASET}\label{persiapan-dan-dataset}}

    \begin{tcolorbox}[breakable, size=fbox, boxrule=1pt, pad at break*=1mm,colback=cellbackground, colframe=cellborder]
\prompt{In}{incolor}{ }{\boxspacing}
\begin{Verbatim}[commandchars=\\\{\}]
\PY{c+c1}{\PYZsh{} Install hidrokit 0.3.x}
\PY{o}{!}pip install hidrokit \PYZhy{}q
\end{Verbatim}
\end{tcolorbox}

    \begin{tcolorbox}[breakable, size=fbox, boxrule=1pt, pad at break*=1mm,colback=cellbackground, colframe=cellborder]
\prompt{In}{incolor}{ }{\boxspacing}
\begin{Verbatim}[commandchars=\\\{\}]
\PY{c+c1}{\PYZsh{} Download sample excel}
\PY{o}{!}wget \PYZhy{}O sample.xlsx \PY{l+s+s2}{\PYZdq{}https://taruma.github.io/assets/hidrokit\PYZus{}dataset/hidrokit\PYZus{}hourly\PYZus{}template.xlsx\PYZdq{}} \PYZhy{}q
\end{Verbatim}
\end{tcolorbox}

    \begin{tcolorbox}[breakable, size=fbox, boxrule=1pt, pad at break*=1mm,colback=cellbackground, colframe=cellborder]
\prompt{In}{incolor}{ }{\boxspacing}
\begin{Verbatim}[commandchars=\\\{\}]
\PY{c+c1}{\PYZsh{} Import library}
\PY{k+kn}{import} \PY{n+nn}{pandas} \PY{k}{as} \PY{n+nn}{pd}
\PY{k+kn}{import} \PY{n+nn}{numpy} \PY{k}{as} \PY{n+nn}{np}
\end{Verbatim}
\end{tcolorbox}

    \hypertarget{kode}{%
\section{KODE}\label{kode}}

    \begin{tcolorbox}[breakable, size=fbox, boxrule=1pt, pad at break*=1mm,colback=cellbackground, colframe=cellborder]
\prompt{In}{incolor}{ }{\boxspacing}
\begin{Verbatim}[commandchars=\\\{\}]
\PY{k+kn}{from} \PY{n+nn}{calendar} \PY{k+kn}{import} \PY{n}{monthrange}
\PY{k+kn}{from} \PY{n+nn}{hidrokit}\PY{n+nn}{.}\PY{n+nn}{contrib}\PY{n+nn}{.}\PY{n+nn}{taruma} \PY{k+kn}{import} \PY{n}{hk43}

\PY{c+c1}{\PYZsh{} ref: https://www.reddit.com/r/learnpython/comments/485h1p/how\PYZus{}do\PYZus{}i\PYZus{}check\PYZus{}if\PYZus{}an\PYZus{}object\PYZus{}is\PYZus{}a\PYZus{}collection\PYZus{}in/}
\PY{k+kn}{from} \PY{n+nn}{collections}\PY{n+nn}{.}\PY{n+nn}{abc} \PY{k+kn}{import} \PY{n}{Sequence}

\PY{k}{def} \PY{n+nf}{\PYZus{}index\PYZus{}hourly}\PY{p}{(}\PY{n}{year}\PY{p}{,} \PY{n}{freq}\PY{o}{=}\PY{l+s+s1}{\PYZsq{}}\PY{l+s+s1}{60min}\PY{l+s+s1}{\PYZsq{}}\PY{p}{)}\PY{p}{:}
    \PY{l+s+sd}{\PYZdq{}\PYZdq{}\PYZdq{}Return object DatetimeIndex\PYZdq{}\PYZdq{}\PYZdq{}}
    \PY{k}{if} \PY{n+nb}{isinstance}\PY{p}{(}\PY{n}{year}\PY{p}{,} \PY{n}{Sequence}\PY{p}{)}\PY{p}{:}
        \PY{n}{year\PYZus{}start}\PY{p}{,} \PY{n}{year\PYZus{}end} \PY{o}{=} \PY{n}{year}
    \PY{k}{else}\PY{p}{:}
        \PY{n}{year\PYZus{}start}\PY{p}{,} \PY{n}{year\PYZus{}end} \PY{o}{=} \PY{n}{year}\PY{p}{,} \PY{n}{year}
    
    \PY{n}{period} \PY{o}{=} \PY{l+s+s1}{\PYZsq{}}\PY{l+s+si}{\PYZob{}\PYZcb{}}\PY{l+s+s1}{0101 00:00,}\PY{l+s+si}{\PYZob{}\PYZcb{}}\PY{l+s+s1}{1231 23:00}\PY{l+s+s1}{\PYZsq{}}\PY{o}{.}\PY{n}{format}\PY{p}{(}\PY{n}{year\PYZus{}start}\PY{p}{,} \PY{n}{year\PYZus{}end}\PY{p}{)}\PY{o}{.}\PY{n}{split}\PY{p}{(}\PY{l+s+s1}{\PYZsq{}}\PY{l+s+s1}{,}\PY{l+s+s1}{\PYZsq{}}\PY{p}{)}
    \PY{k}{return} \PY{n}{pd}\PY{o}{.}\PY{n}{date\PYZus{}range}\PY{p}{(}\PY{o}{*}\PY{n}{period}\PY{p}{,} \PY{n}{freq}\PY{o}{=}\PY{n}{freq}\PY{p}{)}

\PY{k}{def} \PY{n+nf}{\PYZus{}melt\PYZus{}to\PYZus{}array}\PY{p}{(}\PY{n}{df}\PY{p}{)}\PY{p}{:}
    \PY{k}{return} \PY{n}{df}\PY{o}{.}\PY{n}{melt}\PY{p}{(}\PY{p}{)}\PY{o}{.}\PY{n}{drop}\PY{p}{(}\PY{l+s+s1}{\PYZsq{}}\PY{l+s+s1}{variable}\PY{l+s+s1}{\PYZsq{}}\PY{p}{,} \PY{n}{axis}\PY{o}{=}\PY{l+m+mi}{1}\PY{p}{)}\PY{p}{[}\PY{l+s+s1}{\PYZsq{}}\PY{l+s+s1}{value}\PY{l+s+s1}{\PYZsq{}}\PY{p}{]}\PY{o}{.}\PY{n}{values}

\PY{k}{def} \PY{n+nf}{\PYZus{}get\PYZus{}array\PYZus{}in\PYZus{}month}\PY{p}{(}\PY{n}{df}\PY{p}{,} \PY{n}{year}\PY{p}{,} \PY{n}{month}\PY{p}{)}\PY{p}{:}
    \PY{n}{n\PYZus{}days} \PY{o}{=} \PY{n}{monthrange}\PY{p}{(}\PY{n}{year}\PY{p}{,} \PY{n}{month}\PY{p}{)}\PY{p}{[}\PY{l+m+mi}{1}\PY{p}{]}
    \PY{n}{mask\PYZus{}month} \PY{o}{=} \PY{n+nb}{slice}\PY{p}{(}\PY{k+kc}{None}\PY{p}{,} \PY{n}{n\PYZus{}days}\PY{p}{)}
    \PY{n}{df\PYZus{}month} \PY{o}{=} \PY{n}{df}\PY{o}{.}\PY{n}{iloc}\PY{p}{[}\PY{n}{mask\PYZus{}month}\PY{p}{,} \PY{p}{:}\PY{p}{]}\PY{o}{.}\PY{n}{T}
    \PY{k}{return} \PY{n}{\PYZus{}melt\PYZus{}to\PYZus{}array}\PY{p}{(}\PY{n}{df\PYZus{}month}\PY{p}{)}

\PY{k}{def} \PY{n+nf}{\PYZus{}get\PYZus{}year}\PY{p}{(}\PY{n}{df}\PY{p}{,} \PY{n}{loc}\PY{o}{=}\PY{p}{(}\PY{l+m+mi}{0}\PY{p}{,}\PY{l+m+mi}{1}\PY{p}{)}\PY{p}{)}\PY{p}{:}
    \PY{k}{return} \PY{n}{df}\PY{o}{.}\PY{n}{iloc}\PY{p}{[}\PY{n}{loc}\PY{p}{]}

\PY{k}{def} \PY{n+nf}{\PYZus{}get\PYZus{}array\PYZus{}in\PYZus{}year}\PY{p}{(}\PY{n}{df}\PY{p}{,} \PY{n}{year}\PY{p}{)}\PY{p}{:}
    \PY{n}{n\PYZus{}rows}\PY{p}{,} \PY{n}{\PYZus{}} \PY{o}{=} \PY{n}{df}\PY{o}{.}\PY{n}{shape}

    \PY{c+c1}{\PYZsh{} configuration (view the excel)}
    \PY{n}{n\PYZus{}month} \PY{o}{=} \PY{l+m+mi}{1} \PY{c+c1}{\PYZsh{} number of row to monthID}
    \PY{n}{n\PYZus{}gap} \PY{o}{=} \PY{l+m+mi}{2} \PY{c+c1}{\PYZsh{} number of row between month pivot table}
    \PY{n}{n\PYZus{}lines} \PY{o}{=} \PY{l+m+mi}{31} \PY{o}{+} \PY{n}{n\PYZus{}gap} \PY{c+c1}{\PYZsh{} number of row each month}

    \PY{n}{data} \PY{o}{=} \PY{p}{[}\PY{p}{]}
    \PY{k}{for} \PY{n}{row} \PY{o+ow}{in} \PY{n+nb}{range}\PY{p}{(}\PY{l+m+mi}{1}\PY{p}{,} \PY{n}{n\PYZus{}rows}\PY{p}{,} \PY{n}{n\PYZus{}lines}\PY{p}{)}\PY{p}{:}
        \PY{n}{mask\PYZus{}start} \PY{o}{=} \PY{n}{row} \PY{o}{+} \PY{n}{n\PYZus{}month}
        \PY{n}{mask\PYZus{}end} \PY{o}{=} \PY{n}{row} \PY{o}{+} \PY{n}{n\PYZus{}lines}

        \PY{n}{month} \PY{o}{=} \PY{n}{df}\PY{o}{.}\PY{n}{iloc}\PY{p}{[}\PY{n}{mask\PYZus{}start}\PY{p}{,} \PY{l+m+mi}{1}\PY{p}{]}
        \PY{n}{mask\PYZus{}row} \PY{o}{=} \PY{n+nb}{slice}\PY{p}{(}\PY{n}{mask\PYZus{}start}\PY{p}{,} \PY{n}{mask\PYZus{}end}\PY{p}{)}
        
        \PY{n}{df\PYZus{}month} \PY{o}{=} \PY{n}{df}\PY{o}{.}\PY{n}{iloc}\PY{p}{[}\PY{n}{mask\PYZus{}row}\PY{p}{,} \PY{l+m+mi}{4}\PY{p}{:}\PY{p}{]}
        \PY{n}{array\PYZus{}month} \PY{o}{=} \PY{n}{\PYZus{}get\PYZus{}array\PYZus{}in\PYZus{}month}\PY{p}{(}\PY{n}{df\PYZus{}month}\PY{p}{,} \PY{n}{year}\PY{p}{,} \PY{n}{month}\PY{p}{)}
        \PY{n}{data}\PY{o}{.}\PY{n}{append}\PY{p}{(}\PY{n}{array\PYZus{}month}\PY{p}{)}

    \PY{k}{return} \PY{n}{np}\PY{o}{.}\PY{n}{hstack}\PY{p}{(}\PY{n}{data}\PY{p}{)}

\PY{k}{def} \PY{n+nf}{\PYZus{}get\PYZus{}info}\PY{p}{(}\PY{n}{file}\PY{p}{,} \PY{n}{config\PYZus{}sheet}\PY{o}{=}\PY{k+kc}{None}\PY{p}{)}\PY{p}{:}
    \PY{n}{excel} \PY{o}{=} \PY{n}{pd}\PY{o}{.}\PY{n}{ExcelFile}\PY{p}{(}\PY{n}{file}\PY{p}{)}
    \PY{n}{first\PYZus{}sheet} \PY{o}{=} \PY{n}{excel}\PY{o}{.}\PY{n}{sheet\PYZus{}names}\PY{p}{[}\PY{l+m+mi}{0}\PY{p}{]}
    \PY{n}{config\PYZus{}sheet} \PY{o}{=} \PY{n}{first\PYZus{}sheet} \PY{k}{if} \PY{n}{config\PYZus{}sheet} \PY{o+ow}{is} \PY{k+kc}{None} \PY{k}{else} \PY{n}{config\PYZus{}sheet}

    \PY{n}{df} \PY{o}{=} \PY{n}{pd}\PY{o}{.}\PY{n}{read\PYZus{}excel}\PY{p}{(}
        \PY{n}{excel}\PY{p}{,} \PY{n}{sheet\PYZus{}name}\PY{o}{=}\PY{n}{config\PYZus{}sheet}\PY{p}{,} \PY{n}{header}\PY{o}{=}\PY{k+kc}{None}\PY{p}{,} \PY{n}{usecols}\PY{o}{=}\PY{l+s+s1}{\PYZsq{}}\PY{l+s+s1}{A:B}\PY{l+s+s1}{\PYZsq{}}
    \PY{p}{)}
    \PY{n}{info} \PY{o}{=} \PY{p}{\PYZob{}}\PY{p}{\PYZcb{}}

    \PY{k}{for} \PY{n}{index}\PY{p}{,} \PY{n}{\PYZus{}} \PY{o+ow}{in} \PY{n}{df}\PY{o}{.}\PY{n}{iterrows}\PY{p}{(}\PY{p}{)}\PY{p}{:}
        \PY{n}{key} \PY{o}{=} \PY{n}{df}\PY{o}{.}\PY{n}{iloc}\PY{p}{[}\PY{n}{index}\PY{p}{,} \PY{l+m+mi}{0}\PY{p}{]}\PY{o}{.}\PY{n}{lower}\PY{p}{(}\PY{p}{)}
        \PY{n}{value} \PY{o}{=} \PY{n}{df}\PY{o}{.}\PY{n}{iloc}\PY{p}{[}\PY{n}{index}\PY{p}{,} \PY{l+m+mi}{1}\PY{p}{]}
        \PY{n}{info}\PY{p}{[}\PY{n+nb}{str}\PY{p}{(}\PY{n}{key}\PY{p}{)}\PY{p}{]} \PY{o}{=} \PY{n}{value}

    \PY{k}{return} \PY{n}{info}

\PY{k}{def} \PY{n+nf}{read\PYZus{}excel\PYZus{}hourly}\PY{p}{(}\PY{n}{file}\PY{p}{,} \PY{n}{station}\PY{o}{=}\PY{k+kc}{None}\PY{p}{)}\PY{p}{:}
    \PY{n}{excel} \PY{o}{=} \PY{n}{pd}\PY{o}{.}\PY{n}{ExcelFile}\PY{p}{(}\PY{n}{file}\PY{p}{)}

    \PY{c+c1}{\PYZsh{} CONFIG}
    \PY{n}{years} \PY{o}{=} \PY{n}{hk43}\PY{o}{.}\PY{n}{\PYZus{}get\PYZus{}years}\PY{p}{(}\PY{n}{excel}\PY{p}{)}
    \PY{n}{station} \PY{o}{=} \PY{l+s+s1}{\PYZsq{}}\PY{l+s+s1}{NA}\PY{l+s+s1}{\PYZsq{}} \PY{k}{if} \PY{n}{station} \PY{o+ow}{is} \PY{k+kc}{None} \PY{k}{else} \PY{n}{station}

    \PY{c+c1}{\PYZsh{} READ DATA}
    \PY{n}{data} \PY{o}{=} \PY{p}{[}\PY{p}{]}
    \PY{k}{for} \PY{n}{year} \PY{o+ow}{in} \PY{n}{years}\PY{p}{:}
        \PY{n}{sheet} \PY{o}{=} \PY{n}{pd}\PY{o}{.}\PY{n}{read\PYZus{}excel}\PY{p}{(}
            \PY{n}{excel}\PY{p}{,} \PY{n}{sheet\PYZus{}name}\PY{o}{=}\PY{n+nb}{str}\PY{p}{(}\PY{n}{year}\PY{p}{)}\PY{p}{,}
            \PY{n}{header}\PY{o}{=}\PY{k+kc}{None}\PY{p}{,} \PY{n}{nrows}\PY{o}{=}\PY{l+m+mi}{396}\PY{p}{,}
            \PY{n}{usecols}\PY{o}{=}\PY{l+s+s1}{\PYZsq{}}\PY{l+s+s1}{A:AB}\PY{l+s+s1}{\PYZsq{}}
        \PY{p}{)}
        \PY{n}{array} \PY{o}{=} \PY{n}{\PYZus{}get\PYZus{}array\PYZus{}in\PYZus{}year}\PY{p}{(}\PY{n}{sheet}\PY{p}{,} \PY{n}{year}\PY{p}{)}
        \PY{n}{df\PYZus{}year} \PY{o}{=} \PY{n}{pd}\PY{o}{.}\PY{n}{DataFrame}\PY{p}{(}
            \PY{n}{data}\PY{o}{=}\PY{n}{array}\PY{p}{,}
            \PY{n}{columns}\PY{o}{=}\PY{p}{[}\PY{n}{station}\PY{p}{]}\PY{p}{,}
            \PY{n}{index}\PY{o}{=}\PY{n}{\PYZus{}index\PYZus{}hourly}\PY{p}{(}\PY{n}{year}\PY{p}{)}
        \PY{p}{)}
        \PY{n}{data}\PY{o}{.}\PY{n}{append}\PY{p}{(}\PY{n}{df\PYZus{}year}\PY{p}{)}

    \PY{k}{return} \PY{n}{pd}\PY{o}{.}\PY{n}{concat}\PY{p}{(}\PY{n}{data}\PY{p}{,} \PY{n}{axis}\PY{o}{=}\PY{l+m+mi}{0}\PY{p}{)}
\end{Verbatim}
\end{tcolorbox}

    \hypertarget{fungsi}{%
\section{FUNGSI}\label{fungsi}}

    \hypertarget{fungsi-private-_index_hourlyyear-freq60min}{%
\subsection{\texorpdfstring{Fungsi \emph{private}
\texttt{\_index\_hourly(year,\ freq=\textquotesingle{}60min\textquotesingle{})}}{Fungsi private \_index\_hourly(year, freq='60min')}}\label{fungsi-private-_index_hourlyyear-freq60min}}

Tujuan: membuat index menggunakan perintah \texttt{pd.date\_range()}
dengan input \texttt{year} yang berupa bilangan ataupun \emph{sequence}.

    \begin{tcolorbox}[breakable, size=fbox, boxrule=1pt, pad at break*=1mm,colback=cellbackground, colframe=cellborder]
\prompt{In}{incolor}{ }{\boxspacing}
\begin{Verbatim}[commandchars=\\\{\}]
\PY{n}{\PYZus{}index\PYZus{}hourly}\PY{p}{(}\PY{l+m+mi}{2000}\PY{p}{)} \PY{c+c1}{\PYZsh{} jika bilangan harus dalam bentuk integer}
\end{Verbatim}
\end{tcolorbox}

            \begin{tcolorbox}[breakable, size=fbox, boxrule=.5pt, pad at break*=1mm, opacityfill=0]
\prompt{Out}{outcolor}{ }{\boxspacing}
\begin{Verbatim}[commandchars=\\\{\}]
DatetimeIndex(['2000-01-01 00:00:00', '2000-01-01 01:00:00',
               '2000-01-01 02:00:00', '2000-01-01 03:00:00',
               '2000-01-01 04:00:00', '2000-01-01 05:00:00',
               '2000-01-01 06:00:00', '2000-01-01 07:00:00',
               '2000-01-01 08:00:00', '2000-01-01 09:00:00',
               {\ldots}
               '2000-12-31 14:00:00', '2000-12-31 15:00:00',
               '2000-12-31 16:00:00', '2000-12-31 17:00:00',
               '2000-12-31 18:00:00', '2000-12-31 19:00:00',
               '2000-12-31 20:00:00', '2000-12-31 21:00:00',
               '2000-12-31 22:00:00', '2000-12-31 23:00:00'],
              dtype='datetime64[ns]', length=8784, freq='60T')
\end{Verbatim}
\end{tcolorbox}
        
    \begin{tcolorbox}[breakable, size=fbox, boxrule=1pt, pad at break*=1mm,colback=cellbackground, colframe=cellborder]
\prompt{In}{incolor}{ }{\boxspacing}
\begin{Verbatim}[commandchars=\\\{\}]
\PY{n}{\PYZus{}index\PYZus{}hourly}\PY{p}{(}\PY{p}{[}\PY{l+s+s1}{\PYZsq{}}\PY{l+s+s1}{2000}\PY{l+s+s1}{\PYZsq{}}\PY{p}{,} \PY{l+m+mi}{2001}\PY{p}{]}\PY{p}{)} \PY{c+c1}{\PYZsh{} jika dalam seq\PYZhy{}object bisa berupa integer atau string}
\end{Verbatim}
\end{tcolorbox}

            \begin{tcolorbox}[breakable, size=fbox, boxrule=.5pt, pad at break*=1mm, opacityfill=0]
\prompt{Out}{outcolor}{ }{\boxspacing}
\begin{Verbatim}[commandchars=\\\{\}]
DatetimeIndex(['2000-01-01 00:00:00', '2000-01-01 01:00:00',
               '2000-01-01 02:00:00', '2000-01-01 03:00:00',
               '2000-01-01 04:00:00', '2000-01-01 05:00:00',
               '2000-01-01 06:00:00', '2000-01-01 07:00:00',
               '2000-01-01 08:00:00', '2000-01-01 09:00:00',
               {\ldots}
               '2001-12-31 14:00:00', '2001-12-31 15:00:00',
               '2001-12-31 16:00:00', '2001-12-31 17:00:00',
               '2001-12-31 18:00:00', '2001-12-31 19:00:00',
               '2001-12-31 20:00:00', '2001-12-31 21:00:00',
               '2001-12-31 22:00:00', '2001-12-31 23:00:00'],
              dtype='datetime64[ns]', length=17544, freq='60T')
\end{Verbatim}
\end{tcolorbox}
        
    \hypertarget{fungsi-private-_melt_to_arraydf}{%
\subsection{\texorpdfstring{Fungsi \emph{private}
\texttt{\_melt\_to\_array(df)}}{Fungsi private \_melt\_to\_array(df)}}\label{fungsi-private-_melt_to_arraydf}}

Tujuan: perintah
\texttt{df.melt().drop(\textquotesingle{}variable\textquotesingle{},\ axis=1){[}\textquotesingle{}value\textquotesingle{}{]}.values}

Contoh \texttt{pd.melt} bisa lihat pada manual
\href{https://nbviewer.jupyter.org/gist/taruma/a9dd4ea61db2526853b99600909e9c50}{hk43}.

    \hypertarget{fungsi-private-_get_array_in_monthdf-year-month}{%
\subsection{\texorpdfstring{Fungsi \emph{private}
\texttt{\_get\_array\_in\_month(df,\ year,\ month)}}{Fungsi private \_get\_array\_in\_month(df, year, month)}}\label{fungsi-private-_get_array_in_monthdf-year-month}}

Tujuan: mengambil pivot tabel satu bulan dan mengubahnya (\emph{melt})
ke bentuk tabel biasa.

    \hypertarget{fungsi-private-_get_yeardf-loc}{%
\subsection{\texorpdfstring{Fungsi \emph{private}
\texttt{\_get\_year(df,\ loc)}}{Fungsi private \_get\_year(df, loc)}}\label{fungsi-private-_get_yeardf-loc}}

Tujuan: melakukan perintah \texttt{df.iloc{[}loc{]}}

    \hypertarget{fungsi-private-_get_array_in_yeardf-year}{%
\subsection{\texorpdfstring{Fungsi \emph{private}
\texttt{\_get\_array\_in\_year(df,\ year)}}{Fungsi private \_get\_array\_in\_year(df, year)}}\label{fungsi-private-_get_array_in_yeardf-year}}

Tujuan: serupa dengan \texttt{\_get\_array\_in\_month()}, fungsi ini
mengambil seluruh informasi pada \emph{sheet} tunggal.

    \hypertarget{fungsi-private-_get_infofile-config_sheetnone}{%
\subsection{\texorpdfstring{Fungsi \emph{private}
\texttt{\_get\_info(file,\ config\_sheet=None)}}{Fungsi private \_get\_info(file, config\_sheet=None)}}\label{fungsi-private-_get_infofile-config_sheetnone}}

Tujuan: mengambil nilai pada \emph{sheet} pengaturan (\emph{sheet}
pertama pada file) dan mengubahnya ke dalam bentuk \texttt{dictionary}.

    \begin{tcolorbox}[breakable, size=fbox, boxrule=1pt, pad at break*=1mm,colback=cellbackground, colframe=cellborder]
\prompt{In}{incolor}{ }{\boxspacing}
\begin{Verbatim}[commandchars=\\\{\}]
\PY{n}{info\PYZus{}file} \PY{o}{=} \PY{n}{\PYZus{}get\PYZus{}info}\PY{p}{(}\PY{l+s+s1}{\PYZsq{}}\PY{l+s+s1}{sample.xlsx}\PY{l+s+s1}{\PYZsq{}}\PY{p}{,} \PY{n}{config\PYZus{}sheet}\PY{o}{=}\PY{l+s+s1}{\PYZsq{}}\PY{l+s+s1}{\PYZus{}INFO}\PY{l+s+s1}{\PYZsq{}}\PY{p}{)}
\PY{n}{info\PYZus{}file}
\end{Verbatim}
\end{tcolorbox}

            \begin{tcolorbox}[breakable, size=fbox, boxrule=.5pt, pad at break*=1mm, opacityfill=0]
\prompt{Out}{outcolor}{ }{\boxspacing}
\begin{Verbatim}[commandchars=\\\{\}]
\{'key': 'VALUE', 'station\_name': 'AURENE'\}
\end{Verbatim}
\end{tcolorbox}
        
    \begin{tcolorbox}[breakable, size=fbox, boxrule=1pt, pad at break*=1mm,colback=cellbackground, colframe=cellborder]
\prompt{In}{incolor}{ }{\boxspacing}
\begin{Verbatim}[commandchars=\\\{\}]
\PY{n}{info\PYZus{}file}\PY{p}{[}\PY{l+s+s1}{\PYZsq{}}\PY{l+s+s1}{station\PYZus{}name}\PY{l+s+s1}{\PYZsq{}}\PY{p}{]}
\end{Verbatim}
\end{tcolorbox}

            \begin{tcolorbox}[breakable, size=fbox, boxrule=.5pt, pad at break*=1mm, opacityfill=0]
\prompt{Out}{outcolor}{ }{\boxspacing}
\begin{Verbatim}[commandchars=\\\{\}]
'AURENE'
\end{Verbatim}
\end{tcolorbox}
        
    \hypertarget{penerapan}{%
\section{PENERAPAN}\label{penerapan}}

    \hypertarget{fungsi-public-read_excel_hourlyfile-stationnone}{%
\subsection{\texorpdfstring{Fungsi \emph{public}
\texttt{read\_excel\_hourly(file,\ station=None)}}{Fungsi public read\_excel\_hourly(file, station=None)}}\label{fungsi-public-read_excel_hourlyfile-stationnone}}

Tujuan: membaca data jam-jaman yang terdapat pada file lalu mengubahnya
ke dalam bentuk \texttt{pandas.DataFrame} dengan index yang sesuai
dengan tahun kejadiannya.

    \begin{tcolorbox}[breakable, size=fbox, boxrule=1pt, pad at break*=1mm,colback=cellbackground, colframe=cellborder]
\prompt{In}{incolor}{ }{\boxspacing}
\begin{Verbatim}[commandchars=\\\{\}]
\PY{n}{data} \PY{o}{=} \PY{n}{read\PYZus{}excel\PYZus{}hourly}\PY{p}{(}\PY{l+s+s1}{\PYZsq{}}\PY{l+s+s1}{sample.xlsx}\PY{l+s+s1}{\PYZsq{}}\PY{p}{,} \PY{n}{station}\PY{o}{=}\PY{n}{info\PYZus{}file}\PY{p}{[}\PY{l+s+s1}{\PYZsq{}}\PY{l+s+s1}{station\PYZus{}name}\PY{l+s+s1}{\PYZsq{}}\PY{p}{]}\PY{p}{)}
\PY{n}{data}
\end{Verbatim}
\end{tcolorbox}

            \begin{tcolorbox}[breakable, size=fbox, boxrule=.5pt, pad at break*=1mm, opacityfill=0]
\prompt{Out}{outcolor}{ }{\boxspacing}
\begin{Verbatim}[commandchars=\\\{\}]
                    AURENE
2000-01-01 00:00:00      -
2000-01-01 01:00:00    NaN
2000-01-01 02:00:00    NaN
2000-01-01 03:00:00    NaN
2000-01-01 04:00:00    NaN
{\ldots}                    {\ldots}
2002-12-31 19:00:00      -
2002-12-31 20:00:00      -
2002-12-31 21:00:00      -
2002-12-31 22:00:00      -
2002-12-31 23:00:00      -

[26304 rows x 1 columns]
\end{Verbatim}
\end{tcolorbox}
        
    \begin{tcolorbox}[breakable, size=fbox, boxrule=1pt, pad at break*=1mm,colback=cellbackground, colframe=cellborder]
\prompt{In}{incolor}{ }{\boxspacing}
\begin{Verbatim}[commandchars=\\\{\}]
\PY{n}{data}\PY{o}{.}\PY{n}{info}\PY{p}{(}\PY{p}{)}
\end{Verbatim}
\end{tcolorbox}

    \begin{Verbatim}[commandchars=\\\{\}]
<class 'pandas.core.frame.DataFrame'>
DatetimeIndex: 26304 entries, 2000-01-01 00:00:00 to 2002-12-31 23:00:00
Freq: 60T
Data columns (total 1 columns):
AURENE    10918 non-null object
dtypes: object(1)
memory usage: 411.0+ KB
    \end{Verbatim}

    Catatan: data masih berupa \texttt{object}, dan belum diubah ke bentuk
angka. Prapemrosesan ini serupa pada modul
\href{https://nbviewer.jupyter.org/gist/taruma/b00880905f297013f046dad95dc2e284}{hk73}
(untuk membaca berkas bmkg).

    \begin{tcolorbox}[breakable, size=fbox, boxrule=1pt, pad at break*=1mm,colback=cellbackground, colframe=cellborder]
\prompt{In}{incolor}{ }{\boxspacing}
\begin{Verbatim}[commandchars=\\\{\}]
\PY{n}{data}\PY{o}{.}\PY{n}{sample}\PY{p}{(}\PY{n}{n}\PY{o}{=}\PY{l+m+mi}{20}\PY{p}{)} \PY{c+c1}{\PYZsh{} menampilkan sampel 20 baris dalam data secara acak}
\end{Verbatim}
\end{tcolorbox}

            \begin{tcolorbox}[breakable, size=fbox, boxrule=.5pt, pad at break*=1mm, opacityfill=0]
\prompt{Out}{outcolor}{ }{\boxspacing}
\begin{Verbatim}[commandchars=\\\{\}]
                    AURENE
2002-07-26 01:00:00      -
2001-11-09 02:00:00    NaN
2000-08-16 00:00:00      -
2000-11-26 10:00:00    NaN
2001-05-21 09:00:00    NaN
2000-09-16 08:00:00    NaN
2001-06-30 02:00:00    NaN
2000-11-26 00:00:00      -
2001-08-22 19:00:00    NaN
2001-10-27 18:00:00    NaN
2002-03-31 07:00:00      -
2000-10-01 06:00:00    NaN
2002-07-14 09:00:00      -
2000-11-06 16:00:00    NaN
2001-03-11 07:00:00    NaN
2000-10-04 04:00:00    NaN
2002-01-25 15:00:00      -
2001-06-27 13:00:00    NaN
2002-07-15 07:00:00      -
2000-08-08 15:00:00    NaN
\end{Verbatim}
\end{tcolorbox}
        
    \begin{tcolorbox}[breakable, size=fbox, boxrule=1pt, pad at break*=1mm,colback=cellbackground, colframe=cellborder]
\prompt{In}{incolor}{ }{\boxspacing}
\begin{Verbatim}[commandchars=\\\{\}]
\PY{n}{data}\PY{p}{[}\PY{l+s+s1}{\PYZsq{}}\PY{l+s+s1}{2001}\PY{l+s+s1}{\PYZsq{}}\PY{p}{]} \PY{c+c1}{\PYZsh{} menampilkan data pada tahun tertentu}
\end{Verbatim}
\end{tcolorbox}

            \begin{tcolorbox}[breakable, size=fbox, boxrule=.5pt, pad at break*=1mm, opacityfill=0]
\prompt{Out}{outcolor}{ }{\boxspacing}
\begin{Verbatim}[commandchars=\\\{\}]
                    AURENE
2001-01-01 00:00:00    NaN
2001-01-01 01:00:00    NaN
2001-01-01 02:00:00    NaN
2001-01-01 03:00:00    NaN
2001-01-01 04:00:00    NaN
{\ldots}                    {\ldots}
2001-12-31 19:00:00    NaN
2001-12-31 20:00:00    NaN
2001-12-31 21:00:00    NaN
2001-12-31 22:00:00    NaN
2001-12-31 23:00:00    NaN

[8760 rows x 1 columns]
\end{Verbatim}
\end{tcolorbox}
        
    \begin{tcolorbox}[breakable, size=fbox, boxrule=1pt, pad at break*=1mm,colback=cellbackground, colframe=cellborder]
\prompt{In}{incolor}{ }{\boxspacing}
\begin{Verbatim}[commandchars=\\\{\}]
\PY{n}{data}\PY{p}{[}\PY{l+s+s1}{\PYZsq{}}\PY{l+s+s1}{20020101 07:00}\PY{l+s+s1}{\PYZsq{}}\PY{p}{:} \PY{l+s+s1}{\PYZsq{}}\PY{l+s+s1}{20020101 16:00}\PY{l+s+s1}{\PYZsq{}}\PY{p}{]} \PY{c+c1}{\PYZsh{} menampilkan data diantara jam 7.00 sampai 16.00 pada tanggal 1 januari 2002}
\end{Verbatim}
\end{tcolorbox}

            \begin{tcolorbox}[breakable, size=fbox, boxrule=.5pt, pad at break*=1mm, opacityfill=0]
\prompt{Out}{outcolor}{ }{\boxspacing}
\begin{Verbatim}[commandchars=\\\{\}]
                    AURENE
2002-01-01 07:00:00    0.3
2002-01-01 08:00:00      -
2002-01-01 09:00:00      -
2002-01-01 10:00:00      -
2002-01-01 11:00:00      -
2002-01-01 12:00:00      -
2002-01-01 13:00:00    0.5
2002-01-01 14:00:00      -
2002-01-01 15:00:00    1.1
2002-01-01 16:00:00      -
\end{Verbatim}
\end{tcolorbox}
        
    \hypertarget{changelog}{%
\section{Changelog}\label{changelog}}

\begin{verbatim}
- 20191129 - 1.0.0 - Initial
\end{verbatim}

\hypertarget{copyright-2019-taruma-sakti-megariansyah}{%
\paragraph{\texorpdfstring{Copyright © 2019
\href{https://taruma.github.io}{Taruma Sakti
Megariansyah}}{Copyright © 2019 Taruma Sakti Megariansyah}}\label{copyright-2019-taruma-sakti-megariansyah}}

Source code in this notebook is licensed under a
\href{https://choosealicense.com/licenses/mit/}{MIT License}. Data in
this notebook is licensed under a
\href{https://creativecommons.org/licenses/by/4.0/}{Creative Common
Attribution 4.0 International}.


    % Add a bibliography block to the postdoc
    
    
    
\end{document}
