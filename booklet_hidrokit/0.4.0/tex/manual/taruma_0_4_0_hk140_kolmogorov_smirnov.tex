\documentclass[11pt]{article}

    \usepackage[breakable]{tcolorbox}
    \usepackage{parskip} % Stop auto-indenting (to mimic markdown behaviour)
    
    \usepackage{iftex}
    \ifPDFTeX
    	\usepackage[T1]{fontenc}
    	\usepackage{mathpazo}
    \else
    	\usepackage{fontspec}
    \fi

    % Basic figure setup, for now with no caption control since it's done
    % automatically by Pandoc (which extracts ![](path) syntax from Markdown).
    \usepackage{graphicx}
    % Maintain compatibility with old templates. Remove in nbconvert 6.0
    \let\Oldincludegraphics\includegraphics
    % Ensure that by default, figures have no caption (until we provide a
    % proper Figure object with a Caption API and a way to capture that
    % in the conversion process - todo).
    \usepackage{caption}
    \DeclareCaptionFormat{nocaption}{}
    \captionsetup{format=nocaption,aboveskip=0pt,belowskip=0pt}

    \usepackage{float}
    \floatplacement{figure}{H} % forces figures to be placed at the correct location
    \usepackage{xcolor} % Allow colors to be defined
    \usepackage{enumerate} % Needed for markdown enumerations to work
    \usepackage{geometry} % Used to adjust the document margins
    \usepackage{amsmath} % Equations
    \usepackage{amssymb} % Equations
    \usepackage{textcomp} % defines textquotesingle
    % Hack from http://tex.stackexchange.com/a/47451/13684:
    \AtBeginDocument{%
        \def\PYZsq{\textquotesingle}% Upright quotes in Pygmentized code
    }
    \usepackage{upquote} % Upright quotes for verbatim code
    \usepackage{eurosym} % defines \euro
    \usepackage[mathletters]{ucs} % Extended unicode (utf-8) support
    \usepackage{fancyvrb} % verbatim replacement that allows latex
    \usepackage{grffile} % extends the file name processing of package graphics 
                         % to support a larger range
    \makeatletter % fix for old versions of grffile with XeLaTeX
    \@ifpackagelater{grffile}{2019/11/01}
    {
      % Do nothing on new versions
    }
    {
      \def\Gread@@xetex#1{%
        \IfFileExists{"\Gin@base".bb}%
        {\Gread@eps{\Gin@base.bb}}%
        {\Gread@@xetex@aux#1}%
      }
    }
    \makeatother
    \usepackage[Export]{adjustbox} % Used to constrain images to a maximum size
    \adjustboxset{max size={0.9\linewidth}{0.9\paperheight}}

    % The hyperref package gives us a pdf with properly built
    % internal navigation ('pdf bookmarks' for the table of contents,
    % internal cross-reference links, web links for URLs, etc.)
    \usepackage{hyperref}
    % The default LaTeX title has an obnoxious amount of whitespace. By default,
    % titling removes some of it. It also provides customization options.
    \usepackage{titling}
    \usepackage{longtable} % longtable support required by pandoc >1.10
    \usepackage{booktabs}  % table support for pandoc > 1.12.2
    \usepackage[inline]{enumitem} % IRkernel/repr support (it uses the enumerate* environment)
    \usepackage[normalem]{ulem} % ulem is needed to support strikethroughs (\sout)
                                % normalem makes italics be italics, not underlines
    \usepackage{mathrsfs}
    

    
    % Colors for the hyperref package
    \definecolor{urlcolor}{rgb}{0,.145,.698}
    \definecolor{linkcolor}{rgb}{.71,0.21,0.01}
    \definecolor{citecolor}{rgb}{.12,.54,.11}

    % ANSI colors
    \definecolor{ansi-black}{HTML}{3E424D}
    \definecolor{ansi-black-intense}{HTML}{282C36}
    \definecolor{ansi-red}{HTML}{E75C58}
    \definecolor{ansi-red-intense}{HTML}{B22B31}
    \definecolor{ansi-green}{HTML}{00A250}
    \definecolor{ansi-green-intense}{HTML}{007427}
    \definecolor{ansi-yellow}{HTML}{DDB62B}
    \definecolor{ansi-yellow-intense}{HTML}{B27D12}
    \definecolor{ansi-blue}{HTML}{208FFB}
    \definecolor{ansi-blue-intense}{HTML}{0065CA}
    \definecolor{ansi-magenta}{HTML}{D160C4}
    \definecolor{ansi-magenta-intense}{HTML}{A03196}
    \definecolor{ansi-cyan}{HTML}{60C6C8}
    \definecolor{ansi-cyan-intense}{HTML}{258F8F}
    \definecolor{ansi-white}{HTML}{C5C1B4}
    \definecolor{ansi-white-intense}{HTML}{A1A6B2}
    \definecolor{ansi-default-inverse-fg}{HTML}{FFFFFF}
    \definecolor{ansi-default-inverse-bg}{HTML}{000000}

    % common color for the border for error outputs.
    \definecolor{outerrorbackground}{HTML}{FFDFDF}

    % commands and environments needed by pandoc snippets
    % extracted from the output of `pandoc -s`
    \providecommand{\tightlist}{%
      \setlength{\itemsep}{0pt}\setlength{\parskip}{0pt}}
    \DefineVerbatimEnvironment{Highlighting}{Verbatim}{commandchars=\\\{\}}
    % Add ',fontsize=\small' for more characters per line
    \newenvironment{Shaded}{}{}
    \newcommand{\KeywordTok}[1]{\textcolor[rgb]{0.00,0.44,0.13}{\textbf{{#1}}}}
    \newcommand{\DataTypeTok}[1]{\textcolor[rgb]{0.56,0.13,0.00}{{#1}}}
    \newcommand{\DecValTok}[1]{\textcolor[rgb]{0.25,0.63,0.44}{{#1}}}
    \newcommand{\BaseNTok}[1]{\textcolor[rgb]{0.25,0.63,0.44}{{#1}}}
    \newcommand{\FloatTok}[1]{\textcolor[rgb]{0.25,0.63,0.44}{{#1}}}
    \newcommand{\CharTok}[1]{\textcolor[rgb]{0.25,0.44,0.63}{{#1}}}
    \newcommand{\StringTok}[1]{\textcolor[rgb]{0.25,0.44,0.63}{{#1}}}
    \newcommand{\CommentTok}[1]{\textcolor[rgb]{0.38,0.63,0.69}{\textit{{#1}}}}
    \newcommand{\OtherTok}[1]{\textcolor[rgb]{0.00,0.44,0.13}{{#1}}}
    \newcommand{\AlertTok}[1]{\textcolor[rgb]{1.00,0.00,0.00}{\textbf{{#1}}}}
    \newcommand{\FunctionTok}[1]{\textcolor[rgb]{0.02,0.16,0.49}{{#1}}}
    \newcommand{\RegionMarkerTok}[1]{{#1}}
    \newcommand{\ErrorTok}[1]{\textcolor[rgb]{1.00,0.00,0.00}{\textbf{{#1}}}}
    \newcommand{\NormalTok}[1]{{#1}}
    
    % Additional commands for more recent versions of Pandoc
    \newcommand{\ConstantTok}[1]{\textcolor[rgb]{0.53,0.00,0.00}{{#1}}}
    \newcommand{\SpecialCharTok}[1]{\textcolor[rgb]{0.25,0.44,0.63}{{#1}}}
    \newcommand{\VerbatimStringTok}[1]{\textcolor[rgb]{0.25,0.44,0.63}{{#1}}}
    \newcommand{\SpecialStringTok}[1]{\textcolor[rgb]{0.73,0.40,0.53}{{#1}}}
    \newcommand{\ImportTok}[1]{{#1}}
    \newcommand{\DocumentationTok}[1]{\textcolor[rgb]{0.73,0.13,0.13}{\textit{{#1}}}}
    \newcommand{\AnnotationTok}[1]{\textcolor[rgb]{0.38,0.63,0.69}{\textbf{\textit{{#1}}}}}
    \newcommand{\CommentVarTok}[1]{\textcolor[rgb]{0.38,0.63,0.69}{\textbf{\textit{{#1}}}}}
    \newcommand{\VariableTok}[1]{\textcolor[rgb]{0.10,0.09,0.49}{{#1}}}
    \newcommand{\ControlFlowTok}[1]{\textcolor[rgb]{0.00,0.44,0.13}{\textbf{{#1}}}}
    \newcommand{\OperatorTok}[1]{\textcolor[rgb]{0.40,0.40,0.40}{{#1}}}
    \newcommand{\BuiltInTok}[1]{{#1}}
    \newcommand{\ExtensionTok}[1]{{#1}}
    \newcommand{\PreprocessorTok}[1]{\textcolor[rgb]{0.74,0.48,0.00}{{#1}}}
    \newcommand{\AttributeTok}[1]{\textcolor[rgb]{0.49,0.56,0.16}{{#1}}}
    \newcommand{\InformationTok}[1]{\textcolor[rgb]{0.38,0.63,0.69}{\textbf{\textit{{#1}}}}}
    \newcommand{\WarningTok}[1]{\textcolor[rgb]{0.38,0.63,0.69}{\textbf{\textit{{#1}}}}}
    
    
    % Define a nice break command that doesn't care if a line doesn't already
    % exist.
    \def\br{\hspace*{\fill} \\* }
    % Math Jax compatibility definitions
    \def\gt{>}
    \def\lt{<}
    \let\Oldtex\TeX
    \let\Oldlatex\LaTeX
    \renewcommand{\TeX}{\textrm{\Oldtex}}
    \renewcommand{\LaTeX}{\textrm{\Oldlatex}}
    % Document parameters
    % Document title
    \title{taruma\_0\_4\_0\_hk140\_kolmogorov\_smirnov}
    
    
    
    
    
% Pygments definitions
\makeatletter
\def\PY@reset{\let\PY@it=\relax \let\PY@bf=\relax%
    \let\PY@ul=\relax \let\PY@tc=\relax%
    \let\PY@bc=\relax \let\PY@ff=\relax}
\def\PY@tok#1{\csname PY@tok@#1\endcsname}
\def\PY@toks#1+{\ifx\relax#1\empty\else%
    \PY@tok{#1}\expandafter\PY@toks\fi}
\def\PY@do#1{\PY@bc{\PY@tc{\PY@ul{%
    \PY@it{\PY@bf{\PY@ff{#1}}}}}}}
\def\PY#1#2{\PY@reset\PY@toks#1+\relax+\PY@do{#2}}

\@namedef{PY@tok@w}{\def\PY@tc##1{\textcolor[rgb]{0.73,0.73,0.73}{##1}}}
\@namedef{PY@tok@c}{\let\PY@it=\textit\def\PY@tc##1{\textcolor[rgb]{0.24,0.48,0.48}{##1}}}
\@namedef{PY@tok@cp}{\def\PY@tc##1{\textcolor[rgb]{0.61,0.40,0.00}{##1}}}
\@namedef{PY@tok@k}{\let\PY@bf=\textbf\def\PY@tc##1{\textcolor[rgb]{0.00,0.50,0.00}{##1}}}
\@namedef{PY@tok@kp}{\def\PY@tc##1{\textcolor[rgb]{0.00,0.50,0.00}{##1}}}
\@namedef{PY@tok@kt}{\def\PY@tc##1{\textcolor[rgb]{0.69,0.00,0.25}{##1}}}
\@namedef{PY@tok@o}{\def\PY@tc##1{\textcolor[rgb]{0.40,0.40,0.40}{##1}}}
\@namedef{PY@tok@ow}{\let\PY@bf=\textbf\def\PY@tc##1{\textcolor[rgb]{0.67,0.13,1.00}{##1}}}
\@namedef{PY@tok@nb}{\def\PY@tc##1{\textcolor[rgb]{0.00,0.50,0.00}{##1}}}
\@namedef{PY@tok@nf}{\def\PY@tc##1{\textcolor[rgb]{0.00,0.00,1.00}{##1}}}
\@namedef{PY@tok@nc}{\let\PY@bf=\textbf\def\PY@tc##1{\textcolor[rgb]{0.00,0.00,1.00}{##1}}}
\@namedef{PY@tok@nn}{\let\PY@bf=\textbf\def\PY@tc##1{\textcolor[rgb]{0.00,0.00,1.00}{##1}}}
\@namedef{PY@tok@ne}{\let\PY@bf=\textbf\def\PY@tc##1{\textcolor[rgb]{0.80,0.25,0.22}{##1}}}
\@namedef{PY@tok@nv}{\def\PY@tc##1{\textcolor[rgb]{0.10,0.09,0.49}{##1}}}
\@namedef{PY@tok@no}{\def\PY@tc##1{\textcolor[rgb]{0.53,0.00,0.00}{##1}}}
\@namedef{PY@tok@nl}{\def\PY@tc##1{\textcolor[rgb]{0.46,0.46,0.00}{##1}}}
\@namedef{PY@tok@ni}{\let\PY@bf=\textbf\def\PY@tc##1{\textcolor[rgb]{0.44,0.44,0.44}{##1}}}
\@namedef{PY@tok@na}{\def\PY@tc##1{\textcolor[rgb]{0.41,0.47,0.13}{##1}}}
\@namedef{PY@tok@nt}{\let\PY@bf=\textbf\def\PY@tc##1{\textcolor[rgb]{0.00,0.50,0.00}{##1}}}
\@namedef{PY@tok@nd}{\def\PY@tc##1{\textcolor[rgb]{0.67,0.13,1.00}{##1}}}
\@namedef{PY@tok@s}{\def\PY@tc##1{\textcolor[rgb]{0.73,0.13,0.13}{##1}}}
\@namedef{PY@tok@sd}{\let\PY@it=\textit\def\PY@tc##1{\textcolor[rgb]{0.73,0.13,0.13}{##1}}}
\@namedef{PY@tok@si}{\let\PY@bf=\textbf\def\PY@tc##1{\textcolor[rgb]{0.64,0.35,0.47}{##1}}}
\@namedef{PY@tok@se}{\let\PY@bf=\textbf\def\PY@tc##1{\textcolor[rgb]{0.67,0.36,0.12}{##1}}}
\@namedef{PY@tok@sr}{\def\PY@tc##1{\textcolor[rgb]{0.64,0.35,0.47}{##1}}}
\@namedef{PY@tok@ss}{\def\PY@tc##1{\textcolor[rgb]{0.10,0.09,0.49}{##1}}}
\@namedef{PY@tok@sx}{\def\PY@tc##1{\textcolor[rgb]{0.00,0.50,0.00}{##1}}}
\@namedef{PY@tok@m}{\def\PY@tc##1{\textcolor[rgb]{0.40,0.40,0.40}{##1}}}
\@namedef{PY@tok@gh}{\let\PY@bf=\textbf\def\PY@tc##1{\textcolor[rgb]{0.00,0.00,0.50}{##1}}}
\@namedef{PY@tok@gu}{\let\PY@bf=\textbf\def\PY@tc##1{\textcolor[rgb]{0.50,0.00,0.50}{##1}}}
\@namedef{PY@tok@gd}{\def\PY@tc##1{\textcolor[rgb]{0.63,0.00,0.00}{##1}}}
\@namedef{PY@tok@gi}{\def\PY@tc##1{\textcolor[rgb]{0.00,0.52,0.00}{##1}}}
\@namedef{PY@tok@gr}{\def\PY@tc##1{\textcolor[rgb]{0.89,0.00,0.00}{##1}}}
\@namedef{PY@tok@ge}{\let\PY@it=\textit}
\@namedef{PY@tok@gs}{\let\PY@bf=\textbf}
\@namedef{PY@tok@gp}{\let\PY@bf=\textbf\def\PY@tc##1{\textcolor[rgb]{0.00,0.00,0.50}{##1}}}
\@namedef{PY@tok@go}{\def\PY@tc##1{\textcolor[rgb]{0.44,0.44,0.44}{##1}}}
\@namedef{PY@tok@gt}{\def\PY@tc##1{\textcolor[rgb]{0.00,0.27,0.87}{##1}}}
\@namedef{PY@tok@err}{\def\PY@bc##1{{\setlength{\fboxsep}{\string -\fboxrule}\fcolorbox[rgb]{1.00,0.00,0.00}{1,1,1}{\strut ##1}}}}
\@namedef{PY@tok@kc}{\let\PY@bf=\textbf\def\PY@tc##1{\textcolor[rgb]{0.00,0.50,0.00}{##1}}}
\@namedef{PY@tok@kd}{\let\PY@bf=\textbf\def\PY@tc##1{\textcolor[rgb]{0.00,0.50,0.00}{##1}}}
\@namedef{PY@tok@kn}{\let\PY@bf=\textbf\def\PY@tc##1{\textcolor[rgb]{0.00,0.50,0.00}{##1}}}
\@namedef{PY@tok@kr}{\let\PY@bf=\textbf\def\PY@tc##1{\textcolor[rgb]{0.00,0.50,0.00}{##1}}}
\@namedef{PY@tok@bp}{\def\PY@tc##1{\textcolor[rgb]{0.00,0.50,0.00}{##1}}}
\@namedef{PY@tok@fm}{\def\PY@tc##1{\textcolor[rgb]{0.00,0.00,1.00}{##1}}}
\@namedef{PY@tok@vc}{\def\PY@tc##1{\textcolor[rgb]{0.10,0.09,0.49}{##1}}}
\@namedef{PY@tok@vg}{\def\PY@tc##1{\textcolor[rgb]{0.10,0.09,0.49}{##1}}}
\@namedef{PY@tok@vi}{\def\PY@tc##1{\textcolor[rgb]{0.10,0.09,0.49}{##1}}}
\@namedef{PY@tok@vm}{\def\PY@tc##1{\textcolor[rgb]{0.10,0.09,0.49}{##1}}}
\@namedef{PY@tok@sa}{\def\PY@tc##1{\textcolor[rgb]{0.73,0.13,0.13}{##1}}}
\@namedef{PY@tok@sb}{\def\PY@tc##1{\textcolor[rgb]{0.73,0.13,0.13}{##1}}}
\@namedef{PY@tok@sc}{\def\PY@tc##1{\textcolor[rgb]{0.73,0.13,0.13}{##1}}}
\@namedef{PY@tok@dl}{\def\PY@tc##1{\textcolor[rgb]{0.73,0.13,0.13}{##1}}}
\@namedef{PY@tok@s2}{\def\PY@tc##1{\textcolor[rgb]{0.73,0.13,0.13}{##1}}}
\@namedef{PY@tok@sh}{\def\PY@tc##1{\textcolor[rgb]{0.73,0.13,0.13}{##1}}}
\@namedef{PY@tok@s1}{\def\PY@tc##1{\textcolor[rgb]{0.73,0.13,0.13}{##1}}}
\@namedef{PY@tok@mb}{\def\PY@tc##1{\textcolor[rgb]{0.40,0.40,0.40}{##1}}}
\@namedef{PY@tok@mf}{\def\PY@tc##1{\textcolor[rgb]{0.40,0.40,0.40}{##1}}}
\@namedef{PY@tok@mh}{\def\PY@tc##1{\textcolor[rgb]{0.40,0.40,0.40}{##1}}}
\@namedef{PY@tok@mi}{\def\PY@tc##1{\textcolor[rgb]{0.40,0.40,0.40}{##1}}}
\@namedef{PY@tok@il}{\def\PY@tc##1{\textcolor[rgb]{0.40,0.40,0.40}{##1}}}
\@namedef{PY@tok@mo}{\def\PY@tc##1{\textcolor[rgb]{0.40,0.40,0.40}{##1}}}
\@namedef{PY@tok@ch}{\let\PY@it=\textit\def\PY@tc##1{\textcolor[rgb]{0.24,0.48,0.48}{##1}}}
\@namedef{PY@tok@cm}{\let\PY@it=\textit\def\PY@tc##1{\textcolor[rgb]{0.24,0.48,0.48}{##1}}}
\@namedef{PY@tok@cpf}{\let\PY@it=\textit\def\PY@tc##1{\textcolor[rgb]{0.24,0.48,0.48}{##1}}}
\@namedef{PY@tok@c1}{\let\PY@it=\textit\def\PY@tc##1{\textcolor[rgb]{0.24,0.48,0.48}{##1}}}
\@namedef{PY@tok@cs}{\let\PY@it=\textit\def\PY@tc##1{\textcolor[rgb]{0.24,0.48,0.48}{##1}}}

\def\PYZbs{\char`\\}
\def\PYZus{\char`\_}
\def\PYZob{\char`\{}
\def\PYZcb{\char`\}}
\def\PYZca{\char`\^}
\def\PYZam{\char`\&}
\def\PYZlt{\char`\<}
\def\PYZgt{\char`\>}
\def\PYZsh{\char`\#}
\def\PYZpc{\char`\%}
\def\PYZdl{\char`\$}
\def\PYZhy{\char`\-}
\def\PYZsq{\char`\'}
\def\PYZdq{\char`\"}
\def\PYZti{\char`\~}
% for compatibility with earlier versions
\def\PYZat{@}
\def\PYZlb{[}
\def\PYZrb{]}
\makeatother


    % For linebreaks inside Verbatim environment from package fancyvrb. 
    \makeatletter
        \newbox\Wrappedcontinuationbox 
        \newbox\Wrappedvisiblespacebox 
        \newcommand*\Wrappedvisiblespace {\textcolor{red}{\textvisiblespace}} 
        \newcommand*\Wrappedcontinuationsymbol {\textcolor{red}{\llap{\tiny$\m@th\hookrightarrow$}}} 
        \newcommand*\Wrappedcontinuationindent {3ex } 
        \newcommand*\Wrappedafterbreak {\kern\Wrappedcontinuationindent\copy\Wrappedcontinuationbox} 
        % Take advantage of the already applied Pygments mark-up to insert 
        % potential linebreaks for TeX processing. 
        %        {, <, #, %, $, ' and ": go to next line. 
        %        _, }, ^, &, >, - and ~: stay at end of broken line. 
        % Use of \textquotesingle for straight quote. 
        \newcommand*\Wrappedbreaksatspecials {% 
            \def\PYGZus{\discretionary{\char`\_}{\Wrappedafterbreak}{\char`\_}}% 
            \def\PYGZob{\discretionary{}{\Wrappedafterbreak\char`\{}{\char`\{}}% 
            \def\PYGZcb{\discretionary{\char`\}}{\Wrappedafterbreak}{\char`\}}}% 
            \def\PYGZca{\discretionary{\char`\^}{\Wrappedafterbreak}{\char`\^}}% 
            \def\PYGZam{\discretionary{\char`\&}{\Wrappedafterbreak}{\char`\&}}% 
            \def\PYGZlt{\discretionary{}{\Wrappedafterbreak\char`\<}{\char`\<}}% 
            \def\PYGZgt{\discretionary{\char`\>}{\Wrappedafterbreak}{\char`\>}}% 
            \def\PYGZsh{\discretionary{}{\Wrappedafterbreak\char`\#}{\char`\#}}% 
            \def\PYGZpc{\discretionary{}{\Wrappedafterbreak\char`\%}{\char`\%}}% 
            \def\PYGZdl{\discretionary{}{\Wrappedafterbreak\char`\$}{\char`\$}}% 
            \def\PYGZhy{\discretionary{\char`\-}{\Wrappedafterbreak}{\char`\-}}% 
            \def\PYGZsq{\discretionary{}{\Wrappedafterbreak\textquotesingle}{\textquotesingle}}% 
            \def\PYGZdq{\discretionary{}{\Wrappedafterbreak\char`\"}{\char`\"}}% 
            \def\PYGZti{\discretionary{\char`\~}{\Wrappedafterbreak}{\char`\~}}% 
        } 
        % Some characters . , ; ? ! / are not pygmentized. 
        % This macro makes them "active" and they will insert potential linebreaks 
        \newcommand*\Wrappedbreaksatpunct {% 
            \lccode`\~`\.\lowercase{\def~}{\discretionary{\hbox{\char`\.}}{\Wrappedafterbreak}{\hbox{\char`\.}}}% 
            \lccode`\~`\,\lowercase{\def~}{\discretionary{\hbox{\char`\,}}{\Wrappedafterbreak}{\hbox{\char`\,}}}% 
            \lccode`\~`\;\lowercase{\def~}{\discretionary{\hbox{\char`\;}}{\Wrappedafterbreak}{\hbox{\char`\;}}}% 
            \lccode`\~`\:\lowercase{\def~}{\discretionary{\hbox{\char`\:}}{\Wrappedafterbreak}{\hbox{\char`\:}}}% 
            \lccode`\~`\?\lowercase{\def~}{\discretionary{\hbox{\char`\?}}{\Wrappedafterbreak}{\hbox{\char`\?}}}% 
            \lccode`\~`\!\lowercase{\def~}{\discretionary{\hbox{\char`\!}}{\Wrappedafterbreak}{\hbox{\char`\!}}}% 
            \lccode`\~`\/\lowercase{\def~}{\discretionary{\hbox{\char`\/}}{\Wrappedafterbreak}{\hbox{\char`\/}}}% 
            \catcode`\.\active
            \catcode`\,\active 
            \catcode`\;\active
            \catcode`\:\active
            \catcode`\?\active
            \catcode`\!\active
            \catcode`\/\active 
            \lccode`\~`\~ 	
        }
    \makeatother

    \let\OriginalVerbatim=\Verbatim
    \makeatletter
    \renewcommand{\Verbatim}[1][1]{%
        %\parskip\z@skip
        \sbox\Wrappedcontinuationbox {\Wrappedcontinuationsymbol}%
        \sbox\Wrappedvisiblespacebox {\FV@SetupFont\Wrappedvisiblespace}%
        \def\FancyVerbFormatLine ##1{\hsize\linewidth
            \vtop{\raggedright\hyphenpenalty\z@\exhyphenpenalty\z@
                \doublehyphendemerits\z@\finalhyphendemerits\z@
                \strut ##1\strut}%
        }%
        % If the linebreak is at a space, the latter will be displayed as visible
        % space at end of first line, and a continuation symbol starts next line.
        % Stretch/shrink are however usually zero for typewriter font.
        \def\FV@Space {%
            \nobreak\hskip\z@ plus\fontdimen3\font minus\fontdimen4\font
            \discretionary{\copy\Wrappedvisiblespacebox}{\Wrappedafterbreak}
            {\kern\fontdimen2\font}%
        }%
        
        % Allow breaks at special characters using \PYG... macros.
        \Wrappedbreaksatspecials
        % Breaks at punctuation characters . , ; ? ! and / need catcode=\active 	
        \OriginalVerbatim[#1,codes*=\Wrappedbreaksatpunct]%
    }
    \makeatother

    % Exact colors from NB
    \definecolor{incolor}{HTML}{303F9F}
    \definecolor{outcolor}{HTML}{D84315}
    \definecolor{cellborder}{HTML}{CFCFCF}
    \definecolor{cellbackground}{HTML}{F7F7F7}
    
    % prompt
    \makeatletter
    \newcommand{\boxspacing}{\kern\kvtcb@left@rule\kern\kvtcb@boxsep}
    \makeatother
    \newcommand{\prompt}[4]{
        {\ttfamily\llap{{\color{#2}[#3]:\hspace{3pt}#4}}\vspace{-\baselineskip}}
    }
    

    
    % Prevent overflowing lines due to hard-to-break entities
    \sloppy 
    % Setup hyperref package
    \hypersetup{
      breaklinks=true,  % so long urls are correctly broken across lines
      colorlinks=true,
      urlcolor=urlcolor,
      linkcolor=linkcolor,
      citecolor=citecolor,
      }
    % Slightly bigger margins than the latex defaults
    
    \geometry{verbose,tmargin=1in,bmargin=1in,lmargin=1in,rmargin=1in}
    
    

\begin{document}
    
    \maketitle
    
    

    
    Berdasarkan isu
\href{https://github.com/hidrokit/hidrokit/issues/140}{\#140}:
\textbf{Uji Kolmogorov-Smirnov}

Referensi Isu: - Soetopo, W., Montarcih, L., Press, U. B., \& Media, U.
(2017). Rekayasa Statistika untuk Teknik Pengairan. Universitas
Brawijaya Press. https://books.google.co.id/books?id=TzVTDwAAQBAJ -
Soewarno. (1995). hidrologi: Aplikasi Metode Statistik untuk Analisa
Data. NOVA. - Limantara, L. (2018). Rekayasa Hidrologi.

Deskripsi Isu: - Melakukan Uji Kecocokan Distribusi menggunakan Uji
Kolmogorov-Smirnov.

Strategi: - Membuat fungsi \emph{inverse} atau CDF untuk masing-masing
distribusi yang digunakan. (sudah diselesaikan pada isu
\href{https://github.com/hidrokit/hidrokit/issues/179}{\#179}) - Tidak
dibandingkan dengan fungsi \texttt{scipy.stats.kstest}.

    \hypertarget{persiapan-dan-dataset}{%
\section{PERSIAPAN DAN DATASET}\label{persiapan-dan-dataset}}

    \begin{tcolorbox}[breakable, size=fbox, boxrule=1pt, pad at break*=1mm,colback=cellbackground, colframe=cellborder]
\prompt{In}{incolor}{ }{\boxspacing}
\begin{Verbatim}[commandchars=\\\{\}]
\PY{k}{try}\PY{p}{:}
    \PY{k+kn}{import} \PY{n+nn}{hidrokit}
\PY{k}{except} \PY{n+ne}{ModuleNotFoundError}\PY{p}{:}
    \PY{c+c1}{\PYZsh{} saat dibuat menggunakan cabang @dev/dev0.3.7}
    \PY{err}{!}\PY{n}{pip} \PY{n}{install} \PY{n}{git}\PY{o}{+}\PY{n}{https}\PY{p}{:}\PY{o}{/}\PY{o}{/}\PY{n}{github}\PY{o}{.}\PY{n}{com}\PY{o}{/}\PY{n}{taruma}\PY{o}{/}\PY{n}{hidrokit}\PY{o}{.}\PY{n}{git}\PY{n+nd}{@dev}\PY{o}{/}\PY{n}{dev0}\PY{l+m+mf}{.3}\PY{l+m+mf}{.7} \PY{o}{\PYZhy{}}\PY{n}{q}
\end{Verbatim}
\end{tcolorbox}

    \begin{tcolorbox}[breakable, size=fbox, boxrule=1pt, pad at break*=1mm,colback=cellbackground, colframe=cellborder]
\prompt{In}{incolor}{ }{\boxspacing}
\begin{Verbatim}[commandchars=\\\{\}]
\PY{k+kn}{import} \PY{n+nn}{numpy} \PY{k}{as} \PY{n+nn}{np}
\PY{k+kn}{import} \PY{n+nn}{pandas} \PY{k}{as} \PY{n+nn}{pd}
\PY{k+kn}{from} \PY{n+nn}{scipy} \PY{k+kn}{import} \PY{n}{stats}
\PY{k+kn}{from} \PY{n+nn}{hidrokit}\PY{n+nn}{.}\PY{n+nn}{contrib}\PY{n+nn}{.}\PY{n+nn}{taruma} \PY{k+kn}{import} \PY{n}{hk172}\PY{p}{,} \PY{n}{hk124}\PY{p}{,} \PY{n}{hk127}\PY{p}{,} \PY{n}{hk126} 

\PY{n}{frek\PYZus{}normal}\PY{p}{,} \PY{n}{frek\PYZus{}lognormal}\PY{p}{,} \PY{n}{frek\PYZus{}gumbel}\PY{p}{,} \PY{n}{frek\PYZus{}logpearson3} \PY{o}{=} \PY{n}{hk172}\PY{p}{,} \PY{n}{hk124}\PY{p}{,} \PY{n}{hk127}\PY{p}{,} \PY{n}{hk126}
\end{Verbatim}
\end{tcolorbox}

    \begin{tcolorbox}[breakable, size=fbox, boxrule=1pt, pad at break*=1mm,colback=cellbackground, colframe=cellborder]
\prompt{In}{incolor}{ }{\boxspacing}
\begin{Verbatim}[commandchars=\\\{\}]
\PY{c+c1}{\PYZsh{} contoh data diambil dari buku}
\PY{c+c1}{\PYZsh{} limantara hal. 114}

\PY{n}{\PYZus{}HUJAN} \PY{o}{=} \PY{n}{np}\PY{o}{.}\PY{n}{array}\PY{p}{(}\PY{p}{[}\PY{l+m+mi}{85}\PY{p}{,} \PY{l+m+mi}{92}\PY{p}{,} \PY{l+m+mi}{115}\PY{p}{,} \PY{l+m+mi}{116}\PY{p}{,} \PY{l+m+mi}{122}\PY{p}{,} \PY{l+m+mi}{52}\PY{p}{,} \PY{l+m+mi}{69}\PY{p}{,} \PY{l+m+mi}{95}\PY{p}{,} \PY{l+m+mi}{96}\PY{p}{,} \PY{l+m+mi}{105}\PY{p}{]}\PY{p}{)}
\PY{n}{\PYZus{}TAHUN} \PY{o}{=} \PY{n}{np}\PY{o}{.}\PY{n}{arange}\PY{p}{(}\PY{l+m+mi}{1998}\PY{p}{,} \PY{l+m+mi}{2008}\PY{p}{)} \PY{c+c1}{\PYZsh{} 1998\PYZhy{}2007}

\PY{n}{data} \PY{o}{=} \PY{n}{pd}\PY{o}{.}\PY{n}{DataFrame}\PY{p}{(}
    \PY{n}{data}\PY{o}{=}\PY{n}{np}\PY{o}{.}\PY{n}{stack}\PY{p}{(}\PY{p}{[}\PY{n}{\PYZus{}TAHUN}\PY{p}{,} \PY{n}{\PYZus{}HUJAN}\PY{p}{]}\PY{p}{,} \PY{n}{axis}\PY{o}{=}\PY{l+m+mi}{1}\PY{p}{)}\PY{p}{,}
    \PY{n}{columns}\PY{o}{=}\PY{p}{[}\PY{l+s+s1}{\PYZsq{}}\PY{l+s+s1}{tahun}\PY{l+s+s1}{\PYZsq{}}\PY{p}{,} \PY{l+s+s1}{\PYZsq{}}\PY{l+s+s1}{hujan}\PY{l+s+s1}{\PYZsq{}}\PY{p}{]}
\PY{p}{)}
\PY{n}{data}\PY{o}{.}\PY{n}{tahun} \PY{o}{=} \PY{n}{pd}\PY{o}{.}\PY{n}{to\PYZus{}datetime}\PY{p}{(}\PY{n}{data}\PY{o}{.}\PY{n}{tahun}\PY{p}{,} \PY{n+nb}{format}\PY{o}{=}\PY{l+s+s1}{\PYZsq{}}\PY{l+s+s1}{\PYZpc{}}\PY{l+s+s1}{Y}\PY{l+s+s1}{\PYZsq{}}\PY{p}{)}
\PY{n}{data}\PY{o}{.}\PY{n}{set\PYZus{}index}\PY{p}{(}\PY{l+s+s1}{\PYZsq{}}\PY{l+s+s1}{tahun}\PY{l+s+s1}{\PYZsq{}}\PY{p}{,} \PY{n}{inplace}\PY{o}{=}\PY{k+kc}{True}\PY{p}{)}
\PY{n}{data}
\end{Verbatim}
\end{tcolorbox}

            \begin{tcolorbox}[breakable, size=fbox, boxrule=.5pt, pad at break*=1mm, opacityfill=0]
\prompt{Out}{outcolor}{ }{\boxspacing}
\begin{Verbatim}[commandchars=\\\{\}]
            hujan
tahun
1998-01-01     85
1999-01-01     92
2000-01-01    115
2001-01-01    116
2002-01-01    122
2003-01-01     52
2004-01-01     69
2005-01-01     95
2006-01-01     96
2007-01-01    105
\end{Verbatim}
\end{tcolorbox}
        
    \hypertarget{tabel}{%
\section{TABEL}\label{tabel}}

Terdapat 2 tabel untuk modul \texttt{hk140} yaitu: -
\texttt{t\_dcr\_st}: Tabel nilai kritis (Dcr) Untuk Uji
Kolmogorov-Smirnov dari buku \emph{Rekayasa Statistika untuk Teknik
Pengairan} oleh Soetopo. - \texttt{t\_dcr\_sw}: Tabel nilai kritis Do
Untuk Uji Smirnov-Kolmogorov dari buku \emph{hidrologi: Aplikasi Metode
Statistik untuk Analisa Data} oleh Soewarno.

Dalam modul \texttt{hk126} nilai \(\Delta_{kritis}\) akan dibangkitkan
menggunakan fungsi \texttt{scipy.stats.ksone.ppf} secara
\texttt{default}. Mohon diperhatikan jika ingin menggunakan nilai
\(\Delta_{kritis}\) yang berasal dari sumber lain.

    \begin{tcolorbox}[breakable, size=fbox, boxrule=1pt, pad at break*=1mm,colback=cellbackground, colframe=cellborder]
\prompt{In}{incolor}{ }{\boxspacing}
\begin{Verbatim}[commandchars=\\\{\}]
\PY{c+c1}{\PYZsh{} tabel dari soetopo hal. 139}
\PY{c+c1}{\PYZsh{} Tabel Nilai Kritis (Dcr) Untuk Uji Kolmogorov\PYZhy{}Smirnov}

\PY{c+c1}{\PYZsh{} KODE: ST}

\PY{n}{\PYZus{}DATA\PYZus{}ST} \PY{o}{=} \PY{p}{[}
    \PY{p}{[}\PY{l+m+mf}{0.900}\PY{p}{,} \PY{l+m+mf}{0.925}\PY{p}{,} \PY{l+m+mf}{0.950}\PY{p}{,} \PY{l+m+mf}{0.975}\PY{p}{,} \PY{l+m+mf}{0.995}\PY{p}{]}\PY{p}{,}
    \PY{p}{[}\PY{l+m+mf}{0.684}\PY{p}{,} \PY{l+m+mf}{0.726}\PY{p}{,} \PY{l+m+mf}{0.776}\PY{p}{,} \PY{l+m+mf}{0.842}\PY{p}{,} \PY{l+m+mf}{0.929}\PY{p}{]}\PY{p}{,}
    \PY{p}{[}\PY{l+m+mf}{0.565}\PY{p}{,} \PY{l+m+mf}{0.597}\PY{p}{,} \PY{l+m+mf}{0.642}\PY{p}{,} \PY{l+m+mf}{0.708}\PY{p}{,} \PY{l+m+mf}{0.829}\PY{p}{]}\PY{p}{,}
    \PY{p}{[}\PY{l+m+mf}{0.494}\PY{p}{,} \PY{l+m+mf}{0.525}\PY{p}{,} \PY{l+m+mf}{0.564}\PY{p}{,} \PY{l+m+mf}{0.624}\PY{p}{,} \PY{l+m+mf}{0.734}\PY{p}{]}\PY{p}{,}
    \PY{p}{[}\PY{l+m+mf}{0.446}\PY{p}{,} \PY{l+m+mf}{0.474}\PY{p}{,} \PY{l+m+mf}{0.510}\PY{p}{,} \PY{l+m+mf}{0.563}\PY{p}{,} \PY{l+m+mf}{0.669}\PY{p}{]}\PY{p}{,}
    \PY{p}{[}\PY{l+m+mf}{0.410}\PY{p}{,} \PY{l+m+mf}{0.436}\PY{p}{,} \PY{l+m+mf}{0.470}\PY{p}{,} \PY{l+m+mf}{0.521}\PY{p}{,} \PY{l+m+mf}{0.618}\PY{p}{]}\PY{p}{,}
    \PY{p}{[}\PY{l+m+mf}{0.381}\PY{p}{,} \PY{l+m+mf}{0.405}\PY{p}{,} \PY{l+m+mf}{0.438}\PY{p}{,} \PY{l+m+mf}{0.486}\PY{p}{,} \PY{l+m+mf}{0.577}\PY{p}{]}\PY{p}{,}
    \PY{p}{[}\PY{l+m+mf}{0.358}\PY{p}{,} \PY{l+m+mf}{0.381}\PY{p}{,} \PY{l+m+mf}{0.411}\PY{p}{,} \PY{l+m+mf}{0.457}\PY{p}{,} \PY{l+m+mf}{0.543}\PY{p}{]}\PY{p}{,}
    \PY{p}{[}\PY{l+m+mf}{0.339}\PY{p}{,} \PY{l+m+mf}{0.360}\PY{p}{,} \PY{l+m+mf}{0.388}\PY{p}{,} \PY{l+m+mf}{0.432}\PY{p}{,} \PY{l+m+mf}{0.514}\PY{p}{]}\PY{p}{,}
    \PY{p}{[}\PY{l+m+mf}{0.322}\PY{p}{,} \PY{l+m+mf}{0.342}\PY{p}{,} \PY{l+m+mf}{0.368}\PY{p}{,} \PY{l+m+mf}{0.409}\PY{p}{,} \PY{l+m+mf}{0.486}\PY{p}{]}\PY{p}{,}
    \PY{p}{[}\PY{l+m+mf}{0.307}\PY{p}{,} \PY{l+m+mf}{0.326}\PY{p}{,} \PY{l+m+mf}{0.352}\PY{p}{,} \PY{l+m+mf}{0.391}\PY{p}{,} \PY{l+m+mf}{0.468}\PY{p}{]}\PY{p}{,}
    \PY{p}{[}\PY{l+m+mf}{0.295}\PY{p}{,} \PY{l+m+mf}{0.313}\PY{p}{,} \PY{l+m+mf}{0.338}\PY{p}{,} \PY{l+m+mf}{0.375}\PY{p}{,} \PY{l+m+mf}{0.450}\PY{p}{]}\PY{p}{,}
    \PY{p}{[}\PY{l+m+mf}{0.284}\PY{p}{,} \PY{l+m+mf}{0.302}\PY{p}{,} \PY{l+m+mf}{0.325}\PY{p}{,} \PY{l+m+mf}{0.361}\PY{p}{,} \PY{l+m+mf}{0.433}\PY{p}{]}\PY{p}{,}
    \PY{p}{[}\PY{l+m+mf}{0.274}\PY{p}{,} \PY{l+m+mf}{0.292}\PY{p}{,} \PY{l+m+mf}{0.314}\PY{p}{,} \PY{l+m+mf}{0.349}\PY{p}{,} \PY{l+m+mf}{0.418}\PY{p}{]}\PY{p}{,}
    \PY{p}{[}\PY{l+m+mf}{0.266}\PY{p}{,} \PY{l+m+mf}{0.283}\PY{p}{,} \PY{l+m+mf}{0.304}\PY{p}{,} \PY{l+m+mf}{0.338}\PY{p}{,} \PY{l+m+mf}{0.404}\PY{p}{]}\PY{p}{,}
    \PY{p}{[}\PY{l+m+mf}{0.258}\PY{p}{,} \PY{l+m+mf}{0.274}\PY{p}{,} \PY{l+m+mf}{0.295}\PY{p}{,} \PY{l+m+mf}{0.328}\PY{p}{,} \PY{l+m+mf}{0.391}\PY{p}{]}\PY{p}{,}
    \PY{p}{[}\PY{l+m+mf}{0.250}\PY{p}{,} \PY{l+m+mf}{0.266}\PY{p}{,} \PY{l+m+mf}{0.286}\PY{p}{,} \PY{l+m+mf}{0.318}\PY{p}{,} \PY{l+m+mf}{0.380}\PY{p}{]}\PY{p}{,}
    \PY{p}{[}\PY{l+m+mf}{0.244}\PY{p}{,} \PY{l+m+mf}{0.259}\PY{p}{,} \PY{l+m+mf}{0.278}\PY{p}{,} \PY{l+m+mf}{0.309}\PY{p}{,} \PY{l+m+mf}{0.370}\PY{p}{]}\PY{p}{,}
    \PY{p}{[}\PY{l+m+mf}{0.237}\PY{p}{,} \PY{l+m+mf}{0.252}\PY{p}{,} \PY{l+m+mf}{0.272}\PY{p}{,} \PY{l+m+mf}{0.301}\PY{p}{,} \PY{l+m+mf}{0.361}\PY{p}{]}\PY{p}{,}
    \PY{p}{[}\PY{l+m+mf}{0.231}\PY{p}{,} \PY{l+m+mf}{0.246}\PY{p}{,} \PY{l+m+mf}{0.264}\PY{p}{,} \PY{l+m+mf}{0.294}\PY{p}{,} \PY{l+m+mf}{0.352}\PY{p}{]}\PY{p}{,}
\PY{p}{]}

\PY{n}{\PYZus{}INDEX\PYZus{}ST} \PY{o}{=} \PY{n+nb}{range}\PY{p}{(}\PY{l+m+mi}{1}\PY{p}{,} \PY{l+m+mi}{21}\PY{p}{)}

\PY{n}{\PYZus{}COL\PYZus{}ST} \PY{o}{=} \PY{p}{[}\PY{l+m+mf}{0.2}\PY{p}{,} \PY{l+m+mf}{0.15}\PY{p}{,} \PY{l+m+mf}{0.1}\PY{p}{,} \PY{l+m+mf}{0.05}\PY{p}{,} \PY{l+m+mf}{0.01}\PY{p}{]}

\PY{n}{t\PYZus{}dcr\PYZus{}st} \PY{o}{=} \PY{n}{pd}\PY{o}{.}\PY{n}{DataFrame}\PY{p}{(}
    \PY{n}{data}\PY{o}{=}\PY{n}{\PYZus{}DATA\PYZus{}ST}\PY{p}{,} \PY{n}{index}\PY{o}{=}\PY{n}{\PYZus{}INDEX\PYZus{}ST}\PY{p}{,} \PY{n}{columns}\PY{o}{=}\PY{n}{\PYZus{}COL\PYZus{}ST}
\PY{p}{)}
\PY{n}{t\PYZus{}dcr\PYZus{}st}
\end{Verbatim}
\end{tcolorbox}

            \begin{tcolorbox}[breakable, size=fbox, boxrule=.5pt, pad at break*=1mm, opacityfill=0]
\prompt{Out}{outcolor}{ }{\boxspacing}
\begin{Verbatim}[commandchars=\\\{\}]
     0.20   0.15   0.10   0.05   0.01
1   0.900  0.925  0.950  0.975  0.995
2   0.684  0.726  0.776  0.842  0.929
3   0.565  0.597  0.642  0.708  0.829
4   0.494  0.525  0.564  0.624  0.734
5   0.446  0.474  0.510  0.563  0.669
6   0.410  0.436  0.470  0.521  0.618
7   0.381  0.405  0.438  0.486  0.577
8   0.358  0.381  0.411  0.457  0.543
9   0.339  0.360  0.388  0.432  0.514
10  0.322  0.342  0.368  0.409  0.486
11  0.307  0.326  0.352  0.391  0.468
12  0.295  0.313  0.338  0.375  0.450
13  0.284  0.302  0.325  0.361  0.433
14  0.274  0.292  0.314  0.349  0.418
15  0.266  0.283  0.304  0.338  0.404
16  0.258  0.274  0.295  0.328  0.391
17  0.250  0.266  0.286  0.318  0.380
18  0.244  0.259  0.278  0.309  0.370
19  0.237  0.252  0.272  0.301  0.361
20  0.231  0.246  0.264  0.294  0.352
\end{Verbatim}
\end{tcolorbox}
        
    \begin{tcolorbox}[breakable, size=fbox, boxrule=1pt, pad at break*=1mm,colback=cellbackground, colframe=cellborder]
\prompt{In}{incolor}{ }{\boxspacing}
\begin{Verbatim}[commandchars=\\\{\}]
\PY{c+c1}{\PYZsh{} tabel dari soewarno hal. 139}
\PY{c+c1}{\PYZsh{} Tabel Nilai Kritis (Dcr) Untuk Uji Kolmogorov\PYZhy{}Smirnov}

\PY{c+c1}{\PYZsh{} KODE: SW}

\PY{n}{\PYZus{}DATA\PYZus{}SW} \PY{o}{=} \PY{p}{[}
    \PY{p}{[}\PY{l+m+mf}{0.45}\PY{p}{,} \PY{l+m+mf}{0.51}\PY{p}{,} \PY{l+m+mf}{0.56}\PY{p}{,} \PY{l+m+mf}{0.67}\PY{p}{]}\PY{p}{,}
    \PY{p}{[}\PY{l+m+mf}{0.32}\PY{p}{,} \PY{l+m+mf}{0.37}\PY{p}{,} \PY{l+m+mf}{0.41}\PY{p}{,} \PY{l+m+mf}{0.49}\PY{p}{]}\PY{p}{,}
    \PY{p}{[}\PY{l+m+mf}{0.27}\PY{p}{,} \PY{l+m+mf}{0.3} \PY{p}{,} \PY{l+m+mf}{0.34}\PY{p}{,} \PY{l+m+mf}{0.4} \PY{p}{]}\PY{p}{,}
    \PY{p}{[}\PY{l+m+mf}{0.23}\PY{p}{,} \PY{l+m+mf}{0.26}\PY{p}{,} \PY{l+m+mf}{0.29}\PY{p}{,} \PY{l+m+mf}{0.35}\PY{p}{]}\PY{p}{,}
    \PY{p}{[}\PY{l+m+mf}{0.21}\PY{p}{,} \PY{l+m+mf}{0.24}\PY{p}{,} \PY{l+m+mf}{0.26}\PY{p}{,} \PY{l+m+mf}{0.32}\PY{p}{]}\PY{p}{,}
    \PY{p}{[}\PY{l+m+mf}{0.19}\PY{p}{,} \PY{l+m+mf}{0.22}\PY{p}{,} \PY{l+m+mf}{0.24}\PY{p}{,} \PY{l+m+mf}{0.29}\PY{p}{]}\PY{p}{,}
    \PY{p}{[}\PY{l+m+mf}{0.18}\PY{p}{,} \PY{l+m+mf}{0.2} \PY{p}{,} \PY{l+m+mf}{0.22}\PY{p}{,} \PY{l+m+mf}{0.27}\PY{p}{]}\PY{p}{,}
    \PY{p}{[}\PY{l+m+mf}{0.17}\PY{p}{,} \PY{l+m+mf}{0.19}\PY{p}{,} \PY{l+m+mf}{0.21}\PY{p}{,} \PY{l+m+mf}{0.25}\PY{p}{]}\PY{p}{,}
    \PY{p}{[}\PY{l+m+mf}{0.16}\PY{p}{,} \PY{l+m+mf}{0.18}\PY{p}{,} \PY{l+m+mf}{0.2} \PY{p}{,} \PY{l+m+mf}{0.24}\PY{p}{]}\PY{p}{,}
    \PY{p}{[}\PY{l+m+mf}{0.15}\PY{p}{,} \PY{l+m+mf}{0.17}\PY{p}{,} \PY{l+m+mf}{0.19}\PY{p}{,} \PY{l+m+mf}{0.23}\PY{p}{]}
\PY{p}{]}

\PY{n}{\PYZus{}INDEX\PYZus{}SW} \PY{o}{=} \PY{n+nb}{range}\PY{p}{(}\PY{l+m+mi}{5}\PY{p}{,} \PY{l+m+mi}{51}\PY{p}{,} \PY{l+m+mi}{5}\PY{p}{)}

\PY{n}{\PYZus{}COL\PYZus{}SW} \PY{o}{=} \PY{p}{[}\PY{l+m+mf}{0.2}\PY{p}{,} \PY{l+m+mf}{0.1}\PY{p}{,} \PY{l+m+mf}{0.05}\PY{p}{,} \PY{l+m+mf}{0.01}\PY{p}{]}

\PY{n}{t\PYZus{}dcr\PYZus{}sw} \PY{o}{=} \PY{n}{pd}\PY{o}{.}\PY{n}{DataFrame}\PY{p}{(}
    \PY{n}{data}\PY{o}{=}\PY{n}{\PYZus{}DATA\PYZus{}SW}\PY{p}{,} \PY{n}{index}\PY{o}{=}\PY{n}{\PYZus{}INDEX\PYZus{}SW}\PY{p}{,} \PY{n}{columns}\PY{o}{=}\PY{n}{\PYZus{}COL\PYZus{}SW}
\PY{p}{)}
\PY{n}{t\PYZus{}dcr\PYZus{}sw}
\end{Verbatim}
\end{tcolorbox}

            \begin{tcolorbox}[breakable, size=fbox, boxrule=.5pt, pad at break*=1mm, opacityfill=0]
\prompt{Out}{outcolor}{ }{\boxspacing}
\begin{Verbatim}[commandchars=\\\{\}]
    0.20  0.10  0.05  0.01
5   0.45  0.51  0.56  0.67
10  0.32  0.37  0.41  0.49
15  0.27  0.30  0.34  0.40
20  0.23  0.26  0.29  0.35
25  0.21  0.24  0.26  0.32
30  0.19  0.22  0.24  0.29
35  0.18  0.20  0.22  0.27
40  0.17  0.19  0.21  0.25
45  0.16  0.18  0.20  0.24
50  0.15  0.17  0.19  0.23
\end{Verbatim}
\end{tcolorbox}
        
    \hypertarget{kode}{%
\section{KODE}\label{kode}}

    \begin{tcolorbox}[breakable, size=fbox, boxrule=1pt, pad at break*=1mm,colback=cellbackground, colframe=cellborder]
\prompt{In}{incolor}{ }{\boxspacing}
\begin{Verbatim}[commandchars=\\\{\}]
\PY{c+c1}{\PYZsh{} KODE FUNGSI INTERPOLASI DARI TABEL}

\PY{k+kn}{from} \PY{n+nn}{scipy} \PY{k+kn}{import} \PY{n}{interpolate}

\PY{k}{def} \PY{n+nf}{\PYZus{}func\PYZus{}interp\PYZus{}bivariate}\PY{p}{(}\PY{n}{df}\PY{p}{)}\PY{p}{:}
    \PY{l+s+s2}{\PYZdq{}}\PY{l+s+s2}{Membuat fungsi dari tabel untuk interpolasi bilinear}\PY{l+s+s2}{\PYZdq{}}
    \PY{n}{table} \PY{o}{=} \PY{n}{df}\PY{p}{[}\PY{n}{df}\PY{o}{.}\PY{n}{columns}\PY{o}{.}\PY{n}{sort\PYZus{}values}\PY{p}{(}\PY{p}{)}\PY{p}{]}\PY{o}{.}\PY{n}{sort\PYZus{}index}\PY{p}{(}\PY{p}{)}\PY{o}{.}\PY{n}{copy}\PY{p}{(}\PY{p}{)}

    \PY{n}{x} \PY{o}{=} \PY{n}{table}\PY{o}{.}\PY{n}{index}
    \PY{n}{y} \PY{o}{=} \PY{n}{table}\PY{o}{.}\PY{n}{columns}
    \PY{n}{z} \PY{o}{=} \PY{n}{table}\PY{o}{.}\PY{n}{to\PYZus{}numpy}\PY{p}{(}\PY{p}{)}

    \PY{c+c1}{\PYZsh{} penggunaan kx=1, ky=1 untuk interpolasi linear antara 2 titik}
    \PY{c+c1}{\PYZsh{} tidak menggunakan (cubic) spline interpolation}
    \PY{k}{return} \PY{n}{interpolate}\PY{o}{.}\PY{n}{RectBivariateSpline}\PY{p}{(}\PY{n}{x}\PY{p}{,} \PY{n}{y}\PY{p}{,} \PY{n}{z}\PY{p}{,} \PY{n}{kx}\PY{o}{=}\PY{l+m+mi}{1}\PY{p}{,} \PY{n}{ky}\PY{o}{=}\PY{l+m+mi}{1}\PY{p}{)}

\PY{k}{def} \PY{n+nf}{\PYZus{}as\PYZus{}value}\PY{p}{(}\PY{n}{x}\PY{p}{,} \PY{n}{dec}\PY{o}{=}\PY{l+m+mi}{4}\PY{p}{)}\PY{p}{:}
    \PY{n}{x} \PY{o}{=} \PY{n}{np}\PY{o}{.}\PY{n}{around}\PY{p}{(}\PY{n}{x}\PY{p}{,} \PY{n}{dec}\PY{p}{)}
    \PY{k}{return} \PY{n}{x}\PY{o}{.}\PY{n}{flatten}\PY{p}{(}\PY{p}{)} \PY{k}{if} \PY{n}{x}\PY{o}{.}\PY{n}{size} \PY{o}{\PYZgt{}} \PY{l+m+mi}{1} \PY{k}{else} \PY{n}{x}\PY{o}{.}\PY{n}{item}\PY{p}{(}\PY{p}{)}

\PY{k}{def} \PY{n+nf}{\PYZus{}calc\PYZus{}k}\PY{p}{(}\PY{n}{x}\PY{p}{)}\PY{p}{:}
    \PY{k}{return} \PY{p}{(}\PY{n}{x} \PY{o}{\PYZhy{}} \PY{n}{x}\PY{o}{.}\PY{n}{mean}\PY{p}{(}\PY{p}{)}\PY{p}{)} \PY{o}{/} \PY{n}{x}\PY{o}{.}\PY{n}{std}\PY{p}{(}\PY{p}{)}
\end{Verbatim}
\end{tcolorbox}

    \begin{tcolorbox}[breakable, size=fbox, boxrule=1pt, pad at break*=1mm,colback=cellbackground, colframe=cellborder]
\prompt{In}{incolor}{ }{\boxspacing}
\begin{Verbatim}[commandchars=\\\{\}]
\PY{n}{table\PYZus{}source} \PY{o}{=} \PY{p}{\PYZob{}}
    \PY{l+s+s1}{\PYZsq{}}\PY{l+s+s1}{soewarno}\PY{l+s+s1}{\PYZsq{}}\PY{p}{:} \PY{n}{t\PYZus{}dcr\PYZus{}sw}\PY{p}{,}
    \PY{l+s+s1}{\PYZsq{}}\PY{l+s+s1}{soetopo}\PY{l+s+s1}{\PYZsq{}}\PY{p}{:} \PY{n}{t\PYZus{}dcr\PYZus{}st}
\PY{p}{\PYZcb{}}

\PY{n}{anfrek} \PY{o}{=} \PY{p}{\PYZob{}}
    \PY{l+s+s1}{\PYZsq{}}\PY{l+s+s1}{normal}\PY{l+s+s1}{\PYZsq{}}\PY{p}{:} \PY{n}{frek\PYZus{}normal}\PY{p}{,}
    \PY{l+s+s1}{\PYZsq{}}\PY{l+s+s1}{lognormal}\PY{l+s+s1}{\PYZsq{}}\PY{p}{:} \PY{n}{frek\PYZus{}lognormal}\PY{p}{,}
    \PY{l+s+s1}{\PYZsq{}}\PY{l+s+s1}{gumbel}\PY{l+s+s1}{\PYZsq{}}\PY{p}{:} \PY{n}{frek\PYZus{}gumbel}\PY{p}{,}
    \PY{l+s+s1}{\PYZsq{}}\PY{l+s+s1}{logpearson3}\PY{l+s+s1}{\PYZsq{}}\PY{p}{:} \PY{n}{frek\PYZus{}logpearson3}
\PY{p}{\PYZcb{}}

\PY{k}{def} \PY{n+nf}{calc\PYZus{}dcr}\PY{p}{(}\PY{n}{alpha}\PY{p}{,} \PY{n}{n}\PY{p}{,} \PY{n}{source}\PY{o}{=}\PY{l+s+s1}{\PYZsq{}}\PY{l+s+s1}{scipy}\PY{l+s+s1}{\PYZsq{}}\PY{p}{)}\PY{p}{:}
    \PY{n}{alpha} \PY{o}{=} \PY{n}{np}\PY{o}{.}\PY{n}{array}\PY{p}{(}\PY{n}{alpha}\PY{p}{)}
    \PY{k}{if} \PY{n}{source}\PY{o}{.}\PY{n}{lower}\PY{p}{(}\PY{p}{)} \PY{o}{==} \PY{l+s+s1}{\PYZsq{}}\PY{l+s+s1}{scipy}\PY{l+s+s1}{\PYZsq{}}\PY{p}{:}
        \PY{c+c1}{\PYZsh{} ref: https://stackoverflow.com/questions/53509986/}
        \PY{k}{return} \PY{n}{stats}\PY{o}{.}\PY{n}{ksone}\PY{o}{.}\PY{n}{ppf}\PY{p}{(}\PY{l+m+mi}{1}\PY{o}{\PYZhy{}}\PY{n}{alpha}\PY{o}{/}\PY{l+m+mi}{2}\PY{p}{,} \PY{n}{n}\PY{p}{)}
    \PY{k}{elif} \PY{n}{source}\PY{o}{.}\PY{n}{lower}\PY{p}{(}\PY{p}{)} \PY{o+ow}{in} \PY{n}{table\PYZus{}source}\PY{o}{.}\PY{n}{keys}\PY{p}{(}\PY{p}{)}\PY{p}{:}
        \PY{n}{func\PYZus{}table} \PY{o}{=} \PY{n}{\PYZus{}func\PYZus{}interp\PYZus{}bivariate}\PY{p}{(}\PY{n}{table\PYZus{}source}\PY{p}{[}\PY{n}{source}\PY{o}{.}\PY{n}{lower}\PY{p}{(}\PY{p}{)}\PY{p}{]}\PY{p}{)}
        \PY{c+c1}{\PYZsh{} untuk soewarno 2 angka dibelakang koma, dan soetopo = 3}
        \PY{n}{dec} \PY{o}{=} \PY{p}{(}\PY{n}{source}\PY{o}{.}\PY{n}{lower}\PY{p}{(}\PY{p}{)} \PY{o}{==} \PY{l+s+s1}{\PYZsq{}}\PY{l+s+s1}{soetopo}\PY{l+s+s1}{\PYZsq{}}\PY{p}{)} \PY{o}{+} \PY{l+m+mi}{2}
        \PY{k}{return} \PY{n}{\PYZus{}as\PYZus{}value}\PY{p}{(}\PY{n}{func\PYZus{}table}\PY{p}{(}\PY{n}{n}\PY{p}{,} \PY{n}{alpha}\PY{p}{,} \PY{n}{grid}\PY{o}{=}\PY{k+kc}{False}\PY{p}{)}\PY{p}{,} \PY{n}{dec}\PY{p}{)}

\PY{k}{def} \PY{n+nf}{kstest}\PY{p}{(}
    \PY{n}{df}\PY{p}{,} \PY{n}{col}\PY{o}{=}\PY{k+kc}{None}\PY{p}{,} \PY{n}{dist}\PY{o}{=}\PY{l+s+s1}{\PYZsq{}}\PY{l+s+s1}{normal}\PY{l+s+s1}{\PYZsq{}}\PY{p}{,} \PY{n}{source\PYZus{}dist}\PY{o}{=}\PY{l+s+s1}{\PYZsq{}}\PY{l+s+s1}{scipy}\PY{l+s+s1}{\PYZsq{}}\PY{p}{,} 
    \PY{n}{alpha}\PY{o}{=}\PY{l+m+mf}{0.05}\PY{p}{,} \PY{n}{source\PYZus{}dcr}\PY{o}{=}\PY{l+s+s1}{\PYZsq{}}\PY{l+s+s1}{scipy}\PY{l+s+s1}{\PYZsq{}}\PY{p}{,} \PY{n}{show\PYZus{}stat}\PY{o}{=}\PY{k+kc}{True}\PY{p}{,} \PY{n}{report}\PY{o}{=}\PY{l+s+s1}{\PYZsq{}}\PY{l+s+s1}{result}\PY{l+s+s1}{\PYZsq{}}
    \PY{p}{)}\PY{p}{:}

    \PY{n}{source\PYZus{}dist} \PY{o}{=} \PY{l+s+s1}{\PYZsq{}}\PY{l+s+s1}{gumbel}\PY{l+s+s1}{\PYZsq{}} \PY{k}{if} \PY{n}{dist}\PY{o}{.}\PY{n}{lower}\PY{p}{(}\PY{p}{)} \PY{o}{==} \PY{l+s+s1}{\PYZsq{}}\PY{l+s+s1}{gumbel}\PY{l+s+s1}{\PYZsq{}} \PY{k}{else} \PY{n}{source\PYZus{}dist}

    \PY{n}{col} \PY{o}{=} \PY{n}{df}\PY{o}{.}\PY{n}{columns}\PY{p}{[}\PY{l+m+mi}{0}\PY{p}{]} \PY{k}{if} \PY{n}{col} \PY{o+ow}{is} \PY{k+kc}{None} \PY{k}{else} \PY{n}{col}
    \PY{n}{data} \PY{o}{=} \PY{n}{df}\PY{p}{[}\PY{p}{[}\PY{n}{col}\PY{p}{]}\PY{p}{]}\PY{o}{.}\PY{n}{copy}\PY{p}{(}\PY{p}{)}
    \PY{n}{n} \PY{o}{=} \PY{n+nb}{len}\PY{p}{(}\PY{n}{data}\PY{p}{)}
    \PY{n}{data} \PY{o}{=} \PY{n}{data}\PY{o}{.}\PY{n}{rename}\PY{p}{(}\PY{p}{\PYZob{}}\PY{n}{col}\PY{p}{:} \PY{l+s+s1}{\PYZsq{}}\PY{l+s+s1}{x}\PY{l+s+s1}{\PYZsq{}}\PY{p}{\PYZcb{}}\PY{p}{,} \PY{n}{axis}\PY{o}{=}\PY{l+m+mi}{1}\PY{p}{)}
    \PY{n}{data} \PY{o}{=} \PY{n}{data}\PY{o}{.}\PY{n}{sort\PYZus{}values}\PY{p}{(}\PY{l+s+s1}{\PYZsq{}}\PY{l+s+s1}{x}\PY{l+s+s1}{\PYZsq{}}\PY{p}{)}
    \PY{n}{data}\PY{p}{[}\PY{l+s+s1}{\PYZsq{}}\PY{l+s+s1}{no}\PY{l+s+s1}{\PYZsq{}}\PY{p}{]} \PY{o}{=} \PY{n}{np}\PY{o}{.}\PY{n}{arange}\PY{p}{(}\PY{n}{n}\PY{p}{)} \PY{o}{+} \PY{l+m+mi}{1}

    \PY{c+c1}{\PYZsh{} w = weibull}
    \PY{n}{data}\PY{p}{[}\PY{l+s+s1}{\PYZsq{}}\PY{l+s+s1}{p\PYZus{}w}\PY{l+s+s1}{\PYZsq{}}\PY{p}{]} \PY{o}{=} \PY{n}{data}\PY{o}{.}\PY{n}{no} \PY{o}{/} \PY{p}{(}\PY{n}{n}\PY{o}{+}\PY{l+m+mi}{1}\PY{p}{)}
    
    \PY{k}{if} \PY{n}{dist}\PY{o}{.}\PY{n}{lower}\PY{p}{(}\PY{p}{)} \PY{o+ow}{in} \PY{p}{[}\PY{l+s+s1}{\PYZsq{}}\PY{l+s+s1}{normal}\PY{l+s+s1}{\PYZsq{}}\PY{p}{,} \PY{l+s+s1}{\PYZsq{}}\PY{l+s+s1}{gumbel}\PY{l+s+s1}{\PYZsq{}}\PY{p}{]}\PY{p}{:}
        \PY{n}{data}\PY{p}{[}\PY{l+s+s1}{\PYZsq{}}\PY{l+s+s1}{k}\PY{l+s+s1}{\PYZsq{}}\PY{p}{]} \PY{o}{=} \PY{n}{\PYZus{}calc\PYZus{}k}\PY{p}{(}\PY{n}{data}\PY{o}{.}\PY{n}{x}\PY{p}{)}
    \PY{k}{if} \PY{n}{dist}\PY{o}{.}\PY{n}{lower}\PY{p}{(}\PY{p}{)} \PY{o+ow}{in} \PY{p}{[}\PY{l+s+s1}{\PYZsq{}}\PY{l+s+s1}{lognormal}\PY{l+s+s1}{\PYZsq{}}\PY{p}{,} \PY{l+s+s1}{\PYZsq{}}\PY{l+s+s1}{logpearson3}\PY{l+s+s1}{\PYZsq{}}\PY{p}{]}\PY{p}{:}
        \PY{n}{data}\PY{p}{[}\PY{l+s+s1}{\PYZsq{}}\PY{l+s+s1}{log\PYZus{}x}\PY{l+s+s1}{\PYZsq{}}\PY{p}{]} \PY{o}{=} \PY{n}{np}\PY{o}{.}\PY{n}{log10}\PY{p}{(}\PY{n}{data}\PY{o}{.}\PY{n}{x}\PY{p}{)}
        \PY{n}{data}\PY{p}{[}\PY{l+s+s1}{\PYZsq{}}\PY{l+s+s1}{k}\PY{l+s+s1}{\PYZsq{}}\PY{p}{]} \PY{o}{=} \PY{n}{\PYZus{}calc\PYZus{}k}\PY{p}{(}\PY{n}{data}\PY{o}{.}\PY{n}{log\PYZus{}x}\PY{p}{)}

    \PY{n}{func} \PY{o}{=} \PY{n}{anfrek}\PY{p}{[}\PY{n}{dist}\PY{o}{.}\PY{n}{lower}\PY{p}{(}\PY{p}{)}\PY{p}{]}

    \PY{k}{if} \PY{n}{dist}\PY{o}{.}\PY{n}{lower}\PY{p}{(}\PY{p}{)} \PY{o+ow}{in} \PY{p}{[}\PY{l+s+s1}{\PYZsq{}}\PY{l+s+s1}{normal}\PY{l+s+s1}{\PYZsq{}}\PY{p}{,} \PY{l+s+s1}{\PYZsq{}}\PY{l+s+s1}{lognormal}\PY{l+s+s1}{\PYZsq{}}\PY{p}{]}\PY{p}{:}
        \PY{n}{parameter} \PY{o}{=} \PY{p}{(}\PY{p}{)}
    \PY{k}{elif} \PY{n}{dist}\PY{o}{.}\PY{n}{lower}\PY{p}{(}\PY{p}{)} \PY{o}{==} \PY{l+s+s1}{\PYZsq{}}\PY{l+s+s1}{gumbel}\PY{l+s+s1}{\PYZsq{}}\PY{p}{:}
        \PY{n}{parameter} \PY{o}{=} \PY{p}{(}\PY{n}{n}\PY{p}{,}\PY{p}{)}
    \PY{k}{elif} \PY{n}{dist}\PY{o}{.}\PY{n}{lower}\PY{p}{(}\PY{p}{)} \PY{o}{==} \PY{l+s+s1}{\PYZsq{}}\PY{l+s+s1}{logpearson3}\PY{l+s+s1}{\PYZsq{}}\PY{p}{:}
        \PY{n}{parameter} \PY{o}{=} \PY{p}{(}\PY{n}{data}\PY{o}{.}\PY{n}{log\PYZus{}x}\PY{o}{.}\PY{n}{skew}\PY{p}{(}\PY{p}{)}\PY{p}{,}\PY{p}{)}
    
    \PY{c+c1}{\PYZsh{} d = distribusi}
    \PY{n}{data}\PY{p}{[}\PY{l+s+s1}{\PYZsq{}}\PY{l+s+s1}{p\PYZus{}d}\PY{l+s+s1}{\PYZsq{}}\PY{p}{]} \PY{o}{=} \PY{n}{func}\PY{o}{.}\PY{n}{calc\PYZus{}prob}\PY{p}{(}\PY{n}{data}\PY{o}{.}\PY{n}{k}\PY{p}{,} \PY{n}{source}\PY{o}{=}\PY{n}{source\PYZus{}dist}\PY{p}{,} \PY{o}{*}\PY{n}{parameter}\PY{p}{)} 
    \PY{n}{data}\PY{p}{[}\PY{l+s+s1}{\PYZsq{}}\PY{l+s+s1}{d}\PY{l+s+s1}{\PYZsq{}}\PY{p}{]} \PY{o}{=} \PY{p}{(}\PY{n}{data}\PY{o}{.}\PY{n}{p\PYZus{}w} \PY{o}{\PYZhy{}} \PY{n}{data}\PY{o}{.}\PY{n}{p\PYZus{}d}\PY{p}{)}\PY{o}{.}\PY{n}{abs}\PY{p}{(}\PY{p}{)}
    \PY{n}{dmax} \PY{o}{=} \PY{n}{data}\PY{o}{.}\PY{n}{d}\PY{o}{.}\PY{n}{max}\PY{p}{(}\PY{p}{)}
    \PY{n}{dcr} \PY{o}{=} \PY{n}{calc\PYZus{}dcr}\PY{p}{(}\PY{n}{alpha}\PY{p}{,} \PY{n}{n}\PY{p}{,} \PY{n}{source}\PY{o}{=}\PY{n}{source\PYZus{}dcr}\PY{p}{)}
    \PY{n}{result} \PY{o}{=} \PY{n+nb}{int}\PY{p}{(}\PY{n}{dmax} \PY{o}{\PYZlt{}} \PY{n}{dcr}\PY{p}{)}
    \PY{n}{result\PYZus{}text} \PY{o}{=} \PY{p}{[}\PY{l+s+s1}{\PYZsq{}}\PY{l+s+s1}{Distribusi Tidak Diterima}\PY{l+s+s1}{\PYZsq{}}\PY{p}{,} \PY{l+s+s1}{\PYZsq{}}\PY{l+s+s1}{Distribusi Diterima}\PY{l+s+s1}{\PYZsq{}}\PY{p}{]}

    \PY{k}{if} \PY{n}{show\PYZus{}stat}\PY{p}{:}
        \PY{n+nb}{print}\PY{p}{(}\PY{l+s+sa}{f}\PY{l+s+s1}{\PYZsq{}}\PY{l+s+s1}{Periksa Kecocokan Distribusi }\PY{l+s+si}{\PYZob{}}\PY{n}{dist}\PY{o}{.}\PY{n}{title}\PY{p}{(}\PY{p}{)}\PY{l+s+si}{\PYZcb{}}\PY{l+s+s1}{\PYZsq{}}\PY{p}{)}
        \PY{n+nb}{print}\PY{p}{(}\PY{l+s+sa}{f}\PY{l+s+s1}{\PYZsq{}}\PY{l+s+s1}{Delta Kritikal = }\PY{l+s+si}{\PYZob{}}\PY{n}{dcr}\PY{l+s+si}{:}\PY{l+s+s1}{.5f}\PY{l+s+si}{\PYZcb{}}\PY{l+s+s1}{\PYZsq{}}\PY{p}{)}
        \PY{n+nb}{print}\PY{p}{(}\PY{l+s+sa}{f}\PY{l+s+s1}{\PYZsq{}}\PY{l+s+s1}{Delta Max = }\PY{l+s+si}{\PYZob{}}\PY{n}{dmax}\PY{l+s+si}{:}\PY{l+s+s1}{.5f}\PY{l+s+si}{\PYZcb{}}\PY{l+s+s1}{\PYZsq{}}\PY{p}{)}
        \PY{n+nb}{print}\PY{p}{(}\PY{l+s+sa}{f}\PY{l+s+s1}{\PYZsq{}}\PY{l+s+s1}{Result (Dmax \PYZlt{} Dcr) = }\PY{l+s+si}{\PYZob{}}\PY{n}{result\PYZus{}text}\PY{p}{[}\PY{n}{result}\PY{p}{]}\PY{l+s+si}{\PYZcb{}}\PY{l+s+s1}{\PYZsq{}}\PY{p}{)}

    \PY{k}{if} \PY{n}{report}\PY{o}{.}\PY{n}{lower}\PY{p}{(}\PY{p}{)} \PY{o}{==} \PY{l+s+s1}{\PYZsq{}}\PY{l+s+s1}{result}\PY{l+s+s1}{\PYZsq{}}\PY{p}{:}
        \PY{k}{return} \PY{n}{data}\PY{p}{[}\PY{l+s+s1}{\PYZsq{}}\PY{l+s+s1}{no x p\PYZus{}w p\PYZus{}d d}\PY{l+s+s1}{\PYZsq{}}\PY{o}{.}\PY{n}{split}\PY{p}{(}\PY{p}{)}\PY{p}{]}
    \PY{k}{elif} \PY{n}{report}\PY{o}{.}\PY{n}{lower}\PY{p}{(}\PY{p}{)} \PY{o}{==} \PY{l+s+s1}{\PYZsq{}}\PY{l+s+s1}{full}\PY{l+s+s1}{\PYZsq{}}\PY{p}{:}
        \PY{k}{return} \PY{n}{data}
\end{Verbatim}
\end{tcolorbox}

    \hypertarget{fungsi}{%
\section{FUNGSI}\label{fungsi}}

    \hypertarget{fungsi-calc_dcralpha-n-...}{%
\subsection{\texorpdfstring{Fungsi
\texttt{calc\_dcr(alpha,\ n,\ ...)}}{Fungsi calc\_dcr(alpha, n, ...)}}\label{fungsi-calc_dcralpha-n-...}}

Function:
\texttt{calc\_dcr(alpha,\ n,\ source=\textquotesingle{}scipy\textquotesingle{})}

Fungsi \texttt{calc\_dcr(...)} digunakan untuk mencari nilai Delta
kritis (Dcr / \(\Delta_{kritis}\)) dari berbagai sumber berdasarkan
nilai derajat kepercayaan (\emph{level of significance}) \(\alpha\) dan
jumlah banyaknya data \(n\).

\begin{itemize}
\tightlist
\item
  Argumen Posisi:

  \begin{itemize}
  \tightlist
  \item
    \texttt{alpha}: Nilai \emph{level of significance} \(\alpha\). Dalam
    satuan desimal (\(\left(0,1\right) \in \mathbb{R}\)).
  \item
    \texttt{n}: Jumlah banyaknya data.
  \end{itemize}
\item
  Argumen Opsional:

  \begin{itemize}
  \tightlist
  \item
    \texttt{source}: sumber nilai \texttt{Dcr}.
    \texttt{\textquotesingle{}scipy\textquotesingle{}} (default). Sumber
    yang dapat digunakan antara lain: Soetopo
    (\texttt{\textquotesingle{}soetopo\textquotesingle{}}), Soewarno
    (\texttt{\textquotesingle{}soewarno\textquotesingle{}}).
  \end{itemize}
\end{itemize}

Perlu dicatat bahwa batas nilai \(\alpha\) dan \(n\) untuk masing-masing
tabel berbeda-beda. - Untuk \texttt{soetopo} batasan dimulai dari
\(\alpha = \left[0.2,0.01\right]\) dengan \(n = \left[1,20\right]\) -
Untuk \texttt{soewarno} batasan dimulai dari
\(\alpha = \left[0.2,0.01\right]\) dengan \(n = \left[5,50\right]\)

Untuk \(n > 50\) disarankan menggunakan \texttt{scipy}.

    \begin{tcolorbox}[breakable, size=fbox, boxrule=1pt, pad at break*=1mm,colback=cellbackground, colframe=cellborder]
\prompt{In}{incolor}{ }{\boxspacing}
\begin{Verbatim}[commandchars=\\\{\}]
\PY{n}{calc\PYZus{}dcr}\PY{p}{(}\PY{l+m+mf}{0.2}\PY{p}{,} \PY{l+m+mi}{10}\PY{p}{)}
\end{Verbatim}
\end{tcolorbox}

            \begin{tcolorbox}[breakable, size=fbox, boxrule=.5pt, pad at break*=1mm, opacityfill=0]
\prompt{Out}{outcolor}{ }{\boxspacing}
\begin{Verbatim}[commandchars=\\\{\}]
0.32260155962627957
\end{Verbatim}
\end{tcolorbox}
        
    \begin{tcolorbox}[breakable, size=fbox, boxrule=1pt, pad at break*=1mm,colback=cellbackground, colframe=cellborder]
\prompt{In}{incolor}{ }{\boxspacing}
\begin{Verbatim}[commandchars=\\\{\}]
\PY{n}{calc\PYZus{}dcr}\PY{p}{(}\PY{l+m+mf}{0.15}\PY{p}{,} \PY{l+m+mi}{10}\PY{p}{,} \PY{n}{source}\PY{o}{=}\PY{l+s+s1}{\PYZsq{}}\PY{l+s+s1}{soetopo}\PY{l+s+s1}{\PYZsq{}}\PY{p}{)}
\end{Verbatim}
\end{tcolorbox}

            \begin{tcolorbox}[breakable, size=fbox, boxrule=.5pt, pad at break*=1mm, opacityfill=0]
\prompt{Out}{outcolor}{ }{\boxspacing}
\begin{Verbatim}[commandchars=\\\{\}]
0.342
\end{Verbatim}
\end{tcolorbox}
        
    \begin{tcolorbox}[breakable, size=fbox, boxrule=1pt, pad at break*=1mm,colback=cellbackground, colframe=cellborder]
\prompt{In}{incolor}{ }{\boxspacing}
\begin{Verbatim}[commandchars=\\\{\}]
\PY{c+c1}{\PYZsh{} perbandingan antara nilai tabel dan fungsi scipy}

\PY{n}{source\PYZus{}test} \PY{o}{=} \PY{p}{[}\PY{l+s+s1}{\PYZsq{}}\PY{l+s+s1}{soewarno}\PY{l+s+s1}{\PYZsq{}}\PY{p}{,} \PY{l+s+s1}{\PYZsq{}}\PY{l+s+s1}{soetopo}\PY{l+s+s1}{\PYZsq{}}\PY{p}{,} \PY{l+s+s1}{\PYZsq{}}\PY{l+s+s1}{scipy}\PY{l+s+s1}{\PYZsq{}}\PY{p}{]}

\PY{n}{\PYZus{}n} \PY{o}{=} \PY{l+m+mi}{10}
\PY{n}{\PYZus{}alpha} \PY{o}{=} \PY{p}{[}\PY{l+m+mf}{0.2}\PY{p}{,} \PY{l+m+mf}{0.15}\PY{p}{,} \PY{l+m+mf}{0.1}\PY{p}{,} \PY{l+m+mf}{0.07}\PY{p}{,} \PY{l+m+mf}{0.05}\PY{p}{,} \PY{l+m+mf}{0.01}\PY{p}{]}

\PY{k}{for} \PY{n}{\PYZus{}source} \PY{o+ow}{in} \PY{n}{source\PYZus{}test}\PY{p}{:}
    \PY{n+nb}{print}\PY{p}{(}\PY{l+s+sa}{f}\PY{l+s+s1}{\PYZsq{}}\PY{l+s+s1}{Dcr }\PY{l+s+si}{\PYZob{}}\PY{n}{\PYZus{}source}\PY{l+s+si}{:}\PY{l+s+s1}{\PYZlt{}12}\PY{l+s+si}{\PYZcb{}}\PY{l+s+s1}{=}\PY{l+s+s1}{\PYZsq{}}\PY{p}{,} \PY{n}{calc\PYZus{}dcr}\PY{p}{(}\PY{n}{\PYZus{}alpha}\PY{p}{,} \PY{n}{\PYZus{}n}\PY{p}{,} \PY{n}{source}\PY{o}{=}\PY{n}{\PYZus{}source}\PY{p}{)}\PY{p}{)}
\end{Verbatim}
\end{tcolorbox}

    \begin{Verbatim}[commandchars=\\\{\}]
Dcr soewarno    = [0.32 0.35 0.37 0.39 0.41 0.49]
Dcr soetopo     = [0.322 0.342 0.368 0.393 0.409 0.486]
Dcr scipy       = [0.32260156 0.34250845 0.36866333 0.3901533  0.40924614
0.48893166]
    \end{Verbatim}

    \hypertarget{fungsi-kstestdf-...}{%
\subsection{\texorpdfstring{Fungsi
\texttt{kstest(df,\ ...)}}{Fungsi kstest(df, ...)}}\label{fungsi-kstestdf-...}}

Function:
\texttt{kstest(df,\ col=None,\ dist=\textquotesingle{}normal\textquotesingle{},\ source\_dist=\textquotesingle{}scipy\textquotesingle{},\ alpha=0.05,\ source\_dcr=\textquotesingle{}scipy\textquotesingle{},\ show\_stat=True,\ report=\textquotesingle{}result\textquotesingle{})}

Fungsi \texttt{kstest(...)} merupakan fungsi untuk melakukan uji
kolmogorov-smirnov terhadap distribusi yang dibandingkan. Fungsi ini
mengeluarkan objek \texttt{pandas.DataFrame}.

\begin{itemize}
\tightlist
\item
  Argumen Posisi:

  \begin{itemize}
  \tightlist
  \item
    \texttt{df}: \texttt{pandas.DataFrame}.
  \end{itemize}
\item
  Argumen Opsional:

  \begin{itemize}
  \tightlist
  \item
    \texttt{col}: nama kolom, \texttt{None} (default). Jika tidak diisi
    menggunakan kolom pertama dalam \texttt{df} sebagai data masukan.
  \item
    \texttt{dist}: distribusi yang dibandingkan,
    \texttt{\textquotesingle{}normal\textquotesingle{}} (distribusi
    normal) (default). Distribusi yang dapat digunakan antara lain: Log
    Normal (\texttt{\textquotesingle{}lognormal\textquotesingle{}}),
    Gumbel (\texttt{\textquotesingle{}gumbel\textquotesingle{}}), Log
    Pearson 3
    (\texttt{\textquotesingle{}logpearson3\textquotesingle{}}).
  \item
    \texttt{source\_dist}: sumber perhitungan distribusi,
    \texttt{\textquotesingle{}scipy\textquotesingle{}} (default). Lihat
    masing-masing modul analisis frekuensi untuk lebih jelasnya.
  \item
    \texttt{alpha}: nilai \(\alpha\), \texttt{0.05} (default).
  \item
    \texttt{source\_dcr}: sumber nilai Dcr,
    \texttt{\textquotesingle{}scipy\textquotesingle{}} (default). Sumber
    yang dapat digunakan antara lain: Soetopo
    (\texttt{\textquotesingle{}soetopo\textquotesingle{}}), Soewarno
    (\texttt{\textquotesingle{}soewarno\textquotesingle{}}).
  \item
    \texttt{show\_stat}: menampilkan hasil luaran uji, \texttt{True}
    (default).
  \item
    \texttt{report}: opsi kolom luaran dataframe,
    \texttt{\textquotesingle{}result\textquotesingle{}} (default). Untuk
    melihat kolom perhitungan yang lainnya gunakan
    \texttt{\textquotesingle{}full\textquotesingle{}}.
  \end{itemize}
\end{itemize}

    \begin{tcolorbox}[breakable, size=fbox, boxrule=1pt, pad at break*=1mm,colback=cellbackground, colframe=cellborder]
\prompt{In}{incolor}{ }{\boxspacing}
\begin{Verbatim}[commandchars=\\\{\}]
\PY{n}{kstest}\PY{p}{(}\PY{n}{data}\PY{p}{)}
\end{Verbatim}
\end{tcolorbox}

    \begin{Verbatim}[commandchars=\\\{\}]
Periksa Kecocokan Distribusi Normal
Delta Kritikal = 0.40925
Delta Max = 0.09609
Result (Dmax < Dcr) = Distribusi Diterima
    \end{Verbatim}

            \begin{tcolorbox}[breakable, size=fbox, boxrule=.5pt, pad at break*=1mm, opacityfill=0]
\prompt{Out}{outcolor}{ }{\boxspacing}
\begin{Verbatim}[commandchars=\\\{\}]
            no    x       p\_w       p\_d         d
tahun
2003-01-01   1   52  0.090909  0.025435  0.065474
2004-01-01   2   69  0.181818  0.119957  0.061862
1998-01-01   3   85  0.272727  0.328681  0.055953
1999-01-01   4   92  0.363636  0.450869  0.087233
2005-01-01   5   95  0.454545  0.505473  0.050927
2006-01-01   6   96  0.545455  0.523702  0.021753
2007-01-01   7  105  0.636364  0.681178  0.044815
2000-01-01   8  115  0.727273  0.823367  0.096095
2001-01-01   9  116  0.818182  0.834972  0.016790
2002-01-01  10  122  0.909091  0.894052  0.015039
\end{Verbatim}
\end{tcolorbox}
        
    \begin{tcolorbox}[breakable, size=fbox, boxrule=1pt, pad at break*=1mm,colback=cellbackground, colframe=cellborder]
\prompt{In}{incolor}{ }{\boxspacing}
\begin{Verbatim}[commandchars=\\\{\}]
\PY{n}{kstest}\PY{p}{(}\PY{n}{data}\PY{p}{,} \PY{n}{dist}\PY{o}{=}\PY{l+s+s1}{\PYZsq{}}\PY{l+s+s1}{gumbel}\PY{l+s+s1}{\PYZsq{}}\PY{p}{,} \PY{n}{source\PYZus{}dist}\PY{o}{=}\PY{l+s+s1}{\PYZsq{}}\PY{l+s+s1}{soetopo}\PY{l+s+s1}{\PYZsq{}}\PY{p}{)}\PY{p}{;}
\end{Verbatim}
\end{tcolorbox}

    \begin{Verbatim}[commandchars=\\\{\}]
Periksa Kecocokan Distribusi Gumbel
Delta Kritikal = 0.40925
Delta Max = 0.14036
Result (Dmax < Dcr) = Distribusi Diterima
    \end{Verbatim}

    \begin{tcolorbox}[breakable, size=fbox, boxrule=1pt, pad at break*=1mm,colback=cellbackground, colframe=cellborder]
\prompt{In}{incolor}{ }{\boxspacing}
\begin{Verbatim}[commandchars=\\\{\}]
\PY{n}{kstest}\PY{p}{(}\PY{n}{data}\PY{p}{,} \PY{n}{dist}\PY{o}{=}\PY{l+s+s1}{\PYZsq{}}\PY{l+s+s1}{logpearson3}\PY{l+s+s1}{\PYZsq{}}\PY{p}{,} \PY{n}{alpha}\PY{o}{=}\PY{l+m+mf}{0.2}\PY{p}{,} \PY{n}{source\PYZus{}dcr}\PY{o}{=}\PY{l+s+s1}{\PYZsq{}}\PY{l+s+s1}{soetopo}\PY{l+s+s1}{\PYZsq{}}\PY{p}{,} \PY{n}{report}\PY{o}{=}\PY{l+s+s1}{\PYZsq{}}\PY{l+s+s1}{full}\PY{l+s+s1}{\PYZsq{}}\PY{p}{)}
\end{Verbatim}
\end{tcolorbox}

    \begin{Verbatim}[commandchars=\\\{\}]
Periksa Kecocokan Distribusi Logpearson3
Delta Kritikal = 0.32200
Delta Max = 0.07653
Result (Dmax < Dcr) = Distribusi Diterima
    \end{Verbatim}

            \begin{tcolorbox}[breakable, size=fbox, boxrule=.5pt, pad at break*=1mm, opacityfill=0]
\prompt{Out}{outcolor}{ }{\boxspacing}
\begin{Verbatim}[commandchars=\\\{\}]
              x  no       p\_w     log\_x         k       p\_d         d
tahun
2003-01-01   52   1  0.090909  1.716003 -2.179769  0.036195  0.054714
2004-01-01   69   2  0.181818  1.838849 -1.100345  0.131798  0.050021
1998-01-01   85   3  0.272727  1.929419 -0.304525  0.310292  0.037565
1999-01-01   92   4  0.363636  1.963788 -0.002531  0.415463  0.051827
2005-01-01   95   5  0.454545  1.977724  0.119920  0.464456  0.009910
2006-01-01   96   6  0.545455  1.982271  0.159879  0.481195  0.064259
2007-01-01  105   7  0.636364  2.021189  0.501845  0.636948  0.000584
2000-01-01  115   8  0.727273  2.060698  0.849000  0.803798  0.076525
2001-01-01  116   9  0.818182  2.064458  0.882039  0.818980  0.000798
2002-01-01  122  10  0.909091  2.086360  1.074487  0.900271  0.008820
\end{Verbatim}
\end{tcolorbox}
        
    \hypertarget{changelog}{%
\section{Changelog}\label{changelog}}

\begin{verbatim}
- 20220316 - 1.0.0 - Initial
\end{verbatim}

\hypertarget{copyright-2022-taruma-sakti-megariansyah}{%
\paragraph{\texorpdfstring{Copyright © 2022
\href{https://taruma.github.io}{Taruma Sakti
Megariansyah}}{Copyright © 2022 Taruma Sakti Megariansyah}}\label{copyright-2022-taruma-sakti-megariansyah}}

Source code in this notebook is licensed under a
\href{https://choosealicense.com/licenses/mit/}{MIT License}. Data in
this notebook is licensed under a
\href{https://creativecommons.org/licenses/by/4.0/}{Creative Common
Attribution 4.0 International}.


    % Add a bibliography block to the postdoc
    
    
    
\end{document}
