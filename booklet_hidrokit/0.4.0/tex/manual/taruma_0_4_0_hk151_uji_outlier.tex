\documentclass[11pt]{article}

    \usepackage[breakable]{tcolorbox}
    \usepackage{parskip} % Stop auto-indenting (to mimic markdown behaviour)
    
    \usepackage{iftex}
    \ifPDFTeX
    	\usepackage[T1]{fontenc}
    	\usepackage{mathpazo}
    \else
    	\usepackage{fontspec}
    \fi

    % Basic figure setup, for now with no caption control since it's done
    % automatically by Pandoc (which extracts ![](path) syntax from Markdown).
    \usepackage{graphicx}
    % Maintain compatibility with old templates. Remove in nbconvert 6.0
    \let\Oldincludegraphics\includegraphics
    % Ensure that by default, figures have no caption (until we provide a
    % proper Figure object with a Caption API and a way to capture that
    % in the conversion process - todo).
    \usepackage{caption}
    \DeclareCaptionFormat{nocaption}{}
    \captionsetup{format=nocaption,aboveskip=0pt,belowskip=0pt}

    \usepackage{float}
    \floatplacement{figure}{H} % forces figures to be placed at the correct location
    \usepackage{xcolor} % Allow colors to be defined
    \usepackage{enumerate} % Needed for markdown enumerations to work
    \usepackage{geometry} % Used to adjust the document margins
    \usepackage{amsmath} % Equations
    \usepackage{amssymb} % Equations
    \usepackage{textcomp} % defines textquotesingle
    % Hack from http://tex.stackexchange.com/a/47451/13684:
    \AtBeginDocument{%
        \def\PYZsq{\textquotesingle}% Upright quotes in Pygmentized code
    }
    \usepackage{upquote} % Upright quotes for verbatim code
    \usepackage{eurosym} % defines \euro
    \usepackage[mathletters]{ucs} % Extended unicode (utf-8) support
    \usepackage{fancyvrb} % verbatim replacement that allows latex
    \usepackage{grffile} % extends the file name processing of package graphics 
                         % to support a larger range
    \makeatletter % fix for old versions of grffile with XeLaTeX
    \@ifpackagelater{grffile}{2019/11/01}
    {
      % Do nothing on new versions
    }
    {
      \def\Gread@@xetex#1{%
        \IfFileExists{"\Gin@base".bb}%
        {\Gread@eps{\Gin@base.bb}}%
        {\Gread@@xetex@aux#1}%
      }
    }
    \makeatother
    \usepackage[Export]{adjustbox} % Used to constrain images to a maximum size
    \adjustboxset{max size={0.9\linewidth}{0.9\paperheight}}

    % The hyperref package gives us a pdf with properly built
    % internal navigation ('pdf bookmarks' for the table of contents,
    % internal cross-reference links, web links for URLs, etc.)
    \usepackage{hyperref}
    % The default LaTeX title has an obnoxious amount of whitespace. By default,
    % titling removes some of it. It also provides customization options.
    \usepackage{titling}
    \usepackage{longtable} % longtable support required by pandoc >1.10
    \usepackage{booktabs}  % table support for pandoc > 1.12.2
    \usepackage[inline]{enumitem} % IRkernel/repr support (it uses the enumerate* environment)
    \usepackage[normalem]{ulem} % ulem is needed to support strikethroughs (\sout)
                                % normalem makes italics be italics, not underlines
    \usepackage{mathrsfs}
    

    
    % Colors for the hyperref package
    \definecolor{urlcolor}{rgb}{0,.145,.698}
    \definecolor{linkcolor}{rgb}{.71,0.21,0.01}
    \definecolor{citecolor}{rgb}{.12,.54,.11}

    % ANSI colors
    \definecolor{ansi-black}{HTML}{3E424D}
    \definecolor{ansi-black-intense}{HTML}{282C36}
    \definecolor{ansi-red}{HTML}{E75C58}
    \definecolor{ansi-red-intense}{HTML}{B22B31}
    \definecolor{ansi-green}{HTML}{00A250}
    \definecolor{ansi-green-intense}{HTML}{007427}
    \definecolor{ansi-yellow}{HTML}{DDB62B}
    \definecolor{ansi-yellow-intense}{HTML}{B27D12}
    \definecolor{ansi-blue}{HTML}{208FFB}
    \definecolor{ansi-blue-intense}{HTML}{0065CA}
    \definecolor{ansi-magenta}{HTML}{D160C4}
    \definecolor{ansi-magenta-intense}{HTML}{A03196}
    \definecolor{ansi-cyan}{HTML}{60C6C8}
    \definecolor{ansi-cyan-intense}{HTML}{258F8F}
    \definecolor{ansi-white}{HTML}{C5C1B4}
    \definecolor{ansi-white-intense}{HTML}{A1A6B2}
    \definecolor{ansi-default-inverse-fg}{HTML}{FFFFFF}
    \definecolor{ansi-default-inverse-bg}{HTML}{000000}

    % common color for the border for error outputs.
    \definecolor{outerrorbackground}{HTML}{FFDFDF}

    % commands and environments needed by pandoc snippets
    % extracted from the output of `pandoc -s`
    \providecommand{\tightlist}{%
      \setlength{\itemsep}{0pt}\setlength{\parskip}{0pt}}
    \DefineVerbatimEnvironment{Highlighting}{Verbatim}{commandchars=\\\{\}}
    % Add ',fontsize=\small' for more characters per line
    \newenvironment{Shaded}{}{}
    \newcommand{\KeywordTok}[1]{\textcolor[rgb]{0.00,0.44,0.13}{\textbf{{#1}}}}
    \newcommand{\DataTypeTok}[1]{\textcolor[rgb]{0.56,0.13,0.00}{{#1}}}
    \newcommand{\DecValTok}[1]{\textcolor[rgb]{0.25,0.63,0.44}{{#1}}}
    \newcommand{\BaseNTok}[1]{\textcolor[rgb]{0.25,0.63,0.44}{{#1}}}
    \newcommand{\FloatTok}[1]{\textcolor[rgb]{0.25,0.63,0.44}{{#1}}}
    \newcommand{\CharTok}[1]{\textcolor[rgb]{0.25,0.44,0.63}{{#1}}}
    \newcommand{\StringTok}[1]{\textcolor[rgb]{0.25,0.44,0.63}{{#1}}}
    \newcommand{\CommentTok}[1]{\textcolor[rgb]{0.38,0.63,0.69}{\textit{{#1}}}}
    \newcommand{\OtherTok}[1]{\textcolor[rgb]{0.00,0.44,0.13}{{#1}}}
    \newcommand{\AlertTok}[1]{\textcolor[rgb]{1.00,0.00,0.00}{\textbf{{#1}}}}
    \newcommand{\FunctionTok}[1]{\textcolor[rgb]{0.02,0.16,0.49}{{#1}}}
    \newcommand{\RegionMarkerTok}[1]{{#1}}
    \newcommand{\ErrorTok}[1]{\textcolor[rgb]{1.00,0.00,0.00}{\textbf{{#1}}}}
    \newcommand{\NormalTok}[1]{{#1}}
    
    % Additional commands for more recent versions of Pandoc
    \newcommand{\ConstantTok}[1]{\textcolor[rgb]{0.53,0.00,0.00}{{#1}}}
    \newcommand{\SpecialCharTok}[1]{\textcolor[rgb]{0.25,0.44,0.63}{{#1}}}
    \newcommand{\VerbatimStringTok}[1]{\textcolor[rgb]{0.25,0.44,0.63}{{#1}}}
    \newcommand{\SpecialStringTok}[1]{\textcolor[rgb]{0.73,0.40,0.53}{{#1}}}
    \newcommand{\ImportTok}[1]{{#1}}
    \newcommand{\DocumentationTok}[1]{\textcolor[rgb]{0.73,0.13,0.13}{\textit{{#1}}}}
    \newcommand{\AnnotationTok}[1]{\textcolor[rgb]{0.38,0.63,0.69}{\textbf{\textit{{#1}}}}}
    \newcommand{\CommentVarTok}[1]{\textcolor[rgb]{0.38,0.63,0.69}{\textbf{\textit{{#1}}}}}
    \newcommand{\VariableTok}[1]{\textcolor[rgb]{0.10,0.09,0.49}{{#1}}}
    \newcommand{\ControlFlowTok}[1]{\textcolor[rgb]{0.00,0.44,0.13}{\textbf{{#1}}}}
    \newcommand{\OperatorTok}[1]{\textcolor[rgb]{0.40,0.40,0.40}{{#1}}}
    \newcommand{\BuiltInTok}[1]{{#1}}
    \newcommand{\ExtensionTok}[1]{{#1}}
    \newcommand{\PreprocessorTok}[1]{\textcolor[rgb]{0.74,0.48,0.00}{{#1}}}
    \newcommand{\AttributeTok}[1]{\textcolor[rgb]{0.49,0.56,0.16}{{#1}}}
    \newcommand{\InformationTok}[1]{\textcolor[rgb]{0.38,0.63,0.69}{\textbf{\textit{{#1}}}}}
    \newcommand{\WarningTok}[1]{\textcolor[rgb]{0.38,0.63,0.69}{\textbf{\textit{{#1}}}}}
    
    
    % Define a nice break command that doesn't care if a line doesn't already
    % exist.
    \def\br{\hspace*{\fill} \\* }
    % Math Jax compatibility definitions
    \def\gt{>}
    \def\lt{<}
    \let\Oldtex\TeX
    \let\Oldlatex\LaTeX
    \renewcommand{\TeX}{\textrm{\Oldtex}}
    \renewcommand{\LaTeX}{\textrm{\Oldlatex}}
    % Document parameters
    % Document title
    \title{taruma\_0\_4\_0\_hk151\_uji\_outlier}
    
    
    
    
    
% Pygments definitions
\makeatletter
\def\PY@reset{\let\PY@it=\relax \let\PY@bf=\relax%
    \let\PY@ul=\relax \let\PY@tc=\relax%
    \let\PY@bc=\relax \let\PY@ff=\relax}
\def\PY@tok#1{\csname PY@tok@#1\endcsname}
\def\PY@toks#1+{\ifx\relax#1\empty\else%
    \PY@tok{#1}\expandafter\PY@toks\fi}
\def\PY@do#1{\PY@bc{\PY@tc{\PY@ul{%
    \PY@it{\PY@bf{\PY@ff{#1}}}}}}}
\def\PY#1#2{\PY@reset\PY@toks#1+\relax+\PY@do{#2}}

\@namedef{PY@tok@w}{\def\PY@tc##1{\textcolor[rgb]{0.73,0.73,0.73}{##1}}}
\@namedef{PY@tok@c}{\let\PY@it=\textit\def\PY@tc##1{\textcolor[rgb]{0.24,0.48,0.48}{##1}}}
\@namedef{PY@tok@cp}{\def\PY@tc##1{\textcolor[rgb]{0.61,0.40,0.00}{##1}}}
\@namedef{PY@tok@k}{\let\PY@bf=\textbf\def\PY@tc##1{\textcolor[rgb]{0.00,0.50,0.00}{##1}}}
\@namedef{PY@tok@kp}{\def\PY@tc##1{\textcolor[rgb]{0.00,0.50,0.00}{##1}}}
\@namedef{PY@tok@kt}{\def\PY@tc##1{\textcolor[rgb]{0.69,0.00,0.25}{##1}}}
\@namedef{PY@tok@o}{\def\PY@tc##1{\textcolor[rgb]{0.40,0.40,0.40}{##1}}}
\@namedef{PY@tok@ow}{\let\PY@bf=\textbf\def\PY@tc##1{\textcolor[rgb]{0.67,0.13,1.00}{##1}}}
\@namedef{PY@tok@nb}{\def\PY@tc##1{\textcolor[rgb]{0.00,0.50,0.00}{##1}}}
\@namedef{PY@tok@nf}{\def\PY@tc##1{\textcolor[rgb]{0.00,0.00,1.00}{##1}}}
\@namedef{PY@tok@nc}{\let\PY@bf=\textbf\def\PY@tc##1{\textcolor[rgb]{0.00,0.00,1.00}{##1}}}
\@namedef{PY@tok@nn}{\let\PY@bf=\textbf\def\PY@tc##1{\textcolor[rgb]{0.00,0.00,1.00}{##1}}}
\@namedef{PY@tok@ne}{\let\PY@bf=\textbf\def\PY@tc##1{\textcolor[rgb]{0.80,0.25,0.22}{##1}}}
\@namedef{PY@tok@nv}{\def\PY@tc##1{\textcolor[rgb]{0.10,0.09,0.49}{##1}}}
\@namedef{PY@tok@no}{\def\PY@tc##1{\textcolor[rgb]{0.53,0.00,0.00}{##1}}}
\@namedef{PY@tok@nl}{\def\PY@tc##1{\textcolor[rgb]{0.46,0.46,0.00}{##1}}}
\@namedef{PY@tok@ni}{\let\PY@bf=\textbf\def\PY@tc##1{\textcolor[rgb]{0.44,0.44,0.44}{##1}}}
\@namedef{PY@tok@na}{\def\PY@tc##1{\textcolor[rgb]{0.41,0.47,0.13}{##1}}}
\@namedef{PY@tok@nt}{\let\PY@bf=\textbf\def\PY@tc##1{\textcolor[rgb]{0.00,0.50,0.00}{##1}}}
\@namedef{PY@tok@nd}{\def\PY@tc##1{\textcolor[rgb]{0.67,0.13,1.00}{##1}}}
\@namedef{PY@tok@s}{\def\PY@tc##1{\textcolor[rgb]{0.73,0.13,0.13}{##1}}}
\@namedef{PY@tok@sd}{\let\PY@it=\textit\def\PY@tc##1{\textcolor[rgb]{0.73,0.13,0.13}{##1}}}
\@namedef{PY@tok@si}{\let\PY@bf=\textbf\def\PY@tc##1{\textcolor[rgb]{0.64,0.35,0.47}{##1}}}
\@namedef{PY@tok@se}{\let\PY@bf=\textbf\def\PY@tc##1{\textcolor[rgb]{0.67,0.36,0.12}{##1}}}
\@namedef{PY@tok@sr}{\def\PY@tc##1{\textcolor[rgb]{0.64,0.35,0.47}{##1}}}
\@namedef{PY@tok@ss}{\def\PY@tc##1{\textcolor[rgb]{0.10,0.09,0.49}{##1}}}
\@namedef{PY@tok@sx}{\def\PY@tc##1{\textcolor[rgb]{0.00,0.50,0.00}{##1}}}
\@namedef{PY@tok@m}{\def\PY@tc##1{\textcolor[rgb]{0.40,0.40,0.40}{##1}}}
\@namedef{PY@tok@gh}{\let\PY@bf=\textbf\def\PY@tc##1{\textcolor[rgb]{0.00,0.00,0.50}{##1}}}
\@namedef{PY@tok@gu}{\let\PY@bf=\textbf\def\PY@tc##1{\textcolor[rgb]{0.50,0.00,0.50}{##1}}}
\@namedef{PY@tok@gd}{\def\PY@tc##1{\textcolor[rgb]{0.63,0.00,0.00}{##1}}}
\@namedef{PY@tok@gi}{\def\PY@tc##1{\textcolor[rgb]{0.00,0.52,0.00}{##1}}}
\@namedef{PY@tok@gr}{\def\PY@tc##1{\textcolor[rgb]{0.89,0.00,0.00}{##1}}}
\@namedef{PY@tok@ge}{\let\PY@it=\textit}
\@namedef{PY@tok@gs}{\let\PY@bf=\textbf}
\@namedef{PY@tok@gp}{\let\PY@bf=\textbf\def\PY@tc##1{\textcolor[rgb]{0.00,0.00,0.50}{##1}}}
\@namedef{PY@tok@go}{\def\PY@tc##1{\textcolor[rgb]{0.44,0.44,0.44}{##1}}}
\@namedef{PY@tok@gt}{\def\PY@tc##1{\textcolor[rgb]{0.00,0.27,0.87}{##1}}}
\@namedef{PY@tok@err}{\def\PY@bc##1{{\setlength{\fboxsep}{\string -\fboxrule}\fcolorbox[rgb]{1.00,0.00,0.00}{1,1,1}{\strut ##1}}}}
\@namedef{PY@tok@kc}{\let\PY@bf=\textbf\def\PY@tc##1{\textcolor[rgb]{0.00,0.50,0.00}{##1}}}
\@namedef{PY@tok@kd}{\let\PY@bf=\textbf\def\PY@tc##1{\textcolor[rgb]{0.00,0.50,0.00}{##1}}}
\@namedef{PY@tok@kn}{\let\PY@bf=\textbf\def\PY@tc##1{\textcolor[rgb]{0.00,0.50,0.00}{##1}}}
\@namedef{PY@tok@kr}{\let\PY@bf=\textbf\def\PY@tc##1{\textcolor[rgb]{0.00,0.50,0.00}{##1}}}
\@namedef{PY@tok@bp}{\def\PY@tc##1{\textcolor[rgb]{0.00,0.50,0.00}{##1}}}
\@namedef{PY@tok@fm}{\def\PY@tc##1{\textcolor[rgb]{0.00,0.00,1.00}{##1}}}
\@namedef{PY@tok@vc}{\def\PY@tc##1{\textcolor[rgb]{0.10,0.09,0.49}{##1}}}
\@namedef{PY@tok@vg}{\def\PY@tc##1{\textcolor[rgb]{0.10,0.09,0.49}{##1}}}
\@namedef{PY@tok@vi}{\def\PY@tc##1{\textcolor[rgb]{0.10,0.09,0.49}{##1}}}
\@namedef{PY@tok@vm}{\def\PY@tc##1{\textcolor[rgb]{0.10,0.09,0.49}{##1}}}
\@namedef{PY@tok@sa}{\def\PY@tc##1{\textcolor[rgb]{0.73,0.13,0.13}{##1}}}
\@namedef{PY@tok@sb}{\def\PY@tc##1{\textcolor[rgb]{0.73,0.13,0.13}{##1}}}
\@namedef{PY@tok@sc}{\def\PY@tc##1{\textcolor[rgb]{0.73,0.13,0.13}{##1}}}
\@namedef{PY@tok@dl}{\def\PY@tc##1{\textcolor[rgb]{0.73,0.13,0.13}{##1}}}
\@namedef{PY@tok@s2}{\def\PY@tc##1{\textcolor[rgb]{0.73,0.13,0.13}{##1}}}
\@namedef{PY@tok@sh}{\def\PY@tc##1{\textcolor[rgb]{0.73,0.13,0.13}{##1}}}
\@namedef{PY@tok@s1}{\def\PY@tc##1{\textcolor[rgb]{0.73,0.13,0.13}{##1}}}
\@namedef{PY@tok@mb}{\def\PY@tc##1{\textcolor[rgb]{0.40,0.40,0.40}{##1}}}
\@namedef{PY@tok@mf}{\def\PY@tc##1{\textcolor[rgb]{0.40,0.40,0.40}{##1}}}
\@namedef{PY@tok@mh}{\def\PY@tc##1{\textcolor[rgb]{0.40,0.40,0.40}{##1}}}
\@namedef{PY@tok@mi}{\def\PY@tc##1{\textcolor[rgb]{0.40,0.40,0.40}{##1}}}
\@namedef{PY@tok@il}{\def\PY@tc##1{\textcolor[rgb]{0.40,0.40,0.40}{##1}}}
\@namedef{PY@tok@mo}{\def\PY@tc##1{\textcolor[rgb]{0.40,0.40,0.40}{##1}}}
\@namedef{PY@tok@ch}{\let\PY@it=\textit\def\PY@tc##1{\textcolor[rgb]{0.24,0.48,0.48}{##1}}}
\@namedef{PY@tok@cm}{\let\PY@it=\textit\def\PY@tc##1{\textcolor[rgb]{0.24,0.48,0.48}{##1}}}
\@namedef{PY@tok@cpf}{\let\PY@it=\textit\def\PY@tc##1{\textcolor[rgb]{0.24,0.48,0.48}{##1}}}
\@namedef{PY@tok@c1}{\let\PY@it=\textit\def\PY@tc##1{\textcolor[rgb]{0.24,0.48,0.48}{##1}}}
\@namedef{PY@tok@cs}{\let\PY@it=\textit\def\PY@tc##1{\textcolor[rgb]{0.24,0.48,0.48}{##1}}}

\def\PYZbs{\char`\\}
\def\PYZus{\char`\_}
\def\PYZob{\char`\{}
\def\PYZcb{\char`\}}
\def\PYZca{\char`\^}
\def\PYZam{\char`\&}
\def\PYZlt{\char`\<}
\def\PYZgt{\char`\>}
\def\PYZsh{\char`\#}
\def\PYZpc{\char`\%}
\def\PYZdl{\char`\$}
\def\PYZhy{\char`\-}
\def\PYZsq{\char`\'}
\def\PYZdq{\char`\"}
\def\PYZti{\char`\~}
% for compatibility with earlier versions
\def\PYZat{@}
\def\PYZlb{[}
\def\PYZrb{]}
\makeatother


    % For linebreaks inside Verbatim environment from package fancyvrb. 
    \makeatletter
        \newbox\Wrappedcontinuationbox 
        \newbox\Wrappedvisiblespacebox 
        \newcommand*\Wrappedvisiblespace {\textcolor{red}{\textvisiblespace}} 
        \newcommand*\Wrappedcontinuationsymbol {\textcolor{red}{\llap{\tiny$\m@th\hookrightarrow$}}} 
        \newcommand*\Wrappedcontinuationindent {3ex } 
        \newcommand*\Wrappedafterbreak {\kern\Wrappedcontinuationindent\copy\Wrappedcontinuationbox} 
        % Take advantage of the already applied Pygments mark-up to insert 
        % potential linebreaks for TeX processing. 
        %        {, <, #, %, $, ' and ": go to next line. 
        %        _, }, ^, &, >, - and ~: stay at end of broken line. 
        % Use of \textquotesingle for straight quote. 
        \newcommand*\Wrappedbreaksatspecials {% 
            \def\PYGZus{\discretionary{\char`\_}{\Wrappedafterbreak}{\char`\_}}% 
            \def\PYGZob{\discretionary{}{\Wrappedafterbreak\char`\{}{\char`\{}}% 
            \def\PYGZcb{\discretionary{\char`\}}{\Wrappedafterbreak}{\char`\}}}% 
            \def\PYGZca{\discretionary{\char`\^}{\Wrappedafterbreak}{\char`\^}}% 
            \def\PYGZam{\discretionary{\char`\&}{\Wrappedafterbreak}{\char`\&}}% 
            \def\PYGZlt{\discretionary{}{\Wrappedafterbreak\char`\<}{\char`\<}}% 
            \def\PYGZgt{\discretionary{\char`\>}{\Wrappedafterbreak}{\char`\>}}% 
            \def\PYGZsh{\discretionary{}{\Wrappedafterbreak\char`\#}{\char`\#}}% 
            \def\PYGZpc{\discretionary{}{\Wrappedafterbreak\char`\%}{\char`\%}}% 
            \def\PYGZdl{\discretionary{}{\Wrappedafterbreak\char`\$}{\char`\$}}% 
            \def\PYGZhy{\discretionary{\char`\-}{\Wrappedafterbreak}{\char`\-}}% 
            \def\PYGZsq{\discretionary{}{\Wrappedafterbreak\textquotesingle}{\textquotesingle}}% 
            \def\PYGZdq{\discretionary{}{\Wrappedafterbreak\char`\"}{\char`\"}}% 
            \def\PYGZti{\discretionary{\char`\~}{\Wrappedafterbreak}{\char`\~}}% 
        } 
        % Some characters . , ; ? ! / are not pygmentized. 
        % This macro makes them "active" and they will insert potential linebreaks 
        \newcommand*\Wrappedbreaksatpunct {% 
            \lccode`\~`\.\lowercase{\def~}{\discretionary{\hbox{\char`\.}}{\Wrappedafterbreak}{\hbox{\char`\.}}}% 
            \lccode`\~`\,\lowercase{\def~}{\discretionary{\hbox{\char`\,}}{\Wrappedafterbreak}{\hbox{\char`\,}}}% 
            \lccode`\~`\;\lowercase{\def~}{\discretionary{\hbox{\char`\;}}{\Wrappedafterbreak}{\hbox{\char`\;}}}% 
            \lccode`\~`\:\lowercase{\def~}{\discretionary{\hbox{\char`\:}}{\Wrappedafterbreak}{\hbox{\char`\:}}}% 
            \lccode`\~`\?\lowercase{\def~}{\discretionary{\hbox{\char`\?}}{\Wrappedafterbreak}{\hbox{\char`\?}}}% 
            \lccode`\~`\!\lowercase{\def~}{\discretionary{\hbox{\char`\!}}{\Wrappedafterbreak}{\hbox{\char`\!}}}% 
            \lccode`\~`\/\lowercase{\def~}{\discretionary{\hbox{\char`\/}}{\Wrappedafterbreak}{\hbox{\char`\/}}}% 
            \catcode`\.\active
            \catcode`\,\active 
            \catcode`\;\active
            \catcode`\:\active
            \catcode`\?\active
            \catcode`\!\active
            \catcode`\/\active 
            \lccode`\~`\~ 	
        }
    \makeatother

    \let\OriginalVerbatim=\Verbatim
    \makeatletter
    \renewcommand{\Verbatim}[1][1]{%
        %\parskip\z@skip
        \sbox\Wrappedcontinuationbox {\Wrappedcontinuationsymbol}%
        \sbox\Wrappedvisiblespacebox {\FV@SetupFont\Wrappedvisiblespace}%
        \def\FancyVerbFormatLine ##1{\hsize\linewidth
            \vtop{\raggedright\hyphenpenalty\z@\exhyphenpenalty\z@
                \doublehyphendemerits\z@\finalhyphendemerits\z@
                \strut ##1\strut}%
        }%
        % If the linebreak is at a space, the latter will be displayed as visible
        % space at end of first line, and a continuation symbol starts next line.
        % Stretch/shrink are however usually zero for typewriter font.
        \def\FV@Space {%
            \nobreak\hskip\z@ plus\fontdimen3\font minus\fontdimen4\font
            \discretionary{\copy\Wrappedvisiblespacebox}{\Wrappedafterbreak}
            {\kern\fontdimen2\font}%
        }%
        
        % Allow breaks at special characters using \PYG... macros.
        \Wrappedbreaksatspecials
        % Breaks at punctuation characters . , ; ? ! and / need catcode=\active 	
        \OriginalVerbatim[#1,codes*=\Wrappedbreaksatpunct]%
    }
    \makeatother

    % Exact colors from NB
    \definecolor{incolor}{HTML}{303F9F}
    \definecolor{outcolor}{HTML}{D84315}
    \definecolor{cellborder}{HTML}{CFCFCF}
    \definecolor{cellbackground}{HTML}{F7F7F7}
    
    % prompt
    \makeatletter
    \newcommand{\boxspacing}{\kern\kvtcb@left@rule\kern\kvtcb@boxsep}
    \makeatother
    \newcommand{\prompt}[4]{
        {\ttfamily\llap{{\color{#2}[#3]:\hspace{3pt}#4}}\vspace{-\baselineskip}}
    }
    

    
    % Prevent overflowing lines due to hard-to-break entities
    \sloppy 
    % Setup hyperref package
    \hypersetup{
      breaklinks=true,  % so long urls are correctly broken across lines
      colorlinks=true,
      urlcolor=urlcolor,
      linkcolor=linkcolor,
      citecolor=citecolor,
      }
    % Slightly bigger margins than the latex defaults
    
    \geometry{verbose,tmargin=1in,bmargin=1in,lmargin=1in,rmargin=1in}
    
    

\begin{document}
    
    \maketitle
    
    

    
    Berdasarkan isu
\href{https://github.com/hidrokit/hidrokit/issues/151}{\#151}:
\textbf{Uji Outlier}

Referensi Isu: - Te Chow, V. (2010). Applied Hydrology. Tata McGraw-Hill
Education. https://books.google.co.id/books?id=RRwidSsBJrEC p403-405. -
Limantara, Lily M. (2018): Rekayasa Hidrologi, Edisi Revisi. Penerbit
Andi Offset, Yogyakarta. (hal. 89-93)

Deksripsi Isu: - Mencari nilai outlier pada data

Strategi: - Buat tabel lampiran yang digunakan untuk mencari nilai
\(K_n\). - Tabel \texttt{rel\_Kn\_n}, hubungan \(K_n\) dengan jumlah
data \(N\). - Membuat fungsi membaca data kemudian mengeluarkan nilai
batas bawah dan atas - Periksa juga apakah data memiliki outlier atau
tidak. Jika iya, mungkin dikeluarkan bagian mana saja yang memiliki
outlier.

    \hypertarget{persiapan-dan-dataset}{%
\section{PERSIAPAN DAN DATASET}\label{persiapan-dan-dataset}}

    \begin{tcolorbox}[breakable, size=fbox, boxrule=1pt, pad at break*=1mm,colback=cellbackground, colframe=cellborder]
\prompt{In}{incolor}{ }{\boxspacing}
\begin{Verbatim}[commandchars=\\\{\}]
\PY{k+kn}{import} \PY{n+nn}{numpy} \PY{k}{as} \PY{n+nn}{np}
\PY{k+kn}{import} \PY{n+nn}{pandas} \PY{k}{as} \PY{n+nn}{pd}
\end{Verbatim}
\end{tcolorbox}

    \begin{tcolorbox}[breakable, size=fbox, boxrule=1pt, pad at break*=1mm,colback=cellbackground, colframe=cellborder]
\prompt{In}{incolor}{ }{\boxspacing}
\begin{Verbatim}[commandchars=\\\{\}]
\PY{c+c1}{\PYZsh{} contoh diambil dari buku}
\PY{c+c1}{\PYZsh{} Limantara, Lily M. (2018): Rekayasa Hidrologi, Edisi Revisi. }
\PY{c+c1}{\PYZsh{} Penerbit Andi Offset, Yogyakarta. (hal. 90\PYZhy{}91)}

\PY{n}{\PYZus{}hujan} \PY{o}{=} \PY{n}{np}\PY{o}{.}\PY{n}{array}\PY{p}{(}\PY{p}{[}\PY{l+m+mi}{2818}\PY{p}{,} \PY{l+m+mi}{2542}\PY{p}{,} \PY{l+m+mi}{1949}\PY{p}{,} \PY{l+m+mi}{1842}\PY{p}{,} \PY{l+m+mi}{1748}\PY{p}{,} \PY{l+m+mi}{1737}\PY{p}{,} \PY{l+m+mi}{1605}\PY{p}{,} \PY{l+m+mi}{1558}\PY{p}{,} \PY{l+m+mi}{1433}\PY{p}{,} \PY{l+m+mi}{1264}\PY{p}{]}\PY{p}{)}
\PY{n}{\PYZus{}index} \PY{o}{=} \PY{n}{np}\PY{o}{.}\PY{n}{array}\PY{p}{(}\PY{p}{[}\PY{l+m+mi}{2010}\PY{p}{,} \PY{l+m+mi}{2013}\PY{p}{,} \PY{l+m+mi}{2008}\PY{p}{,} \PY{l+m+mi}{2012}\PY{p}{,} \PY{l+m+mi}{2011}\PY{p}{,} \PY{l+m+mi}{2014}\PY{p}{,} \PY{l+m+mi}{2009}\PY{p}{,} \PY{l+m+mi}{2007}\PY{p}{,} \PY{l+m+mi}{2015}\PY{p}{,} \PY{l+m+mi}{2006}\PY{p}{]}\PY{p}{)}
\PY{n}{data} \PY{o}{=} \PY{n}{pd}\PY{o}{.}\PY{n}{DataFrame}\PY{p}{(}
    \PY{n}{data}\PY{o}{=}\PY{n}{np}\PY{o}{.}\PY{n}{stack}\PY{p}{(}\PY{p}{[}\PY{n}{\PYZus{}index}\PY{p}{,} \PY{n}{\PYZus{}hujan}\PY{p}{]}\PY{p}{,} \PY{n}{axis}\PY{o}{=}\PY{l+m+mi}{1}\PY{p}{)}\PY{p}{,} 
    \PY{n}{columns}\PY{o}{=}\PY{p}{[}\PY{l+s+s1}{\PYZsq{}}\PY{l+s+s1}{Tahun}\PY{l+s+s1}{\PYZsq{}}\PY{p}{,}\PY{l+s+s1}{\PYZsq{}}\PY{l+s+s1}{Hujan}\PY{l+s+s1}{\PYZsq{}}\PY{p}{]}
\PY{p}{)}

\PY{c+c1}{\PYZsh{} ubah kolom tahun jadi datetime, dan index}
\PY{n}{data}\PY{o}{.}\PY{n}{Tahun} \PY{o}{=} \PY{n}{pd}\PY{o}{.}\PY{n}{to\PYZus{}datetime}\PY{p}{(}\PY{n}{data}\PY{o}{.}\PY{n}{Tahun}\PY{p}{,} \PY{n+nb}{format}\PY{o}{=}\PY{l+s+s2}{\PYZdq{}}\PY{l+s+s2}{\PYZpc{}}\PY{l+s+s2}{Y}\PY{l+s+s2}{\PYZdq{}}\PY{p}{)}
\PY{n}{data}\PY{o}{.}\PY{n}{set\PYZus{}index}\PY{p}{(}\PY{l+s+s1}{\PYZsq{}}\PY{l+s+s1}{Tahun}\PY{l+s+s1}{\PYZsq{}}\PY{p}{,} \PY{n}{inplace}\PY{o}{=}\PY{k+kc}{True}\PY{p}{)}
\PY{n}{data}
\end{Verbatim}
\end{tcolorbox}

            \begin{tcolorbox}[breakable, size=fbox, boxrule=.5pt, pad at break*=1mm, opacityfill=0]
\prompt{Out}{outcolor}{ }{\boxspacing}
\begin{Verbatim}[commandchars=\\\{\}]
            Hujan
Tahun
2010-01-01   2818
2013-01-01   2542
2008-01-01   1949
2012-01-01   1842
2011-01-01   1748
2014-01-01   1737
2009-01-01   1605
2007-01-01   1558
2015-01-01   1433
2006-01-01   1264
\end{Verbatim}
\end{tcolorbox}
        
    \hypertarget{tabel}{%
\section{TABEL}\label{tabel}}

Nilai tabel mengikuti referensi buku. - Tabel \texttt{t\_rel\_Kn\_n}
diambil pada buku Te Chow, V. (2010). Applied Hydrology. p404.

Tabel yang digunakan dalam perhitungan dibangkitkan dengan kode dibawah
ini:

    \begin{tcolorbox}[breakable, size=fbox, boxrule=1pt, pad at break*=1mm,colback=cellbackground, colframe=cellborder]
\prompt{In}{incolor}{ }{\boxspacing}
\begin{Verbatim}[commandchars=\\\{\}]
\PY{n}{\PYZus{}N} \PY{o}{=} \PY{n}{np}\PY{o}{.}\PY{n}{array}\PY{p}{(}
    \PY{p}{[}\PY{l+m+mi}{10}\PY{p}{,} \PY{l+m+mi}{11}\PY{p}{,} \PY{l+m+mi}{12}\PY{p}{,} \PY{l+m+mi}{13}\PY{p}{,} \PY{l+m+mi}{14}\PY{p}{,} \PY{l+m+mi}{15}\PY{p}{,} \PY{l+m+mi}{16}\PY{p}{,} \PY{l+m+mi}{17}\PY{p}{,} \PY{l+m+mi}{18}\PY{p}{,} \PY{l+m+mi}{19}\PY{p}{,} \PY{l+m+mi}{20}\PY{p}{,} \PY{l+m+mi}{21}\PY{p}{,} \PY{l+m+mi}{22}\PY{p}{,} \PY{l+m+mi}{23}\PY{p}{,} \PY{l+m+mi}{24}\PY{p}{,} \PY{l+m+mi}{25}\PY{p}{,} \PY{l+m+mi}{26}\PY{p}{,} \PY{l+m+mi}{27}\PY{p}{,} 
     \PY{l+m+mi}{28}\PY{p}{,} \PY{l+m+mi}{29}\PY{p}{,} \PY{l+m+mi}{30}\PY{p}{,} \PY{l+m+mi}{31}\PY{p}{,} \PY{l+m+mi}{32}\PY{p}{,} \PY{l+m+mi}{33}\PY{p}{,} \PY{l+m+mi}{34}\PY{p}{,} \PY{l+m+mi}{35}\PY{p}{,} \PY{l+m+mi}{36}\PY{p}{,} \PY{l+m+mi}{37}\PY{p}{,} \PY{l+m+mi}{38}\PY{p}{,} \PY{l+m+mi}{39}\PY{p}{,} \PY{l+m+mi}{40}\PY{p}{,} \PY{l+m+mi}{41}\PY{p}{,} \PY{l+m+mi}{42}\PY{p}{,} \PY{l+m+mi}{43}\PY{p}{,} \PY{l+m+mi}{44}\PY{p}{,} \PY{l+m+mi}{45}\PY{p}{,} 
     \PY{l+m+mi}{46}\PY{p}{,} \PY{l+m+mi}{47}\PY{p}{,} \PY{l+m+mi}{48}\PY{p}{,} \PY{l+m+mi}{49}\PY{p}{,} \PY{l+m+mi}{50}\PY{p}{,} \PY{l+m+mi}{55}\PY{p}{,} \PY{l+m+mi}{60}\PY{p}{,} \PY{l+m+mi}{65}\PY{p}{,} \PY{l+m+mi}{70}\PY{p}{,} \PY{l+m+mi}{75}\PY{p}{,} \PY{l+m+mi}{80}\PY{p}{,} \PY{l+m+mi}{85}\PY{p}{,} \PY{l+m+mi}{90}\PY{p}{,} \PY{l+m+mi}{95}\PY{p}{,} \PY{l+m+mi}{100}\PY{p}{,} \PY{l+m+mi}{110}\PY{p}{,} \PY{l+m+mi}{120}\PY{p}{,} 
     \PY{l+m+mi}{130}\PY{p}{,} \PY{l+m+mi}{140}\PY{p}{]}
\PY{p}{)}

\PY{n}{\PYZus{}Kn} \PY{o}{=} \PY{n}{np}\PY{o}{.}\PY{n}{array}\PY{p}{(}
    \PY{p}{[}\PY{l+m+mf}{2.036}\PY{p}{,} \PY{l+m+mf}{2.088}\PY{p}{,} \PY{l+m+mf}{2.134}\PY{p}{,} \PY{l+m+mf}{2.175}\PY{p}{,} \PY{l+m+mf}{2.213}\PY{p}{,} \PY{l+m+mf}{2.247}\PY{p}{,} \PY{l+m+mf}{2.279}\PY{p}{,} \PY{l+m+mf}{2.309}\PY{p}{,} \PY{l+m+mf}{2.335}\PY{p}{,} \PY{l+m+mf}{2.361}\PY{p}{,} 
     \PY{l+m+mf}{2.385}\PY{p}{,} \PY{l+m+mf}{2.408}\PY{p}{,} \PY{l+m+mf}{2.429}\PY{p}{,} \PY{l+m+mf}{2.448}\PY{p}{,} \PY{l+m+mf}{2.467}\PY{p}{,} \PY{l+m+mf}{2.486}\PY{p}{,} \PY{l+m+mf}{2.502}\PY{p}{,} \PY{l+m+mf}{2.519}\PY{p}{,} \PY{l+m+mf}{2.534}\PY{p}{,} \PY{l+m+mf}{2.549}\PY{p}{,} 
     \PY{l+m+mf}{2.563}\PY{p}{,} \PY{l+m+mf}{2.577}\PY{p}{,} \PY{l+m+mf}{2.591}\PY{p}{,} \PY{l+m+mf}{2.604}\PY{p}{,} \PY{l+m+mf}{2.616}\PY{p}{,} \PY{l+m+mf}{2.628}\PY{p}{,} \PY{l+m+mf}{2.639}\PY{p}{,} \PY{l+m+mf}{2.65}\PY{p}{,} \PY{l+m+mf}{2.661}\PY{p}{,} \PY{l+m+mf}{2.671}\PY{p}{,} 
     \PY{l+m+mf}{2.682}\PY{p}{,} \PY{l+m+mf}{2.692}\PY{p}{,} \PY{l+m+mf}{2.7}\PY{p}{,} \PY{l+m+mf}{2.71}\PY{p}{,} \PY{l+m+mf}{2.719}\PY{p}{,} \PY{l+m+mf}{2.727}\PY{p}{,} \PY{l+m+mf}{2.736}\PY{p}{,} \PY{l+m+mf}{2.744}\PY{p}{,} \PY{l+m+mf}{2.753}\PY{p}{,} \PY{l+m+mf}{2.76}\PY{p}{,} 
     \PY{l+m+mf}{2.768}\PY{p}{,} \PY{l+m+mf}{2.804}\PY{p}{,} \PY{l+m+mf}{2.837}\PY{p}{,} \PY{l+m+mf}{2.866}\PY{p}{,} \PY{l+m+mf}{2.893}\PY{p}{,} \PY{l+m+mf}{2.917}\PY{p}{,} \PY{l+m+mf}{2.94}\PY{p}{,} \PY{l+m+mf}{2.961}\PY{p}{,} \PY{l+m+mf}{2.981}\PY{p}{,} \PY{l+m+mi}{3}\PY{p}{,} 
     \PY{l+m+mf}{3.017}\PY{p}{,} \PY{l+m+mf}{3.049}\PY{p}{,} \PY{l+m+mf}{3.078}\PY{p}{,} \PY{l+m+mf}{3.104}\PY{p}{,} \PY{l+m+mf}{3.129}\PY{p}{]}
\PY{p}{)}

\PY{n}{t\PYZus{}rel\PYZus{}Kn\PYZus{}n} \PY{o}{=} \PY{n}{pd}\PY{o}{.}\PY{n}{DataFrame}\PY{p}{(}\PY{n}{np}\PY{o}{.}\PY{n}{stack}\PY{p}{(}\PY{p}{(}\PY{n}{\PYZus{}N}\PY{p}{,} \PY{n}{\PYZus{}Kn}\PY{p}{)}\PY{p}{,} \PY{n}{axis}\PY{o}{=}\PY{l+m+mi}{1}\PY{p}{)}\PY{p}{,} \PY{n}{columns}\PY{o}{=}\PY{p}{[}\PY{l+s+s1}{\PYZsq{}}\PY{l+s+s1}{N}\PY{l+s+s1}{\PYZsq{}}\PY{p}{,} \PY{l+s+s1}{\PYZsq{}}\PY{l+s+s1}{Kn}\PY{l+s+s1}{\PYZsq{}}\PY{p}{]}\PY{p}{)}
\PY{n}{t\PYZus{}rel\PYZus{}Kn\PYZus{}n}\PY{o}{.}\PY{n}{head}\PY{p}{(}\PY{p}{)}
\end{Verbatim}
\end{tcolorbox}

            \begin{tcolorbox}[breakable, size=fbox, boxrule=.5pt, pad at break*=1mm, opacityfill=0]
\prompt{Out}{outcolor}{ }{\boxspacing}
\begin{Verbatim}[commandchars=\\\{\}]
      N     Kn
0  10.0  2.036
1  11.0  2.088
2  12.0  2.134
3  13.0  2.175
4  14.0  2.213
\end{Verbatim}
\end{tcolorbox}
        
    \hypertarget{kode}{%
\section{KODE}\label{kode}}

    \begin{tcolorbox}[breakable, size=fbox, boxrule=1pt, pad at break*=1mm,colback=cellbackground, colframe=cellborder]
\prompt{In}{incolor}{ }{\boxspacing}
\begin{Verbatim}[commandchars=\\\{\}]
\PY{k}{def} \PY{n+nf}{find\PYZus{}Kn}\PY{p}{(}\PY{n}{n}\PY{p}{,} \PY{n}{table}\PY{o}{=}\PY{n}{t\PYZus{}rel\PYZus{}Kn\PYZus{}n}\PY{p}{)}\PY{p}{:}
    \PY{k}{if} \PY{n}{n} \PY{o}{\PYZlt{}} \PY{l+m+mi}{10} \PY{o+ow}{or} \PY{n}{n} \PY{o}{\PYZgt{}} \PY{l+m+mi}{140}\PY{p}{:}
        \PY{k}{raise} \PY{n+ne}{ValueError}\PY{p}{(}\PY{l+s+s1}{\PYZsq{}}\PY{l+s+s1}{Jumlah data diluar batas bawah (10) / batas atas (140)}\PY{l+s+s1}{\PYZsq{}}\PY{p}{)}
    \PY{k}{else}\PY{p}{:}
        \PY{n}{N} \PY{o}{=} \PY{n}{table}\PY{p}{[}\PY{l+s+s1}{\PYZsq{}}\PY{l+s+s1}{N}\PY{l+s+s1}{\PYZsq{}}\PY{p}{]}\PY{o}{.}\PY{n}{to\PYZus{}numpy}\PY{p}{(}\PY{p}{)}
        \PY{n}{Kn} \PY{o}{=} \PY{n}{table}\PY{p}{[}\PY{l+s+s1}{\PYZsq{}}\PY{l+s+s1}{Kn}\PY{l+s+s1}{\PYZsq{}}\PY{p}{]}\PY{o}{.}\PY{n}{to\PYZus{}numpy}\PY{p}{(}\PY{p}{)}
        \PY{k}{return} \PY{n}{np}\PY{o}{.}\PY{n}{interp}\PY{p}{(}\PY{n}{n}\PY{p}{,} \PY{n}{N}\PY{p}{,} \PY{n}{Kn}\PY{p}{)}

\PY{k}{def} \PY{n+nf}{calc\PYZus{}boundary}\PY{p}{(}\PY{n}{df}\PY{p}{,} \PY{n}{col}\PY{o}{=}\PY{k+kc}{None}\PY{p}{,} \PY{n}{result}\PY{o}{=}\PY{l+s+s1}{\PYZsq{}}\PY{l+s+s1}{value}\PY{l+s+s1}{\PYZsq{}}\PY{p}{,} \PY{n}{show\PYZus{}stat}\PY{o}{=}\PY{k+kc}{False}\PY{p}{)}\PY{p}{:}
    \PY{n}{col} \PY{o}{=} \PY{n}{df}\PY{o}{.}\PY{n}{columns}\PY{p}{[}\PY{l+m+mi}{0}\PY{p}{]} \PY{k}{if} \PY{n}{col} \PY{o+ow}{is} \PY{k+kc}{None} \PY{k}{else} \PY{n}{col}
    
    \PY{n}{x} \PY{o}{=} \PY{n}{df}\PY{p}{[}\PY{n}{col}\PY{p}{]}\PY{o}{.}\PY{n}{to\PYZus{}numpy}\PY{p}{(}\PY{p}{)}
    \PY{n}{n} \PY{o}{=} \PY{n}{x}\PY{o}{.}\PY{n}{size}
    \PY{n}{xlog} \PY{o}{=} \PY{n}{np}\PY{o}{.}\PY{n}{log10}\PY{p}{(}\PY{n}{x}\PY{p}{)}
    \PY{n}{xlogmean} \PY{o}{=} \PY{n}{xlog}\PY{o}{.}\PY{n}{mean}\PY{p}{(}\PY{p}{)}
    \PY{n}{xlogstd} \PY{o}{=} \PY{n}{xlog}\PY{o}{.}\PY{n}{std}\PY{p}{(}\PY{n}{ddof}\PY{o}{=}\PY{l+m+mi}{1}\PY{p}{)}

    \PY{n}{Kn} \PY{o}{=} \PY{n}{find\PYZus{}Kn}\PY{p}{(}\PY{n}{n}\PY{p}{)}

    \PY{c+c1}{\PYZsh{} higher}
    \PY{n}{y\PYZus{}h} \PY{o}{=} \PY{n}{xlogmean} \PY{o}{+} \PY{n}{Kn}\PY{o}{*}\PY{n}{xlogstd}
    \PY{n}{val\PYZus{}h} \PY{o}{=} \PY{l+m+mi}{10}\PY{o}{*}\PY{o}{*}\PY{n}{y\PYZus{}h}

    \PY{c+c1}{\PYZsh{} lower}
    \PY{n}{y\PYZus{}l} \PY{o}{=} \PY{n}{xlogmean} \PY{o}{\PYZhy{}} \PY{n}{Kn}\PY{o}{*}\PY{n}{xlogstd}
    \PY{n}{val\PYZus{}l} \PY{o}{=} \PY{l+m+mi}{10}\PY{o}{*}\PY{o}{*}\PY{n}{y\PYZus{}l}

    \PY{k}{if} \PY{n}{show\PYZus{}stat}\PY{p}{:}
        \PY{n+nb}{print}\PY{p}{(}
            \PY{l+s+sa}{f}\PY{l+s+s1}{\PYZsq{}}\PY{l+s+s1}{Statistik:}\PY{l+s+s1}{\PYZsq{}}\PY{p}{,}
            \PY{l+s+sa}{f}\PY{l+s+s1}{\PYZsq{}}\PY{l+s+s1}{N = }\PY{l+s+si}{\PYZob{}}\PY{n}{n}\PY{l+s+si}{\PYZcb{}}\PY{l+s+s1}{\PYZsq{}}\PY{p}{,}
            \PY{l+s+sa}{f}\PY{l+s+s1}{\PYZsq{}}\PY{l+s+s1}{Mean (log) = }\PY{l+s+si}{\PYZob{}}\PY{n}{xlogmean}\PY{l+s+si}{:}\PY{l+s+s1}{.5f}\PY{l+s+si}{\PYZcb{}}\PY{l+s+s1}{\PYZsq{}}\PY{p}{,}
            \PY{l+s+sa}{f}\PY{l+s+s1}{\PYZsq{}}\PY{l+s+s1}{Std (log) = }\PY{l+s+si}{\PYZob{}}\PY{n}{xlogstd}\PY{l+s+si}{:}\PY{l+s+s1}{.5f}\PY{l+s+si}{\PYZcb{}}\PY{l+s+s1}{\PYZsq{}}\PY{p}{,}
            \PY{l+s+sa}{f}\PY{l+s+s1}{\PYZsq{}}\PY{l+s+s1}{Lower (val) = }\PY{l+s+si}{\PYZob{}}\PY{n}{val\PYZus{}l}\PY{l+s+si}{:}\PY{l+s+s1}{.5f}\PY{l+s+si}{\PYZcb{}}\PY{l+s+s1}{\PYZsq{}}\PY{p}{,}
            \PY{l+s+sa}{f}\PY{l+s+s1}{\PYZsq{}}\PY{l+s+s1}{Higher (val) = }\PY{l+s+si}{\PYZob{}}\PY{n}{val\PYZus{}h}\PY{l+s+si}{:}\PY{l+s+s1}{.5f}\PY{l+s+si}{\PYZcb{}}\PY{l+s+s1}{\PYZsq{}}\PY{p}{,}
            \PY{n}{sep}\PY{o}{=}\PY{l+s+s1}{\PYZsq{}}\PY{l+s+se}{\PYZbs{}n}\PY{l+s+s1}{\PYZsq{}}\PY{p}{,} \PY{n}{end}\PY{o}{=}\PY{l+s+s1}{\PYZsq{}}\PY{l+s+se}{\PYZbs{}n}\PY{l+s+se}{\PYZbs{}n}\PY{l+s+s1}{\PYZsq{}}
        \PY{p}{)}

    \PY{k}{if} \PY{n}{result}\PY{o}{.}\PY{n}{lower}\PY{p}{(}\PY{p}{)} \PY{o}{==} \PY{l+s+s1}{\PYZsq{}}\PY{l+s+s1}{value}\PY{l+s+s1}{\PYZsq{}}\PY{p}{:}
        \PY{k}{return} \PY{p}{(}\PY{n}{val\PYZus{}l}\PY{p}{,} \PY{n}{val\PYZus{}h}\PY{p}{)}
    \PY{k}{elif} \PY{n}{result}\PY{o}{.}\PY{n}{lower}\PY{p}{(}\PY{p}{)} \PY{o}{==} \PY{l+s+s1}{\PYZsq{}}\PY{l+s+s1}{log}\PY{l+s+s1}{\PYZsq{}}\PY{p}{:}
        \PY{k}{return} \PY{p}{(}\PY{n}{y\PYZus{}l}\PY{p}{,} \PY{n}{y\PYZus{}h}\PY{p}{)}

\PY{k}{def} \PY{n+nf}{find\PYZus{}outlier}\PY{p}{(}\PY{n}{df}\PY{p}{,} \PY{n}{col}\PY{o}{=}\PY{k+kc}{None}\PY{p}{,} \PY{n}{verbose}\PY{o}{=}\PY{k+kc}{False}\PY{p}{,} \PY{o}{*}\PY{o}{*}\PY{n}{kwargs}\PY{p}{)}\PY{p}{:}
    
    \PY{n}{low}\PY{p}{,} \PY{n}{high} \PY{o}{=} \PY{n}{calc\PYZus{}boundary}\PY{p}{(}\PY{n}{df}\PY{p}{,} \PY{n}{col}\PY{p}{,} \PY{o}{*}\PY{o}{*}\PY{n}{kwargs}\PY{p}{)}
    
    \PY{n}{col} \PY{o}{=} \PY{n}{df}\PY{o}{.}\PY{n}{columns}\PY{p}{[}\PY{l+m+mi}{0}\PY{p}{]} \PY{k}{if} \PY{n}{col} \PY{o+ow}{is} \PY{k+kc}{None} \PY{k}{else} \PY{n}{col}

    \PY{n}{masklow} \PY{o}{=} \PY{n}{df}\PY{p}{[}\PY{n}{col}\PY{p}{]} \PY{o}{\PYZlt{}} \PY{n}{low}
    \PY{n}{maskhigh} \PY{o}{=} \PY{n}{df}\PY{p}{[}\PY{n}{col}\PY{p}{]} \PY{o}{\PYZgt{}} \PY{n}{high}
    \PY{n}{mask} \PY{o}{=} \PY{n}{masklow} \PY{o}{|} \PY{n}{maskhigh}
    
    \PY{k}{if} \PY{n}{verbose} \PY{o+ow}{and} \PY{n}{masklow}\PY{o}{.}\PY{n}{sum}\PY{p}{(}\PY{p}{)}\PY{p}{:}
        \PY{n+nb}{print}\PY{p}{(}\PY{l+s+sa}{f}\PY{l+s+s1}{\PYZsq{}}\PY{l+s+s1}{Ada outlier dibawah batas bawah sebanyak }\PY{l+s+si}{\PYZob{}}\PY{n}{masklow}\PY{o}{.}\PY{n}{sum}\PY{p}{(}\PY{p}{)}\PY{l+s+si}{\PYZcb{}}\PY{l+s+s1}{.}\PY{l+s+s1}{\PYZsq{}}\PY{p}{)}
    \PY{k}{if} \PY{n}{verbose} \PY{o+ow}{and} \PY{n}{maskhigh}\PY{o}{.}\PY{n}{sum}\PY{p}{(}\PY{p}{)}\PY{p}{:}
        \PY{n+nb}{print}\PY{p}{(}\PY{l+s+sa}{f}\PY{l+s+s1}{\PYZsq{}}\PY{l+s+s1}{Ada outlier diatas batas atas sebanyak }\PY{l+s+si}{\PYZob{}}\PY{n}{maskhigh}\PY{o}{.}\PY{n}{sum}\PY{p}{(}\PY{p}{)}\PY{l+s+si}{\PYZcb{}}\PY{l+s+s1}{.}\PY{l+s+s1}{\PYZsq{}}\PY{p}{)}

    \PY{k}{def} \PY{n+nf}{check\PYZus{}outlier}\PY{p}{(}\PY{n}{x}\PY{p}{)}\PY{p}{:}
        \PY{k}{if} \PY{n}{x} \PY{o}{\PYZlt{}} \PY{n}{low}\PY{p}{:}
            \PY{k}{return} \PY{l+s+s2}{\PYZdq{}}\PY{l+s+s2}{lower}\PY{l+s+s2}{\PYZdq{}}
        \PY{k}{elif} \PY{n}{x} \PY{o}{\PYZgt{}} \PY{n}{high}\PY{p}{:}
            \PY{k}{return} \PY{l+s+s2}{\PYZdq{}}\PY{l+s+s2}{higher}\PY{l+s+s2}{\PYZdq{}}
        \PY{k}{else}\PY{p}{:}
            \PY{k}{return} \PY{n}{pd}\PY{o}{.}\PY{n}{NA}

    \PY{k}{if} \PY{n}{mask}\PY{o}{.}\PY{n}{sum}\PY{p}{(}\PY{p}{)} \PY{o}{!=} \PY{l+m+mi}{0}\PY{p}{:}
        \PY{n}{new\PYZus{}df} \PY{o}{=} \PY{n}{df}\PY{o}{.}\PY{n}{copy}\PY{p}{(}\PY{p}{)}
        \PY{n}{new\PYZus{}df}\PY{p}{[}\PY{l+s+s1}{\PYZsq{}}\PY{l+s+s1}{outlier}\PY{l+s+s1}{\PYZsq{}}\PY{p}{]} \PY{o}{=} \PY{n}{df}\PY{p}{[}\PY{n}{col}\PY{p}{]}\PY{o}{.}\PY{n}{apply}\PY{p}{(}\PY{n}{check\PYZus{}outlier}\PY{p}{)}
        \PY{k}{return} \PY{n}{new\PYZus{}df}\PY{p}{[}\PY{p}{[}\PY{n}{col}\PY{p}{,} \PY{l+s+s1}{\PYZsq{}}\PY{l+s+s1}{outlier}\PY{l+s+s1}{\PYZsq{}}\PY{p}{]}\PY{p}{]}
    \PY{k}{else}\PY{p}{:}
        \PY{n+nb}{print}\PY{p}{(}\PY{l+s+s2}{\PYZdq{}}\PY{l+s+s2}{Tidak ada Outlier}\PY{l+s+s2}{\PYZdq{}}\PY{p}{)}
        \PY{k}{return} \PY{k+kc}{None}
\end{Verbatim}
\end{tcolorbox}

    \hypertarget{fungsi}{%
\section{FUNGSI}\label{fungsi}}

    \hypertarget{fungsi-find_knn}{%
\subsection{\texorpdfstring{Fungsi
\texttt{find\_Kn(n)}}{Fungsi find\_Kn(n)}}\label{fungsi-find_knn}}

Fungsi ini digunakan untuk mencari nilai \(K_n\) untuk perhitungan uji
outlier. - Argumen fungsi: - \texttt{n}: jumlah data
\(\left(10 \le \mathbb{N} \le 140\right)\). Diluar batasan tersebut akan
menghasilkan peringatan \texttt{ValueError}.

    \begin{tcolorbox}[breakable, size=fbox, boxrule=1pt, pad at break*=1mm,colback=cellbackground, colframe=cellborder]
\prompt{In}{incolor}{ }{\boxspacing}
\begin{Verbatim}[commandchars=\\\{\}]
\PY{n}{find\PYZus{}Kn}\PY{p}{(}\PY{n}{n}\PY{o}{=}\PY{l+m+mi}{10}\PY{p}{)}
\end{Verbatim}
\end{tcolorbox}

            \begin{tcolorbox}[breakable, size=fbox, boxrule=.5pt, pad at break*=1mm, opacityfill=0]
\prompt{Out}{outcolor}{ }{\boxspacing}
\begin{Verbatim}[commandchars=\\\{\}]
2.036
\end{Verbatim}
\end{tcolorbox}
        
    \begin{tcolorbox}[breakable, size=fbox, boxrule=1pt, pad at break*=1mm,colback=cellbackground, colframe=cellborder]
\prompt{In}{incolor}{ }{\boxspacing}
\begin{Verbatim}[commandchars=\\\{\}]
\PY{k}{try}\PY{p}{:}
    \PY{n}{find\PYZus{}Kn}\PY{p}{(}\PY{l+m+mi}{141}\PY{p}{)}
\PY{k}{except} \PY{n+ne}{ValueError}\PY{p}{:}
    \PY{n+nb}{print}\PY{p}{(}\PY{l+s+s2}{\PYZdq{}}\PY{l+s+s2}{Hasil akan error}\PY{l+s+s2}{\PYZdq{}}\PY{p}{)}
\end{Verbatim}
\end{tcolorbox}

    \begin{Verbatim}[commandchars=\\\{\}]
Hasil akan error
    \end{Verbatim}

    \hypertarget{fungsi-calc_boundarydf-colnone-resultvalue-show_statfalse}{%
\subsection{\texorpdfstring{Fungsi
\texttt{calc\_boundary(df,\ col=None,\ result=\textquotesingle{}value\textquotesingle{},\ show\_stat=False)}}{Fungsi calc\_boundary(df, col=None, result='value', show\_stat=False)}}\label{fungsi-calc_boundarydf-colnone-resultvalue-show_statfalse}}

Fungsi \texttt{calc\_boundary(...)} digunakan untuk mencari nilai batas
bawah dan batas atas outlier dari data. Keluaran fungsi ini berupa
\emph{tuple} dengan bentuk \texttt{(low\_boundary,\ high\_boundary)}.

\begin{itemize}
\tightlist
\item
  Argumen Posisi:

  \begin{itemize}
  \tightlist
  \item
    \texttt{df}: dataset dalam objek \texttt{pandas.DataFrame}.
  \end{itemize}
\item
  Argumen Opsional:

  \begin{itemize}
  \tightlist
  \item
    \texttt{col=None}: nama kolom data yang akan dicek outlier. Jika
    \texttt{None} maka dipilih kolom pertama dari dataframe.
  \item
    \texttt{result="value"}: keluaran berupa nilai batasan aktual dengan
    skala original. Jika menggunakan \texttt{log}, keluaran berupa nilai
    batasan dalam skala logaritmik.
  \item
    \texttt{show\_stat=False}: jika \texttt{True} akan menampilkan nilai
    statistik berupa jumlah data, rata-rata, standar deviasi, batasan
    bawah dan atas dalam skala original.
  \end{itemize}
\end{itemize}

    \begin{tcolorbox}[breakable, size=fbox, boxrule=1pt, pad at break*=1mm,colback=cellbackground, colframe=cellborder]
\prompt{In}{incolor}{ }{\boxspacing}
\begin{Verbatim}[commandchars=\\\{\}]
\PY{n}{calc\PYZus{}boundary}\PY{p}{(}\PY{n}{data}\PY{p}{)}
\end{Verbatim}
\end{tcolorbox}

            \begin{tcolorbox}[breakable, size=fbox, boxrule=.5pt, pad at break*=1mm, opacityfill=0]
\prompt{Out}{outcolor}{ }{\boxspacing}
\begin{Verbatim}[commandchars=\\\{\}]
(1092.1254115165966, 2961.0550928195808)
\end{Verbatim}
\end{tcolorbox}
        
    \begin{tcolorbox}[breakable, size=fbox, boxrule=1pt, pad at break*=1mm,colback=cellbackground, colframe=cellborder]
\prompt{In}{incolor}{ }{\boxspacing}
\begin{Verbatim}[commandchars=\\\{\}]
\PY{n}{calc\PYZus{}boundary}\PY{p}{(}\PY{n}{data}\PY{p}{,} \PY{n}{col}\PY{o}{=}\PY{l+s+s1}{\PYZsq{}}\PY{l+s+s1}{Hujan}\PY{l+s+s1}{\PYZsq{}}\PY{p}{,} \PY{n}{result}\PY{o}{=}\PY{l+s+s1}{\PYZsq{}}\PY{l+s+s1}{log}\PY{l+s+s1}{\PYZsq{}}\PY{p}{,} \PY{n}{show\PYZus{}stat}\PY{o}{=}\PY{k+kc}{True}\PY{p}{)}
\end{Verbatim}
\end{tcolorbox}

    \begin{Verbatim}[commandchars=\\\{\}]
Statistik:
N = 10
Mean (log) = 3.25486
Std (log) = 0.10638
Lower (val) = 1092.12541
Higher (val) = 2961.05509

    \end{Verbatim}

            \begin{tcolorbox}[breakable, size=fbox, boxrule=.5pt, pad at break*=1mm, opacityfill=0]
\prompt{Out}{outcolor}{ }{\boxspacing}
\begin{Verbatim}[commandchars=\\\{\}]
(3.038272512363497, 3.471446487863784)
\end{Verbatim}
\end{tcolorbox}
        
    \hypertarget{fungsi-find_outlierdf-colnone-verbosefalse-kwargs}{%
\subsection{\texorpdfstring{Fungsi
\texttt{find\_outlier(df,\ col=None,\ verbose=False,\ **kwargs)}}{Fungsi find\_outlier(df, col=None, verbose=False, **kwargs)}}\label{fungsi-find_outlierdf-colnone-verbosefalse-kwargs}}

Fungsi \texttt{find\_outlier(...)} digunakan untuk memeriksa apakah data
memiliki outlier atau tidak dan memberi keluaran berupa dataframe yang
telah ditandai data mana saja yang dikategorikan outlier.

\begin{itemize}
\tightlist
\item
  Argumen Posisi:

  \begin{itemize}
  \tightlist
  \item
    \texttt{df}: dataset dalam objek \texttt{pandas.DataFrame}.
  \end{itemize}
\item
  Argumen Opsional:

  \begin{itemize}
  \tightlist
  \item
    \texttt{col=None}: nama kolom data yang akan dicek outlier. Jika
    \texttt{None} maka dipilih kolom pertama dari dataframe.
  \item
    \texttt{verbose=False}: memberi informasi tambahan jika memiliki
    outlier dan seberapa banyak.
  \item
    \texttt{**kwargs}: \emph{keyword arguments} dari fungsi
    \texttt{.calc\_boundary()}.
  \end{itemize}
\end{itemize}

    \begin{tcolorbox}[breakable, size=fbox, boxrule=1pt, pad at break*=1mm,colback=cellbackground, colframe=cellborder]
\prompt{In}{incolor}{ }{\boxspacing}
\begin{Verbatim}[commandchars=\\\{\}]
\PY{n}{find\PYZus{}outlier}\PY{p}{(}\PY{n}{data}\PY{p}{)}
\end{Verbatim}
\end{tcolorbox}

    \begin{Verbatim}[commandchars=\\\{\}]
Tidak ada Outlier
    \end{Verbatim}

    \begin{tcolorbox}[breakable, size=fbox, boxrule=1pt, pad at break*=1mm,colback=cellbackground, colframe=cellborder]
\prompt{In}{incolor}{ }{\boxspacing}
\begin{Verbatim}[commandchars=\\\{\}]
\PY{n}{find\PYZus{}outlier}\PY{p}{(}\PY{n}{data}\PY{p}{,} \PY{n}{show\PYZus{}stat}\PY{o}{=}\PY{k+kc}{True}\PY{p}{)}
\end{Verbatim}
\end{tcolorbox}

    \begin{Verbatim}[commandchars=\\\{\}]
Statistik:
N = 10
Mean (log) = 3.25486
Std (log) = 0.10638
Lower (val) = 1092.12541
Higher (val) = 2961.05509

Tidak ada Outlier
    \end{Verbatim}

    \begin{tcolorbox}[breakable, size=fbox, boxrule=1pt, pad at break*=1mm,colback=cellbackground, colframe=cellborder]
\prompt{In}{incolor}{ }{\boxspacing}
\begin{Verbatim}[commandchars=\\\{\}]
\PY{c+c1}{\PYZsh{} contoh data dengan outlier}

\PY{n}{data2} \PY{o}{=} \PY{n}{data}\PY{o}{.}\PY{n}{copy}\PY{p}{(}\PY{p}{)}
\PY{n}{data2}\PY{o}{.}\PY{n}{loc}\PY{p}{[}\PY{l+s+s1}{\PYZsq{}}\PY{l+s+s1}{2012}\PY{l+s+s1}{\PYZsq{}}\PY{p}{]} \PY{o}{=} \PY{l+m+mi}{4000}
\end{Verbatim}
\end{tcolorbox}

    \begin{tcolorbox}[breakable, size=fbox, boxrule=1pt, pad at break*=1mm,colback=cellbackground, colframe=cellborder]
\prompt{In}{incolor}{ }{\boxspacing}
\begin{Verbatim}[commandchars=\\\{\}]
\PY{n}{find\PYZus{}outlier}\PY{p}{(}\PY{n}{data2}\PY{p}{,} \PY{l+s+s1}{\PYZsq{}}\PY{l+s+s1}{Hujan}\PY{l+s+s1}{\PYZsq{}}\PY{p}{,} \PY{n}{verbose}\PY{o}{=}\PY{k+kc}{True}\PY{p}{)}
\end{Verbatim}
\end{tcolorbox}

    \begin{Verbatim}[commandchars=\\\{\}]
Ada outlier diatas batas atas sebanyak 1.
    \end{Verbatim}

            \begin{tcolorbox}[breakable, size=fbox, boxrule=.5pt, pad at break*=1mm, opacityfill=0]
\prompt{Out}{outcolor}{ }{\boxspacing}
\begin{Verbatim}[commandchars=\\\{\}]
            Hujan outlier
Tahun
2010-01-01   2818    <NA>
2013-01-01   2542    <NA>
2008-01-01   1949    <NA>
2012-01-01   4000  higher
2011-01-01   1748    <NA>
2014-01-01   1737    <NA>
2009-01-01   1605    <NA>
2007-01-01   1558    <NA>
2015-01-01   1433    <NA>
2006-01-01   1264    <NA>
\end{Verbatim}
\end{tcolorbox}
        
    \begin{tcolorbox}[breakable, size=fbox, boxrule=1pt, pad at break*=1mm,colback=cellbackground, colframe=cellborder]
\prompt{In}{incolor}{ }{\boxspacing}
\begin{Verbatim}[commandchars=\\\{\}]
\PY{n}{data2}\PY{o}{.}\PY{n}{loc}\PY{p}{[}\PY{l+s+s1}{\PYZsq{}}\PY{l+s+s1}{2012}\PY{l+s+s1}{\PYZsq{}}\PY{p}{]} \PY{o}{=} \PY{l+m+mi}{40}

\PY{n}{find\PYZus{}outlier}\PY{p}{(}\PY{n}{data2}\PY{p}{,} \PY{n}{show\PYZus{}stat}\PY{o}{=}\PY{k+kc}{True}\PY{p}{)}
\end{Verbatim}
\end{tcolorbox}

    \begin{Verbatim}[commandchars=\\\{\}]
Statistik:
N = 10
Mean (log) = 3.08854
Std (log) = 0.53301
Lower (val) = 100.77150
Higher (val) = 14918.85009

    \end{Verbatim}

            \begin{tcolorbox}[breakable, size=fbox, boxrule=.5pt, pad at break*=1mm, opacityfill=0]
\prompt{Out}{outcolor}{ }{\boxspacing}
\begin{Verbatim}[commandchars=\\\{\}]
            Hujan outlier
Tahun
2010-01-01   2818    <NA>
2013-01-01   2542    <NA>
2008-01-01   1949    <NA>
2012-01-01     40   lower
2011-01-01   1748    <NA>
2014-01-01   1737    <NA>
2009-01-01   1605    <NA>
2007-01-01   1558    <NA>
2015-01-01   1433    <NA>
2006-01-01   1264    <NA>
\end{Verbatim}
\end{tcolorbox}
        
    \hypertarget{changelog}{%
\section{Changelog}\label{changelog}}

\begin{verbatim}
- 20220303 - 1.0.0 - Initial
\end{verbatim}

\hypertarget{copyright-2022-taruma-sakti-megariansyah}{%
\paragraph{\texorpdfstring{Copyright © 2022
\href{https://taruma.github.io}{Taruma Sakti
Megariansyah}}{Copyright © 2022 Taruma Sakti Megariansyah}}\label{copyright-2022-taruma-sakti-megariansyah}}

Source code in this notebook is licensed under a
\href{https://choosealicense.com/licenses/mit/}{MIT License}. Data in
this notebook is licensed under a
\href{https://creativecommons.org/licenses/by/4.0/}{Creative Common
Attribution 4.0 International}.


    % Add a bibliography block to the postdoc
    
    
    
\end{document}
