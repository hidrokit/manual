\documentclass[11pt]{article}

    \usepackage[breakable]{tcolorbox}
    \usepackage{parskip} % Stop auto-indenting (to mimic markdown behaviour)
    
    \usepackage{iftex}
    \ifPDFTeX
    	\usepackage[T1]{fontenc}
    	\usepackage{mathpazo}
    \else
    	\usepackage{fontspec}
    \fi

    % Basic figure setup, for now with no caption control since it's done
    % automatically by Pandoc (which extracts ![](path) syntax from Markdown).
    \usepackage{graphicx}
    % Maintain compatibility with old templates. Remove in nbconvert 6.0
    \let\Oldincludegraphics\includegraphics
    % Ensure that by default, figures have no caption (until we provide a
    % proper Figure object with a Caption API and a way to capture that
    % in the conversion process - todo).
    \usepackage{caption}
    \DeclareCaptionFormat{nocaption}{}
    \captionsetup{format=nocaption,aboveskip=0pt,belowskip=0pt}

    \usepackage{float}
    \floatplacement{figure}{H} % forces figures to be placed at the correct location
    \usepackage{xcolor} % Allow colors to be defined
    \usepackage{enumerate} % Needed for markdown enumerations to work
    \usepackage{geometry} % Used to adjust the document margins
    \usepackage{amsmath} % Equations
    \usepackage{amssymb} % Equations
    \usepackage{textcomp} % defines textquotesingle
    % Hack from http://tex.stackexchange.com/a/47451/13684:
    \AtBeginDocument{%
        \def\PYZsq{\textquotesingle}% Upright quotes in Pygmentized code
    }
    \usepackage{upquote} % Upright quotes for verbatim code
    \usepackage{eurosym} % defines \euro
    \usepackage[mathletters]{ucs} % Extended unicode (utf-8) support
    \usepackage{fancyvrb} % verbatim replacement that allows latex
    \usepackage{grffile} % extends the file name processing of package graphics 
                         % to support a larger range
    \makeatletter % fix for old versions of grffile with XeLaTeX
    \@ifpackagelater{grffile}{2019/11/01}
    {
      % Do nothing on new versions
    }
    {
      \def\Gread@@xetex#1{%
        \IfFileExists{"\Gin@base".bb}%
        {\Gread@eps{\Gin@base.bb}}%
        {\Gread@@xetex@aux#1}%
      }
    }
    \makeatother
    \usepackage[Export]{adjustbox} % Used to constrain images to a maximum size
    \adjustboxset{max size={0.9\linewidth}{0.9\paperheight}}

    % The hyperref package gives us a pdf with properly built
    % internal navigation ('pdf bookmarks' for the table of contents,
    % internal cross-reference links, web links for URLs, etc.)
    \usepackage{hyperref}
    % The default LaTeX title has an obnoxious amount of whitespace. By default,
    % titling removes some of it. It also provides customization options.
    \usepackage{titling}
    \usepackage{longtable} % longtable support required by pandoc >1.10
    \usepackage{booktabs}  % table support for pandoc > 1.12.2
    \usepackage[inline]{enumitem} % IRkernel/repr support (it uses the enumerate* environment)
    \usepackage[normalem]{ulem} % ulem is needed to support strikethroughs (\sout)
                                % normalem makes italics be italics, not underlines
    \usepackage{mathrsfs}
    

    
    % Colors for the hyperref package
    \definecolor{urlcolor}{rgb}{0,.145,.698}
    \definecolor{linkcolor}{rgb}{.71,0.21,0.01}
    \definecolor{citecolor}{rgb}{.12,.54,.11}

    % ANSI colors
    \definecolor{ansi-black}{HTML}{3E424D}
    \definecolor{ansi-black-intense}{HTML}{282C36}
    \definecolor{ansi-red}{HTML}{E75C58}
    \definecolor{ansi-red-intense}{HTML}{B22B31}
    \definecolor{ansi-green}{HTML}{00A250}
    \definecolor{ansi-green-intense}{HTML}{007427}
    \definecolor{ansi-yellow}{HTML}{DDB62B}
    \definecolor{ansi-yellow-intense}{HTML}{B27D12}
    \definecolor{ansi-blue}{HTML}{208FFB}
    \definecolor{ansi-blue-intense}{HTML}{0065CA}
    \definecolor{ansi-magenta}{HTML}{D160C4}
    \definecolor{ansi-magenta-intense}{HTML}{A03196}
    \definecolor{ansi-cyan}{HTML}{60C6C8}
    \definecolor{ansi-cyan-intense}{HTML}{258F8F}
    \definecolor{ansi-white}{HTML}{C5C1B4}
    \definecolor{ansi-white-intense}{HTML}{A1A6B2}
    \definecolor{ansi-default-inverse-fg}{HTML}{FFFFFF}
    \definecolor{ansi-default-inverse-bg}{HTML}{000000}

    % common color for the border for error outputs.
    \definecolor{outerrorbackground}{HTML}{FFDFDF}

    % commands and environments needed by pandoc snippets
    % extracted from the output of `pandoc -s`
    \providecommand{\tightlist}{%
      \setlength{\itemsep}{0pt}\setlength{\parskip}{0pt}}
    \DefineVerbatimEnvironment{Highlighting}{Verbatim}{commandchars=\\\{\}}
    % Add ',fontsize=\small' for more characters per line
    \newenvironment{Shaded}{}{}
    \newcommand{\KeywordTok}[1]{\textcolor[rgb]{0.00,0.44,0.13}{\textbf{{#1}}}}
    \newcommand{\DataTypeTok}[1]{\textcolor[rgb]{0.56,0.13,0.00}{{#1}}}
    \newcommand{\DecValTok}[1]{\textcolor[rgb]{0.25,0.63,0.44}{{#1}}}
    \newcommand{\BaseNTok}[1]{\textcolor[rgb]{0.25,0.63,0.44}{{#1}}}
    \newcommand{\FloatTok}[1]{\textcolor[rgb]{0.25,0.63,0.44}{{#1}}}
    \newcommand{\CharTok}[1]{\textcolor[rgb]{0.25,0.44,0.63}{{#1}}}
    \newcommand{\StringTok}[1]{\textcolor[rgb]{0.25,0.44,0.63}{{#1}}}
    \newcommand{\CommentTok}[1]{\textcolor[rgb]{0.38,0.63,0.69}{\textit{{#1}}}}
    \newcommand{\OtherTok}[1]{\textcolor[rgb]{0.00,0.44,0.13}{{#1}}}
    \newcommand{\AlertTok}[1]{\textcolor[rgb]{1.00,0.00,0.00}{\textbf{{#1}}}}
    \newcommand{\FunctionTok}[1]{\textcolor[rgb]{0.02,0.16,0.49}{{#1}}}
    \newcommand{\RegionMarkerTok}[1]{{#1}}
    \newcommand{\ErrorTok}[1]{\textcolor[rgb]{1.00,0.00,0.00}{\textbf{{#1}}}}
    \newcommand{\NormalTok}[1]{{#1}}
    
    % Additional commands for more recent versions of Pandoc
    \newcommand{\ConstantTok}[1]{\textcolor[rgb]{0.53,0.00,0.00}{{#1}}}
    \newcommand{\SpecialCharTok}[1]{\textcolor[rgb]{0.25,0.44,0.63}{{#1}}}
    \newcommand{\VerbatimStringTok}[1]{\textcolor[rgb]{0.25,0.44,0.63}{{#1}}}
    \newcommand{\SpecialStringTok}[1]{\textcolor[rgb]{0.73,0.40,0.53}{{#1}}}
    \newcommand{\ImportTok}[1]{{#1}}
    \newcommand{\DocumentationTok}[1]{\textcolor[rgb]{0.73,0.13,0.13}{\textit{{#1}}}}
    \newcommand{\AnnotationTok}[1]{\textcolor[rgb]{0.38,0.63,0.69}{\textbf{\textit{{#1}}}}}
    \newcommand{\CommentVarTok}[1]{\textcolor[rgb]{0.38,0.63,0.69}{\textbf{\textit{{#1}}}}}
    \newcommand{\VariableTok}[1]{\textcolor[rgb]{0.10,0.09,0.49}{{#1}}}
    \newcommand{\ControlFlowTok}[1]{\textcolor[rgb]{0.00,0.44,0.13}{\textbf{{#1}}}}
    \newcommand{\OperatorTok}[1]{\textcolor[rgb]{0.40,0.40,0.40}{{#1}}}
    \newcommand{\BuiltInTok}[1]{{#1}}
    \newcommand{\ExtensionTok}[1]{{#1}}
    \newcommand{\PreprocessorTok}[1]{\textcolor[rgb]{0.74,0.48,0.00}{{#1}}}
    \newcommand{\AttributeTok}[1]{\textcolor[rgb]{0.49,0.56,0.16}{{#1}}}
    \newcommand{\InformationTok}[1]{\textcolor[rgb]{0.38,0.63,0.69}{\textbf{\textit{{#1}}}}}
    \newcommand{\WarningTok}[1]{\textcolor[rgb]{0.38,0.63,0.69}{\textbf{\textit{{#1}}}}}
    
    
    % Define a nice break command that doesn't care if a line doesn't already
    % exist.
    \def\br{\hspace*{\fill} \\* }
    % Math Jax compatibility definitions
    \def\gt{>}
    \def\lt{<}
    \let\Oldtex\TeX
    \let\Oldlatex\LaTeX
    \renewcommand{\TeX}{\textrm{\Oldtex}}
    \renewcommand{\LaTeX}{\textrm{\Oldlatex}}
    % Document parameters
    % Document title
    \title{taruma\_0\_4\_0\_hk124\_log\_normal}
    
    
    
    
    
% Pygments definitions
\makeatletter
\def\PY@reset{\let\PY@it=\relax \let\PY@bf=\relax%
    \let\PY@ul=\relax \let\PY@tc=\relax%
    \let\PY@bc=\relax \let\PY@ff=\relax}
\def\PY@tok#1{\csname PY@tok@#1\endcsname}
\def\PY@toks#1+{\ifx\relax#1\empty\else%
    \PY@tok{#1}\expandafter\PY@toks\fi}
\def\PY@do#1{\PY@bc{\PY@tc{\PY@ul{%
    \PY@it{\PY@bf{\PY@ff{#1}}}}}}}
\def\PY#1#2{\PY@reset\PY@toks#1+\relax+\PY@do{#2}}

\@namedef{PY@tok@w}{\def\PY@tc##1{\textcolor[rgb]{0.73,0.73,0.73}{##1}}}
\@namedef{PY@tok@c}{\let\PY@it=\textit\def\PY@tc##1{\textcolor[rgb]{0.24,0.48,0.48}{##1}}}
\@namedef{PY@tok@cp}{\def\PY@tc##1{\textcolor[rgb]{0.61,0.40,0.00}{##1}}}
\@namedef{PY@tok@k}{\let\PY@bf=\textbf\def\PY@tc##1{\textcolor[rgb]{0.00,0.50,0.00}{##1}}}
\@namedef{PY@tok@kp}{\def\PY@tc##1{\textcolor[rgb]{0.00,0.50,0.00}{##1}}}
\@namedef{PY@tok@kt}{\def\PY@tc##1{\textcolor[rgb]{0.69,0.00,0.25}{##1}}}
\@namedef{PY@tok@o}{\def\PY@tc##1{\textcolor[rgb]{0.40,0.40,0.40}{##1}}}
\@namedef{PY@tok@ow}{\let\PY@bf=\textbf\def\PY@tc##1{\textcolor[rgb]{0.67,0.13,1.00}{##1}}}
\@namedef{PY@tok@nb}{\def\PY@tc##1{\textcolor[rgb]{0.00,0.50,0.00}{##1}}}
\@namedef{PY@tok@nf}{\def\PY@tc##1{\textcolor[rgb]{0.00,0.00,1.00}{##1}}}
\@namedef{PY@tok@nc}{\let\PY@bf=\textbf\def\PY@tc##1{\textcolor[rgb]{0.00,0.00,1.00}{##1}}}
\@namedef{PY@tok@nn}{\let\PY@bf=\textbf\def\PY@tc##1{\textcolor[rgb]{0.00,0.00,1.00}{##1}}}
\@namedef{PY@tok@ne}{\let\PY@bf=\textbf\def\PY@tc##1{\textcolor[rgb]{0.80,0.25,0.22}{##1}}}
\@namedef{PY@tok@nv}{\def\PY@tc##1{\textcolor[rgb]{0.10,0.09,0.49}{##1}}}
\@namedef{PY@tok@no}{\def\PY@tc##1{\textcolor[rgb]{0.53,0.00,0.00}{##1}}}
\@namedef{PY@tok@nl}{\def\PY@tc##1{\textcolor[rgb]{0.46,0.46,0.00}{##1}}}
\@namedef{PY@tok@ni}{\let\PY@bf=\textbf\def\PY@tc##1{\textcolor[rgb]{0.44,0.44,0.44}{##1}}}
\@namedef{PY@tok@na}{\def\PY@tc##1{\textcolor[rgb]{0.41,0.47,0.13}{##1}}}
\@namedef{PY@tok@nt}{\let\PY@bf=\textbf\def\PY@tc##1{\textcolor[rgb]{0.00,0.50,0.00}{##1}}}
\@namedef{PY@tok@nd}{\def\PY@tc##1{\textcolor[rgb]{0.67,0.13,1.00}{##1}}}
\@namedef{PY@tok@s}{\def\PY@tc##1{\textcolor[rgb]{0.73,0.13,0.13}{##1}}}
\@namedef{PY@tok@sd}{\let\PY@it=\textit\def\PY@tc##1{\textcolor[rgb]{0.73,0.13,0.13}{##1}}}
\@namedef{PY@tok@si}{\let\PY@bf=\textbf\def\PY@tc##1{\textcolor[rgb]{0.64,0.35,0.47}{##1}}}
\@namedef{PY@tok@se}{\let\PY@bf=\textbf\def\PY@tc##1{\textcolor[rgb]{0.67,0.36,0.12}{##1}}}
\@namedef{PY@tok@sr}{\def\PY@tc##1{\textcolor[rgb]{0.64,0.35,0.47}{##1}}}
\@namedef{PY@tok@ss}{\def\PY@tc##1{\textcolor[rgb]{0.10,0.09,0.49}{##1}}}
\@namedef{PY@tok@sx}{\def\PY@tc##1{\textcolor[rgb]{0.00,0.50,0.00}{##1}}}
\@namedef{PY@tok@m}{\def\PY@tc##1{\textcolor[rgb]{0.40,0.40,0.40}{##1}}}
\@namedef{PY@tok@gh}{\let\PY@bf=\textbf\def\PY@tc##1{\textcolor[rgb]{0.00,0.00,0.50}{##1}}}
\@namedef{PY@tok@gu}{\let\PY@bf=\textbf\def\PY@tc##1{\textcolor[rgb]{0.50,0.00,0.50}{##1}}}
\@namedef{PY@tok@gd}{\def\PY@tc##1{\textcolor[rgb]{0.63,0.00,0.00}{##1}}}
\@namedef{PY@tok@gi}{\def\PY@tc##1{\textcolor[rgb]{0.00,0.52,0.00}{##1}}}
\@namedef{PY@tok@gr}{\def\PY@tc##1{\textcolor[rgb]{0.89,0.00,0.00}{##1}}}
\@namedef{PY@tok@ge}{\let\PY@it=\textit}
\@namedef{PY@tok@gs}{\let\PY@bf=\textbf}
\@namedef{PY@tok@gp}{\let\PY@bf=\textbf\def\PY@tc##1{\textcolor[rgb]{0.00,0.00,0.50}{##1}}}
\@namedef{PY@tok@go}{\def\PY@tc##1{\textcolor[rgb]{0.44,0.44,0.44}{##1}}}
\@namedef{PY@tok@gt}{\def\PY@tc##1{\textcolor[rgb]{0.00,0.27,0.87}{##1}}}
\@namedef{PY@tok@err}{\def\PY@bc##1{{\setlength{\fboxsep}{\string -\fboxrule}\fcolorbox[rgb]{1.00,0.00,0.00}{1,1,1}{\strut ##1}}}}
\@namedef{PY@tok@kc}{\let\PY@bf=\textbf\def\PY@tc##1{\textcolor[rgb]{0.00,0.50,0.00}{##1}}}
\@namedef{PY@tok@kd}{\let\PY@bf=\textbf\def\PY@tc##1{\textcolor[rgb]{0.00,0.50,0.00}{##1}}}
\@namedef{PY@tok@kn}{\let\PY@bf=\textbf\def\PY@tc##1{\textcolor[rgb]{0.00,0.50,0.00}{##1}}}
\@namedef{PY@tok@kr}{\let\PY@bf=\textbf\def\PY@tc##1{\textcolor[rgb]{0.00,0.50,0.00}{##1}}}
\@namedef{PY@tok@bp}{\def\PY@tc##1{\textcolor[rgb]{0.00,0.50,0.00}{##1}}}
\@namedef{PY@tok@fm}{\def\PY@tc##1{\textcolor[rgb]{0.00,0.00,1.00}{##1}}}
\@namedef{PY@tok@vc}{\def\PY@tc##1{\textcolor[rgb]{0.10,0.09,0.49}{##1}}}
\@namedef{PY@tok@vg}{\def\PY@tc##1{\textcolor[rgb]{0.10,0.09,0.49}{##1}}}
\@namedef{PY@tok@vi}{\def\PY@tc##1{\textcolor[rgb]{0.10,0.09,0.49}{##1}}}
\@namedef{PY@tok@vm}{\def\PY@tc##1{\textcolor[rgb]{0.10,0.09,0.49}{##1}}}
\@namedef{PY@tok@sa}{\def\PY@tc##1{\textcolor[rgb]{0.73,0.13,0.13}{##1}}}
\@namedef{PY@tok@sb}{\def\PY@tc##1{\textcolor[rgb]{0.73,0.13,0.13}{##1}}}
\@namedef{PY@tok@sc}{\def\PY@tc##1{\textcolor[rgb]{0.73,0.13,0.13}{##1}}}
\@namedef{PY@tok@dl}{\def\PY@tc##1{\textcolor[rgb]{0.73,0.13,0.13}{##1}}}
\@namedef{PY@tok@s2}{\def\PY@tc##1{\textcolor[rgb]{0.73,0.13,0.13}{##1}}}
\@namedef{PY@tok@sh}{\def\PY@tc##1{\textcolor[rgb]{0.73,0.13,0.13}{##1}}}
\@namedef{PY@tok@s1}{\def\PY@tc##1{\textcolor[rgb]{0.73,0.13,0.13}{##1}}}
\@namedef{PY@tok@mb}{\def\PY@tc##1{\textcolor[rgb]{0.40,0.40,0.40}{##1}}}
\@namedef{PY@tok@mf}{\def\PY@tc##1{\textcolor[rgb]{0.40,0.40,0.40}{##1}}}
\@namedef{PY@tok@mh}{\def\PY@tc##1{\textcolor[rgb]{0.40,0.40,0.40}{##1}}}
\@namedef{PY@tok@mi}{\def\PY@tc##1{\textcolor[rgb]{0.40,0.40,0.40}{##1}}}
\@namedef{PY@tok@il}{\def\PY@tc##1{\textcolor[rgb]{0.40,0.40,0.40}{##1}}}
\@namedef{PY@tok@mo}{\def\PY@tc##1{\textcolor[rgb]{0.40,0.40,0.40}{##1}}}
\@namedef{PY@tok@ch}{\let\PY@it=\textit\def\PY@tc##1{\textcolor[rgb]{0.24,0.48,0.48}{##1}}}
\@namedef{PY@tok@cm}{\let\PY@it=\textit\def\PY@tc##1{\textcolor[rgb]{0.24,0.48,0.48}{##1}}}
\@namedef{PY@tok@cpf}{\let\PY@it=\textit\def\PY@tc##1{\textcolor[rgb]{0.24,0.48,0.48}{##1}}}
\@namedef{PY@tok@c1}{\let\PY@it=\textit\def\PY@tc##1{\textcolor[rgb]{0.24,0.48,0.48}{##1}}}
\@namedef{PY@tok@cs}{\let\PY@it=\textit\def\PY@tc##1{\textcolor[rgb]{0.24,0.48,0.48}{##1}}}

\def\PYZbs{\char`\\}
\def\PYZus{\char`\_}
\def\PYZob{\char`\{}
\def\PYZcb{\char`\}}
\def\PYZca{\char`\^}
\def\PYZam{\char`\&}
\def\PYZlt{\char`\<}
\def\PYZgt{\char`\>}
\def\PYZsh{\char`\#}
\def\PYZpc{\char`\%}
\def\PYZdl{\char`\$}
\def\PYZhy{\char`\-}
\def\PYZsq{\char`\'}
\def\PYZdq{\char`\"}
\def\PYZti{\char`\~}
% for compatibility with earlier versions
\def\PYZat{@}
\def\PYZlb{[}
\def\PYZrb{]}
\makeatother


    % For linebreaks inside Verbatim environment from package fancyvrb. 
    \makeatletter
        \newbox\Wrappedcontinuationbox 
        \newbox\Wrappedvisiblespacebox 
        \newcommand*\Wrappedvisiblespace {\textcolor{red}{\textvisiblespace}} 
        \newcommand*\Wrappedcontinuationsymbol {\textcolor{red}{\llap{\tiny$\m@th\hookrightarrow$}}} 
        \newcommand*\Wrappedcontinuationindent {3ex } 
        \newcommand*\Wrappedafterbreak {\kern\Wrappedcontinuationindent\copy\Wrappedcontinuationbox} 
        % Take advantage of the already applied Pygments mark-up to insert 
        % potential linebreaks for TeX processing. 
        %        {, <, #, %, $, ' and ": go to next line. 
        %        _, }, ^, &, >, - and ~: stay at end of broken line. 
        % Use of \textquotesingle for straight quote. 
        \newcommand*\Wrappedbreaksatspecials {% 
            \def\PYGZus{\discretionary{\char`\_}{\Wrappedafterbreak}{\char`\_}}% 
            \def\PYGZob{\discretionary{}{\Wrappedafterbreak\char`\{}{\char`\{}}% 
            \def\PYGZcb{\discretionary{\char`\}}{\Wrappedafterbreak}{\char`\}}}% 
            \def\PYGZca{\discretionary{\char`\^}{\Wrappedafterbreak}{\char`\^}}% 
            \def\PYGZam{\discretionary{\char`\&}{\Wrappedafterbreak}{\char`\&}}% 
            \def\PYGZlt{\discretionary{}{\Wrappedafterbreak\char`\<}{\char`\<}}% 
            \def\PYGZgt{\discretionary{\char`\>}{\Wrappedafterbreak}{\char`\>}}% 
            \def\PYGZsh{\discretionary{}{\Wrappedafterbreak\char`\#}{\char`\#}}% 
            \def\PYGZpc{\discretionary{}{\Wrappedafterbreak\char`\%}{\char`\%}}% 
            \def\PYGZdl{\discretionary{}{\Wrappedafterbreak\char`\$}{\char`\$}}% 
            \def\PYGZhy{\discretionary{\char`\-}{\Wrappedafterbreak}{\char`\-}}% 
            \def\PYGZsq{\discretionary{}{\Wrappedafterbreak\textquotesingle}{\textquotesingle}}% 
            \def\PYGZdq{\discretionary{}{\Wrappedafterbreak\char`\"}{\char`\"}}% 
            \def\PYGZti{\discretionary{\char`\~}{\Wrappedafterbreak}{\char`\~}}% 
        } 
        % Some characters . , ; ? ! / are not pygmentized. 
        % This macro makes them "active" and they will insert potential linebreaks 
        \newcommand*\Wrappedbreaksatpunct {% 
            \lccode`\~`\.\lowercase{\def~}{\discretionary{\hbox{\char`\.}}{\Wrappedafterbreak}{\hbox{\char`\.}}}% 
            \lccode`\~`\,\lowercase{\def~}{\discretionary{\hbox{\char`\,}}{\Wrappedafterbreak}{\hbox{\char`\,}}}% 
            \lccode`\~`\;\lowercase{\def~}{\discretionary{\hbox{\char`\;}}{\Wrappedafterbreak}{\hbox{\char`\;}}}% 
            \lccode`\~`\:\lowercase{\def~}{\discretionary{\hbox{\char`\:}}{\Wrappedafterbreak}{\hbox{\char`\:}}}% 
            \lccode`\~`\?\lowercase{\def~}{\discretionary{\hbox{\char`\?}}{\Wrappedafterbreak}{\hbox{\char`\?}}}% 
            \lccode`\~`\!\lowercase{\def~}{\discretionary{\hbox{\char`\!}}{\Wrappedafterbreak}{\hbox{\char`\!}}}% 
            \lccode`\~`\/\lowercase{\def~}{\discretionary{\hbox{\char`\/}}{\Wrappedafterbreak}{\hbox{\char`\/}}}% 
            \catcode`\.\active
            \catcode`\,\active 
            \catcode`\;\active
            \catcode`\:\active
            \catcode`\?\active
            \catcode`\!\active
            \catcode`\/\active 
            \lccode`\~`\~ 	
        }
    \makeatother

    \let\OriginalVerbatim=\Verbatim
    \makeatletter
    \renewcommand{\Verbatim}[1][1]{%
        %\parskip\z@skip
        \sbox\Wrappedcontinuationbox {\Wrappedcontinuationsymbol}%
        \sbox\Wrappedvisiblespacebox {\FV@SetupFont\Wrappedvisiblespace}%
        \def\FancyVerbFormatLine ##1{\hsize\linewidth
            \vtop{\raggedright\hyphenpenalty\z@\exhyphenpenalty\z@
                \doublehyphendemerits\z@\finalhyphendemerits\z@
                \strut ##1\strut}%
        }%
        % If the linebreak is at a space, the latter will be displayed as visible
        % space at end of first line, and a continuation symbol starts next line.
        % Stretch/shrink are however usually zero for typewriter font.
        \def\FV@Space {%
            \nobreak\hskip\z@ plus\fontdimen3\font minus\fontdimen4\font
            \discretionary{\copy\Wrappedvisiblespacebox}{\Wrappedafterbreak}
            {\kern\fontdimen2\font}%
        }%
        
        % Allow breaks at special characters using \PYG... macros.
        \Wrappedbreaksatspecials
        % Breaks at punctuation characters . , ; ? ! and / need catcode=\active 	
        \OriginalVerbatim[#1,codes*=\Wrappedbreaksatpunct]%
    }
    \makeatother

    % Exact colors from NB
    \definecolor{incolor}{HTML}{303F9F}
    \definecolor{outcolor}{HTML}{D84315}
    \definecolor{cellborder}{HTML}{CFCFCF}
    \definecolor{cellbackground}{HTML}{F7F7F7}
    
    % prompt
    \makeatletter
    \newcommand{\boxspacing}{\kern\kvtcb@left@rule\kern\kvtcb@boxsep}
    \makeatother
    \newcommand{\prompt}[4]{
        {\ttfamily\llap{{\color{#2}[#3]:\hspace{3pt}#4}}\vspace{-\baselineskip}}
    }
    

    
    % Prevent overflowing lines due to hard-to-break entities
    \sloppy 
    % Setup hyperref package
    \hypersetup{
      breaklinks=true,  % so long urls are correctly broken across lines
      colorlinks=true,
      urlcolor=urlcolor,
      linkcolor=linkcolor,
      citecolor=citecolor,
      }
    % Slightly bigger margins than the latex defaults
    
    \geometry{verbose,tmargin=1in,bmargin=1in,lmargin=1in,rmargin=1in}
    
    

\begin{document}
    
    \maketitle
    
    

    
    Berdasarkan isu
\href{https://github.com/hidrokit/hidrokit/issues/124}{\#124}:
\textbf{anfrek: Log Normal 2 Parameter}

Referensi Isu: - Soetopo, W., Montarcih, L., Press, U. B., \& Media, U.
(2017). Rekayasa Statistika untuk Teknik Pengairan. Universitas
Brawijaya Press. https://books.google.co.id/books?id=TzVTDwAAQBAJ -
Soewarno. (1995). hidrologi: Aplikasi Metode Statistik untuk Analisa
Data.NOVA.

Deskripsi Isu: - Mencari nilai ekstrim dengan kala ulang tertentu.
Penerapan ini bisa digunakan untuk hujan rancangan atau debit banjir
rancangan.

Diskusi Isu: -
\href{https://github.com/hidrokit/hidrokit/discussions/156}{\#156} -
Bagaimana menghitung periode ulang distribusi (analisis frekuensi) tanpa
melihat tabel?

Strategi: - Akan mengikuti fungsi log pearson
\href{https://github.com/hidrokit/hidrokit/issues/126}{\#126} seperti
pada
\href{https://gist.github.com/taruma/60725ffca91dc6e741daee9a738a978b}{manual}.

    \hypertarget{persiapan-dan-dataset}{%
\section{PERSIAPAN DAN DATASET}\label{persiapan-dan-dataset}}

    \begin{tcolorbox}[breakable, size=fbox, boxrule=1pt, pad at break*=1mm,colback=cellbackground, colframe=cellborder]
\prompt{In}{incolor}{ }{\boxspacing}
\begin{Verbatim}[commandchars=\\\{\}]
\PY{k+kn}{import} \PY{n+nn}{numpy} \PY{k}{as} \PY{n+nn}{np}
\PY{k+kn}{import} \PY{n+nn}{pandas} \PY{k}{as} \PY{n+nn}{pd}
\PY{k+kn}{from} \PY{n+nn}{scipy} \PY{k+kn}{import} \PY{n}{stats}
\end{Verbatim}
\end{tcolorbox}

    \begin{tcolorbox}[breakable, size=fbox, boxrule=1pt, pad at break*=1mm,colback=cellbackground, colframe=cellborder]
\prompt{In}{incolor}{ }{\boxspacing}
\begin{Verbatim}[commandchars=\\\{\}]
\PY{c+c1}{\PYZsh{} contoh data diambil dari buku}
\PY{c+c1}{\PYZsh{} hidrologi: Aplikasi Metode Statistik untuk Analisa Data hal. 144}

\PY{n}{\PYZus{}DEBIT} \PY{o}{=} \PY{p}{[}
    \PY{l+m+mf}{58.3}\PY{p}{,} \PY{l+m+mf}{50.5}\PY{p}{,} \PY{l+m+mf}{46.0}\PY{p}{,} \PY{l+m+mf}{41.8}\PY{p}{,} \PY{l+m+mf}{38.2}\PY{p}{,} \PY{l+m+mf}{37.9}\PY{p}{,} \PY{l+m+mf}{37.7}\PY{p}{,} \PY{l+m+mf}{35.3}\PY{p}{,} \PY{l+m+mf}{35.2}\PY{p}{,} \PY{l+m+mf}{33.4}\PY{p}{,} \PY{l+m+mf}{31.9}\PY{p}{,} 
    \PY{l+m+mf}{31.1}\PY{p}{,} \PY{l+m+mf}{30.9}\PY{p}{,} \PY{l+m+mf}{30.1}\PY{p}{,} \PY{l+m+mf}{28.8}\PY{p}{,} \PY{l+m+mf}{24.7}\PY{p}{,} \PY{l+m+mf}{23.6}\PY{p}{,} \PY{l+m+mf}{23.5}\PY{p}{,} \PY{l+m+mf}{23.1}\PY{p}{,} \PY{l+m+mf}{22.5}\PY{p}{,} \PY{l+m+mf}{21.1}\PY{p}{,} \PY{l+m+mf}{20.5}\PY{p}{,} 
    \PY{l+m+mf}{20.5}\PY{p}{,} \PY{l+m+mf}{20.3}\PY{p}{,} \PY{l+m+mf}{20.2}\PY{p}{,} \PY{l+m+mf}{18.7}\PY{p}{,} \PY{l+m+mf}{17.2}\PY{p}{,} \PY{l+m+mf}{14.9}\PY{p}{,} \PY{l+m+mf}{12.4}\PY{p}{,} \PY{l+m+mf}{11.8}
\PY{p}{]}

\PY{n}{data} \PY{o}{=} \PY{n}{pd}\PY{o}{.}\PY{n}{DataFrame}\PY{p}{(}
    \PY{n}{data}\PY{o}{=}\PY{n}{\PYZus{}DEBIT}\PY{p}{,} \PY{n}{columns}\PY{o}{=}\PY{p}{[}\PY{l+s+s1}{\PYZsq{}}\PY{l+s+s1}{debit}\PY{l+s+s1}{\PYZsq{}}\PY{p}{]}\PY{p}{,} \PY{n}{index}\PY{o}{=}\PY{n+nb}{range}\PY{p}{(}\PY{l+m+mi}{1}\PY{p}{,} \PY{l+m+mi}{31}\PY{p}{)}
\PY{p}{)}

\PY{n}{data}
\end{Verbatim}
\end{tcolorbox}

            \begin{tcolorbox}[breakable, size=fbox, boxrule=.5pt, pad at break*=1mm, opacityfill=0]
\prompt{Out}{outcolor}{ }{\boxspacing}
\begin{Verbatim}[commandchars=\\\{\}]
    debit
1    58.3
2    50.5
3    46.0
4    41.8
5    38.2
6    37.9
7    37.7
8    35.3
9    35.2
10   33.4
11   31.9
12   31.1
13   30.9
14   30.1
15   28.8
16   24.7
17   23.6
18   23.5
19   23.1
20   22.5
21   21.1
22   20.5
23   20.5
24   20.3
25   20.2
26   18.7
27   17.2
28   14.9
29   12.4
30   11.8
\end{Verbatim}
\end{tcolorbox}
        
    \hypertarget{tabel}{%
\section{TABEL}\label{tabel}}

Terdapat 1 tabel untuk modul \texttt{hk124} yaitu: -
\texttt{t\_normal\_sw}: Tabel nilai \(k\) dari Tabel 3.3 Nilai Variabel
Reduksi Gauss. Sumber: hidrologi: Aplikasi Metode Statistik untuk
Analisa Data.

Dalam modul \texttt{hk124} nilai \(k\) akan dibangkitkan menggunakan
\texttt{scipy} secara \texttt{default}. Mohon diperhatikan jika ingin
menggunakan nilai \(k\) yang berasal dari sumber lain.

    \begin{tcolorbox}[breakable, size=fbox, boxrule=1pt, pad at break*=1mm,colback=cellbackground, colframe=cellborder]
\prompt{In}{incolor}{ }{\boxspacing}
\begin{Verbatim}[commandchars=\\\{\}]
\PY{c+c1}{\PYZsh{} Tabel Nilai Variabel Reduksi Gauss}
\PY{c+c1}{\PYZsh{} Dari buku hidrologi: Aplikasi Metode Statistik untuk Analisa Data. hal.119}

\PY{c+c1}{\PYZsh{} KODE: SW}

\PY{n}{\PYZus{}DATA\PYZus{}SW} \PY{o}{=} \PY{p}{[}
    \PY{p}{[}\PY{l+m+mf}{1.001}\PY{p}{,} \PY{l+m+mf}{0.999}\PY{p}{,} \PY{o}{\PYZhy{}}\PY{l+m+mf}{3.050}\PY{p}{]}\PY{p}{,}
    \PY{p}{[}\PY{l+m+mf}{1.005}\PY{p}{,} \PY{l+m+mf}{0.995}\PY{p}{,} \PY{o}{\PYZhy{}}\PY{l+m+mf}{2.580}\PY{p}{]}\PY{p}{,}
    \PY{p}{[}\PY{l+m+mf}{1.010}\PY{p}{,} \PY{l+m+mf}{0.990}\PY{p}{,} \PY{o}{\PYZhy{}}\PY{l+m+mf}{2.330}\PY{p}{]}\PY{p}{,}
    \PY{p}{[}\PY{l+m+mf}{1.050}\PY{p}{,} \PY{l+m+mf}{0.950}\PY{p}{,} \PY{o}{\PYZhy{}}\PY{l+m+mf}{1.640}\PY{p}{]}\PY{p}{,}
    \PY{p}{[}\PY{l+m+mf}{1.110}\PY{p}{,} \PY{l+m+mf}{0.900}\PY{p}{,} \PY{o}{\PYZhy{}}\PY{l+m+mf}{1.280}\PY{p}{]}\PY{p}{,}
    \PY{p}{[}\PY{l+m+mf}{1.250}\PY{p}{,} \PY{l+m+mf}{0.800}\PY{p}{,} \PY{o}{\PYZhy{}}\PY{l+m+mf}{0.840}\PY{p}{]}\PY{p}{,}
    \PY{p}{[}\PY{l+m+mf}{1.330}\PY{p}{,} \PY{l+m+mf}{0.750}\PY{p}{,} \PY{o}{\PYZhy{}}\PY{l+m+mf}{0.670}\PY{p}{]}\PY{p}{,}
    \PY{p}{[}\PY{l+m+mf}{1.430}\PY{p}{,} \PY{l+m+mf}{0.700}\PY{p}{,} \PY{o}{\PYZhy{}}\PY{l+m+mf}{0.520}\PY{p}{]}\PY{p}{,}
    \PY{p}{[}\PY{l+m+mf}{1.670}\PY{p}{,} \PY{l+m+mf}{0.600}\PY{p}{,} \PY{o}{\PYZhy{}}\PY{l+m+mf}{0.250}\PY{p}{]}\PY{p}{,}
    \PY{p}{[}\PY{l+m+mf}{2.000}\PY{p}{,} \PY{l+m+mf}{0.500}\PY{p}{,} \PY{l+m+mf}{0.000}\PY{p}{]}\PY{p}{,}
    \PY{p}{[}\PY{l+m+mf}{2.500}\PY{p}{,} \PY{l+m+mf}{0.400}\PY{p}{,} \PY{l+m+mf}{0.250}\PY{p}{]}\PY{p}{,}
    \PY{p}{[}\PY{l+m+mf}{3.330}\PY{p}{,} \PY{l+m+mf}{0.300}\PY{p}{,} \PY{l+m+mf}{0.520}\PY{p}{]}\PY{p}{,}
    \PY{p}{[}\PY{l+m+mf}{4.000}\PY{p}{,} \PY{l+m+mf}{0.250}\PY{p}{,} \PY{l+m+mf}{0.670}\PY{p}{]}\PY{p}{,}
    \PY{p}{[}\PY{l+m+mf}{5.000}\PY{p}{,} \PY{l+m+mf}{0.200}\PY{p}{,} \PY{l+m+mf}{0.840}\PY{p}{]}\PY{p}{,}
    \PY{p}{[}\PY{l+m+mf}{10.000}\PY{p}{,} \PY{l+m+mf}{0.100}\PY{p}{,} \PY{l+m+mf}{1.280}\PY{p}{]}\PY{p}{,}
    \PY{p}{[}\PY{l+m+mf}{20.000}\PY{p}{,} \PY{l+m+mf}{0.050}\PY{p}{,} \PY{l+m+mf}{1.640}\PY{p}{]}\PY{p}{,}
    \PY{p}{[}\PY{l+m+mf}{50.000}\PY{p}{,} \PY{l+m+mf}{0.200}\PY{p}{,} \PY{l+m+mf}{2.050}\PY{p}{]}\PY{p}{,}
    \PY{p}{[}\PY{l+m+mf}{100.000}\PY{p}{,} \PY{l+m+mf}{0.010}\PY{p}{,} \PY{l+m+mf}{2.330}\PY{p}{]}\PY{p}{,}
    \PY{p}{[}\PY{l+m+mf}{200.000}\PY{p}{,} \PY{l+m+mf}{0.005}\PY{p}{,} \PY{l+m+mf}{2.580}\PY{p}{]}\PY{p}{,}
    \PY{p}{[}\PY{l+m+mf}{500.000}\PY{p}{,} \PY{l+m+mf}{0.002}\PY{p}{,} \PY{l+m+mf}{2.880}\PY{p}{]}\PY{p}{,}
    \PY{p}{[}\PY{l+m+mf}{1000.000}\PY{p}{,} \PY{l+m+mf}{0.001}\PY{p}{,} \PY{l+m+mf}{3.090}\PY{p}{]}\PY{p}{,}
\PY{p}{]}

\PY{n}{\PYZus{}COL\PYZus{}SW} \PY{o}{=} \PY{p}{[}\PY{l+s+s1}{\PYZsq{}}\PY{l+s+s1}{periode\PYZus{}ulang}\PY{l+s+s1}{\PYZsq{}}\PY{p}{,} \PY{l+s+s1}{\PYZsq{}}\PY{l+s+s1}{peluang}\PY{l+s+s1}{\PYZsq{}}\PY{p}{,} \PY{l+s+s1}{\PYZsq{}}\PY{l+s+s1}{k}\PY{l+s+s1}{\PYZsq{}}\PY{p}{]}

\PY{n}{t\PYZus{}normal\PYZus{}sw} \PY{o}{=} \PY{n}{pd}\PY{o}{.}\PY{n}{DataFrame}\PY{p}{(}
    \PY{n}{data}\PY{o}{=}\PY{n}{\PYZus{}DATA\PYZus{}SW}\PY{p}{,} \PY{n}{columns}\PY{o}{=}\PY{n}{\PYZus{}COL\PYZus{}SW}
\PY{p}{)}
\PY{n}{t\PYZus{}normal\PYZus{}sw}
\end{Verbatim}
\end{tcolorbox}

            \begin{tcolorbox}[breakable, size=fbox, boxrule=.5pt, pad at break*=1mm, opacityfill=0]
\prompt{Out}{outcolor}{ }{\boxspacing}
\begin{Verbatim}[commandchars=\\\{\}]
    periode\_ulang  peluang     k
0           1.001    0.999 -3.05
1           1.005    0.995 -2.58
2           1.010    0.990 -2.33
3           1.050    0.950 -1.64
4           1.110    0.900 -1.28
5           1.250    0.800 -0.84
6           1.330    0.750 -0.67
7           1.430    0.700 -0.52
8           1.670    0.600 -0.25
9           2.000    0.500  0.00
10          2.500    0.400  0.25
11          3.330    0.300  0.52
12          4.000    0.250  0.67
13          5.000    0.200  0.84
14         10.000    0.100  1.28
15         20.000    0.050  1.64
16         50.000    0.200  2.05
17        100.000    0.010  2.33
18        200.000    0.005  2.58
19        500.000    0.002  2.88
20       1000.000    0.001  3.09
\end{Verbatim}
\end{tcolorbox}
        
    \hypertarget{kode}{%
\section{KODE}\label{kode}}

    \begin{tcolorbox}[breakable, size=fbox, boxrule=1pt, pad at break*=1mm,colback=cellbackground, colframe=cellborder]
\prompt{In}{incolor}{ }{\boxspacing}
\begin{Verbatim}[commandchars=\\\{\}]
\PY{k}{def} \PY{n+nf}{\PYZus{}find\PYZus{}k\PYZus{}in\PYZus{}table}\PY{p}{(}\PY{n}{return\PYZus{}period}\PY{p}{,} \PY{n}{table}\PY{p}{)}\PY{p}{:}
    \PY{n}{x} \PY{o}{=} \PY{n}{table}\PY{o}{.}\PY{n}{periode\PYZus{}ulang}
    \PY{n}{y} \PY{o}{=} \PY{n}{table}\PY{o}{.}\PY{n}{k}
    \PY{k}{return} \PY{n}{np}\PY{o}{.}\PY{n}{interp}\PY{p}{(}\PY{n}{return\PYZus{}period}\PY{p}{,} \PY{n}{x}\PY{p}{,} \PY{n}{y}\PY{p}{)}

\PY{k}{def} \PY{n+nf}{\PYZus{}calc\PYZus{}prob\PYZus{}in\PYZus{}table}\PY{p}{(}\PY{n}{k}\PY{p}{,} \PY{n}{table}\PY{p}{)}\PY{p}{:}
    \PY{n}{x} \PY{o}{=} \PY{n}{table}\PY{o}{.}\PY{n}{k}
    \PY{n}{y} \PY{o}{=} \PY{n}{table}\PY{o}{.}\PY{n}{peluang}
    \PY{k}{return} \PY{n}{np}\PY{o}{.}\PY{n}{interp}\PY{p}{(}\PY{n}{k}\PY{p}{,} \PY{n}{x}\PY{p}{,} \PY{n}{y}\PY{p}{)}
\end{Verbatim}
\end{tcolorbox}

    \begin{tcolorbox}[breakable, size=fbox, boxrule=1pt, pad at break*=1mm,colback=cellbackground, colframe=cellborder]
\prompt{In}{incolor}{ }{\boxspacing}
\begin{Verbatim}[commandchars=\\\{\}]
\PY{k}{def} \PY{n+nf}{find\PYZus{}K}\PY{p}{(}\PY{n}{return\PYZus{}period}\PY{p}{,} \PY{n}{source}\PY{o}{=}\PY{l+s+s1}{\PYZsq{}}\PY{l+s+s1}{scipy}\PY{l+s+s1}{\PYZsq{}}\PY{p}{)}\PY{p}{:}
    \PY{k}{if} \PY{n}{source}\PY{o}{.}\PY{n}{lower}\PY{p}{(}\PY{p}{)} \PY{o}{==} \PY{l+s+s1}{\PYZsq{}}\PY{l+s+s1}{soewarno}\PY{l+s+s1}{\PYZsq{}}\PY{p}{:}
        \PY{k}{return} \PY{n}{\PYZus{}find\PYZus{}k\PYZus{}in\PYZus{}table}\PY{p}{(}\PY{n}{return\PYZus{}period}\PY{p}{,} \PY{n}{t\PYZus{}normal\PYZus{}sw}\PY{p}{)}
    \PY{k}{elif} \PY{n}{source}\PY{o}{.}\PY{n}{lower}\PY{p}{(}\PY{p}{)} \PY{o}{==} \PY{l+s+s1}{\PYZsq{}}\PY{l+s+s1}{scipy}\PY{l+s+s1}{\PYZsq{}}\PY{p}{:}
        \PY{n}{return\PYZus{}period} \PY{o}{=} \PY{n}{np}\PY{o}{.}\PY{n}{array}\PY{p}{(}\PY{n}{return\PYZus{}period}\PY{p}{)}
        \PY{k}{return} \PY{n}{stats}\PY{o}{.}\PY{n}{norm}\PY{o}{.}\PY{n}{ppf}\PY{p}{(}\PY{l+m+mi}{1} \PY{o}{\PYZhy{}} \PY{l+m+mi}{1}\PY{o}{/}\PY{n}{return\PYZus{}period}\PY{p}{)}

\PY{k}{def} \PY{n+nf}{calc\PYZus{}x\PYZus{}lognormal}\PY{p}{(}\PY{n}{x}\PY{p}{,} \PY{n}{return\PYZus{}period}\PY{o}{=}\PY{p}{[}\PY{l+m+mi}{5}\PY{p}{]}\PY{p}{,} \PY{n}{source}\PY{o}{=}\PY{l+s+s1}{\PYZsq{}}\PY{l+s+s1}{scipy}\PY{l+s+s1}{\PYZsq{}}\PY{p}{,} \PY{n}{show\PYZus{}stat}\PY{o}{=}\PY{k+kc}{False}\PY{p}{)}\PY{p}{:}
    \PY{n}{return\PYZus{}period} \PY{o}{=} \PY{n}{np}\PY{o}{.}\PY{n}{array}\PY{p}{(}\PY{n}{return\PYZus{}period}\PY{p}{)}
    \PY{n}{y} \PY{o}{=} \PY{n}{np}\PY{o}{.}\PY{n}{log10}\PY{p}{(}\PY{n}{x}\PY{p}{)}
    \PY{n}{y\PYZus{}mean} \PY{o}{=} \PY{n}{np}\PY{o}{.}\PY{n}{mean}\PY{p}{(}\PY{n}{y}\PY{p}{)}
    \PY{n}{y\PYZus{}std} \PY{o}{=} \PY{n}{np}\PY{o}{.}\PY{n}{std}\PY{p}{(}\PY{n}{y}\PY{p}{,} \PY{n}{ddof}\PY{o}{=}\PY{l+m+mi}{1}\PY{p}{)}
    \PY{n}{n} \PY{o}{=} \PY{n+nb}{len}\PY{p}{(}\PY{n}{y}\PY{p}{)}

    \PY{n}{k} \PY{o}{=} \PY{n}{find\PYZus{}K}\PY{p}{(}\PY{n}{return\PYZus{}period}\PY{p}{,} \PY{n}{source}\PY{o}{=}\PY{n}{source}\PY{p}{)}

    \PY{k}{if} \PY{n}{show\PYZus{}stat}\PY{p}{:}
        \PY{n+nb}{print}\PY{p}{(}\PY{l+s+sa}{f}\PY{l+s+s1}{\PYZsq{}}\PY{l+s+s1}{y\PYZus{}mean = }\PY{l+s+si}{\PYZob{}}\PY{n}{y\PYZus{}mean}\PY{l+s+si}{:}\PY{l+s+s1}{.5f}\PY{l+s+si}{\PYZcb{}}\PY{l+s+s1}{\PYZsq{}}\PY{p}{)}
        \PY{n+nb}{print}\PY{p}{(}\PY{l+s+sa}{f}\PY{l+s+s1}{\PYZsq{}}\PY{l+s+s1}{y\PYZus{}std = }\PY{l+s+si}{\PYZob{}}\PY{n}{y\PYZus{}std}\PY{l+s+si}{:}\PY{l+s+s1}{.5f}\PY{l+s+si}{\PYZcb{}}\PY{l+s+s1}{\PYZsq{}}\PY{p}{)}
        \PY{n+nb}{print}\PY{p}{(}\PY{l+s+sa}{f}\PY{l+s+s1}{\PYZsq{}}\PY{l+s+s1}{k = }\PY{l+s+si}{\PYZob{}}\PY{n}{k}\PY{l+s+si}{\PYZcb{}}\PY{l+s+s1}{\PYZsq{}}\PY{p}{)}

    \PY{n}{val\PYZus{}y} \PY{o}{=} \PY{n}{y\PYZus{}mean} \PY{o}{+} \PY{n}{k} \PY{o}{*} \PY{n}{y\PYZus{}std}
    \PY{n}{val\PYZus{}x} \PY{o}{=} \PY{n}{np}\PY{o}{.}\PY{n}{power}\PY{p}{(}\PY{l+m+mi}{10}\PY{p}{,} \PY{n}{val\PYZus{}y}\PY{p}{)}
    \PY{k}{return} \PY{n}{val\PYZus{}x}

\PY{k}{def} \PY{n+nf}{freq\PYZus{}lognormal}\PY{p}{(}
    \PY{n}{df}\PY{p}{,} \PY{n}{col}\PY{o}{=}\PY{k+kc}{None}\PY{p}{,}
    \PY{n}{return\PYZus{}period}\PY{o}{=}\PY{p}{[}\PY{l+m+mi}{2}\PY{p}{,} \PY{l+m+mi}{5}\PY{p}{,} \PY{l+m+mi}{10}\PY{p}{,} \PY{l+m+mi}{20}\PY{p}{,} \PY{l+m+mi}{25}\PY{p}{,} \PY{l+m+mi}{50}\PY{p}{,} \PY{l+m+mi}{100}\PY{p}{]}\PY{p}{,} \PY{n}{show\PYZus{}stat}\PY{o}{=}\PY{k+kc}{False}\PY{p}{,} \PY{n}{source}\PY{o}{=}\PY{l+s+s1}{\PYZsq{}}\PY{l+s+s1}{scipy}\PY{l+s+s1}{\PYZsq{}}\PY{p}{,}
    \PY{n}{col\PYZus{}name}\PY{o}{=}\PY{l+s+s1}{\PYZsq{}}\PY{l+s+s1}{Log Normal}\PY{l+s+s1}{\PYZsq{}}\PY{p}{,} \PY{n}{index\PYZus{}name}\PY{o}{=}\PY{l+s+s1}{\PYZsq{}}\PY{l+s+s1}{Kala Ulang}\PY{l+s+s1}{\PYZsq{}}\PY{p}{)}\PY{p}{:}

    \PY{n}{col} \PY{o}{=} \PY{n}{df}\PY{o}{.}\PY{n}{columns}\PY{p}{[}\PY{l+m+mi}{0}\PY{p}{]} \PY{k}{if} \PY{n}{col} \PY{o+ow}{is} \PY{k+kc}{None} \PY{k}{else} \PY{n}{col}

    \PY{n}{x} \PY{o}{=} \PY{n}{df}\PY{p}{[}\PY{n}{col}\PY{p}{]}\PY{o}{.}\PY{n}{copy}\PY{p}{(}\PY{p}{)}

    \PY{n}{arr} \PY{o}{=} \PY{n}{calc\PYZus{}x\PYZus{}lognormal}\PY{p}{(}
        \PY{n}{x}\PY{p}{,} \PY{n}{return\PYZus{}period}\PY{o}{=}\PY{n}{return\PYZus{}period}\PY{p}{,} \PY{n}{show\PYZus{}stat}\PY{o}{=}\PY{n}{show\PYZus{}stat}\PY{p}{,} \PY{n}{source}\PY{o}{=}\PY{n}{source}\PY{p}{)}

    \PY{n}{result} \PY{o}{=} \PY{n}{pd}\PY{o}{.}\PY{n}{DataFrame}\PY{p}{(}
        \PY{n}{data}\PY{o}{=}\PY{n}{arr}\PY{p}{,} \PY{n}{index}\PY{o}{=}\PY{n}{return\PYZus{}period}\PY{p}{,} \PY{n}{columns}\PY{o}{=}\PY{p}{[}\PY{n}{col\PYZus{}name}\PY{p}{]}
    \PY{p}{)}

    \PY{n}{result}\PY{o}{.}\PY{n}{index}\PY{o}{.}\PY{n}{name} \PY{o}{=} \PY{n}{index\PYZus{}name}
    \PY{k}{return} \PY{n}{result}

\PY{k}{def} \PY{n+nf}{calc\PYZus{}prob}\PY{p}{(}\PY{n}{k}\PY{p}{,} \PY{n}{source}\PY{o}{=}\PY{l+s+s1}{\PYZsq{}}\PY{l+s+s1}{scipy}\PY{l+s+s1}{\PYZsq{}}\PY{p}{)}\PY{p}{:}
    \PY{k}{if} \PY{n}{source}\PY{o}{.}\PY{n}{lower}\PY{p}{(}\PY{p}{)} \PY{o}{==} \PY{l+s+s1}{\PYZsq{}}\PY{l+s+s1}{soewarno}\PY{l+s+s1}{\PYZsq{}}\PY{p}{:}
        \PY{n}{k} \PY{o}{=} \PY{n}{np}\PY{o}{.}\PY{n}{array}\PY{p}{(}\PY{n}{k}\PY{p}{)}
        \PY{k}{return} \PY{l+m+mi}{1} \PY{o}{\PYZhy{}} \PY{n}{\PYZus{}calc\PYZus{}prob\PYZus{}in\PYZus{}table}\PY{p}{(}\PY{n}{k}\PY{p}{,} \PY{n}{t\PYZus{}normal\PYZus{}sw}\PY{p}{)}
    \PY{k}{elif} \PY{n}{source}\PY{o}{.}\PY{n}{lower}\PY{p}{(}\PY{p}{)} \PY{o}{==} \PY{l+s+s1}{\PYZsq{}}\PY{l+s+s1}{scipy}\PY{l+s+s1}{\PYZsq{}}\PY{p}{:}
        \PY{k}{return} \PY{n}{stats}\PY{o}{.}\PY{n}{norm}\PY{o}{.}\PY{n}{cdf}\PY{p}{(}\PY{n}{k}\PY{p}{)}
\end{Verbatim}
\end{tcolorbox}

    \hypertarget{fungsi}{%
\section{FUNGSI}\label{fungsi}}

    \hypertarget{fungsi-find_kreturn_period-...}{%
\subsection{\texorpdfstring{Fungsi
\texttt{find\_K(return\_period,\ ...)}}{Fungsi find\_K(return\_period, ...)}}\label{fungsi-find_kreturn_period-...}}

Function:
\texttt{find\_K(return\_period,\ source=\textquotesingle{}scipy\textquotesingle{})}

Fungsi \texttt{find\_K(...)} digunakan untuk mencari nilai \(K\) dari
berbagai sumber berdasarkan kala ulang.

\begin{itemize}
\tightlist
\item
  Argumen Posisi:

  \begin{itemize}
  \tightlist
  \item
    \texttt{return\_period}: kala ulang. Dapat diisi dengan
    \emph{scalar} atau \emph{array\_like}.
  \end{itemize}
\item
  Argumen Opsional:

  \begin{itemize}
  \tightlist
  \item
    \texttt{source}: sumber nilai \(k\), \texttt{scipy} (default).
    Sumber yang dapat digunakan antara lain: Soewarno
    (\texttt{\textquotesingle{}soewarno\textquotesingle{}}).
  \end{itemize}
\end{itemize}

    \begin{tcolorbox}[breakable, size=fbox, boxrule=1pt, pad at break*=1mm,colback=cellbackground, colframe=cellborder]
\prompt{In}{incolor}{ }{\boxspacing}
\begin{Verbatim}[commandchars=\\\{\}]
\PY{n}{find\PYZus{}K}\PY{p}{(}\PY{l+m+mi}{10}\PY{p}{)}
\end{Verbatim}
\end{tcolorbox}

            \begin{tcolorbox}[breakable, size=fbox, boxrule=.5pt, pad at break*=1mm, opacityfill=0]
\prompt{Out}{outcolor}{ }{\boxspacing}
\begin{Verbatim}[commandchars=\\\{\}]
1.2815515655446004
\end{Verbatim}
\end{tcolorbox}
        
    \begin{tcolorbox}[breakable, size=fbox, boxrule=1pt, pad at break*=1mm,colback=cellbackground, colframe=cellborder]
\prompt{In}{incolor}{ }{\boxspacing}
\begin{Verbatim}[commandchars=\\\{\}]
\PY{n}{find\PYZus{}K}\PY{p}{(}\PY{p}{[}\PY{l+m+mi}{2}\PY{p}{,} \PY{l+m+mi}{5}\PY{p}{,} \PY{l+m+mi}{10}\PY{p}{]}\PY{p}{,} \PY{n}{source}\PY{o}{=}\PY{l+s+s1}{\PYZsq{}}\PY{l+s+s1}{soewarno}\PY{l+s+s1}{\PYZsq{}}\PY{p}{)}
\end{Verbatim}
\end{tcolorbox}

            \begin{tcolorbox}[breakable, size=fbox, boxrule=.5pt, pad at break*=1mm, opacityfill=0]
\prompt{Out}{outcolor}{ }{\boxspacing}
\begin{Verbatim}[commandchars=\\\{\}]
array([0.  , 0.84, 1.28])
\end{Verbatim}
\end{tcolorbox}
        
    \begin{tcolorbox}[breakable, size=fbox, boxrule=1pt, pad at break*=1mm,colback=cellbackground, colframe=cellborder]
\prompt{In}{incolor}{ }{\boxspacing}
\begin{Verbatim}[commandchars=\\\{\}]
\PY{c+c1}{\PYZsh{} perbandingan antara masing\PYZhy{}masing sumber}

\PY{n}{\PYZus{}rp} \PY{o}{=} \PY{p}{[}\PY{l+m+mi}{2}\PY{p}{,} \PY{l+m+mi}{5}\PY{p}{,} \PY{l+m+mi}{10}\PY{p}{,} \PY{l+m+mi}{15}\PY{p}{,} \PY{l+m+mi}{20}\PY{p}{,} \PY{l+m+mi}{25}\PY{p}{,} \PY{l+m+mi}{27}\PY{p}{,} \PY{l+m+mi}{50}\PY{p}{,} \PY{l+m+mi}{100}\PY{p}{]}
\PY{n}{source\PYZus{}test} \PY{o}{=} \PY{p}{[}\PY{l+s+s1}{\PYZsq{}}\PY{l+s+s1}{soewarno}\PY{l+s+s1}{\PYZsq{}}\PY{p}{,} \PY{l+s+s1}{\PYZsq{}}\PY{l+s+s1}{scipy}\PY{l+s+s1}{\PYZsq{}}\PY{p}{]}

\PY{k}{for} \PY{n}{\PYZus{}source} \PY{o+ow}{in} \PY{n}{source\PYZus{}test}\PY{p}{:}
    \PY{n+nb}{print}\PY{p}{(}\PY{l+s+sa}{f}\PY{l+s+s1}{\PYZsq{}}\PY{l+s+s1}{k }\PY{l+s+si}{\PYZob{}}\PY{n}{\PYZus{}source}\PY{l+s+si}{:}\PY{l+s+s1}{10}\PY{l+s+si}{\PYZcb{}}\PY{l+s+s1}{= }\PY{l+s+si}{\PYZob{}}\PY{n}{find\PYZus{}K}\PY{p}{(}\PY{n}{\PYZus{}rp}\PY{p}{,} \PY{n}{source}\PY{o}{=}\PY{n}{\PYZus{}source}\PY{p}{)}\PY{l+s+si}{\PYZcb{}}\PY{l+s+s1}{\PYZsq{}}\PY{p}{)}
\end{Verbatim}
\end{tcolorbox}

    \begin{Verbatim}[commandchars=\\\{\}]
k soewarno  = [0.         0.84       1.28       1.46       1.64       1.70833333
 1.73566667 2.05       2.33      ]
k scipy     = [0.         0.84162123 1.28155157 1.50108595 1.64485363 1.75068607
 1.78615556 2.05374891 2.32634787]
    \end{Verbatim}

    \hypertarget{fungsi-calc_x_lognormalx-...}{%
\subsection{\texorpdfstring{Fungsi
\texttt{calc\_x\_lognormal(x,\ ...)}}{Fungsi calc\_x\_lognormal(x, ...)}}\label{fungsi-calc_x_lognormalx-...}}

Function:
\texttt{calc\_x\_lognormal(x,\ return\_period={[}5{]},\ source=\textquotesingle{}scipy\textquotesingle{},\ show\_stat=False)}

Fungsi \texttt{calc\_x\_lognormal(...)} digunakan untuk mencari besar
\(X\) berdasarkan kala ulang (\emph{return period}), yang hasilnya dalam
bentuk \texttt{numpy.array}.

\begin{itemize}
\tightlist
\item
  Argumen Posisi:

  \begin{itemize}
  \tightlist
  \item
    \texttt{x}: \emph{array}.
  \end{itemize}
\item
  Argumen Opsional:

  \begin{itemize}
  \tightlist
  \item
    \texttt{return\_period}: kala ulang (tahun), \texttt{{[}5{]}}
    (default).
  \item
    \texttt{source}: sumber nilai \(k\),
    \texttt{\textquotesingle{}scipy\textquotesingle{}} (default). Sumber
    yang dapat digunakan antara lain: Soewarno
    (\texttt{\textquotesingle{}soewarno\textquotesingle{}}).
  \item
    \texttt{show\_stat}: menampilkan parameter statistik. \texttt{False}
    (default).
  \end{itemize}
\end{itemize}

    \begin{tcolorbox}[breakable, size=fbox, boxrule=1pt, pad at break*=1mm,colback=cellbackground, colframe=cellborder]
\prompt{In}{incolor}{ }{\boxspacing}
\begin{Verbatim}[commandchars=\\\{\}]
\PY{n}{calc\PYZus{}x\PYZus{}lognormal}\PY{p}{(}\PY{n}{data}\PY{o}{.}\PY{n}{debit}\PY{p}{)}
\end{Verbatim}
\end{tcolorbox}

            \begin{tcolorbox}[breakable, size=fbox, boxrule=.5pt, pad at break*=1mm, opacityfill=0]
\prompt{Out}{outcolor}{ }{\boxspacing}
\begin{Verbatim}[commandchars=\\\{\}]
array([37.23254445])
\end{Verbatim}
\end{tcolorbox}
        
    \begin{tcolorbox}[breakable, size=fbox, boxrule=1pt, pad at break*=1mm,colback=cellbackground, colframe=cellborder]
\prompt{In}{incolor}{ }{\boxspacing}
\begin{Verbatim}[commandchars=\\\{\}]
\PY{n}{calc\PYZus{}x\PYZus{}lognormal}\PY{p}{(}\PY{n}{data}\PY{o}{.}\PY{n}{debit}\PY{p}{,} \PY{n}{show\PYZus{}stat}\PY{o}{=}\PY{k+kc}{True}\PY{p}{)}
\end{Verbatim}
\end{tcolorbox}

    \begin{Verbatim}[commandchars=\\\{\}]
y\_mean = 1.42638
y\_std = 0.17174
k = [0.84162123]
    \end{Verbatim}

            \begin{tcolorbox}[breakable, size=fbox, boxrule=.5pt, pad at break*=1mm, opacityfill=0]
\prompt{Out}{outcolor}{ }{\boxspacing}
\begin{Verbatim}[commandchars=\\\{\}]
array([37.23254445])
\end{Verbatim}
\end{tcolorbox}
        
    \begin{tcolorbox}[breakable, size=fbox, boxrule=1pt, pad at break*=1mm,colback=cellbackground, colframe=cellborder]
\prompt{In}{incolor}{ }{\boxspacing}
\begin{Verbatim}[commandchars=\\\{\}]
\PY{n}{calc\PYZus{}x\PYZus{}lognormal}\PY{p}{(}\PY{n}{data}\PY{o}{.}\PY{n}{debit}\PY{p}{,} \PY{n}{return\PYZus{}period}\PY{o}{=}\PY{p}{[}\PY{l+m+mi}{5}\PY{p}{,} \PY{l+m+mi}{10}\PY{p}{,} \PY{l+m+mi}{15}\PY{p}{,} \PY{l+m+mi}{20}\PY{p}{,} \PY{l+m+mi}{21}\PY{p}{]}\PY{p}{,} \PY{n}{show\PYZus{}stat}\PY{o}{=}\PY{k+kc}{True}\PY{p}{)}
\end{Verbatim}
\end{tcolorbox}

    \begin{Verbatim}[commandchars=\\\{\}]
y\_mean = 1.42638
y\_std = 0.17174
k = [0.84162123 1.28155157 1.50108595 1.64485363 1.66839119]
    \end{Verbatim}

            \begin{tcolorbox}[breakable, size=fbox, boxrule=.5pt, pad at break*=1mm, opacityfill=0]
\prompt{Out}{outcolor}{ }{\boxspacing}
\begin{Verbatim}[commandchars=\\\{\}]
array([37.23254445, 44.30748222, 48.32593502, 51.15300628, 51.63135738])
\end{Verbatim}
\end{tcolorbox}
        
    \hypertarget{fungsi-freq_lognormaldf-...}{%
\subsection{\texorpdfstring{Fungsi
\texttt{freq\_lognormal(df,\ ...)}}{Fungsi freq\_lognormal(df, ...)}}\label{fungsi-freq_lognormaldf-...}}

Function:
\texttt{freq\_lognormal(df,\ col=None,\ return\_period={[}2,\ 5,\ 10,\ 20,\ 25,\ 50,\ 100{]},\ show\_stat=False,\ source=\textquotesingle{}scipy\textquotesingle{},\ col\_name=\textquotesingle{}Log\ Normal\textquotesingle{})}

Fungsi \texttt{freq\_lognormal(...)} merupakan fungsi kembangan lebih
lanjut dari \texttt{calc\_x\_lognormal(...)} yang menerima input
\texttt{pandas.DataFrame} dan memiliki luaran berupa
\texttt{pandas.DataFrame}.

\begin{itemize}
\tightlist
\item
  Argumen Posisi:

  \begin{itemize}
  \tightlist
  \item
    \texttt{df}: \texttt{pandas.DataFrame}.
  \end{itemize}
\item
  Argumen Opsional:

  \begin{itemize}
  \tightlist
  \item
    \texttt{col}: nama kolom, \texttt{None} (default). Jika tidak diisi
    menggunakan kolom pertama dalam \texttt{df} sebagai data masukan.
  \item
    \texttt{return\_period}: kala ulang (tahun),
    \texttt{{[}2,\ 5,\ 10,\ 20,\ 25,\ 50,\ 100{]}} (default).
  \item
    \texttt{source}: sumber nilai \(k\),
    \texttt{\textquotesingle{}scipy\textquotesingle{}} (default). Sumber
    yang dapat digunakan antara lain: Soewarno
    (\texttt{\textquotesingle{}soewarno\textquotesingle{}}).
  \item
    \texttt{show\_stat}: menampilkan parameter statistik. \texttt{False}
    (default).
  \item
    \texttt{col\_name}: nama kolom luaran, \texttt{Log\ Normal}
    (default).
  \end{itemize}
\end{itemize}

    \begin{tcolorbox}[breakable, size=fbox, boxrule=1pt, pad at break*=1mm,colback=cellbackground, colframe=cellborder]
\prompt{In}{incolor}{ }{\boxspacing}
\begin{Verbatim}[commandchars=\\\{\}]
\PY{n}{freq\PYZus{}lognormal}\PY{p}{(}\PY{n}{data}\PY{p}{)}
\end{Verbatim}
\end{tcolorbox}

            \begin{tcolorbox}[breakable, size=fbox, boxrule=.5pt, pad at break*=1mm, opacityfill=0]
\prompt{Out}{outcolor}{ }{\boxspacing}
\begin{Verbatim}[commandchars=\\\{\}]
            Log Normal
Kala Ulang
2            26.692012
5            37.232544
10           44.307482
20           51.153006
25           53.339262
50           60.130596
100          66.974904
\end{Verbatim}
\end{tcolorbox}
        
    \begin{tcolorbox}[breakable, size=fbox, boxrule=1pt, pad at break*=1mm,colback=cellbackground, colframe=cellborder]
\prompt{In}{incolor}{ }{\boxspacing}
\begin{Verbatim}[commandchars=\\\{\}]
\PY{n}{freq\PYZus{}lognormal}\PY{p}{(}\PY{n}{data}\PY{p}{,} \PY{n}{source}\PY{o}{=}\PY{l+s+s1}{\PYZsq{}}\PY{l+s+s1}{soewarno}\PY{l+s+s1}{\PYZsq{}}\PY{p}{,} \PY{n}{col\PYZus{}name}\PY{o}{=}\PY{l+s+s1}{\PYZsq{}}\PY{l+s+s1}{Log Normal (Soewarno)}\PY{l+s+s1}{\PYZsq{}}\PY{p}{)}
\end{Verbatim}
\end{tcolorbox}

            \begin{tcolorbox}[breakable, size=fbox, boxrule=.5pt, pad at break*=1mm, opacityfill=0]
\prompt{Out}{outcolor}{ }{\boxspacing}
\begin{Verbatim}[commandchars=\\\{\}]
            Log Normal (Soewarno)
Kala Ulang
2                       26.692012
5                       37.208682
10                      44.280305
20                      51.054919
25                      52.453355
50                      60.041518
100                     67.071701
\end{Verbatim}
\end{tcolorbox}
        
    \begin{tcolorbox}[breakable, size=fbox, boxrule=1pt, pad at break*=1mm,colback=cellbackground, colframe=cellborder]
\prompt{In}{incolor}{ }{\boxspacing}
\begin{Verbatim}[commandchars=\\\{\}]
\PY{n}{freq\PYZus{}lognormal}\PY{p}{(}\PY{n}{data}\PY{p}{,} \PY{l+s+s1}{\PYZsq{}}\PY{l+s+s1}{debit}\PY{l+s+s1}{\PYZsq{}}\PY{p}{,} \PY{n}{source}\PY{o}{=}\PY{l+s+s1}{\PYZsq{}}\PY{l+s+s1}{scipy}\PY{l+s+s1}{\PYZsq{}}\PY{p}{,} \PY{n}{col\PYZus{}name}\PY{o}{=}\PY{l+s+sa}{f}\PY{l+s+s1}{\PYZsq{}}\PY{l+s+s1}{Log Normal (scipy)}\PY{l+s+s1}{\PYZsq{}}\PY{p}{,} \PY{n}{show\PYZus{}stat}\PY{o}{=}\PY{k+kc}{True}\PY{p}{)}
\end{Verbatim}
\end{tcolorbox}

    \begin{Verbatim}[commandchars=\\\{\}]
y\_mean = 1.42638
y\_std = 0.17174
k = [0.         0.84162123 1.28155157 1.64485363 1.75068607 2.05374891
 2.32634787]
    \end{Verbatim}

            \begin{tcolorbox}[breakable, size=fbox, boxrule=.5pt, pad at break*=1mm, opacityfill=0]
\prompt{Out}{outcolor}{ }{\boxspacing}
\begin{Verbatim}[commandchars=\\\{\}]
            Log Normal (scipy)
Kala Ulang
2                    26.692012
5                    37.232544
10                   44.307482
20                   51.153006
25                   53.339262
50                   60.130596
100                  66.974904
\end{Verbatim}
\end{tcolorbox}
        
    \begin{tcolorbox}[breakable, size=fbox, boxrule=1pt, pad at break*=1mm,colback=cellbackground, colframe=cellborder]
\prompt{In}{incolor}{ }{\boxspacing}
\begin{Verbatim}[commandchars=\\\{\}]
\PY{n}{\PYZus{}res} \PY{o}{=} \PY{p}{[}\PY{p}{]}

\PY{k}{for} \PY{n}{\PYZus{}s} \PY{o+ow}{in} \PY{p}{[}\PY{l+s+s1}{\PYZsq{}}\PY{l+s+s1}{soewarno}\PY{l+s+s1}{\PYZsq{}}\PY{p}{,} \PY{l+s+s1}{\PYZsq{}}\PY{l+s+s1}{scipy}\PY{l+s+s1}{\PYZsq{}}\PY{p}{]}\PY{p}{:}
    \PY{n}{\PYZus{}res} \PY{o}{+}\PY{o}{=} \PY{p}{[}\PY{n}{freq\PYZus{}lognormal}\PY{p}{(}\PY{n}{data}\PY{p}{,} \PY{l+s+s1}{\PYZsq{}}\PY{l+s+s1}{debit}\PY{l+s+s1}{\PYZsq{}}\PY{p}{,} \PY{n}{source}\PY{o}{=}\PY{n}{\PYZus{}s}\PY{p}{,} \PY{n}{col\PYZus{}name}\PY{o}{=}\PY{l+s+sa}{f}\PY{l+s+s1}{\PYZsq{}}\PY{l+s+s1}{Log Normal (}\PY{l+s+si}{\PYZob{}}\PY{n}{\PYZus{}s}\PY{l+s+si}{\PYZcb{}}\PY{l+s+s1}{)}\PY{l+s+s1}{\PYZsq{}}\PY{p}{)}\PY{p}{]}

\PY{n}{pd}\PY{o}{.}\PY{n}{concat}\PY{p}{(}\PY{n}{\PYZus{}res}\PY{p}{,} \PY{n}{axis}\PY{o}{=}\PY{l+m+mi}{1}\PY{p}{)}
\end{Verbatim}
\end{tcolorbox}

            \begin{tcolorbox}[breakable, size=fbox, boxrule=.5pt, pad at break*=1mm, opacityfill=0]
\prompt{Out}{outcolor}{ }{\boxspacing}
\begin{Verbatim}[commandchars=\\\{\}]
            Log Normal (soewarno)  Log Normal (scipy)
Kala Ulang
2                       26.692012           26.692012
5                       37.208682           37.232544
10                      44.280305           44.307482
20                      51.054919           51.153006
25                      52.453355           53.339262
50                      60.041518           60.130596
100                     67.071701           66.974904
\end{Verbatim}
\end{tcolorbox}
        
    \hypertarget{fungsi-calc_probk-...}{%
\subsection{\texorpdfstring{Fungsi
\texttt{calc\_prob(k,\ ...)}}{Fungsi calc\_prob(k, ...)}}\label{fungsi-calc_probk-...}}

Function:
\texttt{calc\_prob(k,\ source=\textquotesingle{}scipy\textquotesingle{})}

Fungsi \texttt{calc\_prob(...)} digunakan untuk mencari nilai
peluang/probabilitas dari berbagai sumber berdasarkan nilai \(K\).

\begin{itemize}
\tightlist
\item
  Argumen Posisi:

  \begin{itemize}
  \tightlist
  \item
    \texttt{k}: \(K\) (faktor frekuensi).
  \end{itemize}
\item
  Argumen Opsional:

  \begin{itemize}
  \tightlist
  \item
    \texttt{source}: sumber nilai peluang, \texttt{scipy} (default).
    Sumber yang dapat digunakan antara lain: Soewarno
    (\texttt{\textquotesingle{}soewarno\textquotesingle{}}).
  \end{itemize}
\end{itemize}

Catatan: Fungsi ini sama saja dengan yang di modul
\href{https://gist.github.com/taruma/91b9fcd8fb92c12f4ea2639320ead116}{hk172}
atau \href{https://github.com/hidrokit/hidrokit/issues/172}{\#172}
(Anfrek: Normal).

    \begin{tcolorbox}[breakable, size=fbox, boxrule=1pt, pad at break*=1mm,colback=cellbackground, colframe=cellborder]
\prompt{In}{incolor}{ }{\boxspacing}
\begin{Verbatim}[commandchars=\\\{\}]
\PY{n}{calc\PYZus{}prob}\PY{p}{(}\PY{o}{\PYZhy{}}\PY{l+m+mf}{0.25}\PY{p}{)}
\end{Verbatim}
\end{tcolorbox}

            \begin{tcolorbox}[breakable, size=fbox, boxrule=.5pt, pad at break*=1mm, opacityfill=0]
\prompt{Out}{outcolor}{ }{\boxspacing}
\begin{Verbatim}[commandchars=\\\{\}]
0.4012936743170763
\end{Verbatim}
\end{tcolorbox}
        
    \begin{tcolorbox}[breakable, size=fbox, boxrule=1pt, pad at break*=1mm,colback=cellbackground, colframe=cellborder]
\prompt{In}{incolor}{ }{\boxspacing}
\begin{Verbatim}[commandchars=\\\{\}]
\PY{n}{calc\PYZus{}prob}\PY{p}{(}\PY{l+m+mf}{0.52}\PY{p}{,} \PY{n}{source}\PY{o}{=}\PY{l+s+s1}{\PYZsq{}}\PY{l+s+s1}{soewarno}\PY{l+s+s1}{\PYZsq{}}\PY{p}{)}
\end{Verbatim}
\end{tcolorbox}

            \begin{tcolorbox}[breakable, size=fbox, boxrule=.5pt, pad at break*=1mm, opacityfill=0]
\prompt{Out}{outcolor}{ }{\boxspacing}
\begin{Verbatim}[commandchars=\\\{\}]
0.7
\end{Verbatim}
\end{tcolorbox}
        
    \begin{tcolorbox}[breakable, size=fbox, boxrule=1pt, pad at break*=1mm,colback=cellbackground, colframe=cellborder]
\prompt{In}{incolor}{ }{\boxspacing}
\begin{Verbatim}[commandchars=\\\{\}]
\PY{c+c1}{\PYZsh{} perbandingan antara masing\PYZhy{}masing sumber}

\PY{n}{\PYZus{}k} \PY{o}{=} \PY{p}{[}
    \PY{o}{\PYZhy{}}\PY{l+m+mf}{3.09}\PY{p}{,} \PY{o}{\PYZhy{}}\PY{l+m+mf}{2.58}\PY{p}{,} \PY{o}{\PYZhy{}}\PY{l+m+mf}{2.33}\PY{p}{,} \PY{o}{\PYZhy{}}\PY{l+m+mf}{1.67}\PY{p}{,}  \PY{l+m+mf}{0.}  \PY{p}{,}  \PY{l+m+mf}{0.84}\PY{p}{,}  \PY{l+m+mf}{1.28}\PY{p}{,}  \PY{l+m+mf}{1.5} \PY{p}{,}  \PY{l+m+mf}{1.64}\PY{p}{,}
    \PY{l+m+mf}{1.75}\PY{p}{,}  \PY{l+m+mf}{1.79}\PY{p}{,}  \PY{l+m+mf}{2.05}\PY{p}{,}  \PY{l+m+mf}{2.33}
\PY{p}{]}

\PY{n}{source\PYZus{}test} \PY{o}{=} \PY{p}{[}\PY{l+s+s1}{\PYZsq{}}\PY{l+s+s1}{soewarno}\PY{l+s+s1}{\PYZsq{}}\PY{p}{,} \PY{l+s+s1}{\PYZsq{}}\PY{l+s+s1}{scipy}\PY{l+s+s1}{\PYZsq{}}\PY{p}{]}

\PY{k}{for} \PY{n}{\PYZus{}source} \PY{o+ow}{in} \PY{n}{source\PYZus{}test}\PY{p}{:}
    \PY{n+nb}{print}\PY{p}{(}\PY{l+s+sa}{f}\PY{l+s+s1}{\PYZsq{}}\PY{l+s+s1}{prob }\PY{l+s+si}{\PYZob{}}\PY{n}{\PYZus{}source}\PY{l+s+si}{:}\PY{l+s+s1}{10}\PY{l+s+si}{\PYZcb{}}\PY{l+s+s1}{= }\PY{l+s+si}{\PYZob{}}\PY{n}{calc\PYZus{}prob}\PY{p}{(}\PY{n}{\PYZus{}k}\PY{p}{,} \PY{n}{source}\PY{o}{=}\PY{n}{\PYZus{}source}\PY{p}{)}\PY{l+s+si}{\PYZcb{}}\PY{l+s+s1}{\PYZsq{}}\PY{p}{)}
\end{Verbatim}
\end{tcolorbox}

    \begin{Verbatim}[commandchars=\\\{\}]
prob soewarno  = [0.001      0.005      0.01       0.04826087 0.5        0.8
 0.9        0.93055556 0.95       0.9097561  0.89512195 0.8
 0.99      ]
prob scipy     = [0.00100078 0.00494002 0.00990308 0.04745968 0.5
0.79954581
 0.89972743 0.9331928  0.94949742 0.95994084 0.96327304 0.97981778
 0.99009692]
    \end{Verbatim}

    \hypertarget{changelog}{%
\section{Changelog}\label{changelog}}

\begin{verbatim}
- 20220323 - 1.1.0 - tambah argumen index_name="Kala Ulang" pada fungsi freq_lognormal() untuk penamaan index
- 20220316 - 1.0.3 - ubah fungsi _calc_prob(...) (hasil menjadi 1-P)
- 20220315 - 1.0.2 - ubah nama fungsi find_prob -> calc_prob(...)
- 20220314 - 1.0.1 - Tambah Fungsi find_prob(...)
- 20220310 - 1.0.0 - Initial
\end{verbatim}

\hypertarget{copyright-2022-taruma-sakti-megariansyah}{%
\paragraph{\texorpdfstring{Copyright © 2022
\href{https://taruma.github.io}{Taruma Sakti
Megariansyah}}{Copyright © 2022 Taruma Sakti Megariansyah}}\label{copyright-2022-taruma-sakti-megariansyah}}

Source code in this notebook is licensed under a
\href{https://choosealicense.com/licenses/mit/}{MIT License}. Data in
this notebook is licensed under a
\href{https://creativecommons.org/licenses/by/4.0/}{Creative Common
Attribution 4.0 International}.


    % Add a bibliography block to the postdoc
    
    
    
\end{document}
