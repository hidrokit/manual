\documentclass[11pt]{article}

    \usepackage[breakable]{tcolorbox}
    \usepackage{parskip} % Stop auto-indenting (to mimic markdown behaviour)
    
    \usepackage{iftex}
    \ifPDFTeX
    	\usepackage[T1]{fontenc}
    	\usepackage{mathpazo}
    \else
    	\usepackage{fontspec}
    \fi

    % Basic figure setup, for now with no caption control since it's done
    % automatically by Pandoc (which extracts ![](path) syntax from Markdown).
    \usepackage{graphicx}
    % Maintain compatibility with old templates. Remove in nbconvert 6.0
    \let\Oldincludegraphics\includegraphics
    % Ensure that by default, figures have no caption (until we provide a
    % proper Figure object with a Caption API and a way to capture that
    % in the conversion process - todo).
    \usepackage{caption}
    \DeclareCaptionFormat{nocaption}{}
    \captionsetup{format=nocaption,aboveskip=0pt,belowskip=0pt}

    \usepackage{float}
    \floatplacement{figure}{H} % forces figures to be placed at the correct location
    \usepackage{xcolor} % Allow colors to be defined
    \usepackage{enumerate} % Needed for markdown enumerations to work
    \usepackage{geometry} % Used to adjust the document margins
    \usepackage{amsmath} % Equations
    \usepackage{amssymb} % Equations
    \usepackage{textcomp} % defines textquotesingle
    % Hack from http://tex.stackexchange.com/a/47451/13684:
    \AtBeginDocument{%
        \def\PYZsq{\textquotesingle}% Upright quotes in Pygmentized code
    }
    \usepackage{upquote} % Upright quotes for verbatim code
    \usepackage{eurosym} % defines \euro
    \usepackage[mathletters]{ucs} % Extended unicode (utf-8) support
    \usepackage{fancyvrb} % verbatim replacement that allows latex
    \usepackage{grffile} % extends the file name processing of package graphics 
                         % to support a larger range
    \makeatletter % fix for old versions of grffile with XeLaTeX
    \@ifpackagelater{grffile}{2019/11/01}
    {
      % Do nothing on new versions
    }
    {
      \def\Gread@@xetex#1{%
        \IfFileExists{"\Gin@base".bb}%
        {\Gread@eps{\Gin@base.bb}}%
        {\Gread@@xetex@aux#1}%
      }
    }
    \makeatother
    \usepackage[Export]{adjustbox} % Used to constrain images to a maximum size
    \adjustboxset{max size={0.9\linewidth}{0.9\paperheight}}

    % The hyperref package gives us a pdf with properly built
    % internal navigation ('pdf bookmarks' for the table of contents,
    % internal cross-reference links, web links for URLs, etc.)
    \usepackage{hyperref}
    % The default LaTeX title has an obnoxious amount of whitespace. By default,
    % titling removes some of it. It also provides customization options.
    \usepackage{titling}
    \usepackage{longtable} % longtable support required by pandoc >1.10
    \usepackage{booktabs}  % table support for pandoc > 1.12.2
    \usepackage[inline]{enumitem} % IRkernel/repr support (it uses the enumerate* environment)
    \usepackage[normalem]{ulem} % ulem is needed to support strikethroughs (\sout)
                                % normalem makes italics be italics, not underlines
    \usepackage{mathrsfs}
    

    
    % Colors for the hyperref package
    \definecolor{urlcolor}{rgb}{0,.145,.698}
    \definecolor{linkcolor}{rgb}{.71,0.21,0.01}
    \definecolor{citecolor}{rgb}{.12,.54,.11}

    % ANSI colors
    \definecolor{ansi-black}{HTML}{3E424D}
    \definecolor{ansi-black-intense}{HTML}{282C36}
    \definecolor{ansi-red}{HTML}{E75C58}
    \definecolor{ansi-red-intense}{HTML}{B22B31}
    \definecolor{ansi-green}{HTML}{00A250}
    \definecolor{ansi-green-intense}{HTML}{007427}
    \definecolor{ansi-yellow}{HTML}{DDB62B}
    \definecolor{ansi-yellow-intense}{HTML}{B27D12}
    \definecolor{ansi-blue}{HTML}{208FFB}
    \definecolor{ansi-blue-intense}{HTML}{0065CA}
    \definecolor{ansi-magenta}{HTML}{D160C4}
    \definecolor{ansi-magenta-intense}{HTML}{A03196}
    \definecolor{ansi-cyan}{HTML}{60C6C8}
    \definecolor{ansi-cyan-intense}{HTML}{258F8F}
    \definecolor{ansi-white}{HTML}{C5C1B4}
    \definecolor{ansi-white-intense}{HTML}{A1A6B2}
    \definecolor{ansi-default-inverse-fg}{HTML}{FFFFFF}
    \definecolor{ansi-default-inverse-bg}{HTML}{000000}

    % common color for the border for error outputs.
    \definecolor{outerrorbackground}{HTML}{FFDFDF}

    % commands and environments needed by pandoc snippets
    % extracted from the output of `pandoc -s`
    \providecommand{\tightlist}{%
      \setlength{\itemsep}{0pt}\setlength{\parskip}{0pt}}
    \DefineVerbatimEnvironment{Highlighting}{Verbatim}{commandchars=\\\{\}}
    % Add ',fontsize=\small' for more characters per line
    \newenvironment{Shaded}{}{}
    \newcommand{\KeywordTok}[1]{\textcolor[rgb]{0.00,0.44,0.13}{\textbf{{#1}}}}
    \newcommand{\DataTypeTok}[1]{\textcolor[rgb]{0.56,0.13,0.00}{{#1}}}
    \newcommand{\DecValTok}[1]{\textcolor[rgb]{0.25,0.63,0.44}{{#1}}}
    \newcommand{\BaseNTok}[1]{\textcolor[rgb]{0.25,0.63,0.44}{{#1}}}
    \newcommand{\FloatTok}[1]{\textcolor[rgb]{0.25,0.63,0.44}{{#1}}}
    \newcommand{\CharTok}[1]{\textcolor[rgb]{0.25,0.44,0.63}{{#1}}}
    \newcommand{\StringTok}[1]{\textcolor[rgb]{0.25,0.44,0.63}{{#1}}}
    \newcommand{\CommentTok}[1]{\textcolor[rgb]{0.38,0.63,0.69}{\textit{{#1}}}}
    \newcommand{\OtherTok}[1]{\textcolor[rgb]{0.00,0.44,0.13}{{#1}}}
    \newcommand{\AlertTok}[1]{\textcolor[rgb]{1.00,0.00,0.00}{\textbf{{#1}}}}
    \newcommand{\FunctionTok}[1]{\textcolor[rgb]{0.02,0.16,0.49}{{#1}}}
    \newcommand{\RegionMarkerTok}[1]{{#1}}
    \newcommand{\ErrorTok}[1]{\textcolor[rgb]{1.00,0.00,0.00}{\textbf{{#1}}}}
    \newcommand{\NormalTok}[1]{{#1}}
    
    % Additional commands for more recent versions of Pandoc
    \newcommand{\ConstantTok}[1]{\textcolor[rgb]{0.53,0.00,0.00}{{#1}}}
    \newcommand{\SpecialCharTok}[1]{\textcolor[rgb]{0.25,0.44,0.63}{{#1}}}
    \newcommand{\VerbatimStringTok}[1]{\textcolor[rgb]{0.25,0.44,0.63}{{#1}}}
    \newcommand{\SpecialStringTok}[1]{\textcolor[rgb]{0.73,0.40,0.53}{{#1}}}
    \newcommand{\ImportTok}[1]{{#1}}
    \newcommand{\DocumentationTok}[1]{\textcolor[rgb]{0.73,0.13,0.13}{\textit{{#1}}}}
    \newcommand{\AnnotationTok}[1]{\textcolor[rgb]{0.38,0.63,0.69}{\textbf{\textit{{#1}}}}}
    \newcommand{\CommentVarTok}[1]{\textcolor[rgb]{0.38,0.63,0.69}{\textbf{\textit{{#1}}}}}
    \newcommand{\VariableTok}[1]{\textcolor[rgb]{0.10,0.09,0.49}{{#1}}}
    \newcommand{\ControlFlowTok}[1]{\textcolor[rgb]{0.00,0.44,0.13}{\textbf{{#1}}}}
    \newcommand{\OperatorTok}[1]{\textcolor[rgb]{0.40,0.40,0.40}{{#1}}}
    \newcommand{\BuiltInTok}[1]{{#1}}
    \newcommand{\ExtensionTok}[1]{{#1}}
    \newcommand{\PreprocessorTok}[1]{\textcolor[rgb]{0.74,0.48,0.00}{{#1}}}
    \newcommand{\AttributeTok}[1]{\textcolor[rgb]{0.49,0.56,0.16}{{#1}}}
    \newcommand{\InformationTok}[1]{\textcolor[rgb]{0.38,0.63,0.69}{\textbf{\textit{{#1}}}}}
    \newcommand{\WarningTok}[1]{\textcolor[rgb]{0.38,0.63,0.69}{\textbf{\textit{{#1}}}}}
    
    
    % Define a nice break command that doesn't care if a line doesn't already
    % exist.
    \def\br{\hspace*{\fill} \\* }
    % Math Jax compatibility definitions
    \def\gt{>}
    \def\lt{<}
    \let\Oldtex\TeX
    \let\Oldlatex\LaTeX
    \renewcommand{\TeX}{\textrm{\Oldtex}}
    \renewcommand{\LaTeX}{\textrm{\Oldlatex}}
    % Document parameters
    % Document title
    \title{taruma\_0\_3\_2\_hk73\_bmkg}
    
    
    
    
    
% Pygments definitions
\makeatletter
\def\PY@reset{\let\PY@it=\relax \let\PY@bf=\relax%
    \let\PY@ul=\relax \let\PY@tc=\relax%
    \let\PY@bc=\relax \let\PY@ff=\relax}
\def\PY@tok#1{\csname PY@tok@#1\endcsname}
\def\PY@toks#1+{\ifx\relax#1\empty\else%
    \PY@tok{#1}\expandafter\PY@toks\fi}
\def\PY@do#1{\PY@bc{\PY@tc{\PY@ul{%
    \PY@it{\PY@bf{\PY@ff{#1}}}}}}}
\def\PY#1#2{\PY@reset\PY@toks#1+\relax+\PY@do{#2}}

\@namedef{PY@tok@w}{\def\PY@tc##1{\textcolor[rgb]{0.73,0.73,0.73}{##1}}}
\@namedef{PY@tok@c}{\let\PY@it=\textit\def\PY@tc##1{\textcolor[rgb]{0.24,0.48,0.48}{##1}}}
\@namedef{PY@tok@cp}{\def\PY@tc##1{\textcolor[rgb]{0.61,0.40,0.00}{##1}}}
\@namedef{PY@tok@k}{\let\PY@bf=\textbf\def\PY@tc##1{\textcolor[rgb]{0.00,0.50,0.00}{##1}}}
\@namedef{PY@tok@kp}{\def\PY@tc##1{\textcolor[rgb]{0.00,0.50,0.00}{##1}}}
\@namedef{PY@tok@kt}{\def\PY@tc##1{\textcolor[rgb]{0.69,0.00,0.25}{##1}}}
\@namedef{PY@tok@o}{\def\PY@tc##1{\textcolor[rgb]{0.40,0.40,0.40}{##1}}}
\@namedef{PY@tok@ow}{\let\PY@bf=\textbf\def\PY@tc##1{\textcolor[rgb]{0.67,0.13,1.00}{##1}}}
\@namedef{PY@tok@nb}{\def\PY@tc##1{\textcolor[rgb]{0.00,0.50,0.00}{##1}}}
\@namedef{PY@tok@nf}{\def\PY@tc##1{\textcolor[rgb]{0.00,0.00,1.00}{##1}}}
\@namedef{PY@tok@nc}{\let\PY@bf=\textbf\def\PY@tc##1{\textcolor[rgb]{0.00,0.00,1.00}{##1}}}
\@namedef{PY@tok@nn}{\let\PY@bf=\textbf\def\PY@tc##1{\textcolor[rgb]{0.00,0.00,1.00}{##1}}}
\@namedef{PY@tok@ne}{\let\PY@bf=\textbf\def\PY@tc##1{\textcolor[rgb]{0.80,0.25,0.22}{##1}}}
\@namedef{PY@tok@nv}{\def\PY@tc##1{\textcolor[rgb]{0.10,0.09,0.49}{##1}}}
\@namedef{PY@tok@no}{\def\PY@tc##1{\textcolor[rgb]{0.53,0.00,0.00}{##1}}}
\@namedef{PY@tok@nl}{\def\PY@tc##1{\textcolor[rgb]{0.46,0.46,0.00}{##1}}}
\@namedef{PY@tok@ni}{\let\PY@bf=\textbf\def\PY@tc##1{\textcolor[rgb]{0.44,0.44,0.44}{##1}}}
\@namedef{PY@tok@na}{\def\PY@tc##1{\textcolor[rgb]{0.41,0.47,0.13}{##1}}}
\@namedef{PY@tok@nt}{\let\PY@bf=\textbf\def\PY@tc##1{\textcolor[rgb]{0.00,0.50,0.00}{##1}}}
\@namedef{PY@tok@nd}{\def\PY@tc##1{\textcolor[rgb]{0.67,0.13,1.00}{##1}}}
\@namedef{PY@tok@s}{\def\PY@tc##1{\textcolor[rgb]{0.73,0.13,0.13}{##1}}}
\@namedef{PY@tok@sd}{\let\PY@it=\textit\def\PY@tc##1{\textcolor[rgb]{0.73,0.13,0.13}{##1}}}
\@namedef{PY@tok@si}{\let\PY@bf=\textbf\def\PY@tc##1{\textcolor[rgb]{0.64,0.35,0.47}{##1}}}
\@namedef{PY@tok@se}{\let\PY@bf=\textbf\def\PY@tc##1{\textcolor[rgb]{0.67,0.36,0.12}{##1}}}
\@namedef{PY@tok@sr}{\def\PY@tc##1{\textcolor[rgb]{0.64,0.35,0.47}{##1}}}
\@namedef{PY@tok@ss}{\def\PY@tc##1{\textcolor[rgb]{0.10,0.09,0.49}{##1}}}
\@namedef{PY@tok@sx}{\def\PY@tc##1{\textcolor[rgb]{0.00,0.50,0.00}{##1}}}
\@namedef{PY@tok@m}{\def\PY@tc##1{\textcolor[rgb]{0.40,0.40,0.40}{##1}}}
\@namedef{PY@tok@gh}{\let\PY@bf=\textbf\def\PY@tc##1{\textcolor[rgb]{0.00,0.00,0.50}{##1}}}
\@namedef{PY@tok@gu}{\let\PY@bf=\textbf\def\PY@tc##1{\textcolor[rgb]{0.50,0.00,0.50}{##1}}}
\@namedef{PY@tok@gd}{\def\PY@tc##1{\textcolor[rgb]{0.63,0.00,0.00}{##1}}}
\@namedef{PY@tok@gi}{\def\PY@tc##1{\textcolor[rgb]{0.00,0.52,0.00}{##1}}}
\@namedef{PY@tok@gr}{\def\PY@tc##1{\textcolor[rgb]{0.89,0.00,0.00}{##1}}}
\@namedef{PY@tok@ge}{\let\PY@it=\textit}
\@namedef{PY@tok@gs}{\let\PY@bf=\textbf}
\@namedef{PY@tok@gp}{\let\PY@bf=\textbf\def\PY@tc##1{\textcolor[rgb]{0.00,0.00,0.50}{##1}}}
\@namedef{PY@tok@go}{\def\PY@tc##1{\textcolor[rgb]{0.44,0.44,0.44}{##1}}}
\@namedef{PY@tok@gt}{\def\PY@tc##1{\textcolor[rgb]{0.00,0.27,0.87}{##1}}}
\@namedef{PY@tok@err}{\def\PY@bc##1{{\setlength{\fboxsep}{\string -\fboxrule}\fcolorbox[rgb]{1.00,0.00,0.00}{1,1,1}{\strut ##1}}}}
\@namedef{PY@tok@kc}{\let\PY@bf=\textbf\def\PY@tc##1{\textcolor[rgb]{0.00,0.50,0.00}{##1}}}
\@namedef{PY@tok@kd}{\let\PY@bf=\textbf\def\PY@tc##1{\textcolor[rgb]{0.00,0.50,0.00}{##1}}}
\@namedef{PY@tok@kn}{\let\PY@bf=\textbf\def\PY@tc##1{\textcolor[rgb]{0.00,0.50,0.00}{##1}}}
\@namedef{PY@tok@kr}{\let\PY@bf=\textbf\def\PY@tc##1{\textcolor[rgb]{0.00,0.50,0.00}{##1}}}
\@namedef{PY@tok@bp}{\def\PY@tc##1{\textcolor[rgb]{0.00,0.50,0.00}{##1}}}
\@namedef{PY@tok@fm}{\def\PY@tc##1{\textcolor[rgb]{0.00,0.00,1.00}{##1}}}
\@namedef{PY@tok@vc}{\def\PY@tc##1{\textcolor[rgb]{0.10,0.09,0.49}{##1}}}
\@namedef{PY@tok@vg}{\def\PY@tc##1{\textcolor[rgb]{0.10,0.09,0.49}{##1}}}
\@namedef{PY@tok@vi}{\def\PY@tc##1{\textcolor[rgb]{0.10,0.09,0.49}{##1}}}
\@namedef{PY@tok@vm}{\def\PY@tc##1{\textcolor[rgb]{0.10,0.09,0.49}{##1}}}
\@namedef{PY@tok@sa}{\def\PY@tc##1{\textcolor[rgb]{0.73,0.13,0.13}{##1}}}
\@namedef{PY@tok@sb}{\def\PY@tc##1{\textcolor[rgb]{0.73,0.13,0.13}{##1}}}
\@namedef{PY@tok@sc}{\def\PY@tc##1{\textcolor[rgb]{0.73,0.13,0.13}{##1}}}
\@namedef{PY@tok@dl}{\def\PY@tc##1{\textcolor[rgb]{0.73,0.13,0.13}{##1}}}
\@namedef{PY@tok@s2}{\def\PY@tc##1{\textcolor[rgb]{0.73,0.13,0.13}{##1}}}
\@namedef{PY@tok@sh}{\def\PY@tc##1{\textcolor[rgb]{0.73,0.13,0.13}{##1}}}
\@namedef{PY@tok@s1}{\def\PY@tc##1{\textcolor[rgb]{0.73,0.13,0.13}{##1}}}
\@namedef{PY@tok@mb}{\def\PY@tc##1{\textcolor[rgb]{0.40,0.40,0.40}{##1}}}
\@namedef{PY@tok@mf}{\def\PY@tc##1{\textcolor[rgb]{0.40,0.40,0.40}{##1}}}
\@namedef{PY@tok@mh}{\def\PY@tc##1{\textcolor[rgb]{0.40,0.40,0.40}{##1}}}
\@namedef{PY@tok@mi}{\def\PY@tc##1{\textcolor[rgb]{0.40,0.40,0.40}{##1}}}
\@namedef{PY@tok@il}{\def\PY@tc##1{\textcolor[rgb]{0.40,0.40,0.40}{##1}}}
\@namedef{PY@tok@mo}{\def\PY@tc##1{\textcolor[rgb]{0.40,0.40,0.40}{##1}}}
\@namedef{PY@tok@ch}{\let\PY@it=\textit\def\PY@tc##1{\textcolor[rgb]{0.24,0.48,0.48}{##1}}}
\@namedef{PY@tok@cm}{\let\PY@it=\textit\def\PY@tc##1{\textcolor[rgb]{0.24,0.48,0.48}{##1}}}
\@namedef{PY@tok@cpf}{\let\PY@it=\textit\def\PY@tc##1{\textcolor[rgb]{0.24,0.48,0.48}{##1}}}
\@namedef{PY@tok@c1}{\let\PY@it=\textit\def\PY@tc##1{\textcolor[rgb]{0.24,0.48,0.48}{##1}}}
\@namedef{PY@tok@cs}{\let\PY@it=\textit\def\PY@tc##1{\textcolor[rgb]{0.24,0.48,0.48}{##1}}}

\def\PYZbs{\char`\\}
\def\PYZus{\char`\_}
\def\PYZob{\char`\{}
\def\PYZcb{\char`\}}
\def\PYZca{\char`\^}
\def\PYZam{\char`\&}
\def\PYZlt{\char`\<}
\def\PYZgt{\char`\>}
\def\PYZsh{\char`\#}
\def\PYZpc{\char`\%}
\def\PYZdl{\char`\$}
\def\PYZhy{\char`\-}
\def\PYZsq{\char`\'}
\def\PYZdq{\char`\"}
\def\PYZti{\char`\~}
% for compatibility with earlier versions
\def\PYZat{@}
\def\PYZlb{[}
\def\PYZrb{]}
\makeatother


    % For linebreaks inside Verbatim environment from package fancyvrb. 
    \makeatletter
        \newbox\Wrappedcontinuationbox 
        \newbox\Wrappedvisiblespacebox 
        \newcommand*\Wrappedvisiblespace {\textcolor{red}{\textvisiblespace}} 
        \newcommand*\Wrappedcontinuationsymbol {\textcolor{red}{\llap{\tiny$\m@th\hookrightarrow$}}} 
        \newcommand*\Wrappedcontinuationindent {3ex } 
        \newcommand*\Wrappedafterbreak {\kern\Wrappedcontinuationindent\copy\Wrappedcontinuationbox} 
        % Take advantage of the already applied Pygments mark-up to insert 
        % potential linebreaks for TeX processing. 
        %        {, <, #, %, $, ' and ": go to next line. 
        %        _, }, ^, &, >, - and ~: stay at end of broken line. 
        % Use of \textquotesingle for straight quote. 
        \newcommand*\Wrappedbreaksatspecials {% 
            \def\PYGZus{\discretionary{\char`\_}{\Wrappedafterbreak}{\char`\_}}% 
            \def\PYGZob{\discretionary{}{\Wrappedafterbreak\char`\{}{\char`\{}}% 
            \def\PYGZcb{\discretionary{\char`\}}{\Wrappedafterbreak}{\char`\}}}% 
            \def\PYGZca{\discretionary{\char`\^}{\Wrappedafterbreak}{\char`\^}}% 
            \def\PYGZam{\discretionary{\char`\&}{\Wrappedafterbreak}{\char`\&}}% 
            \def\PYGZlt{\discretionary{}{\Wrappedafterbreak\char`\<}{\char`\<}}% 
            \def\PYGZgt{\discretionary{\char`\>}{\Wrappedafterbreak}{\char`\>}}% 
            \def\PYGZsh{\discretionary{}{\Wrappedafterbreak\char`\#}{\char`\#}}% 
            \def\PYGZpc{\discretionary{}{\Wrappedafterbreak\char`\%}{\char`\%}}% 
            \def\PYGZdl{\discretionary{}{\Wrappedafterbreak\char`\$}{\char`\$}}% 
            \def\PYGZhy{\discretionary{\char`\-}{\Wrappedafterbreak}{\char`\-}}% 
            \def\PYGZsq{\discretionary{}{\Wrappedafterbreak\textquotesingle}{\textquotesingle}}% 
            \def\PYGZdq{\discretionary{}{\Wrappedafterbreak\char`\"}{\char`\"}}% 
            \def\PYGZti{\discretionary{\char`\~}{\Wrappedafterbreak}{\char`\~}}% 
        } 
        % Some characters . , ; ? ! / are not pygmentized. 
        % This macro makes them "active" and they will insert potential linebreaks 
        \newcommand*\Wrappedbreaksatpunct {% 
            \lccode`\~`\.\lowercase{\def~}{\discretionary{\hbox{\char`\.}}{\Wrappedafterbreak}{\hbox{\char`\.}}}% 
            \lccode`\~`\,\lowercase{\def~}{\discretionary{\hbox{\char`\,}}{\Wrappedafterbreak}{\hbox{\char`\,}}}% 
            \lccode`\~`\;\lowercase{\def~}{\discretionary{\hbox{\char`\;}}{\Wrappedafterbreak}{\hbox{\char`\;}}}% 
            \lccode`\~`\:\lowercase{\def~}{\discretionary{\hbox{\char`\:}}{\Wrappedafterbreak}{\hbox{\char`\:}}}% 
            \lccode`\~`\?\lowercase{\def~}{\discretionary{\hbox{\char`\?}}{\Wrappedafterbreak}{\hbox{\char`\?}}}% 
            \lccode`\~`\!\lowercase{\def~}{\discretionary{\hbox{\char`\!}}{\Wrappedafterbreak}{\hbox{\char`\!}}}% 
            \lccode`\~`\/\lowercase{\def~}{\discretionary{\hbox{\char`\/}}{\Wrappedafterbreak}{\hbox{\char`\/}}}% 
            \catcode`\.\active
            \catcode`\,\active 
            \catcode`\;\active
            \catcode`\:\active
            \catcode`\?\active
            \catcode`\!\active
            \catcode`\/\active 
            \lccode`\~`\~ 	
        }
    \makeatother

    \let\OriginalVerbatim=\Verbatim
    \makeatletter
    \renewcommand{\Verbatim}[1][1]{%
        %\parskip\z@skip
        \sbox\Wrappedcontinuationbox {\Wrappedcontinuationsymbol}%
        \sbox\Wrappedvisiblespacebox {\FV@SetupFont\Wrappedvisiblespace}%
        \def\FancyVerbFormatLine ##1{\hsize\linewidth
            \vtop{\raggedright\hyphenpenalty\z@\exhyphenpenalty\z@
                \doublehyphendemerits\z@\finalhyphendemerits\z@
                \strut ##1\strut}%
        }%
        % If the linebreak is at a space, the latter will be displayed as visible
        % space at end of first line, and a continuation symbol starts next line.
        % Stretch/shrink are however usually zero for typewriter font.
        \def\FV@Space {%
            \nobreak\hskip\z@ plus\fontdimen3\font minus\fontdimen4\font
            \discretionary{\copy\Wrappedvisiblespacebox}{\Wrappedafterbreak}
            {\kern\fontdimen2\font}%
        }%
        
        % Allow breaks at special characters using \PYG... macros.
        \Wrappedbreaksatspecials
        % Breaks at punctuation characters . , ; ? ! and / need catcode=\active 	
        \OriginalVerbatim[#1,codes*=\Wrappedbreaksatpunct]%
    }
    \makeatother

    % Exact colors from NB
    \definecolor{incolor}{HTML}{303F9F}
    \definecolor{outcolor}{HTML}{D84315}
    \definecolor{cellborder}{HTML}{CFCFCF}
    \definecolor{cellbackground}{HTML}{F7F7F7}
    
    % prompt
    \makeatletter
    \newcommand{\boxspacing}{\kern\kvtcb@left@rule\kern\kvtcb@boxsep}
    \makeatother
    \newcommand{\prompt}[4]{
        {\ttfamily\llap{{\color{#2}[#3]:\hspace{3pt}#4}}\vspace{-\baselineskip}}
    }
    

    
    % Prevent overflowing lines due to hard-to-break entities
    \sloppy 
    % Setup hyperref package
    \hypersetup{
      breaklinks=true,  % so long urls are correctly broken across lines
      colorlinks=true,
      urlcolor=urlcolor,
      linkcolor=linkcolor,
      citecolor=citecolor,
      }
    % Slightly bigger margins than the latex defaults
    
    \geometry{verbose,tmargin=1in,bmargin=1in,lmargin=1in,rmargin=1in}
    
    

\begin{document}
    
    \maketitle
    
    

    
    Berdasarkan isu
\href{https://github.com/taruma/hidrokit/issues/73}{\#73}:
\textbf{request: mengolah berkas dari data bmkg}

Deskripsi: - mengolah berkas excel yang diperoleh dari data online bmkg
untuk siap dipakai - memeriksa kondisi data

Fungsi yang diharapkan:

\textbf{Umum / General} - Memeriksa apakah data lengkap atau tidak? Jika
tidak, data apa dan pada tanggal berapa? - Memeriksa apakah data tidak
ada data / tidak ada pengukuran (9999) atau data tidak diukur (8888)?
Jika ada, data apa dan pada tanggal berapa? - Menampilkan ``potongan''
baris yang tidak memiliki data / tidak melakukan pengukuran?

    \hypertarget{dataset}{%
\section{DATASET}\label{dataset}}

    \begin{tcolorbox}[breakable, size=fbox, boxrule=1pt, pad at break*=1mm,colback=cellbackground, colframe=cellborder]
\prompt{In}{incolor}{ }{\boxspacing}
\begin{Verbatim}[commandchars=\\\{\}]
\PY{c+c1}{\PYZsh{} AKSES GOOGLE DRIVE }
\PY{k+kn}{from} \PY{n+nn}{google}\PY{n+nn}{.}\PY{n+nn}{colab} \PY{k+kn}{import} \PY{n}{drive}
\PY{n}{drive}\PY{o}{.}\PY{n}{mount}\PY{p}{(}\PY{l+s+s1}{\PYZsq{}}\PY{l+s+s1}{/content/gdrive}\PY{l+s+s1}{\PYZsq{}}\PY{p}{)}
\end{Verbatim}
\end{tcolorbox}

    \begin{Verbatim}[commandchars=\\\{\}]
Drive already mounted at /content/gdrive; to attempt to forcibly remount, call
drive.mount("/content/gdrive", force\_remount=True).
    \end{Verbatim}

    \begin{tcolorbox}[breakable, size=fbox, boxrule=1pt, pad at break*=1mm,colback=cellbackground, colframe=cellborder]
\prompt{In}{incolor}{ }{\boxspacing}
\begin{Verbatim}[commandchars=\\\{\}]
\PY{c+c1}{\PYZsh{} DRIVE PATH}
\PY{n}{DRIVE\PYZus{}DROP\PYZus{}PATH} \PY{o}{=} \PY{l+s+s1}{\PYZsq{}}\PY{l+s+s1}{/content/gdrive/My Drive/Colab Notebooks/\PYZus{}dropbox}\PY{l+s+s1}{\PYZsq{}}
\PY{n}{DRIVE\PYZus{}DATASET\PYZus{}PATH} \PY{o}{=} \PY{l+s+s1}{\PYZsq{}}\PY{l+s+s1}{/content/gdrive/My Drive/Colab Notebooks/\PYZus{}dataset/uma\PYZus{}pamarayan}\PY{l+s+s1}{\PYZsq{}}
\end{Verbatim}
\end{tcolorbox}

    \begin{tcolorbox}[breakable, size=fbox, boxrule=1pt, pad at break*=1mm,colback=cellbackground, colframe=cellborder]
\prompt{In}{incolor}{ }{\boxspacing}
\begin{Verbatim}[commandchars=\\\{\}]
\PY{n}{DATASET\PYZus{}PATH} \PY{o}{=} \PY{n}{DRIVE\PYZus{}DATASET\PYZus{}PATH} \PY{o}{+} \PY{l+s+s1}{\PYZsq{}}\PY{l+s+s1}{/klimatologi\PYZus{}geofisika\PYZus{}tangerang\PYZus{}1998\PYZus{}2009.xlsx}\PY{l+s+s1}{\PYZsq{}}
\end{Verbatim}
\end{tcolorbox}

    \hypertarget{fungsi}{%
\section{FUNGSI}\label{fungsi}}

    \begin{tcolorbox}[breakable, size=fbox, boxrule=1pt, pad at break*=1mm,colback=cellbackground, colframe=cellborder]
\prompt{In}{incolor}{ }{\boxspacing}
\begin{Verbatim}[commandchars=\\\{\}]
\PY{k+kn}{import} \PY{n+nn}{pandas} \PY{k}{as} \PY{n+nn}{pd}
\PY{k+kn}{import} \PY{n+nn}{numpy} \PY{k}{as} \PY{n+nn}{np}
\PY{k+kn}{from} \PY{n+nn}{operator} \PY{k+kn}{import} \PY{n}{itemgetter}
\PY{k+kn}{from} \PY{n+nn}{itertools} \PY{k+kn}{import} \PY{n}{groupby}

\PY{k}{def} \PY{n+nf}{\PYZus{}read\PYZus{}bmkg}\PY{p}{(}\PY{n}{io}\PY{p}{)}\PY{p}{:}
    \PY{k}{return} \PY{n}{pd}\PY{o}{.}\PY{n}{read\PYZus{}excel}\PY{p}{(}
        \PY{n}{io}\PY{p}{,} \PY{n}{skiprows}\PY{o}{=}\PY{l+m+mi}{8}\PY{p}{,} \PY{n}{skipfooter}\PY{o}{=}\PY{l+m+mi}{16}\PY{p}{,} \PY{n}{header}\PY{o}{=}\PY{l+m+mi}{0}\PY{p}{,} \PY{n}{index\PYZus{}col}\PY{o}{=}\PY{l+m+mi}{0}\PY{p}{,} \PY{n}{parse\PYZus{}dates}\PY{o}{=}\PY{k+kc}{True}\PY{p}{,}
        \PY{n}{date\PYZus{}parser}\PY{o}{=}\PY{k}{lambda} \PY{n}{x}\PY{p}{:} \PY{n}{pd}\PY{o}{.}\PY{n}{to\PYZus{}datetime}\PY{p}{(}\PY{n}{x}\PY{p}{,} \PY{n+nb}{format}\PY{o}{=}\PY{l+s+s1}{\PYZsq{}}\PY{l+s+si}{\PYZpc{}d}\PY{l+s+s1}{\PYZhy{}}\PY{l+s+s1}{\PYZpc{}}\PY{l+s+s1}{m\PYZhy{}}\PY{l+s+s1}{\PYZpc{}}\PY{l+s+s1}{Y}\PY{l+s+s1}{\PYZsq{}}\PY{p}{)}
    \PY{p}{)}

\PY{k}{def} \PY{n+nf}{\PYZus{}have\PYZus{}nan}\PY{p}{(}\PY{n}{dataset}\PY{p}{)}\PY{p}{:}
    \PY{k}{if} \PY{n}{dataset}\PY{o}{.}\PY{n}{isna}\PY{p}{(}\PY{p}{)}\PY{o}{.}\PY{n}{any}\PY{p}{(}\PY{p}{)}\PY{o}{.}\PY{n}{any}\PY{p}{(}\PY{p}{)}\PY{p}{:}
        \PY{k}{return} \PY{k+kc}{True}
    \PY{k}{else}\PY{p}{:}
        \PY{k}{return} \PY{k+kc}{False}

\PY{k}{def} \PY{n+nf}{\PYZus{}get\PYZus{}index1D}\PY{p}{(}\PY{n}{array1D\PYZus{}bool}\PY{p}{)}\PY{p}{:}
    \PY{k}{return} \PY{n}{np}\PY{o}{.}\PY{n}{argwhere}\PY{p}{(}\PY{n}{array1D\PYZus{}bool}\PY{p}{)}\PY{o}{.}\PY{n}{reshape}\PY{p}{(}\PY{o}{\PYZhy{}}\PY{l+m+mi}{1}\PY{p}{,}\PY{p}{)}

\PY{k}{def} \PY{n+nf}{\PYZus{}get\PYZus{}nan}\PY{p}{(}\PY{n}{dataset}\PY{p}{)}\PY{p}{:}
    \PY{n}{nan} \PY{o}{=} \PY{p}{\PYZob{}}\PY{p}{\PYZcb{}}

    \PY{k}{for} \PY{n}{col} \PY{o+ow}{in} \PY{n}{dataset}\PY{o}{.}\PY{n}{columns}\PY{p}{:}
        \PY{n}{nan}\PY{p}{[}\PY{n}{col}\PY{p}{]} \PY{o}{=} \PY{n}{\PYZus{}get\PYZus{}index1D}\PY{p}{(}\PY{n}{dataset}\PY{p}{[}\PY{n}{col}\PY{p}{]}\PY{o}{.}\PY{n}{isna}\PY{p}{(}\PY{p}{)}\PY{o}{.}\PY{n}{values}\PY{p}{)}\PY{o}{.}\PY{n}{tolist}\PY{p}{(}\PY{p}{)}

    \PY{k}{return} \PY{n}{nan}

\PY{k}{def} \PY{n+nf}{\PYZus{}get\PYZus{}missing}\PY{p}{(}\PY{n}{dataset}\PY{p}{)}\PY{p}{:}
    \PY{n}{missing} \PY{o}{=} \PY{p}{\PYZob{}}\PY{p}{\PYZcb{}}

    \PY{k}{for} \PY{n}{col} \PY{o+ow}{in} \PY{n}{dataset}\PY{o}{.}\PY{n}{columns}\PY{p}{:}
        \PY{n}{masking} \PY{o}{=} \PY{p}{(}\PY{n}{dataset}\PY{p}{[}\PY{n}{col}\PY{p}{]} \PY{o}{==} \PY{l+m+mi}{8888}\PY{p}{)} \PY{o}{|} \PY{p}{(}\PY{n}{dataset}\PY{p}{[}\PY{n}{col}\PY{p}{]} \PY{o}{==} \PY{l+m+mi}{9999}\PY{p}{)}
        \PY{n}{missing}\PY{p}{[}\PY{n}{col}\PY{p}{]} \PY{o}{=} \PY{n}{\PYZus{}get\PYZus{}index1D}\PY{p}{(}\PY{n}{masking}\PY{o}{.}\PY{n}{values}\PY{p}{)}
    
    \PY{k}{return} \PY{n}{missing}

\PY{k}{def} \PY{n+nf}{\PYZus{}check\PYZus{}nan}\PY{p}{(}\PY{n}{dataset}\PY{p}{)}\PY{p}{:}
    \PY{k}{if} \PY{n}{\PYZus{}have\PYZus{}nan}\PY{p}{(}\PY{n}{dataset}\PY{p}{)}\PY{p}{:}
        \PY{k}{return} \PY{n}{\PYZus{}get\PYZus{}nan}\PY{p}{(}\PY{n}{dataset}\PY{p}{)}
    \PY{k}{else}\PY{p}{:}
        \PY{k}{return} \PY{k+kc}{None}

\PY{k}{def} \PY{n+nf}{\PYZus{}get\PYZus{}nan\PYZus{}columns}\PY{p}{(}\PY{n}{dataset}\PY{p}{)}\PY{p}{:}
    \PY{k}{return} \PY{n}{dataset}\PY{o}{.}\PY{n}{columns}\PY{p}{[}\PY{n}{dataset}\PY{o}{.}\PY{n}{isna}\PY{p}{(}\PY{p}{)}\PY{o}{.}\PY{n}{any}\PY{p}{(}\PY{p}{)}\PY{p}{]}\PY{o}{.}\PY{n}{tolist}\PY{p}{(}\PY{p}{)}

\PY{k}{def} \PY{n+nf}{\PYZus{}group\PYZus{}as\PYZus{}list}\PY{p}{(}\PY{n}{array}\PY{p}{)}\PY{p}{:}

    \PY{c+c1}{\PYZsh{} based on https://stackoverflow.com/a/15276206 }
    \PY{n}{group\PYZus{}list} \PY{o}{=} \PY{p}{[}\PY{p}{]}
    \PY{k}{for} \PY{n}{\PYZus{}}\PY{p}{,} \PY{n}{g} \PY{o+ow}{in} \PY{n}{groupby}\PY{p}{(}\PY{n+nb}{enumerate}\PY{p}{(}\PY{n}{array}\PY{p}{)}\PY{p}{,} \PY{k}{lambda} \PY{n}{x}\PY{p}{:} \PY{n}{x}\PY{p}{[}\PY{l+m+mi}{0}\PY{p}{]}\PY{o}{\PYZhy{}}\PY{n}{x}\PY{p}{[}\PY{l+m+mi}{1}\PY{p}{]}\PY{p}{)}\PY{p}{:}
        \PY{n}{single\PYZus{}list} \PY{o}{=} \PY{n+nb}{sorted}\PY{p}{(}\PY{n+nb}{list}\PY{p}{(}\PY{n+nb}{map}\PY{p}{(}\PY{n}{itemgetter}\PY{p}{(}\PY{l+m+mi}{1}\PY{p}{)}\PY{p}{,} \PY{n}{g}\PY{p}{)}\PY{p}{)}\PY{p}{)}
        \PY{n}{group\PYZus{}list}\PY{o}{.}\PY{n}{append}\PY{p}{(}\PY{n}{single\PYZus{}list}\PY{p}{)}
    
    \PY{k}{return} \PY{n}{group\PYZus{}list}

\PY{k}{def} \PY{n+nf}{\PYZus{}group\PYZus{}as\PYZus{}index}\PY{p}{(}
    \PY{n}{group\PYZus{}list}\PY{p}{,} \PY{n}{index}\PY{o}{=}\PY{k+kc}{None}\PY{p}{,} \PY{n}{date\PYZus{}format}\PY{o}{=}\PY{l+s+s1}{\PYZsq{}}\PY{l+s+s1}{\PYZpc{}}\PY{l+s+s1}{Y}\PY{l+s+s1}{\PYZpc{}}\PY{l+s+s1}{m}\PY{l+s+si}{\PYZpc{}d}\PY{l+s+s1}{\PYZsq{}}\PY{p}{,}
    \PY{n}{format\PYZus{}date} \PY{o}{=} \PY{l+s+s1}{\PYZsq{}}\PY{l+s+si}{\PYZob{}\PYZcb{}}\PY{l+s+s1}{\PYZhy{}}\PY{l+s+si}{\PYZob{}\PYZcb{}}\PY{l+s+s1}{\PYZsq{}}
\PY{p}{)}\PY{p}{:}
    \PY{n}{group\PYZus{}index} \PY{o}{=} \PY{p}{[}\PY{p}{]}
    \PY{n}{date\PYZus{}index} \PY{o}{=} \PY{n+nb}{isinstance}\PY{p}{(}\PY{n}{index}\PY{p}{,} \PY{n}{pd}\PY{o}{.}\PY{n}{DatetimeIndex}\PY{p}{)}

    \PY{k}{for} \PY{n}{item} \PY{o+ow}{in} \PY{n}{group\PYZus{}list}\PY{p}{:}
        \PY{k}{if} \PY{n+nb}{len}\PY{p}{(}\PY{n}{item}\PY{p}{)} \PY{o}{==} \PY{l+m+mi}{1}\PY{p}{:}
            \PY{k}{if} \PY{n}{date\PYZus{}index}\PY{p}{:}
                \PY{n}{group\PYZus{}index}\PY{o}{.}\PY{n}{append}\PY{p}{(}\PY{n}{index}\PY{p}{[}\PY{n}{item}\PY{p}{[}\PY{l+m+mi}{0}\PY{p}{]}\PY{p}{]}\PY{o}{.}\PY{n}{strftime}\PY{p}{(}\PY{n}{date\PYZus{}format}\PY{p}{)}\PY{p}{)}
            \PY{k}{else}\PY{p}{:}
                \PY{n}{group\PYZus{}index}\PY{o}{.}\PY{n}{append}\PY{p}{(}\PY{n}{index}\PY{p}{[}\PY{n}{item}\PY{p}{[}\PY{l+m+mi}{0}\PY{p}{]}\PY{p}{]}\PY{p}{)}
        \PY{k}{else}\PY{p}{:}
            \PY{k}{if} \PY{n}{date\PYZus{}index}\PY{p}{:}
                \PY{n}{group\PYZus{}index}\PY{o}{.}\PY{n}{append}\PY{p}{(}
                    \PY{n}{format\PYZus{}date}\PY{o}{.}\PY{n}{format}\PY{p}{(}
                        \PY{n}{index}\PY{p}{[}\PY{n}{item}\PY{p}{[}\PY{l+m+mi}{0}\PY{p}{]}\PY{p}{]}\PY{o}{.}\PY{n}{strftime}\PY{p}{(}\PY{n}{date\PYZus{}format}\PY{p}{)}\PY{p}{,}
                        \PY{n}{index}\PY{p}{[}\PY{n}{item}\PY{p}{[}\PY{o}{\PYZhy{}}\PY{l+m+mi}{1}\PY{p}{]}\PY{p}{]}\PY{o}{.}\PY{n}{strftime}\PY{p}{(}\PY{n}{date\PYZus{}format}\PY{p}{)}
                    \PY{p}{)}
                \PY{p}{)}
            \PY{k}{else}\PY{p}{:}
                \PY{n}{group\PYZus{}index}\PY{o}{.}\PY{n}{append}\PY{p}{(}
                    \PY{n}{format\PYZus{}date}\PY{o}{.}\PY{n}{format}\PY{p}{(}
                        \PY{n}{index}\PY{p}{[}\PY{n}{item}\PY{p}{[}\PY{l+m+mi}{0}\PY{p}{]}\PY{p}{]}\PY{p}{,} \PY{n}{index}\PY{p}{[}\PY{n}{item}\PY{p}{[}\PY{o}{\PYZhy{}}\PY{l+m+mi}{1}\PY{p}{]}\PY{p}{]}
                    \PY{p}{)}
                \PY{p}{)}
            
    \PY{k}{return} \PY{n}{group\PYZus{}index}
\end{Verbatim}
\end{tcolorbox}

    \hypertarget{penggunaan}{%
\section{PENGGUNAAN}\label{penggunaan}}

    \hypertarget{fungsi-_read_bmkg}{%
\subsection{\texorpdfstring{Fungsi
\texttt{\_read\_bmkg}}{Fungsi \_read\_bmkg}}\label{fungsi-_read_bmkg}}

Tujuan: Impor berkas excel bmkg ke dataframe

    \begin{tcolorbox}[breakable, size=fbox, boxrule=1pt, pad at break*=1mm,colback=cellbackground, colframe=cellborder]
\prompt{In}{incolor}{ }{\boxspacing}
\begin{Verbatim}[commandchars=\\\{\}]
\PY{n}{dataset} \PY{o}{=} \PY{n}{\PYZus{}read\PYZus{}bmkg}\PY{p}{(}\PY{n}{DATASET\PYZus{}PATH}\PY{p}{)}
\PY{n}{dataset}\PY{o}{.}\PY{n}{head}\PY{p}{(}\PY{p}{)}
\end{Verbatim}
\end{tcolorbox}

            \begin{tcolorbox}[breakable, size=fbox, boxrule=.5pt, pad at break*=1mm, opacityfill=0]
\prompt{Out}{outcolor}{ }{\boxspacing}
\begin{Verbatim}[commandchars=\\\{\}]
              Tn    Tx  Tavg  RH\_avg    RR   ss  ff\_x  ddd\_x  ff\_avg ddd\_car
Tanggal
1998-03-01  24.0  32.0  27.7    89.0   0.0  5.0   2.0   78.0     2.0      W
1998-03-02  23.9  31.2  27.6    92.0  16.0  2.0   4.0  283.0     3.0      SE
1998-03-03  24.2  33.0  27.3    90.0   7.0  2.0   5.0  330.0     2.0      S
1998-03-04  24.0  30.4  26.9    90.0   1.0  0.2   3.0  285.0     0.0      SW
1998-03-05  24.0  32.0  27.4    89.0   0.0  3.7   0.0  345.0     0.0      E
\end{Verbatim}
\end{tcolorbox}
        
    \begin{tcolorbox}[breakable, size=fbox, boxrule=1pt, pad at break*=1mm,colback=cellbackground, colframe=cellborder]
\prompt{In}{incolor}{ }{\boxspacing}
\begin{Verbatim}[commandchars=\\\{\}]
\PY{n}{dataset}\PY{o}{.}\PY{n}{tail}\PY{p}{(}\PY{p}{)}
\end{Verbatim}
\end{tcolorbox}

            \begin{tcolorbox}[breakable, size=fbox, boxrule=.5pt, pad at break*=1mm, opacityfill=0]
\prompt{Out}{outcolor}{ }{\boxspacing}
\begin{Verbatim}[commandchars=\\\{\}]
              Tn    Tx  Tavg  RH\_avg    RR   ss  ff\_x  ddd\_x  ff\_avg ddd\_car
Tanggal
2008-12-28  24.7  33.0  28.6    78.0   0.0  5.1  13.0  270.0     3.0      W
2008-12-29  23.2  29.4  26.7    84.0   1.0  0.0  15.0  270.0     8.0      W
2008-12-30  24.6  31.2  27.2    82.0   2.0  1.5  13.0  315.0     4.0      W
2008-12-31  24.6  32.2  27.5    82.0   3.0  1.5  13.0  270.0     4.0      W
2009-01-01  24.5  32.1  26.2    90.0  18.0  1.6   8.0  270.0     2.0      W
\end{Verbatim}
\end{tcolorbox}
        
    \hypertarget{fungsi-_have_nan}{%
\subsection{\texorpdfstring{Fungsi
\texttt{\_have\_nan()}}{Fungsi \_have\_nan()}}\label{fungsi-_have_nan}}

Tujuan: Memeriksa apakah di dalam tabel memiliki nilai yang hilang
(np.nan)

    \begin{tcolorbox}[breakable, size=fbox, boxrule=1pt, pad at break*=1mm,colback=cellbackground, colframe=cellborder]
\prompt{In}{incolor}{ }{\boxspacing}
\begin{Verbatim}[commandchars=\\\{\}]
\PY{n}{\PYZus{}have\PYZus{}nan}\PY{p}{(}\PY{n}{dataset}\PY{p}{)}
\end{Verbatim}
\end{tcolorbox}

            \begin{tcolorbox}[breakable, size=fbox, boxrule=.5pt, pad at break*=1mm, opacityfill=0]
\prompt{Out}{outcolor}{ }{\boxspacing}
\begin{Verbatim}[commandchars=\\\{\}]
True
\end{Verbatim}
\end{tcolorbox}
        
    \hypertarget{fungsi-_get_index1d}{%
\subsection{\texorpdfstring{Fungsi
\texttt{\_get\_index1D()}}{Fungsi \_get\_index1D()}}\label{fungsi-_get_index1d}}

Tujuan: Memperoleh index data yang hilang untuk setiap array

    \begin{tcolorbox}[breakable, size=fbox, boxrule=1pt, pad at break*=1mm,colback=cellbackground, colframe=cellborder]
\prompt{In}{incolor}{ }{\boxspacing}
\begin{Verbatim}[commandchars=\\\{\}]
\PY{n}{\PYZus{}get\PYZus{}index1D}\PY{p}{(}\PY{n}{dataset}\PY{p}{[}\PY{l+s+s1}{\PYZsq{}}\PY{l+s+s1}{RH\PYZus{}avg}\PY{l+s+s1}{\PYZsq{}}\PY{p}{]}\PY{o}{.}\PY{n}{isna}\PY{p}{(}\PY{p}{)}\PY{o}{.}\PY{n}{values}\PY{p}{)}
\end{Verbatim}
\end{tcolorbox}

            \begin{tcolorbox}[breakable, size=fbox, boxrule=.5pt, pad at break*=1mm, opacityfill=0]
\prompt{Out}{outcolor}{ }{\boxspacing}
\begin{Verbatim}[commandchars=\\\{\}]
array([ 852, 1037, 1038, 1039, 1040, 1041, 1042, 1043, 1044, 1045, 1046,
       1047, 1048, 1049, 1050, 1051, 1052, 1053, 1054, 1055, 1056, 1057,
       1058, 1059, 1060, 1061, 1062, 1063, 1064, 1065, 1066, 1067, 1220,
       1221, 1222, 1223, 1224, 1628, 1629, 1697, 2657])
\end{Verbatim}
\end{tcolorbox}
        
    \hypertarget{fungsi-_get_nan}{%
\subsection{\texorpdfstring{Fungsi
\texttt{\_get\_nan()}}{Fungsi \_get\_nan()}}\label{fungsi-_get_nan}}

Tujuan: Memperoleh index data yang hilang untuk setiap kolom dalam
bentuk \texttt{dictionary}

    \begin{tcolorbox}[breakable, size=fbox, boxrule=1pt, pad at break*=1mm,colback=cellbackground, colframe=cellborder]
\prompt{In}{incolor}{ }{\boxspacing}
\begin{Verbatim}[commandchars=\\\{\}]
\PY{n}{\PYZus{}get\PYZus{}nan}\PY{p}{(}\PY{n}{dataset}\PY{p}{)}\PY{o}{.}\PY{n}{keys}\PY{p}{(}\PY{p}{)}
\end{Verbatim}
\end{tcolorbox}

            \begin{tcolorbox}[breakable, size=fbox, boxrule=.5pt, pad at break*=1mm, opacityfill=0]
\prompt{Out}{outcolor}{ }{\boxspacing}
\begin{Verbatim}[commandchars=\\\{\}]
dict\_keys(['Tn', 'Tx', 'Tavg', 'RH\_avg', 'RR', 'ss', 'ff\_x', 'ddd\_x', 'ff\_avg',
'ddd\_car'])
\end{Verbatim}
\end{tcolorbox}
        
    \begin{tcolorbox}[breakable, size=fbox, boxrule=1pt, pad at break*=1mm,colback=cellbackground, colframe=cellborder]
\prompt{In}{incolor}{ }{\boxspacing}
\begin{Verbatim}[commandchars=\\\{\}]
\PY{n+nb}{print}\PY{p}{(}\PY{n}{\PYZus{}get\PYZus{}nan}\PY{p}{(}\PY{n}{dataset}\PY{p}{)}\PY{p}{[}\PY{l+s+s1}{\PYZsq{}}\PY{l+s+s1}{RH\PYZus{}avg}\PY{l+s+s1}{\PYZsq{}}\PY{p}{]}\PY{p}{)}
\end{Verbatim}
\end{tcolorbox}

    \begin{Verbatim}[commandchars=\\\{\}]
[852, 1037, 1038, 1039, 1040, 1041, 1042, 1043, 1044, 1045, 1046, 1047, 1048,
1049, 1050, 1051, 1052, 1053, 1054, 1055, 1056, 1057, 1058, 1059, 1060, 1061,
1062, 1063, 1064, 1065, 1066, 1067, 1220, 1221, 1222, 1223, 1224, 1628, 1629,
1697, 2657]
    \end{Verbatim}

    \hypertarget{fungsi-_get_nan_columns}{%
\subsection{\texorpdfstring{Fungsi
\texttt{\_get\_nan\_columns()}}{Fungsi \_get\_nan\_columns()}}\label{fungsi-_get_nan_columns}}

Tujuan: Memperoleh nama kolom yang memiliki nilai yang hilang
\texttt{NaN}.

    \begin{tcolorbox}[breakable, size=fbox, boxrule=1pt, pad at break*=1mm,colback=cellbackground, colframe=cellborder]
\prompt{In}{incolor}{ }{\boxspacing}
\begin{Verbatim}[commandchars=\\\{\}]
\PY{n}{\PYZus{}get\PYZus{}nan\PYZus{}columns}\PY{p}{(}\PY{n}{dataset}\PY{p}{)}
\end{Verbatim}
\end{tcolorbox}

            \begin{tcolorbox}[breakable, size=fbox, boxrule=.5pt, pad at break*=1mm, opacityfill=0]
\prompt{Out}{outcolor}{ }{\boxspacing}
\begin{Verbatim}[commandchars=\\\{\}]
['RH\_avg', 'ss', 'ff\_x', 'ddd\_x', 'ff\_avg', 'ddd\_car']
\end{Verbatim}
\end{tcolorbox}
        
    \hypertarget{fungsi-_check_nan}{%
\subsection{\texorpdfstring{Fungsi
\texttt{\_check\_nan()}}{Fungsi \_check\_nan()}}\label{fungsi-_check_nan}}

Tujuan: Gabungan dari \texttt{\_have\_nan()} dan \texttt{\_get\_nan()}.
Memeriksa apakah dataset memiliki \texttt{NaN}, jika iya, memberikan
nilai hasil \texttt{\_get\_nan()}, jika tidak memberikan nilai
\texttt{None}.

    \begin{tcolorbox}[breakable, size=fbox, boxrule=1pt, pad at break*=1mm,colback=cellbackground, colframe=cellborder]
\prompt{In}{incolor}{ }{\boxspacing}
\begin{Verbatim}[commandchars=\\\{\}]
\PY{n}{\PYZus{}check\PYZus{}nan}\PY{p}{(}\PY{n}{dataset}\PY{p}{)}\PY{o}{.}\PY{n}{items}\PY{p}{(}\PY{p}{)}
\end{Verbatim}
\end{tcolorbox}

            \begin{tcolorbox}[breakable, size=fbox, boxrule=.5pt, pad at break*=1mm, opacityfill=0]
\prompt{Out}{outcolor}{ }{\boxspacing}
\begin{Verbatim}[commandchars=\\\{\}]
dict\_items([('Tn', []), ('Tx', []), ('Tavg', []), ('RH\_avg', [852, 1037, 1038,
1039, 1040, 1041, 1042, 1043, 1044, 1045, 1046, 1047, 1048, 1049, 1050, 1051,
1052, 1053, 1054, 1055, 1056, 1057, 1058, 1059, 1060, 1061, 1062, 1063, 1064,
1065, 1066, 1067, 1220, 1221, 1222, 1223, 1224, 1628, 1629, 1697, 2657]), ('RR',
[]), ('ss', [97, 103, 113, 117, 131, 132, 133, 139, 148, 207, 212, 238, 273,
308, 313, 314, 315, 324, 332, 333, 340, 343, 344, 349, 354, 357, 359, 360, 363,
379, 604, 609, 610, 616, 617, 623, 642, 655, 656, 663, 664, 665, 669, 678, 687,
690, 696, 700, 701, 708, 709, 710, 712, 721, 733, 768, 824, 825, 840, 949, 976,
977, 978, 984, 985, 986, 1000, 1068, 1073, 1074, 1077, 1080, 1084, 1087, 1089,
1101, 1125, 1135, 1174, 1182, 1185, 1187, 1199, 1355, 1363, 1366, 1384, 1400,
1401, 1408, 1411, 1419, 1429, 1430, 1431, 1436, 1437, 1441, 1443, 1445, 1446,
1454, 1456, 1468, 1470, 1479, 1492, 1531, 1749, 1762, 1764, 1792, 1803, 1810,
1818, 1823, 1840, 1844, 1864, 1938, 2028, 2070, 2086, 2090, 2091, 2096, 2105,
2112, 2117, 2120, 2121, 2122, 2125, 2129, 2157, 2165, 2166, 2167, 2169, 2174,
2175, 2177, 2178, 2179, 2180, 2181, 2182, 2196, 2198, 2199, 2209, 2254, 2293,
2295, 2415, 2416, 2417, 2457, 2461, 2467, 2489, 2500, 2517, 2522, 2526, 2534,
2538, 2539, 2540, 2550, 2553, 2565, 2575, 2663, 2664, 2665, 2666, 2710, 2712,
2714, 2757, 2787, 2788, 2826, 2827, 2846, 2847, 2848, 2849, 2857, 2859, 2865,
2879, 2883, 2885, 2921, 2922, 2923, 2925, 2940, 2941, 3004, 3020, 3028, 3220,
3259, 3594, 3595, 3608, 3624, 3630, 3631, 3632, 3633, 3634, 3635, 3636, 3637,
3638, 3641, 3642, 3644, 3645, 3646, 3648, 3649, 3651, 3652, 3695, 3697, 3705,
3709, 3758, 3759, 3761, 3890]), ('ff\_x', [1037, 1038, 1039, 1040, 1041, 1042,
1043, 1044, 1045, 1046, 1047, 1048, 1049, 1050, 1051, 1052, 1053, 1054, 1055,
1056, 1057, 1058, 1059, 1060, 1061, 1062, 1063, 1064, 1065, 1066, 1067, 1706,
1707, 1708, 1709, 1710, 1711, 1712, 1713, 1714, 1715, 1716, 1717, 1718, 1719,
1720, 1721, 1722, 1723, 1724, 1725, 1726, 1727, 1728, 1729, 1730, 1731, 1732,
1733, 1734, 1735, 1736, 1737, 1738, 1739, 1740, 1741, 1742, 1743, 1744, 1745,
1746, 1747, 1748, 1749, 1750, 1751, 1752, 1753, 1754, 1755, 1756, 1757, 1758,
1759, 1760, 1761, 1762, 1763, 1764, 1765, 1766]), ('ddd\_x', [1037, 1038, 1039,
1040, 1041, 1042, 1043, 1044, 1045, 1046, 1047, 1048, 1049, 1050, 1051, 1052,
1053, 1054, 1055, 1056, 1057, 1058, 1059, 1060, 1061, 1062, 1063, 1064, 1065,
1066, 1067, 1706, 1707, 1708, 1709, 1710, 1711, 1712, 1713, 1714, 1715, 1716,
1717, 1718, 1719, 1720, 1721, 1722, 1723, 1724, 1725, 1726, 1727, 1728, 1729,
1730, 1731, 1732, 1733, 1734, 1735, 1736, 1737, 1738, 1739, 1740, 1741, 1742,
1743, 1744, 1745, 1746, 1747, 1748, 1749, 1750, 1751, 1752, 1753, 1754, 1755,
1756, 1757, 1758, 1759, 1760, 1761, 1762, 1763, 1764, 1765, 1766, 3286]),
('ff\_avg', [1037, 1038, 1039, 1040, 1041, 1042, 1043, 1044, 1045, 1046, 1047,
1048, 1049, 1050, 1051, 1052, 1053, 1054, 1055, 1056, 1057, 1058, 1059, 1060,
1061, 1062, 1063, 1064, 1065, 1066, 1067, 1286, 1699, 1705, 1706, 1707, 1708,
1709, 1710, 1711, 1712, 1713, 1714, 1715, 1716, 1717, 1718, 1719, 1720, 1721,
1722, 1723, 1724, 1725, 1726, 1727, 1728, 1729, 1730, 1731, 1732, 1733, 1734,
1735, 1736, 1737, 1738, 1739, 1740, 1741, 1742, 1743, 1744, 1745, 1746, 1747,
1748, 1749, 1750, 1751, 1752, 1753, 1754, 1755, 1756, 1757, 1758, 1759, 1760,
1761, 1762, 1763, 1764, 1765, 1766, 3396]), ('ddd\_car', [1037, 1038, 1039, 1040,
1041, 1042, 1043, 1044, 1045, 1046, 1047, 1048, 1049, 1050, 1051, 1052, 1053,
1054, 1055, 1056, 1057, 1058, 1059, 1060, 1061, 1062, 1063, 1064, 1065, 1066,
1067, 1286, 1699, 1705, 1706, 1707, 1708, 1709, 1710, 1711, 1712, 1713, 1714,
1715, 1716, 1717, 1718, 1719, 1720, 1721, 1722, 1723, 1724, 1725, 1726, 1727,
1728, 1729, 1730, 1731, 1732, 1733, 1734, 1735, 1736, 1737, 1738, 1739, 1740,
1741, 1742, 1743, 1744, 1745, 1746, 1747, 1748, 1749, 1750, 1751, 1752, 1753,
1754, 1755, 1756, 1757, 1758, 1759, 1760, 1761, 1762, 1763, 1764, 1765, 1766])])
\end{Verbatim}
\end{tcolorbox}
        
    \begin{tcolorbox}[breakable, size=fbox, boxrule=1pt, pad at break*=1mm,colback=cellbackground, colframe=cellborder]
\prompt{In}{incolor}{ }{\boxspacing}
\begin{Verbatim}[commandchars=\\\{\}]
\PY{c+c1}{\PYZsh{} Jika tidak memiliki nilai nan}
\PY{n+nb}{print}\PY{p}{(}\PY{n}{\PYZus{}check\PYZus{}nan}\PY{p}{(}\PY{n}{dataset}\PY{o}{.}\PY{n}{drop}\PY{p}{(}\PY{n}{\PYZus{}get\PYZus{}nan\PYZus{}columns}\PY{p}{(}\PY{n}{dataset}\PY{p}{)}\PY{p}{,} \PY{n}{axis}\PY{o}{=}\PY{l+m+mi}{1}\PY{p}{)}\PY{p}{)}\PY{p}{)}
\end{Verbatim}
\end{tcolorbox}

    \begin{Verbatim}[commandchars=\\\{\}]
None
    \end{Verbatim}

    \hypertarget{fungsi-_group_as_list}{%
\subsection{\texorpdfstring{Fungsi
\texttt{\_group\_as\_list()}}{Fungsi \_group\_as\_list()}}\label{fungsi-_group_as_list}}

Tujuan: Mengelompokkan kelompok array yang bersifat kontinu (nilainya
berurutan) dalam masing-masing list.

Referensi: https://stackoverflow.com/a/15276206 (dimodifikasi untuk
Python 3.x dan kemudahan membaca)

    \begin{tcolorbox}[breakable, size=fbox, boxrule=1pt, pad at break*=1mm,colback=cellbackground, colframe=cellborder]
\prompt{In}{incolor}{ }{\boxspacing}
\begin{Verbatim}[commandchars=\\\{\}]
\PY{n}{missing\PYZus{}dict} \PY{o}{=} \PY{n}{\PYZus{}get\PYZus{}nan}\PY{p}{(}\PY{n}{dataset}\PY{p}{)}
\PY{n}{missing\PYZus{}RH\PYZus{}avg} \PY{o}{=} \PY{n}{missing\PYZus{}dict}\PY{p}{[}\PY{l+s+s1}{\PYZsq{}}\PY{l+s+s1}{RH\PYZus{}avg}\PY{l+s+s1}{\PYZsq{}}\PY{p}{]}
\PY{n+nb}{print}\PY{p}{(}\PY{n}{missing\PYZus{}RH\PYZus{}avg}\PY{p}{)}
\end{Verbatim}
\end{tcolorbox}

    \begin{Verbatim}[commandchars=\\\{\}]
[852, 1037, 1038, 1039, 1040, 1041, 1042, 1043, 1044, 1045, 1046, 1047, 1048,
1049, 1050, 1051, 1052, 1053, 1054, 1055, 1056, 1057, 1058, 1059, 1060, 1061,
1062, 1063, 1064, 1065, 1066, 1067, 1220, 1221, 1222, 1223, 1224, 1628, 1629,
1697, 2657]
    \end{Verbatim}

    \begin{tcolorbox}[breakable, size=fbox, boxrule=1pt, pad at break*=1mm,colback=cellbackground, colframe=cellborder]
\prompt{In}{incolor}{ }{\boxspacing}
\begin{Verbatim}[commandchars=\\\{\}]
\PY{n+nb}{print}\PY{p}{(}\PY{n}{\PYZus{}group\PYZus{}as\PYZus{}list}\PY{p}{(}\PY{n}{missing\PYZus{}RH\PYZus{}avg}\PY{p}{)}\PY{p}{)}
\end{Verbatim}
\end{tcolorbox}

    \begin{Verbatim}[commandchars=\\\{\}]
[[852], [1037, 1038, 1039, 1040, 1041, 1042, 1043, 1044, 1045, 1046, 1047, 1048,
1049, 1050, 1051, 1052, 1053, 1054, 1055, 1056, 1057, 1058, 1059, 1060, 1061,
1062, 1063, 1064, 1065, 1066, 1067], [1220, 1221, 1222, 1223, 1224], [1628,
1629], [1697], [2657]]
    \end{Verbatim}

    \hypertarget{fungsi-_group_as_index}{%
\subsection{\texorpdfstring{Fungsi
\texttt{\_group\_as\_index()}}{Fungsi \_group\_as\_index()}}\label{fungsi-_group_as_index}}

Tujuan: Mengubah hasil pengelompokkan menjadi jenis index dataset (dalam
kasus ini dalam bentuk tanggal dibandingkan dalam bentuk angka-index
dataset).

    \begin{tcolorbox}[breakable, size=fbox, boxrule=1pt, pad at break*=1mm,colback=cellbackground, colframe=cellborder]
\prompt{In}{incolor}{ }{\boxspacing}
\begin{Verbatim}[commandchars=\\\{\}]
\PY{n}{\PYZus{}group\PYZus{}as\PYZus{}index}\PY{p}{(}\PY{n}{\PYZus{}group\PYZus{}as\PYZus{}list}\PY{p}{(}\PY{n}{missing\PYZus{}RH\PYZus{}avg}\PY{p}{)}\PY{p}{,} \PY{n}{index}\PY{o}{=}\PY{n}{dataset}\PY{o}{.}\PY{n}{index}\PY{p}{,} \PY{n}{date\PYZus{}format}\PY{o}{=}\PY{l+s+s1}{\PYZsq{}}\PY{l+s+si}{\PYZpc{}d}\PY{l+s+s1}{ }\PY{l+s+s1}{\PYZpc{}}\PY{l+s+s1}{b }\PY{l+s+s1}{\PYZpc{}}\PY{l+s+s1}{Y}\PY{l+s+s1}{\PYZsq{}}\PY{p}{)}
\end{Verbatim}
\end{tcolorbox}

            \begin{tcolorbox}[breakable, size=fbox, boxrule=.5pt, pad at break*=1mm, opacityfill=0]
\prompt{Out}{outcolor}{ }{\boxspacing}
\begin{Verbatim}[commandchars=\\\{\}]
['30 Jun 2000',
 '01 Jan 2001-31 Jan 2001',
 '03 Jul 2001-07 Jul 2001',
 '15 Aug 2002-16 Aug 2002',
 '23 Oct 2002',
 '09 Jun 2005']
\end{Verbatim}
\end{tcolorbox}
        
    \hypertarget{fungsi-_get_missing}{%
\subsection{\texorpdfstring{Fungsi
\texttt{\_get\_missing()}}{Fungsi \_get\_missing()}}\label{fungsi-_get_missing}}

Tujuan: Memperoleh index yang memiliki nilai tidak terukur (bernilai
\texttt{8888} atau \texttt{9999}) untuk setiap kolomnya

    \begin{tcolorbox}[breakable, size=fbox, boxrule=1pt, pad at break*=1mm,colback=cellbackground, colframe=cellborder]
\prompt{In}{incolor}{ }{\boxspacing}
\begin{Verbatim}[commandchars=\\\{\}]
\PY{n}{\PYZus{}get\PYZus{}missing}\PY{p}{(}\PY{n}{dataset}\PY{p}{)}
\end{Verbatim}
\end{tcolorbox}

            \begin{tcolorbox}[breakable, size=fbox, boxrule=.5pt, pad at break*=1mm, opacityfill=0]
\prompt{Out}{outcolor}{ }{\boxspacing}
\begin{Verbatim}[commandchars=\\\{\}]
\{'RH\_avg': array([], dtype=int64),
 'RR': array([2090, 2467, 2468, 2489, 2506, 2606, 3208, 3209, 3229, 3712, 3876,
        3904, 3946]),
 'Tavg': array([], dtype=int64),
 'Tn': array([], dtype=int64),
 'Tx': array([], dtype=int64),
 'ddd\_car': array([], dtype=int64),
 'ddd\_x': array([], dtype=int64),
 'ff\_avg': array([], dtype=int64),
 'ff\_x': array([], dtype=int64),
 'ss': array([], dtype=int64)\}
\end{Verbatim}
\end{tcolorbox}
        
    \hypertarget{penerapan}{%
\section{Penerapan}\label{penerapan}}

    \hypertarget{menampilkan-index-yang-bermasalah}{%
\subsection{Menampilkan index yang
bermasalah}\label{menampilkan-index-yang-bermasalah}}

Tujuan: Setelah memperoleh index dari hasil \texttt{\_get\_missing()}
atau \texttt{\_get\_nan()}, bisa menampilkan potongan index tersebut
dalam dataframe.

    \begin{tcolorbox}[breakable, size=fbox, boxrule=1pt, pad at break*=1mm,colback=cellbackground, colframe=cellborder]
\prompt{In}{incolor}{ }{\boxspacing}
\begin{Verbatim}[commandchars=\\\{\}]
\PY{n}{dataset}\PY{o}{.}\PY{n}{iloc}\PY{p}{[}\PY{n}{\PYZus{}get\PYZus{}missing}\PY{p}{(}\PY{n}{dataset}\PY{p}{)}\PY{p}{[}\PY{l+s+s1}{\PYZsq{}}\PY{l+s+s1}{RR}\PY{l+s+s1}{\PYZsq{}}\PY{p}{]}\PY{p}{]}
\end{Verbatim}
\end{tcolorbox}

            \begin{tcolorbox}[breakable, size=fbox, boxrule=.5pt, pad at break*=1mm, opacityfill=0]
\prompt{Out}{outcolor}{ }{\boxspacing}
\begin{Verbatim}[commandchars=\\\{\}]
              Tn    Tx  Tavg  RH\_avg      RR   ss  ff\_x  ddd\_x  ff\_avg ddd\_car
Tanggal
2003-11-20  24.1  28.8  26.2    85.0  8888.0  NaN   1.0   90.0     1.0      E
2004-12-01  23.7  29.9  26.2    87.0  8888.0  NaN  10.0  270.0     0.0      W
2004-12-02  24.0  34.0  28.5    78.0  8888.0  2.0  10.0  270.0     0.0      W
2004-12-23  23.6  31.2  27.1    86.0  8888.0  NaN   6.0  270.0     0.0      W
2005-01-09  23.2  30.6  26.5    89.0  8888.0  4.4   3.0  270.0     0.0      W
2005-04-19  24.8  33.0  27.9    84.0  8888.0  5.1  10.0  315.0     0.0      E
2006-12-12  25.4  32.8  28.0    81.0  8888.0  5.0   5.0   45.0     2.0      N
2006-12-13  23.2  33.2  28.5    80.0  8888.0  4.2   5.0   45.0     2.0      N
2007-01-02  22.2  31.6  26.6    83.0  8888.0  5.6  15.0  270.0     4.0      W
2008-04-29  21.0  32.4  28.1    82.0  8888.0  6.0   2.0  360.0     0.0      N
2008-10-10  24.6  35.1  29.6    72.0  8888.0  7.3  10.0  360.0     4.0      N
2008-11-07  25.0  31.6  27.5    83.0  8888.0  2.5   3.0  270.0     0.0      W
2008-12-19  25.2  31.6  27.3    81.0  8888.0  4.0   5.0  360.0     2.0      N
\end{Verbatim}
\end{tcolorbox}
        
    \begin{tcolorbox}[breakable, size=fbox, boxrule=1pt, pad at break*=1mm,colback=cellbackground, colframe=cellborder]
\prompt{In}{incolor}{ }{\boxspacing}
\begin{Verbatim}[commandchars=\\\{\}]
\PY{n}{\PYZus{}group\PYZus{}as\PYZus{}list}\PY{p}{(}\PY{n}{\PYZus{}get\PYZus{}missing}\PY{p}{(}\PY{n}{dataset}\PY{p}{)}\PY{p}{[}\PY{l+s+s1}{\PYZsq{}}\PY{l+s+s1}{RR}\PY{l+s+s1}{\PYZsq{}}\PY{p}{]}\PY{p}{)}
\end{Verbatim}
\end{tcolorbox}

            \begin{tcolorbox}[breakable, size=fbox, boxrule=.5pt, pad at break*=1mm, opacityfill=0]
\prompt{Out}{outcolor}{ }{\boxspacing}
\begin{Verbatim}[commandchars=\\\{\}]
[[2090],
 [2467, 2468],
 [2489],
 [2506],
 [2606],
 [3208, 3209],
 [3229],
 [3712],
 [3876],
 [3904],
 [3946]]
\end{Verbatim}
\end{tcolorbox}
        
    \begin{tcolorbox}[breakable, size=fbox, boxrule=1pt, pad at break*=1mm,colback=cellbackground, colframe=cellborder]
\prompt{In}{incolor}{ }{\boxspacing}
\begin{Verbatim}[commandchars=\\\{\}]
\PY{n}{\PYZus{}group\PYZus{}as\PYZus{}index}\PY{p}{(}\PY{n}{\PYZus{}}\PY{p}{,} \PY{n}{index}\PY{o}{=}\PY{n}{dataset}\PY{o}{.}\PY{n}{index}\PY{p}{,} \PY{n}{date\PYZus{}format}\PY{o}{=}\PY{l+s+s1}{\PYZsq{}}\PY{l+s+si}{\PYZpc{}d}\PY{l+s+s1}{ }\PY{l+s+s1}{\PYZpc{}}\PY{l+s+s1}{b }\PY{l+s+s1}{\PYZpc{}}\PY{l+s+s1}{Y}\PY{l+s+s1}{\PYZsq{}}\PY{p}{,} \PY{n}{format\PYZus{}date}\PY{o}{=}\PY{l+s+s1}{\PYZsq{}}\PY{l+s+si}{\PYZob{}\PYZcb{}}\PY{l+s+s1}{ sampai }\PY{l+s+si}{\PYZob{}\PYZcb{}}\PY{l+s+s1}{\PYZsq{}}\PY{p}{)}
\end{Verbatim}
\end{tcolorbox}

            \begin{tcolorbox}[breakable, size=fbox, boxrule=.5pt, pad at break*=1mm, opacityfill=0]
\prompt{Out}{outcolor}{ }{\boxspacing}
\begin{Verbatim}[commandchars=\\\{\}]
['20 Nov 2003',
 '01 Dec 2004 sampai 02 Dec 2004',
 '23 Dec 2004',
 '09 Jan 2005',
 '19 Apr 2005',
 '12 Dec 2006 sampai 13 Dec 2006',
 '02 Jan 2007',
 '29 Apr 2008',
 '10 Oct 2008',
 '07 Nov 2008',
 '19 Dec 2008']
\end{Verbatim}
\end{tcolorbox}
        
    \hypertarget{changelog}{%
\section{Changelog}\label{changelog}}

\begin{verbatim}
- 20190928 - 1.0.0 - Initial
\end{verbatim}

\hypertarget{copyright-2019-taruma-sakti-megariansyah}{%
\paragraph{\texorpdfstring{Copyright © 2019
\href{https://taruma.github.io}{Taruma Sakti
Megariansyah}}{Copyright © 2019 Taruma Sakti Megariansyah}}\label{copyright-2019-taruma-sakti-megariansyah}}

Source code in this notebook is licensed under a
\href{https://choosealicense.com/licenses/mit/}{MIT License}. Data in
this notebook is licensed under a
\href{https://creativecommons.org/licenses/by/4.0/}{Creative Common
Attribution 4.0 International}.


    % Add a bibliography block to the postdoc
    
    
    
\end{document}
