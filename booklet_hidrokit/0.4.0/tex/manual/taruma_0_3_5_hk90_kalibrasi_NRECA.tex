\documentclass[11pt]{article}

    \usepackage[breakable]{tcolorbox}
    \usepackage{parskip} % Stop auto-indenting (to mimic markdown behaviour)
    
    \usepackage{iftex}
    \ifPDFTeX
    	\usepackage[T1]{fontenc}
    	\usepackage{mathpazo}
    \else
    	\usepackage{fontspec}
    \fi

    % Basic figure setup, for now with no caption control since it's done
    % automatically by Pandoc (which extracts ![](path) syntax from Markdown).
    \usepackage{graphicx}
    % Maintain compatibility with old templates. Remove in nbconvert 6.0
    \let\Oldincludegraphics\includegraphics
    % Ensure that by default, figures have no caption (until we provide a
    % proper Figure object with a Caption API and a way to capture that
    % in the conversion process - todo).
    \usepackage{caption}
    \DeclareCaptionFormat{nocaption}{}
    \captionsetup{format=nocaption,aboveskip=0pt,belowskip=0pt}

    \usepackage{float}
    \floatplacement{figure}{H} % forces figures to be placed at the correct location
    \usepackage{xcolor} % Allow colors to be defined
    \usepackage{enumerate} % Needed for markdown enumerations to work
    \usepackage{geometry} % Used to adjust the document margins
    \usepackage{amsmath} % Equations
    \usepackage{amssymb} % Equations
    \usepackage{textcomp} % defines textquotesingle
    % Hack from http://tex.stackexchange.com/a/47451/13684:
    \AtBeginDocument{%
        \def\PYZsq{\textquotesingle}% Upright quotes in Pygmentized code
    }
    \usepackage{upquote} % Upright quotes for verbatim code
    \usepackage{eurosym} % defines \euro
    \usepackage[mathletters]{ucs} % Extended unicode (utf-8) support
    \usepackage{fancyvrb} % verbatim replacement that allows latex
    \usepackage{grffile} % extends the file name processing of package graphics 
                         % to support a larger range
    \makeatletter % fix for old versions of grffile with XeLaTeX
    \@ifpackagelater{grffile}{2019/11/01}
    {
      % Do nothing on new versions
    }
    {
      \def\Gread@@xetex#1{%
        \IfFileExists{"\Gin@base".bb}%
        {\Gread@eps{\Gin@base.bb}}%
        {\Gread@@xetex@aux#1}%
      }
    }
    \makeatother
    \usepackage[Export]{adjustbox} % Used to constrain images to a maximum size
    \adjustboxset{max size={0.9\linewidth}{0.9\paperheight}}

    % The hyperref package gives us a pdf with properly built
    % internal navigation ('pdf bookmarks' for the table of contents,
    % internal cross-reference links, web links for URLs, etc.)
    \usepackage{hyperref}
    % The default LaTeX title has an obnoxious amount of whitespace. By default,
    % titling removes some of it. It also provides customization options.
    \usepackage{titling}
    \usepackage{longtable} % longtable support required by pandoc >1.10
    \usepackage{booktabs}  % table support for pandoc > 1.12.2
    \usepackage[inline]{enumitem} % IRkernel/repr support (it uses the enumerate* environment)
    \usepackage[normalem]{ulem} % ulem is needed to support strikethroughs (\sout)
                                % normalem makes italics be italics, not underlines
    \usepackage{mathrsfs}
    

    
    % Colors for the hyperref package
    \definecolor{urlcolor}{rgb}{0,.145,.698}
    \definecolor{linkcolor}{rgb}{.71,0.21,0.01}
    \definecolor{citecolor}{rgb}{.12,.54,.11}

    % ANSI colors
    \definecolor{ansi-black}{HTML}{3E424D}
    \definecolor{ansi-black-intense}{HTML}{282C36}
    \definecolor{ansi-red}{HTML}{E75C58}
    \definecolor{ansi-red-intense}{HTML}{B22B31}
    \definecolor{ansi-green}{HTML}{00A250}
    \definecolor{ansi-green-intense}{HTML}{007427}
    \definecolor{ansi-yellow}{HTML}{DDB62B}
    \definecolor{ansi-yellow-intense}{HTML}{B27D12}
    \definecolor{ansi-blue}{HTML}{208FFB}
    \definecolor{ansi-blue-intense}{HTML}{0065CA}
    \definecolor{ansi-magenta}{HTML}{D160C4}
    \definecolor{ansi-magenta-intense}{HTML}{A03196}
    \definecolor{ansi-cyan}{HTML}{60C6C8}
    \definecolor{ansi-cyan-intense}{HTML}{258F8F}
    \definecolor{ansi-white}{HTML}{C5C1B4}
    \definecolor{ansi-white-intense}{HTML}{A1A6B2}
    \definecolor{ansi-default-inverse-fg}{HTML}{FFFFFF}
    \definecolor{ansi-default-inverse-bg}{HTML}{000000}

    % common color for the border for error outputs.
    \definecolor{outerrorbackground}{HTML}{FFDFDF}

    % commands and environments needed by pandoc snippets
    % extracted from the output of `pandoc -s`
    \providecommand{\tightlist}{%
      \setlength{\itemsep}{0pt}\setlength{\parskip}{0pt}}
    \DefineVerbatimEnvironment{Highlighting}{Verbatim}{commandchars=\\\{\}}
    % Add ',fontsize=\small' for more characters per line
    \newenvironment{Shaded}{}{}
    \newcommand{\KeywordTok}[1]{\textcolor[rgb]{0.00,0.44,0.13}{\textbf{{#1}}}}
    \newcommand{\DataTypeTok}[1]{\textcolor[rgb]{0.56,0.13,0.00}{{#1}}}
    \newcommand{\DecValTok}[1]{\textcolor[rgb]{0.25,0.63,0.44}{{#1}}}
    \newcommand{\BaseNTok}[1]{\textcolor[rgb]{0.25,0.63,0.44}{{#1}}}
    \newcommand{\FloatTok}[1]{\textcolor[rgb]{0.25,0.63,0.44}{{#1}}}
    \newcommand{\CharTok}[1]{\textcolor[rgb]{0.25,0.44,0.63}{{#1}}}
    \newcommand{\StringTok}[1]{\textcolor[rgb]{0.25,0.44,0.63}{{#1}}}
    \newcommand{\CommentTok}[1]{\textcolor[rgb]{0.38,0.63,0.69}{\textit{{#1}}}}
    \newcommand{\OtherTok}[1]{\textcolor[rgb]{0.00,0.44,0.13}{{#1}}}
    \newcommand{\AlertTok}[1]{\textcolor[rgb]{1.00,0.00,0.00}{\textbf{{#1}}}}
    \newcommand{\FunctionTok}[1]{\textcolor[rgb]{0.02,0.16,0.49}{{#1}}}
    \newcommand{\RegionMarkerTok}[1]{{#1}}
    \newcommand{\ErrorTok}[1]{\textcolor[rgb]{1.00,0.00,0.00}{\textbf{{#1}}}}
    \newcommand{\NormalTok}[1]{{#1}}
    
    % Additional commands for more recent versions of Pandoc
    \newcommand{\ConstantTok}[1]{\textcolor[rgb]{0.53,0.00,0.00}{{#1}}}
    \newcommand{\SpecialCharTok}[1]{\textcolor[rgb]{0.25,0.44,0.63}{{#1}}}
    \newcommand{\VerbatimStringTok}[1]{\textcolor[rgb]{0.25,0.44,0.63}{{#1}}}
    \newcommand{\SpecialStringTok}[1]{\textcolor[rgb]{0.73,0.40,0.53}{{#1}}}
    \newcommand{\ImportTok}[1]{{#1}}
    \newcommand{\DocumentationTok}[1]{\textcolor[rgb]{0.73,0.13,0.13}{\textit{{#1}}}}
    \newcommand{\AnnotationTok}[1]{\textcolor[rgb]{0.38,0.63,0.69}{\textbf{\textit{{#1}}}}}
    \newcommand{\CommentVarTok}[1]{\textcolor[rgb]{0.38,0.63,0.69}{\textbf{\textit{{#1}}}}}
    \newcommand{\VariableTok}[1]{\textcolor[rgb]{0.10,0.09,0.49}{{#1}}}
    \newcommand{\ControlFlowTok}[1]{\textcolor[rgb]{0.00,0.44,0.13}{\textbf{{#1}}}}
    \newcommand{\OperatorTok}[1]{\textcolor[rgb]{0.40,0.40,0.40}{{#1}}}
    \newcommand{\BuiltInTok}[1]{{#1}}
    \newcommand{\ExtensionTok}[1]{{#1}}
    \newcommand{\PreprocessorTok}[1]{\textcolor[rgb]{0.74,0.48,0.00}{{#1}}}
    \newcommand{\AttributeTok}[1]{\textcolor[rgb]{0.49,0.56,0.16}{{#1}}}
    \newcommand{\InformationTok}[1]{\textcolor[rgb]{0.38,0.63,0.69}{\textbf{\textit{{#1}}}}}
    \newcommand{\WarningTok}[1]{\textcolor[rgb]{0.38,0.63,0.69}{\textbf{\textit{{#1}}}}}
    
    
    % Define a nice break command that doesn't care if a line doesn't already
    % exist.
    \def\br{\hspace*{\fill} \\* }
    % Math Jax compatibility definitions
    \def\gt{>}
    \def\lt{<}
    \let\Oldtex\TeX
    \let\Oldlatex\LaTeX
    \renewcommand{\TeX}{\textrm{\Oldtex}}
    \renewcommand{\LaTeX}{\textrm{\Oldlatex}}
    % Document parameters
    % Document title
    \title{taruma\_0\_3\_5\_hk90\_kalibrasi\_NRECA}
    
    
    
    
    
% Pygments definitions
\makeatletter
\def\PY@reset{\let\PY@it=\relax \let\PY@bf=\relax%
    \let\PY@ul=\relax \let\PY@tc=\relax%
    \let\PY@bc=\relax \let\PY@ff=\relax}
\def\PY@tok#1{\csname PY@tok@#1\endcsname}
\def\PY@toks#1+{\ifx\relax#1\empty\else%
    \PY@tok{#1}\expandafter\PY@toks\fi}
\def\PY@do#1{\PY@bc{\PY@tc{\PY@ul{%
    \PY@it{\PY@bf{\PY@ff{#1}}}}}}}
\def\PY#1#2{\PY@reset\PY@toks#1+\relax+\PY@do{#2}}

\@namedef{PY@tok@w}{\def\PY@tc##1{\textcolor[rgb]{0.73,0.73,0.73}{##1}}}
\@namedef{PY@tok@c}{\let\PY@it=\textit\def\PY@tc##1{\textcolor[rgb]{0.24,0.48,0.48}{##1}}}
\@namedef{PY@tok@cp}{\def\PY@tc##1{\textcolor[rgb]{0.61,0.40,0.00}{##1}}}
\@namedef{PY@tok@k}{\let\PY@bf=\textbf\def\PY@tc##1{\textcolor[rgb]{0.00,0.50,0.00}{##1}}}
\@namedef{PY@tok@kp}{\def\PY@tc##1{\textcolor[rgb]{0.00,0.50,0.00}{##1}}}
\@namedef{PY@tok@kt}{\def\PY@tc##1{\textcolor[rgb]{0.69,0.00,0.25}{##1}}}
\@namedef{PY@tok@o}{\def\PY@tc##1{\textcolor[rgb]{0.40,0.40,0.40}{##1}}}
\@namedef{PY@tok@ow}{\let\PY@bf=\textbf\def\PY@tc##1{\textcolor[rgb]{0.67,0.13,1.00}{##1}}}
\@namedef{PY@tok@nb}{\def\PY@tc##1{\textcolor[rgb]{0.00,0.50,0.00}{##1}}}
\@namedef{PY@tok@nf}{\def\PY@tc##1{\textcolor[rgb]{0.00,0.00,1.00}{##1}}}
\@namedef{PY@tok@nc}{\let\PY@bf=\textbf\def\PY@tc##1{\textcolor[rgb]{0.00,0.00,1.00}{##1}}}
\@namedef{PY@tok@nn}{\let\PY@bf=\textbf\def\PY@tc##1{\textcolor[rgb]{0.00,0.00,1.00}{##1}}}
\@namedef{PY@tok@ne}{\let\PY@bf=\textbf\def\PY@tc##1{\textcolor[rgb]{0.80,0.25,0.22}{##1}}}
\@namedef{PY@tok@nv}{\def\PY@tc##1{\textcolor[rgb]{0.10,0.09,0.49}{##1}}}
\@namedef{PY@tok@no}{\def\PY@tc##1{\textcolor[rgb]{0.53,0.00,0.00}{##1}}}
\@namedef{PY@tok@nl}{\def\PY@tc##1{\textcolor[rgb]{0.46,0.46,0.00}{##1}}}
\@namedef{PY@tok@ni}{\let\PY@bf=\textbf\def\PY@tc##1{\textcolor[rgb]{0.44,0.44,0.44}{##1}}}
\@namedef{PY@tok@na}{\def\PY@tc##1{\textcolor[rgb]{0.41,0.47,0.13}{##1}}}
\@namedef{PY@tok@nt}{\let\PY@bf=\textbf\def\PY@tc##1{\textcolor[rgb]{0.00,0.50,0.00}{##1}}}
\@namedef{PY@tok@nd}{\def\PY@tc##1{\textcolor[rgb]{0.67,0.13,1.00}{##1}}}
\@namedef{PY@tok@s}{\def\PY@tc##1{\textcolor[rgb]{0.73,0.13,0.13}{##1}}}
\@namedef{PY@tok@sd}{\let\PY@it=\textit\def\PY@tc##1{\textcolor[rgb]{0.73,0.13,0.13}{##1}}}
\@namedef{PY@tok@si}{\let\PY@bf=\textbf\def\PY@tc##1{\textcolor[rgb]{0.64,0.35,0.47}{##1}}}
\@namedef{PY@tok@se}{\let\PY@bf=\textbf\def\PY@tc##1{\textcolor[rgb]{0.67,0.36,0.12}{##1}}}
\@namedef{PY@tok@sr}{\def\PY@tc##1{\textcolor[rgb]{0.64,0.35,0.47}{##1}}}
\@namedef{PY@tok@ss}{\def\PY@tc##1{\textcolor[rgb]{0.10,0.09,0.49}{##1}}}
\@namedef{PY@tok@sx}{\def\PY@tc##1{\textcolor[rgb]{0.00,0.50,0.00}{##1}}}
\@namedef{PY@tok@m}{\def\PY@tc##1{\textcolor[rgb]{0.40,0.40,0.40}{##1}}}
\@namedef{PY@tok@gh}{\let\PY@bf=\textbf\def\PY@tc##1{\textcolor[rgb]{0.00,0.00,0.50}{##1}}}
\@namedef{PY@tok@gu}{\let\PY@bf=\textbf\def\PY@tc##1{\textcolor[rgb]{0.50,0.00,0.50}{##1}}}
\@namedef{PY@tok@gd}{\def\PY@tc##1{\textcolor[rgb]{0.63,0.00,0.00}{##1}}}
\@namedef{PY@tok@gi}{\def\PY@tc##1{\textcolor[rgb]{0.00,0.52,0.00}{##1}}}
\@namedef{PY@tok@gr}{\def\PY@tc##1{\textcolor[rgb]{0.89,0.00,0.00}{##1}}}
\@namedef{PY@tok@ge}{\let\PY@it=\textit}
\@namedef{PY@tok@gs}{\let\PY@bf=\textbf}
\@namedef{PY@tok@gp}{\let\PY@bf=\textbf\def\PY@tc##1{\textcolor[rgb]{0.00,0.00,0.50}{##1}}}
\@namedef{PY@tok@go}{\def\PY@tc##1{\textcolor[rgb]{0.44,0.44,0.44}{##1}}}
\@namedef{PY@tok@gt}{\def\PY@tc##1{\textcolor[rgb]{0.00,0.27,0.87}{##1}}}
\@namedef{PY@tok@err}{\def\PY@bc##1{{\setlength{\fboxsep}{\string -\fboxrule}\fcolorbox[rgb]{1.00,0.00,0.00}{1,1,1}{\strut ##1}}}}
\@namedef{PY@tok@kc}{\let\PY@bf=\textbf\def\PY@tc##1{\textcolor[rgb]{0.00,0.50,0.00}{##1}}}
\@namedef{PY@tok@kd}{\let\PY@bf=\textbf\def\PY@tc##1{\textcolor[rgb]{0.00,0.50,0.00}{##1}}}
\@namedef{PY@tok@kn}{\let\PY@bf=\textbf\def\PY@tc##1{\textcolor[rgb]{0.00,0.50,0.00}{##1}}}
\@namedef{PY@tok@kr}{\let\PY@bf=\textbf\def\PY@tc##1{\textcolor[rgb]{0.00,0.50,0.00}{##1}}}
\@namedef{PY@tok@bp}{\def\PY@tc##1{\textcolor[rgb]{0.00,0.50,0.00}{##1}}}
\@namedef{PY@tok@fm}{\def\PY@tc##1{\textcolor[rgb]{0.00,0.00,1.00}{##1}}}
\@namedef{PY@tok@vc}{\def\PY@tc##1{\textcolor[rgb]{0.10,0.09,0.49}{##1}}}
\@namedef{PY@tok@vg}{\def\PY@tc##1{\textcolor[rgb]{0.10,0.09,0.49}{##1}}}
\@namedef{PY@tok@vi}{\def\PY@tc##1{\textcolor[rgb]{0.10,0.09,0.49}{##1}}}
\@namedef{PY@tok@vm}{\def\PY@tc##1{\textcolor[rgb]{0.10,0.09,0.49}{##1}}}
\@namedef{PY@tok@sa}{\def\PY@tc##1{\textcolor[rgb]{0.73,0.13,0.13}{##1}}}
\@namedef{PY@tok@sb}{\def\PY@tc##1{\textcolor[rgb]{0.73,0.13,0.13}{##1}}}
\@namedef{PY@tok@sc}{\def\PY@tc##1{\textcolor[rgb]{0.73,0.13,0.13}{##1}}}
\@namedef{PY@tok@dl}{\def\PY@tc##1{\textcolor[rgb]{0.73,0.13,0.13}{##1}}}
\@namedef{PY@tok@s2}{\def\PY@tc##1{\textcolor[rgb]{0.73,0.13,0.13}{##1}}}
\@namedef{PY@tok@sh}{\def\PY@tc##1{\textcolor[rgb]{0.73,0.13,0.13}{##1}}}
\@namedef{PY@tok@s1}{\def\PY@tc##1{\textcolor[rgb]{0.73,0.13,0.13}{##1}}}
\@namedef{PY@tok@mb}{\def\PY@tc##1{\textcolor[rgb]{0.40,0.40,0.40}{##1}}}
\@namedef{PY@tok@mf}{\def\PY@tc##1{\textcolor[rgb]{0.40,0.40,0.40}{##1}}}
\@namedef{PY@tok@mh}{\def\PY@tc##1{\textcolor[rgb]{0.40,0.40,0.40}{##1}}}
\@namedef{PY@tok@mi}{\def\PY@tc##1{\textcolor[rgb]{0.40,0.40,0.40}{##1}}}
\@namedef{PY@tok@il}{\def\PY@tc##1{\textcolor[rgb]{0.40,0.40,0.40}{##1}}}
\@namedef{PY@tok@mo}{\def\PY@tc##1{\textcolor[rgb]{0.40,0.40,0.40}{##1}}}
\@namedef{PY@tok@ch}{\let\PY@it=\textit\def\PY@tc##1{\textcolor[rgb]{0.24,0.48,0.48}{##1}}}
\@namedef{PY@tok@cm}{\let\PY@it=\textit\def\PY@tc##1{\textcolor[rgb]{0.24,0.48,0.48}{##1}}}
\@namedef{PY@tok@cpf}{\let\PY@it=\textit\def\PY@tc##1{\textcolor[rgb]{0.24,0.48,0.48}{##1}}}
\@namedef{PY@tok@c1}{\let\PY@it=\textit\def\PY@tc##1{\textcolor[rgb]{0.24,0.48,0.48}{##1}}}
\@namedef{PY@tok@cs}{\let\PY@it=\textit\def\PY@tc##1{\textcolor[rgb]{0.24,0.48,0.48}{##1}}}

\def\PYZbs{\char`\\}
\def\PYZus{\char`\_}
\def\PYZob{\char`\{}
\def\PYZcb{\char`\}}
\def\PYZca{\char`\^}
\def\PYZam{\char`\&}
\def\PYZlt{\char`\<}
\def\PYZgt{\char`\>}
\def\PYZsh{\char`\#}
\def\PYZpc{\char`\%}
\def\PYZdl{\char`\$}
\def\PYZhy{\char`\-}
\def\PYZsq{\char`\'}
\def\PYZdq{\char`\"}
\def\PYZti{\char`\~}
% for compatibility with earlier versions
\def\PYZat{@}
\def\PYZlb{[}
\def\PYZrb{]}
\makeatother


    % For linebreaks inside Verbatim environment from package fancyvrb. 
    \makeatletter
        \newbox\Wrappedcontinuationbox 
        \newbox\Wrappedvisiblespacebox 
        \newcommand*\Wrappedvisiblespace {\textcolor{red}{\textvisiblespace}} 
        \newcommand*\Wrappedcontinuationsymbol {\textcolor{red}{\llap{\tiny$\m@th\hookrightarrow$}}} 
        \newcommand*\Wrappedcontinuationindent {3ex } 
        \newcommand*\Wrappedafterbreak {\kern\Wrappedcontinuationindent\copy\Wrappedcontinuationbox} 
        % Take advantage of the already applied Pygments mark-up to insert 
        % potential linebreaks for TeX processing. 
        %        {, <, #, %, $, ' and ": go to next line. 
        %        _, }, ^, &, >, - and ~: stay at end of broken line. 
        % Use of \textquotesingle for straight quote. 
        \newcommand*\Wrappedbreaksatspecials {% 
            \def\PYGZus{\discretionary{\char`\_}{\Wrappedafterbreak}{\char`\_}}% 
            \def\PYGZob{\discretionary{}{\Wrappedafterbreak\char`\{}{\char`\{}}% 
            \def\PYGZcb{\discretionary{\char`\}}{\Wrappedafterbreak}{\char`\}}}% 
            \def\PYGZca{\discretionary{\char`\^}{\Wrappedafterbreak}{\char`\^}}% 
            \def\PYGZam{\discretionary{\char`\&}{\Wrappedafterbreak}{\char`\&}}% 
            \def\PYGZlt{\discretionary{}{\Wrappedafterbreak\char`\<}{\char`\<}}% 
            \def\PYGZgt{\discretionary{\char`\>}{\Wrappedafterbreak}{\char`\>}}% 
            \def\PYGZsh{\discretionary{}{\Wrappedafterbreak\char`\#}{\char`\#}}% 
            \def\PYGZpc{\discretionary{}{\Wrappedafterbreak\char`\%}{\char`\%}}% 
            \def\PYGZdl{\discretionary{}{\Wrappedafterbreak\char`\$}{\char`\$}}% 
            \def\PYGZhy{\discretionary{\char`\-}{\Wrappedafterbreak}{\char`\-}}% 
            \def\PYGZsq{\discretionary{}{\Wrappedafterbreak\textquotesingle}{\textquotesingle}}% 
            \def\PYGZdq{\discretionary{}{\Wrappedafterbreak\char`\"}{\char`\"}}% 
            \def\PYGZti{\discretionary{\char`\~}{\Wrappedafterbreak}{\char`\~}}% 
        } 
        % Some characters . , ; ? ! / are not pygmentized. 
        % This macro makes them "active" and they will insert potential linebreaks 
        \newcommand*\Wrappedbreaksatpunct {% 
            \lccode`\~`\.\lowercase{\def~}{\discretionary{\hbox{\char`\.}}{\Wrappedafterbreak}{\hbox{\char`\.}}}% 
            \lccode`\~`\,\lowercase{\def~}{\discretionary{\hbox{\char`\,}}{\Wrappedafterbreak}{\hbox{\char`\,}}}% 
            \lccode`\~`\;\lowercase{\def~}{\discretionary{\hbox{\char`\;}}{\Wrappedafterbreak}{\hbox{\char`\;}}}% 
            \lccode`\~`\:\lowercase{\def~}{\discretionary{\hbox{\char`\:}}{\Wrappedafterbreak}{\hbox{\char`\:}}}% 
            \lccode`\~`\?\lowercase{\def~}{\discretionary{\hbox{\char`\?}}{\Wrappedafterbreak}{\hbox{\char`\?}}}% 
            \lccode`\~`\!\lowercase{\def~}{\discretionary{\hbox{\char`\!}}{\Wrappedafterbreak}{\hbox{\char`\!}}}% 
            \lccode`\~`\/\lowercase{\def~}{\discretionary{\hbox{\char`\/}}{\Wrappedafterbreak}{\hbox{\char`\/}}}% 
            \catcode`\.\active
            \catcode`\,\active 
            \catcode`\;\active
            \catcode`\:\active
            \catcode`\?\active
            \catcode`\!\active
            \catcode`\/\active 
            \lccode`\~`\~ 	
        }
    \makeatother

    \let\OriginalVerbatim=\Verbatim
    \makeatletter
    \renewcommand{\Verbatim}[1][1]{%
        %\parskip\z@skip
        \sbox\Wrappedcontinuationbox {\Wrappedcontinuationsymbol}%
        \sbox\Wrappedvisiblespacebox {\FV@SetupFont\Wrappedvisiblespace}%
        \def\FancyVerbFormatLine ##1{\hsize\linewidth
            \vtop{\raggedright\hyphenpenalty\z@\exhyphenpenalty\z@
                \doublehyphendemerits\z@\finalhyphendemerits\z@
                \strut ##1\strut}%
        }%
        % If the linebreak is at a space, the latter will be displayed as visible
        % space at end of first line, and a continuation symbol starts next line.
        % Stretch/shrink are however usually zero for typewriter font.
        \def\FV@Space {%
            \nobreak\hskip\z@ plus\fontdimen3\font minus\fontdimen4\font
            \discretionary{\copy\Wrappedvisiblespacebox}{\Wrappedafterbreak}
            {\kern\fontdimen2\font}%
        }%
        
        % Allow breaks at special characters using \PYG... macros.
        \Wrappedbreaksatspecials
        % Breaks at punctuation characters . , ; ? ! and / need catcode=\active 	
        \OriginalVerbatim[#1,codes*=\Wrappedbreaksatpunct]%
    }
    \makeatother

    % Exact colors from NB
    \definecolor{incolor}{HTML}{303F9F}
    \definecolor{outcolor}{HTML}{D84315}
    \definecolor{cellborder}{HTML}{CFCFCF}
    \definecolor{cellbackground}{HTML}{F7F7F7}
    
    % prompt
    \makeatletter
    \newcommand{\boxspacing}{\kern\kvtcb@left@rule\kern\kvtcb@boxsep}
    \makeatother
    \newcommand{\prompt}[4]{
        {\ttfamily\llap{{\color{#2}[#3]:\hspace{3pt}#4}}\vspace{-\baselineskip}}
    }
    

    
    % Prevent overflowing lines due to hard-to-break entities
    \sloppy 
    % Setup hyperref package
    \hypersetup{
      breaklinks=true,  % so long urls are correctly broken across lines
      colorlinks=true,
      urlcolor=urlcolor,
      linkcolor=linkcolor,
      citecolor=citecolor,
      }
    % Slightly bigger margins than the latex defaults
    
    \geometry{verbose,tmargin=1in,bmargin=1in,lmargin=1in,rmargin=1in}
    
    

\begin{document}
    
    \maketitle
    
    

    
    Berdasarkan isu
\href{https://github.com/taruma/hidrokit/issues/90}{\#90}:
\textbf{kalibrasi model NRECA}

Referensi isu: - \texttt{hidrokit.contrib.taruma.hk89}
\href{https://github.com/taruma/hidrokit/issues/89}{\#89}.
(\href{https://gist.github.com/taruma/1502a7aa67cf074969d806cd3ffdf35c}{manual/notebook}).
\textbf{Pemodelan NRECA} - Roberts, W., Williams, G., Jackson, E.,
Nelson, E., Ames, D., 2018. Hydrostats: A Python Package for
Characterizing Errors between Observed and Predicted Time Series.
Hydrology 5(4) 66, doi:10.3390/hydrology5040066

Deskripsi permasalahan: - Mencari parameter model NRECA terbaik
berdasarkan metrik yang digunakan (RMSE, NSE, MAE, dll)

Strategi permasalahan: - Mengikuti ide \texttt{GridSearchCV} pada
\texttt{scikit-learn}, yang membuat \emph{parameter grid}, dan melakukan
pemodelan berdasarkan seluruh kombinasi yang memungkinkan dalam
\emph{parameter grid}. - Input dalam kalibrasi berupa fungsi model
(NRECA) dan metrik. Sehingga membuat kebebasan untuk menggunakan model
atau metrik yang dikembangkan sendiri.

Catatan: - Fungsi metrik bisa dikembangkan sendiri dengan bentuk
\texttt{nama\_metrik(simulasi,\ prediksi)}. - Dalam \emph{notebook} ini,
metrik hidrologi menggunakan paket \emph{HydroErr} yang telah
dikembangkan oleh BYU-Hydroinformatics
(\href{https://github.com/BYU-Hydroinformatics/HydroErr/}{github}).

    \hypertarget{persiapan-dan-dataset}{%
\section{PERSIAPAN DAN DATASET}\label{persiapan-dan-dataset}}

    \begin{tcolorbox}[breakable, size=fbox, boxrule=1pt, pad at break*=1mm,colback=cellbackground, colframe=cellborder]
\prompt{In}{incolor}{ }{\boxspacing}
\begin{Verbatim}[commandchars=\\\{\}]
\PY{k+kn}{import} \PY{n+nn}{numpy} \PY{k}{as} \PY{n+nn}{np}
\PY{k+kn}{import} \PY{n+nn}{pandas} \PY{k}{as} \PY{n+nn}{pd}
\PY{k+kn}{import} \PY{n+nn}{matplotlib}\PY{n+nn}{.}\PY{n+nn}{pyplot} \PY{k}{as} \PY{n+nn}{plt}
\end{Verbatim}
\end{tcolorbox}

    \begin{tcolorbox}[breakable, size=fbox, boxrule=1pt, pad at break*=1mm,colback=cellbackground, colframe=cellborder]
\prompt{In}{incolor}{ }{\boxspacing}
\begin{Verbatim}[commandchars=\\\{\}]
\PY{k}{try}\PY{p}{:}
    \PY{k+kn}{import} \PY{n+nn}{hidrokit}
\PY{k}{except} \PY{n+ne}{ModuleNotFoundError}\PY{p}{:}
    \PY{o}{!}pip install git+https://github.com/taruma/hidrokit.git@latest \PYZhy{}q
    \PY{k+kn}{import} \PY{n+nn}{hidrokit}
\PY{n+nb}{print}\PY{p}{(}\PY{l+s+sa}{f}\PY{l+s+s1}{\PYZsq{}}\PY{l+s+s1}{hidrokit version: }\PY{l+s+si}{\PYZob{}}\PY{n}{hidrokit}\PY{o}{.}\PY{n}{\PYZus{}\PYZus{}version\PYZus{}\PYZus{}}\PY{l+s+si}{\PYZcb{}}\PY{l+s+s1}{\PYZsq{}}\PY{p}{)}
\end{Verbatim}
\end{tcolorbox}

    \begin{Verbatim}[commandchars=\\\{\}]
  Building wheel for hidrokit (setup.py) {\ldots} done
hidrokit version: 0.3.5-beta.2
    \end{Verbatim}

    \begin{tcolorbox}[breakable, size=fbox, boxrule=1pt, pad at break*=1mm,colback=cellbackground, colframe=cellborder]
\prompt{In}{incolor}{ }{\boxspacing}
\begin{Verbatim}[commandchars=\\\{\}]
\PY{k}{try}\PY{p}{:}
    \PY{k+kn}{import} \PY{n+nn}{HydroErr} \PY{k}{as} \PY{n+nn}{he}
\PY{k}{except} \PY{n+ne}{ModuleNotFoundError}\PY{p}{:}
    \PY{o}{!}pip install HydroErr \PYZhy{}q
    \PY{k+kn}{import} \PY{n+nn}{HydroErr} \PY{k}{as} \PY{n+nn}{he}
\end{Verbatim}
\end{tcolorbox}

    \begin{Verbatim}[commandchars=\\\{\}]
  Building wheel for HydroErr (setup.py) {\ldots} done
    \end{Verbatim}

    \begin{tcolorbox}[breakable, size=fbox, boxrule=1pt, pad at break*=1mm,colback=cellbackground, colframe=cellborder]
\prompt{In}{incolor}{ }{\boxspacing}
\begin{Verbatim}[commandchars=\\\{\}]
\PY{c+c1}{\PYZsh{} LOAD DATASET}
\PY{k}{try}\PY{p}{:}
    \PY{n}{pd}\PY{o}{.}\PY{n}{read\PYZus{}excel}\PY{p}{(}\PY{l+s+s1}{\PYZsq{}}\PY{l+s+s1}{NRECA\PYZus{}sample.xlsx}\PY{l+s+s1}{\PYZsq{}}\PY{p}{)}
    \PY{n}{dataset\PYZus{}path} \PY{o}{=} \PY{l+s+s1}{\PYZsq{}}\PY{l+s+s1}{NRECA\PYZus{}sample.xlsx}\PY{l+s+s1}{\PYZsq{}}
\PY{k}{except}\PY{p}{:}
    \PY{k+kn}{from} \PY{n+nn}{google}\PY{n+nn}{.}\PY{n+nn}{colab} \PY{k+kn}{import} \PY{n}{files}
    \PY{n}{dataset\PYZus{}path} \PY{o}{=} \PY{n+nb}{list}\PY{p}{(}\PY{n}{files}\PY{o}{.}\PY{n}{upload}\PY{p}{(}\PY{p}{)}\PY{o}{.}\PY{n}{keys}\PY{p}{(}\PY{p}{)}\PY{p}{)}\PY{p}{[}\PY{l+m+mi}{0}\PY{p}{]}
\end{Verbatim}
\end{tcolorbox}

    
    \begin{Verbatim}[commandchars=\\\{\}]
<IPython.core.display.HTML object>
    \end{Verbatim}

    
    \begin{Verbatim}[commandchars=\\\{\}]
Saving NRECA\_sample.xlsx to NRECA\_sample.xlsx
    \end{Verbatim}

    \begin{tcolorbox}[breakable, size=fbox, boxrule=1pt, pad at break*=1mm,colback=cellbackground, colframe=cellborder]
\prompt{In}{incolor}{ }{\boxspacing}
\begin{Verbatim}[commandchars=\\\{\}]
\PY{n}{dataset} \PY{o}{=} \PY{n}{pd}\PY{o}{.}\PY{n}{read\PYZus{}excel}\PY{p}{(}\PY{n}{dataset\PYZus{}path}\PY{p}{,} \PY{n}{header}\PY{o}{=}\PY{l+m+mi}{0}\PY{p}{,} \PY{n}{index\PYZus{}col}\PY{o}{=}\PY{l+m+mi}{0}\PY{p}{,} \PY{n}{parse\PYZus{}dates}\PY{o}{=}\PY{k+kc}{True}\PY{p}{)}
\PY{n}{dataset}\PY{o}{.}\PY{n}{info}\PY{p}{(}\PY{p}{)}
\PY{n}{dataset}\PY{o}{.}\PY{n}{head}\PY{p}{(}\PY{p}{)}
\end{Verbatim}
\end{tcolorbox}

    \begin{Verbatim}[commandchars=\\\{\}]
<class 'pandas.core.frame.DataFrame'>
DatetimeIndex: 120 entries, 1999-01-01 to 2008-12-01
Data columns (total 3 columns):
PRECIP    120 non-null float64
PET       120 non-null float64
OBS       120 non-null float64
dtypes: float64(3)
memory usage: 3.8 KB
    \end{Verbatim}

            \begin{tcolorbox}[breakable, size=fbox, boxrule=.5pt, pad at break*=1mm, opacityfill=0]
\prompt{Out}{outcolor}{ }{\boxspacing}
\begin{Verbatim}[commandchars=\\\{\}]
                PRECIP     PET     OBS
1999-01-01  507.000000  142.63  259.00
1999-02-01  374.228918  128.84  164.22
1999-03-01  211.762683  138.19   86.26
1999-04-01  219.793874  138.32   60.21
1999-05-01  132.121255  125.36   64.82
\end{Verbatim}
\end{tcolorbox}
        
    \hypertarget{kode}{%
\section{KODE}\label{kode}}

    \begin{tcolorbox}[breakable, size=fbox, boxrule=1pt, pad at break*=1mm,colback=cellbackground, colframe=cellborder]
\prompt{In}{incolor}{ }{\boxspacing}
\begin{Verbatim}[commandchars=\\\{\}]
\PY{k+kn}{from} \PY{n+nn}{itertools} \PY{k+kn}{import} \PY{n}{product}

\PY{c+c1}{\PYZsh{}ref: sklearn.model\PYZus{}selection.ParameterGrid}
\PY{k}{def} \PY{n+nf}{\PYZus{}parameter\PYZus{}grid}\PY{p}{(}\PY{n}{parameter}\PY{p}{)}\PY{p}{:}
    \PY{n}{items} \PY{o}{=} \PY{n}{parameter}\PY{o}{.}\PY{n}{items}\PY{p}{(}\PY{p}{)}
    \PY{n}{keys}\PY{p}{,} \PY{n}{values} \PY{o}{=} \PY{n+nb}{zip}\PY{p}{(}\PY{o}{*}\PY{n}{items}\PY{p}{)}
    \PY{k}{for} \PY{n}{combination} \PY{o+ow}{in} \PY{n}{product}\PY{p}{(}\PY{o}{*}\PY{n}{values}\PY{p}{)}\PY{p}{:}
        \PY{n}{grid} \PY{o}{=} \PY{n+nb}{dict}\PY{p}{(}\PY{n+nb}{zip}\PY{p}{(}\PY{n}{keys}\PY{p}{,} \PY{n}{combination}\PY{p}{)}\PY{p}{)}
        \PY{k}{yield} \PY{n}{grid}

\PY{k}{def} \PY{n+nf}{\PYZus{}best\PYZus{}parameter}\PY{p}{(}\PY{n}{results}\PY{p}{,} \PY{n}{calibration\PYZus{}parameter}\PY{p}{)}\PY{p}{:}
    \PY{n}{key} \PY{o}{=} \PY{n+nb}{list}\PY{p}{(}\PY{n}{calibration\PYZus{}parameter}\PY{o}{.}\PY{n}{keys}\PY{p}{(}\PY{p}{)}\PY{p}{)}
    \PY{k}{return} \PY{n+nb}{dict}\PY{p}{(}\PY{n+nb}{zip}\PY{p}{(}\PY{n}{key}\PY{p}{,} \PY{n}{results}\PY{o}{.}\PY{n}{iloc}\PY{p}{[}\PY{l+m+mi}{0}\PY{p}{]}\PY{p}{[}\PY{n}{key}\PY{p}{]}\PY{o}{.}\PY{n}{values}\PY{p}{)}\PY{p}{)}

\PY{k}{def} \PY{n+nf}{calibration}\PY{p}{(}\PY{n}{observed}\PY{p}{,} \PY{n}{func}\PY{p}{,} \PY{n}{calibration\PYZus{}parameter}\PY{p}{,} \PY{n}{func\PYZus{}parameter}\PY{p}{,}
                \PY{n}{metrics}\PY{p}{,} \PY{n}{met\PYZus{}names}\PY{p}{,} \PY{n}{met\PYZus{}sort}\PY{p}{,} \PY{n}{met\PYZus{}min}\PY{o}{=}\PY{k+kc}{True}\PY{p}{,} \PY{n}{observed\PYZus{}func}\PY{o}{=}\PY{k+kc}{None}\PY{p}{)}\PY{p}{:}
    
    \PY{n}{metrics} \PY{o}{=} \PY{n}{metrics} \PY{k}{if} \PY{n+nb}{isinstance}\PY{p}{(}\PY{n}{metrics}\PY{p}{,} \PY{p}{(}\PY{n+nb}{list}\PY{p}{,} \PY{n+nb}{tuple}\PY{p}{)}\PY{p}{)} \PY{k}{else} \PY{p}{[}\PY{n}{metrics}\PY{p}{]}
    \PY{n}{met\PYZus{}names} \PY{o}{=} \PY{p}{(}\PY{n}{met\PYZus{}names} \PY{k}{if} \PY{n+nb}{isinstance}\PY{p}{(}\PY{n}{met\PYZus{}names}\PY{p}{,} \PY{p}{(}\PY{n+nb}{list}\PY{p}{)}\PY{p}{)} 
                           \PY{k}{else} \PY{p}{[}\PY{n}{met\PYZus{}names}\PY{p}{]}\PY{p}{)}

    \PY{n}{param\PYZus{}grid} \PY{o}{=} \PY{n+nb}{list}\PY{p}{(}\PY{n}{\PYZus{}parameter\PYZus{}grid}\PY{p}{(}\PY{n}{calibration\PYZus{}parameter}\PY{p}{)}\PY{p}{)}
    \PY{n}{n\PYZus{}param} \PY{o}{=} \PY{n+nb}{len}\PY{p}{(}\PY{n}{param\PYZus{}grid}\PY{p}{)}
    \PY{n+nb}{print}\PY{p}{(}\PY{l+s+s1}{\PYZsq{}}\PY{l+s+s1}{N = }\PY{l+s+si}{\PYZob{}\PYZcb{}}\PY{l+s+s1}{\PYZsq{}}\PY{o}{.}\PY{n}{format}\PY{p}{(}\PY{n}{n\PYZus{}param}\PY{p}{)}\PY{p}{)}

    \PY{n}{observed} \PY{o}{=} \PY{p}{(}
        \PY{n}{observed\PYZus{}func}\PY{p}{(}\PY{n}{observed}\PY{p}{)} \PY{k}{if} \PY{n}{observed\PYZus{}func} \PY{o+ow}{is} \PY{o+ow}{not} \PY{k+kc}{None} \PY{k}{else} \PY{n}{observed}
    \PY{p}{)}

    \PY{n}{results} \PY{o}{=} \PY{p}{[}\PY{p}{]}

    \PY{n+nb}{print}\PY{p}{(}\PY{l+s+s1}{\PYZsq{}}\PY{l+s+s1}{PROGRESS 0 [\PYZhy{}x\PYZhy{}\PYZhy{}xx\PYZhy{}\PYZhy{}x\PYZhy{}] 100}\PY{l+s+s1}{\PYZsq{}}\PY{p}{)}
    \PY{n+nb}{print}\PY{p}{(}\PY{l+s+s1}{\PYZsq{}}\PY{l+s+s1}{\PYZhy{}\PYZhy{}\PYZhy{}\PYZhy{}\PYZhy{}\PYZhy{}\PYZhy{}\PYZhy{}\PYZhy{}\PYZgt{} [}\PY{l+s+s1}{\PYZsq{}}\PY{p}{,} \PY{n}{end}\PY{o}{=}\PY{l+s+s1}{\PYZsq{}}\PY{l+s+s1}{\PYZsq{}}\PY{p}{)}

    \PY{k}{for} \PY{n}{i}\PY{p}{,} \PY{n}{p} \PY{o+ow}{in} \PY{n+nb}{enumerate}\PY{p}{(}\PY{n}{param\PYZus{}grid}\PY{p}{,} \PY{n}{start}\PY{o}{=}\PY{l+m+mi}{1}\PY{p}{)}\PY{p}{:}
        \PY{n}{simulated} \PY{o}{=} \PY{n}{func}\PY{p}{(}\PY{o}{*}\PY{o}{*}\PY{n}{p}\PY{p}{,} \PY{o}{*}\PY{o}{*}\PY{n}{func\PYZus{}parameter}\PY{p}{)}
        \PY{n}{met\PYZus{}res} \PY{o}{=} \PY{p}{[}\PY{n}{m}\PY{p}{(}\PY{n}{simulated}\PY{p}{,} \PY{n}{observed}\PY{p}{)} \PY{k}{for} \PY{n}{m} \PY{o+ow}{in} \PY{n}{metrics}\PY{p}{]}
        \PY{n}{results}\PY{o}{.}\PY{n}{append}\PY{p}{(}
            \PY{n+nb}{list}\PY{p}{(}\PY{n}{p}\PY{o}{.}\PY{n}{values}\PY{p}{(}\PY{p}{)}\PY{p}{)} \PY{o}{+} \PY{n}{met\PYZus{}res}
        \PY{p}{)}
        \PY{k}{if} \PY{p}{(}\PY{n}{i} \PY{o}{\PYZpc{}} \PY{p}{(}\PY{n}{n\PYZus{}param} \PY{o}{/}\PY{o}{/} \PY{l+m+mi}{10}\PY{p}{)}\PY{p}{)} \PY{o}{==} \PY{l+m+mi}{0}\PY{p}{:}
            \PY{n+nb}{print}\PY{p}{(}\PY{l+s+s1}{\PYZsq{}}\PY{l+s+s1}{=}\PY{l+s+s1}{\PYZsq{}}\PY{p}{,} \PY{n}{end}\PY{o}{=}\PY{l+s+s1}{\PYZsq{}}\PY{l+s+s1}{\PYZsq{}}\PY{p}{)}
    
    \PY{n+nb}{print}\PY{p}{(}\PY{l+s+s1}{\PYZsq{}}\PY{l+s+s1}{] DONE}\PY{l+s+s1}{\PYZsq{}}\PY{p}{)}

    \PY{n}{columns\PYZus{}name} \PY{o}{=} \PY{n+nb}{list}\PY{p}{(}\PY{n}{p}\PY{o}{.}\PY{n}{keys}\PY{p}{(}\PY{p}{)}\PY{p}{)} \PY{o}{+} \PY{n}{met\PYZus{}names}

    \PY{n}{results} \PY{o}{=} \PY{p}{(}\PY{n}{pd}\PY{o}{.}\PY{n}{DataFrame}\PY{p}{(}\PY{n}{results}\PY{p}{,} \PY{n}{columns}\PY{o}{=}\PY{n}{columns\PYZus{}name}\PY{p}{)}
                 \PY{o}{.}\PY{n}{sort\PYZus{}values}\PY{p}{(}\PY{n}{by}\PY{o}{=}\PY{n}{met\PYZus{}sort}\PY{p}{,} \PY{n}{ascending}\PY{o}{=}\PY{n}{met\PYZus{}min}\PY{p}{)}
                 \PY{o}{.}\PY{n}{reset\PYZus{}index}\PY{p}{(}\PY{n}{drop}\PY{o}{=}\PY{k+kc}{True}\PY{p}{)}\PY{p}{)}

    \PY{k}{return} \PY{n}{results}
\end{Verbatim}
\end{tcolorbox}

    \hypertarget{fungsi}{%
\section{FUNGSI}\label{fungsi}}

    \hypertarget{fungsi-calibration}{%
\subsection{\texorpdfstring{Fungsi
\texttt{calibration()}}{Fungsi calibration()}}\label{fungsi-calibration}}

Fungsi \texttt{calibration()} mengiterasi seluruh kombinasi parameter
yang ada dan memeriksa hasilnya dengan metrik yang diberikan, kemudian
menyusunnya dan mengurutkan berdasarkan metrik pilihan dan disajikan
dalam bentuk \texttt{pandas.DataFrame}.

Fungsi ini tidak terbatas dengan fungsi model \texttt{model\_NRECA()},
fungsi ini dirancang untuk menerima input fungsi model yang dikembangkan
sendiri dengan syarat fungsi tersebut memberi keluaran dalam bentuk yang
diminta oleh fungsi metrik.

    \hypertarget{argumen-calibration}{%
\subsubsection{\texorpdfstring{Argumen
\texttt{calibration()}}{Argumen calibration()}}\label{argumen-calibration}}

Fungsi meminta 7 argumen posisi dan 2 argumen opsional. Argumen ini bisa
dibagi menjadi beberapa bagian:

\hypertarget{argumen-fungsi}{%
\paragraph{Argumen fungsi}\label{argumen-fungsi}}

\begin{itemize}
\tightlist
\item
  \texttt{observed=}, argumen ini meminta data observasi atau data yang
  akan dibandingkan hasilnya dengan nilai dari pemodelan.
\item
  \texttt{observed\_func=} (\emph{default}=\texttt{None}), argumen
  opsional ini membungkus nilai \texttt{observed} yang disediakan agar
  nilai \texttt{observed} menyesuaikan dengan jenis masukan yang diminta
  oleh fungsi metrik.
\end{itemize}

\hypertarget{argumen-model}{%
\paragraph{Argumen model}\label{argumen-model}}

\begin{itemize}
\tightlist
\item
  \texttt{func=}, argumen ini meminta fungsi model. Contoh: untuk NRECA
  maka \texttt{func=model\_NRECA}.
\item
  \texttt{calibration\_parameter=}, argumen ini meminta
  \emph{dictionary} daftar argumen yang akan digunakan sebagai input
  \texttt{func} dan diiterasi seluruh kombinasi dari parameter yang
  disediakan.
\item
  \texttt{func\_parameter=}, argumen ini serupa dengan
  \texttt{calibration\_parameter}, akan tetapi argumen ini meminta
  \emph{dictionary} daftar input \texttt{func} yang tidak akan
  diiterasi.
\end{itemize}

\hypertarget{argumen-metrik}{%
\paragraph{Argumen metrik}\label{argumen-metrik}}

\begin{itemize}
\tightlist
\item
  \texttt{metrics=}, argumen ini meminta \emph{list} fungsi atau fungsi
  yang digunakan untuk mengevaluasi nilai simulasi dengan observasi.
  Contoh: metrik NSE \texttt{metrics=he.NSE} atau dalam bentuk list
  \texttt{metrics={[}he.NSE,\ he.rmse{]}}.
\item
  \texttt{met\_names=}, argumen ini meminta \emph{list} nama dari fungsi
  metrik.
\item
  \texttt{met\_sort=}, argumen ini meminta nama metrik yang akan
  diurutkan.
\item
  \texttt{met\_min=} (\emph{default}=\texttt{True}), mengurutkan kolom
  \texttt{met\_sort} dari kecil ke besar atau besar ke kecil jika
  \texttt{False}.
\end{itemize}

    \hypertarget{penggunaan}{%
\section{PENGGUNAAN}\label{penggunaan}}

Dalam \emph{notebook} ini akan menggunakan metrik yang telah disediakan
pada paket HydroErr.

    \hypertarget{parameter-metrik}{%
\subsection{Parameter metrik}\label{parameter-metrik}}

Fungsi \texttt{calibration()} membutuhkan argumen untuk parameter metrik
berupa \texttt{metrics}, \texttt{met\_names}, dan \texttt{met\_sort};
argumen opsional \texttt{met\_min}.

Metrik dapat dikembangkan sendiri dengan bentuk
\texttt{nama\_metrik(simulasi,\ observasi)}. Untuk penggunaan praktis,
telah tersedia paket HydroErr oleh BYU-Hydroinformatics yang
mempersiapkan daftar metrik dalam hidrologi. Daftar metrik yang tersedia
bisa baca
\href{https://hydroerr.readthedocs.io/en/stable/list_of_metrics.html}{disini}.

Dalam \emph{notebook} ini akan diberikan contoh penggunaan fungsi metrik
sendiri dan menggunakan metrik dari HydroErr. Metrik yang akan digunakan
antara lain: \texttt{nama\_metrik} (metrik buatan sendiri),
\texttt{he.rmse} \emph{root mean square error}, \texttt{he.r\_squared}
\emph{coefficient of determination}, \texttt{he.nse}
\emph{Nash-Sutcliffe Efficiency}, \texttt{he.kge\_2012}
\emph{Kling-Gupta efficiency} (2012).

Hasil kalibrasi akan diurutkan berdasarkan \texttt{NSE}. Karena performa
\texttt{NSE} dinyatakan lebih baik jika nilainya mendekati maksimum (1),
maka argumen \texttt{met\_min=False} agar pengurutan nilai dari besar ke
kecil.

Fungsi metrik hanya menerima input dalam bentuk \texttt{numpy.array}
atau \emph{list}. Sehingga keluaran dari model harus disesuaikan dengan
permintaan metrik.

Isian untuk \texttt{metrics} dan \texttt{met\_names} \textbf{harus
berupa \emph{list}}. Panjang (\emph{list}) \texttt{met\_names} dan
\texttt{metrics} harus sama.

    \begin{tcolorbox}[breakable, size=fbox, boxrule=1pt, pad at break*=1mm,colback=cellbackground, colframe=cellborder]
\prompt{In}{incolor}{ }{\boxspacing}
\begin{Verbatim}[commandchars=\\\{\}]
\PY{k+kn}{import} \PY{n+nn}{HydroErr} \PY{k}{as} \PY{n+nn}{he}

\PY{k}{def} \PY{n+nf}{nama\PYZus{}metrik}\PY{p}{(}\PY{n}{simulasi}\PY{p}{,} \PY{n}{observasi}\PY{p}{)}\PY{p}{:}
    \PY{k}{return} \PY{n}{np}\PY{o}{.}\PY{n}{mean}\PY{p}{(}\PY{n}{simulasi}\PY{p}{)}\PY{o}{/}\PY{n}{np}\PY{o}{.}\PY{n}{mean}\PY{p}{(}\PY{n}{observasi}\PY{p}{)}

\PY{c+c1}{\PYZsh{} Dalam bentuk dictionary agar lebih mudah dibaca}
\PY{n}{metrics\PYZus{}parameter} \PY{o}{=} \PY{p}{\PYZob{}}
    \PY{l+s+s1}{\PYZsq{}}\PY{l+s+s1}{metrics}\PY{l+s+s1}{\PYZsq{}}\PY{p}{:} \PY{p}{[}\PY{n}{nama\PYZus{}metrik}\PY{p}{,} \PY{n}{he}\PY{o}{.}\PY{n}{rmse}\PY{p}{,} \PY{n}{he}\PY{o}{.}\PY{n}{r\PYZus{}squared}\PY{p}{,} \PY{n}{he}\PY{o}{.}\PY{n}{nse}\PY{p}{,} \PY{n}{he}\PY{o}{.}\PY{n}{kge\PYZus{}2012}\PY{p}{]}\PY{p}{,}
    \PY{l+s+s1}{\PYZsq{}}\PY{l+s+s1}{met\PYZus{}names}\PY{l+s+s1}{\PYZsq{}}\PY{p}{:} \PY{p}{[}\PY{l+s+s1}{\PYZsq{}}\PY{l+s+s1}{NAMA\PYZus{}METRIK}\PY{l+s+s1}{\PYZsq{}}\PY{p}{,} \PY{l+s+s1}{\PYZsq{}}\PY{l+s+s1}{RMSE}\PY{l+s+s1}{\PYZsq{}}\PY{p}{,} \PY{l+s+s1}{\PYZsq{}}\PY{l+s+s1}{R2}\PY{l+s+s1}{\PYZsq{}}\PY{p}{,} \PY{l+s+s1}{\PYZsq{}}\PY{l+s+s1}{NSE}\PY{l+s+s1}{\PYZsq{}}\PY{p}{,} \PY{l+s+s1}{\PYZsq{}}\PY{l+s+s1}{KGE\PYZus{}2012}\PY{l+s+s1}{\PYZsq{}}\PY{p}{]}\PY{p}{,}
    \PY{l+s+s1}{\PYZsq{}}\PY{l+s+s1}{met\PYZus{}sort}\PY{l+s+s1}{\PYZsq{}}\PY{p}{:} \PY{l+s+s1}{\PYZsq{}}\PY{l+s+s1}{NSE}\PY{l+s+s1}{\PYZsq{}}\PY{p}{,}
    \PY{l+s+s1}{\PYZsq{}}\PY{l+s+s1}{met\PYZus{}min}\PY{l+s+s1}{\PYZsq{}}\PY{p}{:} \PY{k+kc}{False}
\PY{p}{\PYZcb{}}
\end{Verbatim}
\end{tcolorbox}

    \hypertarget{parameter-model-nreca}{%
\subsection{Parameter Model NRECA}\label{parameter-model-nreca}}

Diketahui bahwa \texttt{model\_NRECA()} membutuhkan 10 argumen dan 2
argumen opsional. Argumen tersebut dapat dibagi menjadi dua kategori
yaitu \texttt{calibration\_parameter} dan \texttt{func\_parameter}.

    \hypertarget{calibration_parameter}{%
\subsubsection{\texorpdfstring{\texttt{calibration\_parameter}}{calibration\_parameter}}\label{calibration_parameter}}

Dalam model NRECA, ingin mencari tahu kombinasi parameter terbaik dari
\texttt{MSTOR}, \texttt{GSTOR}, \texttt{PSUB}, \texttt{GWF}. Isian
setiap \emph{key} (\emph{values}) \textbf{harus berupa \emph{list} /
\emph{iterable}}.

    \begin{tcolorbox}[breakable, size=fbox, boxrule=1pt, pad at break*=1mm,colback=cellbackground, colframe=cellborder]
\prompt{In}{incolor}{ }{\boxspacing}
\begin{Verbatim}[commandchars=\\\{\}]
\PY{n}{calibration\PYZus{}parameter} \PY{o}{=} \PY{p}{\PYZob{}}
    \PY{l+s+s1}{\PYZsq{}}\PY{l+s+s1}{MSTOR}\PY{l+s+s1}{\PYZsq{}}\PY{p}{:} \PY{p}{[}\PY{l+m+mi}{1000}\PY{p}{,} \PY{l+m+mi}{1100}\PY{p}{,} \PY{l+m+mi}{1200}\PY{p}{]}\PY{p}{,} \PY{c+c1}{\PYZsh{} dalam bentuk list}
    \PY{l+s+s1}{\PYZsq{}}\PY{l+s+s1}{GSTOR}\PY{l+s+s1}{\PYZsq{}}\PY{p}{:} \PY{p}{(}\PY{l+m+mi}{100}\PY{p}{,} \PY{l+m+mi}{110}\PY{p}{,} \PY{l+m+mi}{120}\PY{p}{)}\PY{p}{,} \PY{c+c1}{\PYZsh{} dalam bentuk tuple}
    \PY{l+s+s1}{\PYZsq{}}\PY{l+s+s1}{PSUB}\PY{l+s+s1}{\PYZsq{}}\PY{p}{:} \PY{n}{np}\PY{o}{.}\PY{n}{linspace}\PY{p}{(}\PY{l+m+mf}{0.3}\PY{p}{,} \PY{l+m+mf}{0.8}\PY{p}{,} \PY{l+m+mi}{31}\PY{p}{)}\PY{p}{,} \PY{c+c1}{\PYZsh{} menggunakan numpy.linspace}
    \PY{l+s+s1}{\PYZsq{}}\PY{l+s+s1}{GWF}\PY{l+s+s1}{\PYZsq{}}\PY{p}{:} \PY{n}{np}\PY{o}{.}\PY{n}{arange}\PY{p}{(}\PY{l+m+mf}{0.2}\PY{p}{,} \PY{l+m+mf}{0.91}\PY{p}{,} \PY{l+m+mf}{0.1}\PY{p}{)}\PY{p}{,} \PY{c+c1}{\PYZsh{} menggunakan numpy.arange}
\PY{p}{\PYZcb{}}
\end{Verbatim}
\end{tcolorbox}

    \hypertarget{func_parameter}{%
\subsubsection{\texorpdfstring{\texttt{func\_parameter}}{func\_parameter}}\label{func_parameter}}

Argumen model yang diperlukan oleh \texttt{model\_NRECA()} disertakan
dalam \texttt{func\_parameter}. Argumen \texttt{as\_df=False} memastikan
hasil keluaran model berbentuk \texttt{numpy.ndarray} karena permintaan
dari fungsi metrik.

    \begin{tcolorbox}[breakable, size=fbox, boxrule=1pt, pad at break*=1mm,colback=cellbackground, colframe=cellborder]
\prompt{In}{incolor}{ }{\boxspacing}
\begin{Verbatim}[commandchars=\\\{\}]
\PY{n}{func\PYZus{}parameter} \PY{o}{=} \PY{p}{\PYZob{}}
    \PY{l+s+s1}{\PYZsq{}}\PY{l+s+s1}{df}\PY{l+s+s1}{\PYZsq{}}\PY{p}{:} \PY{n}{dataset}\PY{p}{,}
    \PY{l+s+s1}{\PYZsq{}}\PY{l+s+s1}{precip\PYZus{}col}\PY{l+s+s1}{\PYZsq{}}\PY{p}{:} \PY{l+s+s1}{\PYZsq{}}\PY{l+s+s1}{PRECIP}\PY{l+s+s1}{\PYZsq{}}\PY{p}{,}
    \PY{l+s+s1}{\PYZsq{}}\PY{l+s+s1}{pet\PYZus{}col}\PY{l+s+s1}{\PYZsq{}}\PY{p}{:} \PY{l+s+s1}{\PYZsq{}}\PY{l+s+s1}{PET}\PY{l+s+s1}{\PYZsq{}}\PY{p}{,}
    
    \PY{l+s+s1}{\PYZsq{}}\PY{l+s+s1}{AREA}\PY{l+s+s1}{\PYZsq{}}\PY{p}{:} \PY{l+m+mf}{1450.6e6}\PY{p}{,}
    \PY{l+s+s1}{\PYZsq{}}\PY{l+s+s1}{CF}\PY{l+s+s1}{\PYZsq{}}\PY{p}{:} \PY{l+m+mf}{0.6}\PY{p}{,}
    \PY{l+s+s1}{\PYZsq{}}\PY{l+s+s1}{C}\PY{l+s+s1}{\PYZsq{}}\PY{p}{:} \PY{l+m+mf}{0.25}\PY{p}{,}

    \PY{l+s+s1}{\PYZsq{}}\PY{l+s+s1}{as\PYZus{}df}\PY{l+s+s1}{\PYZsq{}}\PY{p}{:} \PY{k+kc}{False}
\PY{p}{\PYZcb{}}
\end{Verbatim}
\end{tcolorbox}

    \hypertarget{kalibrasi}{%
\subsection{Kalibrasi}\label{kalibrasi}}

Nilai observasi disimpan pada \texttt{dataset} kolom \texttt{OBS}.
Karena fungsi metrik meminta input dalam bentuk \texttt{numpy.array}
maka nilai observasi harus diubah sebelum disertakan ke dalam fungsi
\texttt{calibration()}.

    \begin{tcolorbox}[breakable, size=fbox, boxrule=1pt, pad at break*=1mm,colback=cellbackground, colframe=cellborder]
\prompt{In}{incolor}{ }{\boxspacing}
\begin{Verbatim}[commandchars=\\\{\}]
\PY{k+kn}{from} \PY{n+nn}{hidrokit}\PY{n+nn}{.}\PY{n+nn}{contrib}\PY{n+nn}{.}\PY{n+nn}{taruma}\PY{n+nn}{.}\PY{n+nn}{hk89} \PY{k+kn}{import} \PY{n}{model\PYZus{}NRECA}

\PY{n}{observed\PYZus{}values} \PY{o}{=} \PY{n}{dataset}\PY{o}{.}\PY{n}{loc}\PY{p}{[}\PY{p}{:}\PY{p}{,} \PY{l+s+s1}{\PYZsq{}}\PY{l+s+s1}{OBS}\PY{l+s+s1}{\PYZsq{}}\PY{p}{]}\PY{o}{.}\PY{n}{values}

\PY{n}{results} \PY{o}{=} \PY{n}{calibration}\PY{p}{(}\PY{n}{observed\PYZus{}values}\PY{p}{,}
                      \PY{n}{model\PYZus{}NRECA}\PY{p}{,} \PY{n}{calibration\PYZus{}parameter}\PY{p}{,} \PY{n}{func\PYZus{}parameter}\PY{p}{,}
                      \PY{o}{*}\PY{o}{*}\PY{n}{metrics\PYZus{}parameter}\PY{p}{)}
\end{Verbatim}
\end{tcolorbox}

    \begin{Verbatim}[commandchars=\\\{\}]
N = 2232
PROGRESS 0 [-x--xx--x-] 100
---------> [==========] DONE
    \end{Verbatim}

    \hypertarget{hasil-kalibrasi}{%
\subsection{Hasil Kalibrasi}\label{hasil-kalibrasi}}

Hasil kalibrasi disimpan di \texttt{results} dalam bentuk
\texttt{pandas.DataFrame} yang telah diurutkan berdasarkan nilai
\texttt{NSE}. Berikut nilai 10 NSE terbaik beserta parameter yang
dikalibrasi disertai metrik yang lain:

    \begin{tcolorbox}[breakable, size=fbox, boxrule=1pt, pad at break*=1mm,colback=cellbackground, colframe=cellborder]
\prompt{In}{incolor}{ }{\boxspacing}
\begin{Verbatim}[commandchars=\\\{\}]
\PY{n}{results}\PY{o}{.}\PY{n}{head}\PY{p}{(}\PY{l+m+mi}{10}\PY{p}{)}
\end{Verbatim}
\end{tcolorbox}

            \begin{tcolorbox}[breakable, size=fbox, boxrule=.5pt, pad at break*=1mm, opacityfill=0]
\prompt{Out}{outcolor}{ }{\boxspacing}
\begin{Verbatim}[commandchars=\\\{\}]
   MSTOR  GSTOR      PSUB  GWF  {\ldots}       RMSE        R2       NSE  KGE\_2012
0   1200    120  0.400000  0.2  {\ldots}  36.735131  0.553446  0.526348  0.701803
1   1200    120  0.416667  0.2  {\ldots}  36.739909  0.549800  0.526225  0.693170
2   1200    120  0.383333  0.2  {\ldots}  36.753441  0.556862  0.525876  0.709968
3   1200    120  0.433333  0.2  {\ldots}  36.767768  0.545910  0.525506  0.684108
4   1200    110  0.400000  0.2  {\ldots}  36.771415  0.552507  0.525412  0.701492
5   1200    110  0.416667  0.2  {\ldots}  36.777133  0.548837  0.525264  0.692858
6   1200    110  0.383333  0.2  {\ldots}  36.788763  0.555946  0.524964  0.709658
7   1200    120  0.366667  0.2  {\ldots}  36.794806  0.560062  0.524808  0.717623
8   1200    110  0.433333  0.2  {\ldots}  36.805908  0.544922  0.524521  0.683794
9   1200    100  0.400000  0.2  {\ldots}  36.808446  0.551555  0.524455  0.701176

[10 rows x 9 columns]
\end{Verbatim}
\end{tcolorbox}
        
    Mengambil nilai parameter terbaik (baris pertama) dengan fungsi
\texttt{\_best\_parameter()}

    \begin{tcolorbox}[breakable, size=fbox, boxrule=1pt, pad at break*=1mm,colback=cellbackground, colframe=cellborder]
\prompt{In}{incolor}{ }{\boxspacing}
\begin{Verbatim}[commandchars=\\\{\}]
\PY{n}{best\PYZus{}param} \PY{o}{=} \PY{n}{\PYZus{}best\PYZus{}parameter}\PY{p}{(}\PY{n}{results}\PY{p}{,} \PY{n}{calibration\PYZus{}parameter}\PY{p}{)}
\PY{n}{best\PYZus{}param}
\end{Verbatim}
\end{tcolorbox}

            \begin{tcolorbox}[breakable, size=fbox, boxrule=.5pt, pad at break*=1mm, opacityfill=0]
\prompt{Out}{outcolor}{ }{\boxspacing}
\begin{Verbatim}[commandchars=\\\{\}]
\{'GSTOR': 120.0, 'GWF': 0.2, 'MSTOR': 1200.0, 'PSUB': 0.4\}
\end{Verbatim}
\end{tcolorbox}
        
    Memodelkan dengan parameter terbaik yang diperoleh dari hasil kalibrasi.
Karena pada \texttt{func\_parameter} argumen \texttt{as\_df=False} maka
dibuat \emph{dictionary} baru yang memberikan argumen
\texttt{as\_df=True}.

    \begin{tcolorbox}[breakable, size=fbox, boxrule=1pt, pad at break*=1mm,colback=cellbackground, colframe=cellborder]
\prompt{In}{incolor}{ }{\boxspacing}
\begin{Verbatim}[commandchars=\\\{\}]
\PY{n}{func\PYZus{}parameter\PYZus{}df} \PY{o}{=} \PY{p}{\PYZob{}}
    \PY{l+s+s1}{\PYZsq{}}\PY{l+s+s1}{df}\PY{l+s+s1}{\PYZsq{}}\PY{p}{:} \PY{n}{dataset}\PY{p}{,}
    \PY{l+s+s1}{\PYZsq{}}\PY{l+s+s1}{precip\PYZus{}col}\PY{l+s+s1}{\PYZsq{}}\PY{p}{:} \PY{l+s+s1}{\PYZsq{}}\PY{l+s+s1}{PRECIP}\PY{l+s+s1}{\PYZsq{}}\PY{p}{,}
    \PY{l+s+s1}{\PYZsq{}}\PY{l+s+s1}{pet\PYZus{}col}\PY{l+s+s1}{\PYZsq{}}\PY{p}{:} \PY{l+s+s1}{\PYZsq{}}\PY{l+s+s1}{PET}\PY{l+s+s1}{\PYZsq{}}\PY{p}{,}
    
    \PY{l+s+s1}{\PYZsq{}}\PY{l+s+s1}{AREA}\PY{l+s+s1}{\PYZsq{}}\PY{p}{:} \PY{l+m+mf}{1450.6e6}\PY{p}{,}
    \PY{l+s+s1}{\PYZsq{}}\PY{l+s+s1}{CF}\PY{l+s+s1}{\PYZsq{}}\PY{p}{:} \PY{l+m+mf}{0.6}\PY{p}{,}
    \PY{l+s+s1}{\PYZsq{}}\PY{l+s+s1}{C}\PY{l+s+s1}{\PYZsq{}}\PY{p}{:} \PY{l+m+mf}{0.25}\PY{p}{,}

    \PY{l+s+s1}{\PYZsq{}}\PY{l+s+s1}{as\PYZus{}df}\PY{l+s+s1}{\PYZsq{}}\PY{p}{:} \PY{k+kc}{True}
\PY{p}{\PYZcb{}}

\PY{c+c1}{\PYZsh{} results with best parameter}
\PY{n}{model\PYZus{}NRECA}\PY{p}{(}\PY{o}{*}\PY{o}{*}\PY{n}{func\PYZus{}parameter\PYZus{}df}\PY{p}{,} 
            \PY{o}{*}\PY{o}{*}\PY{n}{best\PYZus{}param}\PY{p}{,} \PY{n}{report}\PY{o}{=}\PY{l+s+s1}{\PYZsq{}}\PY{l+s+s1}{full}\PY{l+s+s1}{\PYZsq{}}\PY{p}{)}
\end{Verbatim}
\end{tcolorbox}

            \begin{tcolorbox}[breakable, size=fbox, boxrule=.5pt, pad at break*=1mm, opacityfill=0]
\prompt{Out}{outcolor}{ }{\boxspacing}
\begin{Verbatim}[commandchars=\\\{\}]
            DAYS      PRECIP     PET  {\ldots}       DFLOW        FLOW   DISCHARGE
1999-01-01  31.0  507.000000  142.63  {\ldots}  231.359678  286.207635  155.007764
1999-02-01  28.0  374.228918  128.84  {\ldots}  166.292510  232.343210  139.317568
1999-03-01  31.0  211.762683  138.19  {\ldots}   72.873100  135.430074   73.347844
1999-04-01  30.0  219.793874  138.32  {\ldots}   77.639273  138.036755   77.251588
1999-05-01  31.0  132.121255  125.36  {\ldots}   32.403866   85.042367   46.058265
{\ldots}          {\ldots}         {\ldots}     {\ldots}  {\ldots}         {\ldots}         {\ldots}         {\ldots}
2008-08-01  31.0  117.425578  137.29  {\ldots}   20.532130   51.771615   28.039092
2008-09-01  30.0   66.624410  161.50  {\ldots}    0.000000   24.991588   13.986419
2008-10-01  31.0  265.947354  166.58  {\ldots}   89.869858  121.845776   65.990697
2008-11-01  30.0  249.151252  150.56  {\ldots}   86.862225  124.024589   69.409749
2008-12-01  31.0  199.088447  144.15  {\ldots}   62.100769  100.110763   54.219188

[120 rows x 18 columns]
\end{Verbatim}
\end{tcolorbox}
        
    \hypertarget{changelog}{%
\section{Changelog}\label{changelog}}

\begin{verbatim}
- 20191215 - 1.0.0 - Initial
\end{verbatim}

\hypertarget{copyright-2019-taruma-sakti-megariansyah}{%
\paragraph{\texorpdfstring{Copyright © 2019
\href{https://taruma.github.io}{Taruma Sakti
Megariansyah}}{Copyright © 2019 Taruma Sakti Megariansyah}}\label{copyright-2019-taruma-sakti-megariansyah}}

Source code in this notebook is licensed under a
\href{https://choosealicense.com/licenses/mit/}{MIT License}. Data in
this notebook is licensed under a
\href{https://creativecommons.org/licenses/by/4.0/}{Creative Common
Attribution 4.0 International}.


    % Add a bibliography block to the postdoc
    
    
    
\end{document}
