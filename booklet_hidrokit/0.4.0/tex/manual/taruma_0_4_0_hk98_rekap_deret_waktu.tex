\documentclass[11pt]{article}

    \usepackage[breakable]{tcolorbox}
    \usepackage{parskip} % Stop auto-indenting (to mimic markdown behaviour)
    
    \usepackage{iftex}
    \ifPDFTeX
    	\usepackage[T1]{fontenc}
    	\usepackage{mathpazo}
    \else
    	\usepackage{fontspec}
    \fi

    % Basic figure setup, for now with no caption control since it's done
    % automatically by Pandoc (which extracts ![](path) syntax from Markdown).
    \usepackage{graphicx}
    % Maintain compatibility with old templates. Remove in nbconvert 6.0
    \let\Oldincludegraphics\includegraphics
    % Ensure that by default, figures have no caption (until we provide a
    % proper Figure object with a Caption API and a way to capture that
    % in the conversion process - todo).
    \usepackage{caption}
    \DeclareCaptionFormat{nocaption}{}
    \captionsetup{format=nocaption,aboveskip=0pt,belowskip=0pt}

    \usepackage{float}
    \floatplacement{figure}{H} % forces figures to be placed at the correct location
    \usepackage{xcolor} % Allow colors to be defined
    \usepackage{enumerate} % Needed for markdown enumerations to work
    \usepackage{geometry} % Used to adjust the document margins
    \usepackage{amsmath} % Equations
    \usepackage{amssymb} % Equations
    \usepackage{textcomp} % defines textquotesingle
    % Hack from http://tex.stackexchange.com/a/47451/13684:
    \AtBeginDocument{%
        \def\PYZsq{\textquotesingle}% Upright quotes in Pygmentized code
    }
    \usepackage{upquote} % Upright quotes for verbatim code
    \usepackage{eurosym} % defines \euro
    \usepackage[mathletters]{ucs} % Extended unicode (utf-8) support
    \usepackage{fancyvrb} % verbatim replacement that allows latex
    \usepackage{grffile} % extends the file name processing of package graphics 
                         % to support a larger range
    \makeatletter % fix for old versions of grffile with XeLaTeX
    \@ifpackagelater{grffile}{2019/11/01}
    {
      % Do nothing on new versions
    }
    {
      \def\Gread@@xetex#1{%
        \IfFileExists{"\Gin@base".bb}%
        {\Gread@eps{\Gin@base.bb}}%
        {\Gread@@xetex@aux#1}%
      }
    }
    \makeatother
    \usepackage[Export]{adjustbox} % Used to constrain images to a maximum size
    \adjustboxset{max size={0.9\linewidth}{0.9\paperheight}}

    % The hyperref package gives us a pdf with properly built
    % internal navigation ('pdf bookmarks' for the table of contents,
    % internal cross-reference links, web links for URLs, etc.)
    \usepackage{hyperref}
    % The default LaTeX title has an obnoxious amount of whitespace. By default,
    % titling removes some of it. It also provides customization options.
    \usepackage{titling}
    \usepackage{longtable} % longtable support required by pandoc >1.10
    \usepackage{booktabs}  % table support for pandoc > 1.12.2
    \usepackage[inline]{enumitem} % IRkernel/repr support (it uses the enumerate* environment)
    \usepackage[normalem]{ulem} % ulem is needed to support strikethroughs (\sout)
                                % normalem makes italics be italics, not underlines
    \usepackage{mathrsfs}
    

    
    % Colors for the hyperref package
    \definecolor{urlcolor}{rgb}{0,.145,.698}
    \definecolor{linkcolor}{rgb}{.71,0.21,0.01}
    \definecolor{citecolor}{rgb}{.12,.54,.11}

    % ANSI colors
    \definecolor{ansi-black}{HTML}{3E424D}
    \definecolor{ansi-black-intense}{HTML}{282C36}
    \definecolor{ansi-red}{HTML}{E75C58}
    \definecolor{ansi-red-intense}{HTML}{B22B31}
    \definecolor{ansi-green}{HTML}{00A250}
    \definecolor{ansi-green-intense}{HTML}{007427}
    \definecolor{ansi-yellow}{HTML}{DDB62B}
    \definecolor{ansi-yellow-intense}{HTML}{B27D12}
    \definecolor{ansi-blue}{HTML}{208FFB}
    \definecolor{ansi-blue-intense}{HTML}{0065CA}
    \definecolor{ansi-magenta}{HTML}{D160C4}
    \definecolor{ansi-magenta-intense}{HTML}{A03196}
    \definecolor{ansi-cyan}{HTML}{60C6C8}
    \definecolor{ansi-cyan-intense}{HTML}{258F8F}
    \definecolor{ansi-white}{HTML}{C5C1B4}
    \definecolor{ansi-white-intense}{HTML}{A1A6B2}
    \definecolor{ansi-default-inverse-fg}{HTML}{FFFFFF}
    \definecolor{ansi-default-inverse-bg}{HTML}{000000}

    % common color for the border for error outputs.
    \definecolor{outerrorbackground}{HTML}{FFDFDF}

    % commands and environments needed by pandoc snippets
    % extracted from the output of `pandoc -s`
    \providecommand{\tightlist}{%
      \setlength{\itemsep}{0pt}\setlength{\parskip}{0pt}}
    \DefineVerbatimEnvironment{Highlighting}{Verbatim}{commandchars=\\\{\}}
    % Add ',fontsize=\small' for more characters per line
    \newenvironment{Shaded}{}{}
    \newcommand{\KeywordTok}[1]{\textcolor[rgb]{0.00,0.44,0.13}{\textbf{{#1}}}}
    \newcommand{\DataTypeTok}[1]{\textcolor[rgb]{0.56,0.13,0.00}{{#1}}}
    \newcommand{\DecValTok}[1]{\textcolor[rgb]{0.25,0.63,0.44}{{#1}}}
    \newcommand{\BaseNTok}[1]{\textcolor[rgb]{0.25,0.63,0.44}{{#1}}}
    \newcommand{\FloatTok}[1]{\textcolor[rgb]{0.25,0.63,0.44}{{#1}}}
    \newcommand{\CharTok}[1]{\textcolor[rgb]{0.25,0.44,0.63}{{#1}}}
    \newcommand{\StringTok}[1]{\textcolor[rgb]{0.25,0.44,0.63}{{#1}}}
    \newcommand{\CommentTok}[1]{\textcolor[rgb]{0.38,0.63,0.69}{\textit{{#1}}}}
    \newcommand{\OtherTok}[1]{\textcolor[rgb]{0.00,0.44,0.13}{{#1}}}
    \newcommand{\AlertTok}[1]{\textcolor[rgb]{1.00,0.00,0.00}{\textbf{{#1}}}}
    \newcommand{\FunctionTok}[1]{\textcolor[rgb]{0.02,0.16,0.49}{{#1}}}
    \newcommand{\RegionMarkerTok}[1]{{#1}}
    \newcommand{\ErrorTok}[1]{\textcolor[rgb]{1.00,0.00,0.00}{\textbf{{#1}}}}
    \newcommand{\NormalTok}[1]{{#1}}
    
    % Additional commands for more recent versions of Pandoc
    \newcommand{\ConstantTok}[1]{\textcolor[rgb]{0.53,0.00,0.00}{{#1}}}
    \newcommand{\SpecialCharTok}[1]{\textcolor[rgb]{0.25,0.44,0.63}{{#1}}}
    \newcommand{\VerbatimStringTok}[1]{\textcolor[rgb]{0.25,0.44,0.63}{{#1}}}
    \newcommand{\SpecialStringTok}[1]{\textcolor[rgb]{0.73,0.40,0.53}{{#1}}}
    \newcommand{\ImportTok}[1]{{#1}}
    \newcommand{\DocumentationTok}[1]{\textcolor[rgb]{0.73,0.13,0.13}{\textit{{#1}}}}
    \newcommand{\AnnotationTok}[1]{\textcolor[rgb]{0.38,0.63,0.69}{\textbf{\textit{{#1}}}}}
    \newcommand{\CommentVarTok}[1]{\textcolor[rgb]{0.38,0.63,0.69}{\textbf{\textit{{#1}}}}}
    \newcommand{\VariableTok}[1]{\textcolor[rgb]{0.10,0.09,0.49}{{#1}}}
    \newcommand{\ControlFlowTok}[1]{\textcolor[rgb]{0.00,0.44,0.13}{\textbf{{#1}}}}
    \newcommand{\OperatorTok}[1]{\textcolor[rgb]{0.40,0.40,0.40}{{#1}}}
    \newcommand{\BuiltInTok}[1]{{#1}}
    \newcommand{\ExtensionTok}[1]{{#1}}
    \newcommand{\PreprocessorTok}[1]{\textcolor[rgb]{0.74,0.48,0.00}{{#1}}}
    \newcommand{\AttributeTok}[1]{\textcolor[rgb]{0.49,0.56,0.16}{{#1}}}
    \newcommand{\InformationTok}[1]{\textcolor[rgb]{0.38,0.63,0.69}{\textbf{\textit{{#1}}}}}
    \newcommand{\WarningTok}[1]{\textcolor[rgb]{0.38,0.63,0.69}{\textbf{\textit{{#1}}}}}
    
    
    % Define a nice break command that doesn't care if a line doesn't already
    % exist.
    \def\br{\hspace*{\fill} \\* }
    % Math Jax compatibility definitions
    \def\gt{>}
    \def\lt{<}
    \let\Oldtex\TeX
    \let\Oldlatex\LaTeX
    \renewcommand{\TeX}{\textrm{\Oldtex}}
    \renewcommand{\LaTeX}{\textrm{\Oldlatex}}
    % Document parameters
    % Document title
    \title{taruma\_0\_4\_0\_hk98\_rekap\_deret\_waktu}
    
    
    
    
    
% Pygments definitions
\makeatletter
\def\PY@reset{\let\PY@it=\relax \let\PY@bf=\relax%
    \let\PY@ul=\relax \let\PY@tc=\relax%
    \let\PY@bc=\relax \let\PY@ff=\relax}
\def\PY@tok#1{\csname PY@tok@#1\endcsname}
\def\PY@toks#1+{\ifx\relax#1\empty\else%
    \PY@tok{#1}\expandafter\PY@toks\fi}
\def\PY@do#1{\PY@bc{\PY@tc{\PY@ul{%
    \PY@it{\PY@bf{\PY@ff{#1}}}}}}}
\def\PY#1#2{\PY@reset\PY@toks#1+\relax+\PY@do{#2}}

\@namedef{PY@tok@w}{\def\PY@tc##1{\textcolor[rgb]{0.73,0.73,0.73}{##1}}}
\@namedef{PY@tok@c}{\let\PY@it=\textit\def\PY@tc##1{\textcolor[rgb]{0.24,0.48,0.48}{##1}}}
\@namedef{PY@tok@cp}{\def\PY@tc##1{\textcolor[rgb]{0.61,0.40,0.00}{##1}}}
\@namedef{PY@tok@k}{\let\PY@bf=\textbf\def\PY@tc##1{\textcolor[rgb]{0.00,0.50,0.00}{##1}}}
\@namedef{PY@tok@kp}{\def\PY@tc##1{\textcolor[rgb]{0.00,0.50,0.00}{##1}}}
\@namedef{PY@tok@kt}{\def\PY@tc##1{\textcolor[rgb]{0.69,0.00,0.25}{##1}}}
\@namedef{PY@tok@o}{\def\PY@tc##1{\textcolor[rgb]{0.40,0.40,0.40}{##1}}}
\@namedef{PY@tok@ow}{\let\PY@bf=\textbf\def\PY@tc##1{\textcolor[rgb]{0.67,0.13,1.00}{##1}}}
\@namedef{PY@tok@nb}{\def\PY@tc##1{\textcolor[rgb]{0.00,0.50,0.00}{##1}}}
\@namedef{PY@tok@nf}{\def\PY@tc##1{\textcolor[rgb]{0.00,0.00,1.00}{##1}}}
\@namedef{PY@tok@nc}{\let\PY@bf=\textbf\def\PY@tc##1{\textcolor[rgb]{0.00,0.00,1.00}{##1}}}
\@namedef{PY@tok@nn}{\let\PY@bf=\textbf\def\PY@tc##1{\textcolor[rgb]{0.00,0.00,1.00}{##1}}}
\@namedef{PY@tok@ne}{\let\PY@bf=\textbf\def\PY@tc##1{\textcolor[rgb]{0.80,0.25,0.22}{##1}}}
\@namedef{PY@tok@nv}{\def\PY@tc##1{\textcolor[rgb]{0.10,0.09,0.49}{##1}}}
\@namedef{PY@tok@no}{\def\PY@tc##1{\textcolor[rgb]{0.53,0.00,0.00}{##1}}}
\@namedef{PY@tok@nl}{\def\PY@tc##1{\textcolor[rgb]{0.46,0.46,0.00}{##1}}}
\@namedef{PY@tok@ni}{\let\PY@bf=\textbf\def\PY@tc##1{\textcolor[rgb]{0.44,0.44,0.44}{##1}}}
\@namedef{PY@tok@na}{\def\PY@tc##1{\textcolor[rgb]{0.41,0.47,0.13}{##1}}}
\@namedef{PY@tok@nt}{\let\PY@bf=\textbf\def\PY@tc##1{\textcolor[rgb]{0.00,0.50,0.00}{##1}}}
\@namedef{PY@tok@nd}{\def\PY@tc##1{\textcolor[rgb]{0.67,0.13,1.00}{##1}}}
\@namedef{PY@tok@s}{\def\PY@tc##1{\textcolor[rgb]{0.73,0.13,0.13}{##1}}}
\@namedef{PY@tok@sd}{\let\PY@it=\textit\def\PY@tc##1{\textcolor[rgb]{0.73,0.13,0.13}{##1}}}
\@namedef{PY@tok@si}{\let\PY@bf=\textbf\def\PY@tc##1{\textcolor[rgb]{0.64,0.35,0.47}{##1}}}
\@namedef{PY@tok@se}{\let\PY@bf=\textbf\def\PY@tc##1{\textcolor[rgb]{0.67,0.36,0.12}{##1}}}
\@namedef{PY@tok@sr}{\def\PY@tc##1{\textcolor[rgb]{0.64,0.35,0.47}{##1}}}
\@namedef{PY@tok@ss}{\def\PY@tc##1{\textcolor[rgb]{0.10,0.09,0.49}{##1}}}
\@namedef{PY@tok@sx}{\def\PY@tc##1{\textcolor[rgb]{0.00,0.50,0.00}{##1}}}
\@namedef{PY@tok@m}{\def\PY@tc##1{\textcolor[rgb]{0.40,0.40,0.40}{##1}}}
\@namedef{PY@tok@gh}{\let\PY@bf=\textbf\def\PY@tc##1{\textcolor[rgb]{0.00,0.00,0.50}{##1}}}
\@namedef{PY@tok@gu}{\let\PY@bf=\textbf\def\PY@tc##1{\textcolor[rgb]{0.50,0.00,0.50}{##1}}}
\@namedef{PY@tok@gd}{\def\PY@tc##1{\textcolor[rgb]{0.63,0.00,0.00}{##1}}}
\@namedef{PY@tok@gi}{\def\PY@tc##1{\textcolor[rgb]{0.00,0.52,0.00}{##1}}}
\@namedef{PY@tok@gr}{\def\PY@tc##1{\textcolor[rgb]{0.89,0.00,0.00}{##1}}}
\@namedef{PY@tok@ge}{\let\PY@it=\textit}
\@namedef{PY@tok@gs}{\let\PY@bf=\textbf}
\@namedef{PY@tok@gp}{\let\PY@bf=\textbf\def\PY@tc##1{\textcolor[rgb]{0.00,0.00,0.50}{##1}}}
\@namedef{PY@tok@go}{\def\PY@tc##1{\textcolor[rgb]{0.44,0.44,0.44}{##1}}}
\@namedef{PY@tok@gt}{\def\PY@tc##1{\textcolor[rgb]{0.00,0.27,0.87}{##1}}}
\@namedef{PY@tok@err}{\def\PY@bc##1{{\setlength{\fboxsep}{\string -\fboxrule}\fcolorbox[rgb]{1.00,0.00,0.00}{1,1,1}{\strut ##1}}}}
\@namedef{PY@tok@kc}{\let\PY@bf=\textbf\def\PY@tc##1{\textcolor[rgb]{0.00,0.50,0.00}{##1}}}
\@namedef{PY@tok@kd}{\let\PY@bf=\textbf\def\PY@tc##1{\textcolor[rgb]{0.00,0.50,0.00}{##1}}}
\@namedef{PY@tok@kn}{\let\PY@bf=\textbf\def\PY@tc##1{\textcolor[rgb]{0.00,0.50,0.00}{##1}}}
\@namedef{PY@tok@kr}{\let\PY@bf=\textbf\def\PY@tc##1{\textcolor[rgb]{0.00,0.50,0.00}{##1}}}
\@namedef{PY@tok@bp}{\def\PY@tc##1{\textcolor[rgb]{0.00,0.50,0.00}{##1}}}
\@namedef{PY@tok@fm}{\def\PY@tc##1{\textcolor[rgb]{0.00,0.00,1.00}{##1}}}
\@namedef{PY@tok@vc}{\def\PY@tc##1{\textcolor[rgb]{0.10,0.09,0.49}{##1}}}
\@namedef{PY@tok@vg}{\def\PY@tc##1{\textcolor[rgb]{0.10,0.09,0.49}{##1}}}
\@namedef{PY@tok@vi}{\def\PY@tc##1{\textcolor[rgb]{0.10,0.09,0.49}{##1}}}
\@namedef{PY@tok@vm}{\def\PY@tc##1{\textcolor[rgb]{0.10,0.09,0.49}{##1}}}
\@namedef{PY@tok@sa}{\def\PY@tc##1{\textcolor[rgb]{0.73,0.13,0.13}{##1}}}
\@namedef{PY@tok@sb}{\def\PY@tc##1{\textcolor[rgb]{0.73,0.13,0.13}{##1}}}
\@namedef{PY@tok@sc}{\def\PY@tc##1{\textcolor[rgb]{0.73,0.13,0.13}{##1}}}
\@namedef{PY@tok@dl}{\def\PY@tc##1{\textcolor[rgb]{0.73,0.13,0.13}{##1}}}
\@namedef{PY@tok@s2}{\def\PY@tc##1{\textcolor[rgb]{0.73,0.13,0.13}{##1}}}
\@namedef{PY@tok@sh}{\def\PY@tc##1{\textcolor[rgb]{0.73,0.13,0.13}{##1}}}
\@namedef{PY@tok@s1}{\def\PY@tc##1{\textcolor[rgb]{0.73,0.13,0.13}{##1}}}
\@namedef{PY@tok@mb}{\def\PY@tc##1{\textcolor[rgb]{0.40,0.40,0.40}{##1}}}
\@namedef{PY@tok@mf}{\def\PY@tc##1{\textcolor[rgb]{0.40,0.40,0.40}{##1}}}
\@namedef{PY@tok@mh}{\def\PY@tc##1{\textcolor[rgb]{0.40,0.40,0.40}{##1}}}
\@namedef{PY@tok@mi}{\def\PY@tc##1{\textcolor[rgb]{0.40,0.40,0.40}{##1}}}
\@namedef{PY@tok@il}{\def\PY@tc##1{\textcolor[rgb]{0.40,0.40,0.40}{##1}}}
\@namedef{PY@tok@mo}{\def\PY@tc##1{\textcolor[rgb]{0.40,0.40,0.40}{##1}}}
\@namedef{PY@tok@ch}{\let\PY@it=\textit\def\PY@tc##1{\textcolor[rgb]{0.24,0.48,0.48}{##1}}}
\@namedef{PY@tok@cm}{\let\PY@it=\textit\def\PY@tc##1{\textcolor[rgb]{0.24,0.48,0.48}{##1}}}
\@namedef{PY@tok@cpf}{\let\PY@it=\textit\def\PY@tc##1{\textcolor[rgb]{0.24,0.48,0.48}{##1}}}
\@namedef{PY@tok@c1}{\let\PY@it=\textit\def\PY@tc##1{\textcolor[rgb]{0.24,0.48,0.48}{##1}}}
\@namedef{PY@tok@cs}{\let\PY@it=\textit\def\PY@tc##1{\textcolor[rgb]{0.24,0.48,0.48}{##1}}}

\def\PYZbs{\char`\\}
\def\PYZus{\char`\_}
\def\PYZob{\char`\{}
\def\PYZcb{\char`\}}
\def\PYZca{\char`\^}
\def\PYZam{\char`\&}
\def\PYZlt{\char`\<}
\def\PYZgt{\char`\>}
\def\PYZsh{\char`\#}
\def\PYZpc{\char`\%}
\def\PYZdl{\char`\$}
\def\PYZhy{\char`\-}
\def\PYZsq{\char`\'}
\def\PYZdq{\char`\"}
\def\PYZti{\char`\~}
% for compatibility with earlier versions
\def\PYZat{@}
\def\PYZlb{[}
\def\PYZrb{]}
\makeatother


    % For linebreaks inside Verbatim environment from package fancyvrb. 
    \makeatletter
        \newbox\Wrappedcontinuationbox 
        \newbox\Wrappedvisiblespacebox 
        \newcommand*\Wrappedvisiblespace {\textcolor{red}{\textvisiblespace}} 
        \newcommand*\Wrappedcontinuationsymbol {\textcolor{red}{\llap{\tiny$\m@th\hookrightarrow$}}} 
        \newcommand*\Wrappedcontinuationindent {3ex } 
        \newcommand*\Wrappedafterbreak {\kern\Wrappedcontinuationindent\copy\Wrappedcontinuationbox} 
        % Take advantage of the already applied Pygments mark-up to insert 
        % potential linebreaks for TeX processing. 
        %        {, <, #, %, $, ' and ": go to next line. 
        %        _, }, ^, &, >, - and ~: stay at end of broken line. 
        % Use of \textquotesingle for straight quote. 
        \newcommand*\Wrappedbreaksatspecials {% 
            \def\PYGZus{\discretionary{\char`\_}{\Wrappedafterbreak}{\char`\_}}% 
            \def\PYGZob{\discretionary{}{\Wrappedafterbreak\char`\{}{\char`\{}}% 
            \def\PYGZcb{\discretionary{\char`\}}{\Wrappedafterbreak}{\char`\}}}% 
            \def\PYGZca{\discretionary{\char`\^}{\Wrappedafterbreak}{\char`\^}}% 
            \def\PYGZam{\discretionary{\char`\&}{\Wrappedafterbreak}{\char`\&}}% 
            \def\PYGZlt{\discretionary{}{\Wrappedafterbreak\char`\<}{\char`\<}}% 
            \def\PYGZgt{\discretionary{\char`\>}{\Wrappedafterbreak}{\char`\>}}% 
            \def\PYGZsh{\discretionary{}{\Wrappedafterbreak\char`\#}{\char`\#}}% 
            \def\PYGZpc{\discretionary{}{\Wrappedafterbreak\char`\%}{\char`\%}}% 
            \def\PYGZdl{\discretionary{}{\Wrappedafterbreak\char`\$}{\char`\$}}% 
            \def\PYGZhy{\discretionary{\char`\-}{\Wrappedafterbreak}{\char`\-}}% 
            \def\PYGZsq{\discretionary{}{\Wrappedafterbreak\textquotesingle}{\textquotesingle}}% 
            \def\PYGZdq{\discretionary{}{\Wrappedafterbreak\char`\"}{\char`\"}}% 
            \def\PYGZti{\discretionary{\char`\~}{\Wrappedafterbreak}{\char`\~}}% 
        } 
        % Some characters . , ; ? ! / are not pygmentized. 
        % This macro makes them "active" and they will insert potential linebreaks 
        \newcommand*\Wrappedbreaksatpunct {% 
            \lccode`\~`\.\lowercase{\def~}{\discretionary{\hbox{\char`\.}}{\Wrappedafterbreak}{\hbox{\char`\.}}}% 
            \lccode`\~`\,\lowercase{\def~}{\discretionary{\hbox{\char`\,}}{\Wrappedafterbreak}{\hbox{\char`\,}}}% 
            \lccode`\~`\;\lowercase{\def~}{\discretionary{\hbox{\char`\;}}{\Wrappedafterbreak}{\hbox{\char`\;}}}% 
            \lccode`\~`\:\lowercase{\def~}{\discretionary{\hbox{\char`\:}}{\Wrappedafterbreak}{\hbox{\char`\:}}}% 
            \lccode`\~`\?\lowercase{\def~}{\discretionary{\hbox{\char`\?}}{\Wrappedafterbreak}{\hbox{\char`\?}}}% 
            \lccode`\~`\!\lowercase{\def~}{\discretionary{\hbox{\char`\!}}{\Wrappedafterbreak}{\hbox{\char`\!}}}% 
            \lccode`\~`\/\lowercase{\def~}{\discretionary{\hbox{\char`\/}}{\Wrappedafterbreak}{\hbox{\char`\/}}}% 
            \catcode`\.\active
            \catcode`\,\active 
            \catcode`\;\active
            \catcode`\:\active
            \catcode`\?\active
            \catcode`\!\active
            \catcode`\/\active 
            \lccode`\~`\~ 	
        }
    \makeatother

    \let\OriginalVerbatim=\Verbatim
    \makeatletter
    \renewcommand{\Verbatim}[1][1]{%
        %\parskip\z@skip
        \sbox\Wrappedcontinuationbox {\Wrappedcontinuationsymbol}%
        \sbox\Wrappedvisiblespacebox {\FV@SetupFont\Wrappedvisiblespace}%
        \def\FancyVerbFormatLine ##1{\hsize\linewidth
            \vtop{\raggedright\hyphenpenalty\z@\exhyphenpenalty\z@
                \doublehyphendemerits\z@\finalhyphendemerits\z@
                \strut ##1\strut}%
        }%
        % If the linebreak is at a space, the latter will be displayed as visible
        % space at end of first line, and a continuation symbol starts next line.
        % Stretch/shrink are however usually zero for typewriter font.
        \def\FV@Space {%
            \nobreak\hskip\z@ plus\fontdimen3\font minus\fontdimen4\font
            \discretionary{\copy\Wrappedvisiblespacebox}{\Wrappedafterbreak}
            {\kern\fontdimen2\font}%
        }%
        
        % Allow breaks at special characters using \PYG... macros.
        \Wrappedbreaksatspecials
        % Breaks at punctuation characters . , ; ? ! and / need catcode=\active 	
        \OriginalVerbatim[#1,codes*=\Wrappedbreaksatpunct]%
    }
    \makeatother

    % Exact colors from NB
    \definecolor{incolor}{HTML}{303F9F}
    \definecolor{outcolor}{HTML}{D84315}
    \definecolor{cellborder}{HTML}{CFCFCF}
    \definecolor{cellbackground}{HTML}{F7F7F7}
    
    % prompt
    \makeatletter
    \newcommand{\boxspacing}{\kern\kvtcb@left@rule\kern\kvtcb@boxsep}
    \makeatother
    \newcommand{\prompt}[4]{
        {\ttfamily\llap{{\color{#2}[#3]:\hspace{3pt}#4}}\vspace{-\baselineskip}}
    }
    

    
    % Prevent overflowing lines due to hard-to-break entities
    \sloppy 
    % Setup hyperref package
    \hypersetup{
      breaklinks=true,  % so long urls are correctly broken across lines
      colorlinks=true,
      urlcolor=urlcolor,
      linkcolor=linkcolor,
      citecolor=citecolor,
      }
    % Slightly bigger margins than the latex defaults
    
    \geometry{verbose,tmargin=1in,bmargin=1in,lmargin=1in,rmargin=1in}
    
    

\begin{document}
    
    \maketitle
    
    

    
    Berdasarkan isu
\href{https://github.com/taruma/hidrokit/issues/98}{\#98}: \textbf{buat
ringkasan/rekap data deret waktu}

Deskripsi permasalahan: - Membuat ringkasan/rekapitulasi/laporan dari
data deret waktu (\emph{time series}).

Strategi Penyelesaian: - Membuat fungsi yang memudahkan kostumisasi saat
menggunakan fungsi buatan sendiri.

Catatan: - Fungsi ini hanya diuji pada data harian dengan kepentingan
merekapitulasi setiap bulannya.

    \hypertarget{persiapan-dan-dataset}{%
\section{PERSIAPAN DAN DATASET}\label{persiapan-dan-dataset}}

    \begin{tcolorbox}[breakable, size=fbox, boxrule=1pt, pad at break*=1mm,colback=cellbackground, colframe=cellborder]
\prompt{In}{incolor}{ }{\boxspacing}
\begin{Verbatim}[commandchars=\\\{\}]
\PY{k+kn}{import} \PY{n+nn}{numpy} \PY{k}{as} \PY{n+nn}{np}
\PY{k+kn}{import} \PY{n+nn}{pandas} \PY{k}{as} \PY{n+nn}{pd}
\end{Verbatim}
\end{tcolorbox}

    \begin{tcolorbox}[breakable, size=fbox, boxrule=1pt, pad at break*=1mm,colback=cellbackground, colframe=cellborder]
\prompt{In}{incolor}{ }{\boxspacing}
\begin{Verbatim}[commandchars=\\\{\}]
\PY{k}{try}\PY{p}{:}
    \PY{k+kn}{import} \PY{n+nn}{hidrokit}
\PY{k}{except} \PY{n+ne}{ModuleNotFoundError}\PY{p}{:}
    \PY{o}{!}pip install git+https://github.com/taruma/hidrokit.git@latest \PYZhy{}q
    \PY{k+kn}{import} \PY{n+nn}{hidrokit}

\PY{n+nb}{print}\PY{p}{(}\PY{l+s+sa}{f}\PY{l+s+s1}{\PYZsq{}}\PY{l+s+s1}{hidrokit version: }\PY{l+s+si}{\PYZob{}}\PY{n}{hidrokit}\PY{o}{.}\PY{n}{\PYZus{}\PYZus{}version\PYZus{}\PYZus{}}\PY{l+s+si}{\PYZcb{}}\PY{l+s+s1}{\PYZsq{}}\PY{p}{)}
\end{Verbatim}
\end{tcolorbox}

    \begin{Verbatim}[commandchars=\\\{\}]
  Building wheel for hidrokit (setup.py) {\ldots} done
hidrokit version: 0.3.6
    \end{Verbatim}

    \begin{tcolorbox}[breakable, size=fbox, boxrule=1pt, pad at break*=1mm,colback=cellbackground, colframe=cellborder]
\prompt{In}{incolor}{ }{\boxspacing}
\begin{Verbatim}[commandchars=\\\{\}]
\PY{o}{!}wget \PYZhy{}O sample.xlsx \PY{l+s+s2}{\PYZdq{}https://taruma.github.io/assets/hidrokit\PYZus{}dataset/data\PYZus{}daily\PYZus{}sample.xlsx\PYZdq{}} \PYZhy{}q
\PY{n}{dataset\PYZus{}path} \PY{o}{=} \PY{l+s+s1}{\PYZsq{}}\PY{l+s+s1}{sample.xlsx}\PY{l+s+s1}{\PYZsq{}}
\end{Verbatim}
\end{tcolorbox}

    \begin{tcolorbox}[breakable, size=fbox, boxrule=1pt, pad at break*=1mm,colback=cellbackground, colframe=cellborder]
\prompt{In}{incolor}{ }{\boxspacing}
\begin{Verbatim}[commandchars=\\\{\}]
\PY{k+kn}{from} \PY{n+nn}{hidrokit}\PY{n+nn}{.}\PY{n+nn}{contrib}\PY{n+nn}{.}\PY{n+nn}{taruma} \PY{k+kn}{import} \PY{n}{hk88}

\PY{n}{\PYZus{}data} \PY{o}{=} \PY{n}{hk88}\PY{o}{.}\PY{n}{read\PYZus{}workbook}\PY{p}{(}\PY{n}{dataset\PYZus{}path}\PY{p}{,} \PY{p}{[}\PY{l+s+s1}{\PYZsq{}}\PY{l+s+s1}{STA\PYZus{}A}\PY{l+s+s1}{\PYZsq{}}\PY{p}{,} \PY{l+s+s1}{\PYZsq{}}\PY{l+s+s1}{STA\PYZus{}B}\PY{l+s+s1}{\PYZsq{}}\PY{p}{,} \PY{l+s+s1}{\PYZsq{}}\PY{l+s+s1}{STA\PYZus{}C}\PY{l+s+s1}{\PYZsq{}}\PY{p}{]}\PY{p}{,} 
                           \PY{n}{as\PYZus{}df}\PY{o}{=}\PY{k+kc}{False}\PY{p}{)}
\PY{n}{dataset} \PY{o}{=} \PY{n}{pd}\PY{o}{.}\PY{n}{concat}\PY{p}{(}\PY{n}{\PYZus{}data}\PY{p}{,} \PY{n}{sort}\PY{o}{=}\PY{k+kc}{True}\PY{p}{,} \PY{n}{axis}\PY{o}{=}\PY{l+m+mi}{1}\PY{p}{)}\PY{o}{.}\PY{n}{infer\PYZus{}objects}\PY{p}{(}\PY{p}{)}
\PY{n}{dataset}\PY{o}{.}\PY{n}{info}\PY{p}{(}\PY{p}{)}
\PY{n}{dataset}\PY{o}{.}\PY{n}{head}\PY{p}{(}\PY{p}{)}
\end{Verbatim}
\end{tcolorbox}

    \begin{Verbatim}[commandchars=\\\{\}]
<class 'pandas.core.frame.DataFrame'>
DatetimeIndex: 5478 entries, 2001-01-01 to 2015-12-31
Freq: D
Data columns (total 3 columns):
 \#   Column  Non-Null Count  Dtype
---  ------  --------------  -----
 0   STA\_A   5477 non-null   float64
 1   STA\_B   5470 non-null   float64
 2   STA\_C   5475 non-null   float64
dtypes: float64(3)
memory usage: 171.2 KB
    \end{Verbatim}

            \begin{tcolorbox}[breakable, size=fbox, boxrule=.5pt, pad at break*=1mm, opacityfill=0]
\prompt{Out}{outcolor}{ }{\boxspacing}
\begin{Verbatim}[commandchars=\\\{\}]
            STA\_A  STA\_B  STA\_C
2001-01-01    0.0    0.0   0.00
2001-01-02    0.0    0.0   0.65
2001-01-03    0.0   45.0   9.16
2001-01-04    0.0    0.0   0.00
2001-01-05    0.0    5.0   1.03
\end{Verbatim}
\end{tcolorbox}
        
    \hypertarget{kode}{%
\section{KODE}\label{kode}}

    \begin{tcolorbox}[breakable, size=fbox, boxrule=1pt, pad at break*=1mm,colback=cellbackground, colframe=cellborder]
\prompt{In}{incolor}{ }{\boxspacing}
\begin{Verbatim}[commandchars=\\\{\}]
\PY{k}{def} \PY{n+nf}{summary\PYZus{}station}\PY{p}{(}\PY{n}{dataset}\PY{p}{,} \PY{n}{column}\PY{p}{,} \PY{n}{ufunc}\PY{p}{,} \PY{n}{ufunc\PYZus{}col}\PY{p}{,} \PY{n}{n\PYZus{}days}\PY{o}{=}\PY{l+s+s1}{\PYZsq{}}\PY{l+s+s1}{M}\PY{l+s+s1}{\PYZsq{}}\PY{p}{)}\PY{p}{:}
    \PY{n}{grouped} \PY{o}{=} \PY{p}{[}\PY{n}{dataset}\PY{o}{.}\PY{n}{index}\PY{o}{.}\PY{n}{year}\PY{p}{,} \PY{n}{dataset}\PY{o}{.}\PY{n}{index}\PY{o}{.}\PY{n}{month}\PY{p}{]}

    \PY{n}{ufunc} \PY{o}{=} \PY{n}{ufunc} \PY{k}{if} \PY{n+nb}{isinstance}\PY{p}{(}\PY{n}{ufunc}\PY{p}{,} \PY{p}{(}\PY{n+nb}{list}\PY{p}{,} \PY{n+nb}{tuple}\PY{p}{)}\PY{p}{)} \PY{k}{else} \PY{p}{(}\PY{n}{ufunc}\PY{p}{,}\PY{p}{)}
    \PY{n}{ufunc\PYZus{}col} \PY{o}{=} \PY{p}{(}\PY{n}{ufunc\PYZus{}col} 
                 \PY{k}{if} \PY{n+nb}{isinstance}\PY{p}{(}\PY{n}{ufunc\PYZus{}col}\PY{p}{,} \PY{p}{(}\PY{n+nb}{list}\PY{p}{,} \PY{n+nb}{tuple}\PY{p}{)}\PY{p}{)} \PY{k}{else} \PY{p}{(}\PY{n}{ufunc\PYZus{}col}\PY{p}{,}\PY{p}{)}\PY{p}{)}

    \PY{k}{if} \PY{n+nb}{len}\PY{p}{(}\PY{n}{ufunc}\PY{p}{)} \PY{o}{!=} \PY{n+nb}{len}\PY{p}{(}\PY{n}{ufunc\PYZus{}col}\PY{p}{)}\PY{p}{:}
        \PY{k}{raise} \PY{n+ne}{ValueError}\PY{p}{(}\PY{l+s+s1}{\PYZsq{}}\PY{l+s+s1}{length ufunc and ufunc\PYZus{}col are not matched.}\PY{l+s+s1}{\PYZsq{}}\PY{p}{)}

    \PY{n}{ix\PYZus{}month} \PY{o}{=} \PY{p}{[}\PY{p}{]}
    \PY{n}{val\PYZus{}month} \PY{o}{=} \PY{p}{[}\PY{p}{]}
    \PY{k}{for} \PY{n}{i}\PY{p}{,} \PY{n}{x} \PY{o+ow}{in} \PY{n}{dataset}\PY{p}{[}\PY{n}{column}\PY{p}{]}\PY{o}{.}\PY{n}{groupby}\PY{p}{(}\PY{n}{by}\PY{o}{=}\PY{n}{grouped}\PY{p}{)}\PY{p}{:}
        \PY{n}{each\PYZus{}month} \PY{o}{=} \PY{n}{x}\PY{o}{.}\PY{n}{groupby}\PY{p}{(}\PY{n}{pd}\PY{o}{.}\PY{n}{Grouper}\PY{p}{(}\PY{n}{freq}\PY{o}{=}\PY{n}{n\PYZus{}days}\PY{p}{)}\PY{p}{)}\PY{o}{.}\PY{n}{agg}\PY{p}{(}\PY{n}{ufunc}\PY{p}{)}
        \PY{n}{val\PYZus{}month}\PY{o}{.}\PY{n}{append}\PY{p}{(}\PY{n}{each\PYZus{}month}\PY{o}{.}\PY{n}{values}\PY{p}{)}
        \PY{n}{ix\PYZus{}month} \PY{o}{+}\PY{o}{=} \PY{n}{each\PYZus{}month}\PY{o}{.}\PY{n}{index}
    \PY{k}{return} \PY{n}{pd}\PY{o}{.}\PY{n}{DataFrame}\PY{p}{(}
        \PY{n}{data}\PY{o}{=}\PY{n}{np}\PY{o}{.}\PY{n}{vstack}\PY{p}{(}\PY{n}{val\PYZus{}month}\PY{p}{)}\PY{p}{,} \PY{n}{index}\PY{o}{=}\PY{n}{ix\PYZus{}month}\PY{p}{,} 
        \PY{n}{columns}\PY{o}{=}\PY{n}{pd}\PY{o}{.}\PY{n}{MultiIndex}\PY{o}{.}\PY{n}{from\PYZus{}product}\PY{p}{(}\PY{p}{[}\PY{p}{[}\PY{n}{column}\PY{p}{]}\PY{p}{,} \PY{n}{ufunc\PYZus{}col}\PY{p}{]}\PY{p}{)}
    \PY{p}{)}

\PY{k}{def} \PY{n+nf}{summary\PYZus{}all}\PY{p}{(}\PY{n}{dataset}\PY{p}{,} \PY{n}{ufunc}\PY{p}{,} \PY{n}{ufunc\PYZus{}col}\PY{p}{,} \PY{n}{columns}\PY{o}{=}\PY{k+kc}{None}\PY{p}{,} \PY{n}{n\PYZus{}days}\PY{o}{=}\PY{l+s+s1}{\PYZsq{}}\PY{l+s+s1}{MS}\PY{l+s+s1}{\PYZsq{}}\PY{p}{,} \PY{n}{verbose}\PY{o}{=}\PY{k+kc}{False}\PY{p}{)}\PY{p}{:}
    \PY{n}{res} \PY{o}{=} \PY{p}{[}\PY{p}{]}

    \PY{n}{columns} \PY{o}{=} \PY{n}{columns} \PY{k}{if} \PY{n}{columns} \PY{o+ow}{is} \PY{o+ow}{not} \PY{k+kc}{None} \PY{k}{else} \PY{n+nb}{list}\PY{p}{(}\PY{n}{dataset}\PY{o}{.}\PY{n}{columns}\PY{p}{)}
    \PY{n}{columns} \PY{o}{=} \PY{n}{columns} \PY{k}{if} \PY{n+nb}{isinstance}\PY{p}{(}\PY{n}{columns}\PY{p}{,} \PY{p}{(}\PY{n+nb}{list}\PY{p}{,} \PY{n+nb}{tuple}\PY{p}{)}\PY{p}{)} \PY{k}{else} \PY{p}{[}\PY{n}{columns}\PY{p}{]}

    \PY{k}{for} \PY{n}{column} \PY{o+ow}{in} \PY{n}{columns}\PY{p}{:}
        \PY{k}{if} \PY{n}{verbose}\PY{p}{:}
            \PY{n+nb}{print}\PY{p}{(}\PY{l+s+s1}{\PYZsq{}}\PY{l+s+s1}{PROCESSING:}\PY{l+s+s1}{\PYZsq{}}\PY{p}{,} \PY{n}{column}\PY{p}{)}
        \PY{n}{res}\PY{o}{.}\PY{n}{append}\PY{p}{(}
            \PY{n}{summary\PYZus{}station}\PY{p}{(}\PY{n}{dataset}\PY{p}{,} \PY{n}{column}\PY{p}{,} \PY{n}{ufunc}\PY{p}{,} \PY{n}{ufunc\PYZus{}col}\PY{p}{,} \PY{n}{n\PYZus{}days}\PY{o}{=}\PY{n}{n\PYZus{}days}\PY{p}{)}
        \PY{p}{)}
    \PY{k}{return} \PY{n}{pd}\PY{o}{.}\PY{n}{concat}\PY{p}{(}\PY{n}{res}\PY{p}{,} \PY{n}{axis}\PY{o}{=}\PY{l+m+mi}{1}\PY{p}{)}
\end{Verbatim}
\end{tcolorbox}

    \hypertarget{fungsi}{%
\section{FUNGSI}\label{fungsi}}

    \hypertarget{fungsi-summary_station}{%
\subsection{\texorpdfstring{Fungsi
\texttt{summary\_station()}}{Fungsi summary\_station()}}\label{fungsi-summary_station}}

Fungsi ini membuat rekap untuk stasiun/kolom tunggal dalam bentuk
keluaran \texttt{pandas.DataFrame}. Argumen yang dibutuhkan antara lain:

\begin{itemize}
\tightlist
\item
  \texttt{dataset}: DataFrame dataset. Isian berupa
  \texttt{pandas.DataFrame}.
\item
  \texttt{column}: kolom tunggal yang akan diproses. Isian berupa
  \emph{string}.
\item
  \texttt{ufunc}: fungsi atau \emph{list} fungsi yang akan digunakan.
  Isian berupa \texttt{object} atau \emph{list of \texttt{object}}.
\item
  \texttt{ufunc\_col}: nama atau \emph{list} nama dari fungsi
  \texttt{ufunc}. Isian berupa \emph{list of string}.
\item
  \texttt{n\_days=\textquotesingle{}M\textquotesingle{}}: indikator
  jumlah hari/bulan yang diproses. Isian merupakan isian valid untuk
  parameter \texttt{freq} pada objek \texttt{pd.Grouper}
  (\href{https://pandas.pydata.org/pandas-docs/stable/user_guide/timeseries.html\#offset-aliases}{referensi}).
  Isian \texttt{\textquotesingle{}M\textquotesingle{}} berarti setiap
  bulan (\emph{M}onth), isian
  \texttt{\textquotesingle{}9D\textquotesingle{}} berarti setiap 9 hari
  (\emph{D}ays).
\end{itemize}

    \hypertarget{argumen-ufunc-dan-ufunc_col}{%
\subsubsection{\texorpdfstring{Argumen \texttt{ufunc} dan
\texttt{ufunc\_col}}{Argumen ufunc dan ufunc\_col}}\label{argumen-ufunc-dan-ufunc_col}}

Pengguna diberi kebebasan dalam melakukan perhitungan pada fungsi
\texttt{summary\_station}. Dalam \emph{notebook} ini akan diberikan
contoh menggunakan fungsi yang tersedia pada python, numpy, dan
membuatnya sendiri.

    \begin{tcolorbox}[breakable, size=fbox, boxrule=1pt, pad at break*=1mm,colback=cellbackground, colframe=cellborder]
\prompt{In}{incolor}{ }{\boxspacing}
\begin{Verbatim}[commandchars=\\\{\}]
\PY{c+c1}{\PYZsh{} Fungsi buatan sendiri}
\PY{k}{def} \PY{n+nf}{n\PYZus{}rain}\PY{p}{(}\PY{n}{x}\PY{p}{)}\PY{p}{:}
    \PY{l+s+s2}{\PYZdq{}}\PY{l+s+s2}{Jumlah hari hujan}\PY{l+s+s2}{\PYZdq{}}
    \PY{k}{return} \PY{p}{(}\PY{n}{x} \PY{o}{\PYZgt{}} \PY{l+m+mi}{0}\PY{p}{)}\PY{o}{.}\PY{n}{sum}\PY{p}{(}\PY{p}{)}

\PY{n}{myfunc} \PY{o}{=} \PY{p}{[}\PY{n}{np}\PY{o}{.}\PY{n}{sum}\PY{p}{,} \PY{n}{n\PYZus{}rain}\PY{p}{,} \PY{n+nb}{len}\PY{p}{]}
\PY{n}{myfunc\PYZus{}col} \PY{o}{=} \PY{p}{[}\PY{l+s+s1}{\PYZsq{}}\PY{l+s+s1}{sum}\PY{l+s+s1}{\PYZsq{}}\PY{p}{,} \PY{l+s+s1}{\PYZsq{}}\PY{l+s+s1}{n\PYZus{}rain}\PY{l+s+s1}{\PYZsq{}}\PY{p}{,} \PY{l+s+s1}{\PYZsq{}}\PY{l+s+s1}{n\PYZus{}days}\PY{l+s+s1}{\PYZsq{}}\PY{p}{]}
\end{Verbatim}
\end{tcolorbox}

    \hypertarget{penggunaan-default}{%
\subsubsection{\texorpdfstring{Penggunaan
(\emph{default})}{Penggunaan (default)}}\label{penggunaan-default}}

Jika tidak diberi argumen \texttt{n\_days} maka fungsi ini akan
memproses data setiap bulan.

    \begin{tcolorbox}[breakable, size=fbox, boxrule=1pt, pad at break*=1mm,colback=cellbackground, colframe=cellborder]
\prompt{In}{incolor}{ }{\boxspacing}
\begin{Verbatim}[commandchars=\\\{\}]
\PY{n}{summary\PYZus{}station}\PY{p}{(}
    \PY{n}{dataset}\PY{o}{=}\PY{n}{dataset}\PY{p}{,} \PY{n}{column}\PY{o}{=}\PY{l+s+s1}{\PYZsq{}}\PY{l+s+s1}{STA\PYZus{}B}\PY{l+s+s1}{\PYZsq{}}\PY{p}{,} 
    \PY{n}{ufunc}\PY{o}{=}\PY{n}{myfunc}\PY{p}{,} \PY{n}{ufunc\PYZus{}col}\PY{o}{=}\PY{n}{myfunc\PYZus{}col}\PY{p}{)}
\end{Verbatim}
\end{tcolorbox}

            \begin{tcolorbox}[breakable, size=fbox, boxrule=.5pt, pad at break*=1mm, opacityfill=0]
\prompt{Out}{outcolor}{ }{\boxspacing}
\begin{Verbatim}[commandchars=\\\{\}]
            STA\_B
              sum n\_rain n\_days
2001-01-31  454.0   18.0   31.0
2001-02-28  298.0   12.0   28.0
2001-03-31  475.0   18.0   31.0
2001-04-30  272.0   12.0   30.0
2001-05-31   86.0    4.0   31.0
{\ldots}           {\ldots}    {\ldots}    {\ldots}
2015-08-31    0.0    0.0   31.0
2015-09-30    0.0    0.0   30.0
2015-10-31   14.0    1.0   31.0
2015-11-30  165.0    3.0   30.0
2015-12-31  216.0   11.0   31.0

[180 rows x 3 columns]
\end{Verbatim}
\end{tcolorbox}
        
    \hypertarget{argumen-n_days}{%
\subsubsection{\texorpdfstring{Argumen
\texttt{n\_days}}{Argumen n\_days}}\label{argumen-n_days}}

\texttt{n\_days} bisa diisi dengan jumlah hari yang ingin diproses
\textbf{setiap bulan}-nya.

    \begin{tcolorbox}[breakable, size=fbox, boxrule=1pt, pad at break*=1mm,colback=cellbackground, colframe=cellborder]
\prompt{In}{incolor}{ }{\boxspacing}
\begin{Verbatim}[commandchars=\\\{\}]
\PY{c+c1}{\PYZsh{} Setiap 8 Hari}
\PY{n}{summary\PYZus{}station}\PY{p}{(}
    \PY{n}{dataset}\PY{o}{=}\PY{n}{dataset}\PY{p}{,} \PY{n}{column}\PY{o}{=}\PY{l+s+s1}{\PYZsq{}}\PY{l+s+s1}{STA\PYZus{}B}\PY{l+s+s1}{\PYZsq{}}\PY{p}{,} 
    \PY{n}{ufunc}\PY{o}{=}\PY{n}{myfunc}\PY{p}{,} \PY{n}{ufunc\PYZus{}col}\PY{o}{=}\PY{n}{myfunc\PYZus{}col}\PY{p}{,}
    \PY{n}{n\PYZus{}days}\PY{o}{=}\PY{l+s+s1}{\PYZsq{}}\PY{l+s+s1}{8D}\PY{l+s+s1}{\PYZsq{}}\PY{p}{)}
\end{Verbatim}
\end{tcolorbox}

            \begin{tcolorbox}[breakable, size=fbox, boxrule=.5pt, pad at break*=1mm, opacityfill=0]
\prompt{Out}{outcolor}{ }{\boxspacing}
\begin{Verbatim}[commandchars=\\\{\}]
            STA\_B
              sum n\_rain n\_days
2001-01-01   90.0    4.0    8.0
2001-01-09  123.0    5.0    8.0
2001-01-17  192.0    6.0    8.0
2001-01-25   49.0    3.0    7.0
2001-02-01  129.0    5.0    8.0
{\ldots}           {\ldots}    {\ldots}    {\ldots}
2015-11-25   48.0    1.0    6.0
2015-12-01   78.0    2.0    8.0
2015-12-09   48.0    4.0    8.0
2015-12-17   52.0    3.0    8.0
2015-12-25   38.0    2.0    7.0

[720 rows x 3 columns]
\end{Verbatim}
\end{tcolorbox}
        
    \begin{tcolorbox}[breakable, size=fbox, boxrule=1pt, pad at break*=1mm,colback=cellbackground, colframe=cellborder]
\prompt{In}{incolor}{ }{\boxspacing}
\begin{Verbatim}[commandchars=\\\{\}]
\PY{c+c1}{\PYZsh{} Setiap 15 Hari}
\PY{n}{summary\PYZus{}station}\PY{p}{(}
    \PY{n}{dataset}\PY{o}{=}\PY{n}{dataset}\PY{p}{,} \PY{n}{column}\PY{o}{=}\PY{l+s+s1}{\PYZsq{}}\PY{l+s+s1}{STA\PYZus{}C}\PY{l+s+s1}{\PYZsq{}}\PY{p}{,} 
    \PY{n}{ufunc}\PY{o}{=}\PY{n}{myfunc}\PY{p}{,} \PY{n}{ufunc\PYZus{}col}\PY{o}{=}\PY{n}{myfunc\PYZus{}col}\PY{p}{,}
    \PY{n}{n\PYZus{}days}\PY{o}{=}\PY{l+s+s1}{\PYZsq{}}\PY{l+s+s1}{15D}\PY{l+s+s1}{\PYZsq{}}\PY{p}{)}
\end{Verbatim}
\end{tcolorbox}

            \begin{tcolorbox}[breakable, size=fbox, boxrule=.5pt, pad at break*=1mm, opacityfill=0]
\prompt{Out}{outcolor}{ }{\boxspacing}
\begin{Verbatim}[commandchars=\\\{\}]
             STA\_C
               sum n\_rain n\_days
2001-01-01  158.08   13.0   15.0
2001-01-16  146.94   14.0   15.0
2001-01-31   22.96    1.0    1.0
2001-02-01  157.80   12.0   15.0
2001-02-16   77.45   11.0   13.0
{\ldots}            {\ldots}    {\ldots}    {\ldots}
2015-11-01  152.00    7.0   15.0
2015-11-16   76.00    4.0   15.0
2015-12-01   23.00    1.0   15.0
2015-12-16   46.00    8.0   15.0
2015-12-31    0.00    0.0    1.0

[465 rows x 3 columns]
\end{Verbatim}
\end{tcolorbox}
        
    \hypertarget{fungsi-summary_all}{%
\subsection{\texorpdfstring{Fungsi
\texttt{summary\_all()}}{Fungsi summary\_all()}}\label{fungsi-summary_all}}

Fungsi ini hanya melakukan proses \texttt{summary\_station()} untuk
seluruh kolom atau kolom tertentu yang diatur dengan argumen
\texttt{columns}. Argumen \texttt{dataset}, \texttt{ufunc},
\texttt{ufunc\_col},
\texttt{n\_days=\textquotesingle{}M\textquotesingle{}} sama dengan
\texttt{summary\_station()}, yang membedakan adalah argumen
\texttt{columns}.

Baru di versi \texttt{0.4.0} (isu
\href{https://github.com/hidrokit/hidrokit/issues/115}{\#115}):
Penambahan argumen \texttt{verbose} untuk menampilkan/tidak menampilkan
tulisan \texttt{PROCESSING:\ ...}. Nilai default \texttt{False} (tidak
ditampilkan).

    \begin{tcolorbox}[breakable, size=fbox, boxrule=1pt, pad at break*=1mm,colback=cellbackground, colframe=cellborder]
\prompt{In}{incolor}{ }{\boxspacing}
\begin{Verbatim}[commandchars=\\\{\}]
\PY{c+c1}{\PYZsh{} Menggunakan fungsi yang lebih banyak}
\PY{k}{def} \PY{n+nf}{n\PYZus{}rain}\PY{p}{(}\PY{n}{x}\PY{p}{)}\PY{p}{:}
    \PY{l+s+s2}{\PYZdq{}}\PY{l+s+s2}{Jumlah hari hujan}\PY{l+s+s2}{\PYZdq{}}
    \PY{k}{return} \PY{p}{(}\PY{n}{x} \PY{o}{\PYZgt{}} \PY{l+m+mi}{0}\PY{p}{)}\PY{o}{.}\PY{n}{sum}\PY{p}{(}\PY{p}{)}

\PY{k}{def} \PY{n+nf}{n\PYZus{}dry}\PY{p}{(}\PY{n}{x}\PY{p}{)}\PY{p}{:}
    \PY{l+s+s2}{\PYZdq{}}\PY{l+s+s2}{Jumlah hari kering}\PY{l+s+s2}{\PYZdq{}}
    \PY{k}{return} \PY{n}{np}\PY{o}{.}\PY{n}{logical\PYZus{}or}\PY{p}{(}\PY{n}{x} \PY{o}{==} \PY{l+m+mi}{0}\PY{p}{,} \PY{n}{x}\PY{o}{.}\PY{n}{isna}\PY{p}{(}\PY{p}{)}\PY{p}{)}\PY{o}{.}\PY{n}{sum}\PY{p}{(}\PY{p}{)}

\PY{n}{myfunc\PYZus{}all} \PY{o}{=} \PY{p}{[}\PY{n+nb}{len}\PY{p}{,} \PY{n}{n\PYZus{}rain}\PY{p}{,} \PY{n}{n\PYZus{}dry}\PY{p}{,} \PY{n}{np}\PY{o}{.}\PY{n}{sum}\PY{p}{,} \PY{n}{np}\PY{o}{.}\PY{n}{mean}\PY{p}{,} \PY{n}{np}\PY{o}{.}\PY{n}{std}\PY{p}{]}
\PY{n}{myfunc\PYZus{}all\PYZus{}col} \PY{o}{=} \PY{p}{[}\PY{l+s+s1}{\PYZsq{}}\PY{l+s+s1}{n\PYZus{}days}\PY{l+s+s1}{\PYZsq{}}\PY{p}{,} \PY{l+s+s1}{\PYZsq{}}\PY{l+s+s1}{n\PYZus{}rain}\PY{l+s+s1}{\PYZsq{}}\PY{p}{,} \PY{l+s+s1}{\PYZsq{}}\PY{l+s+s1}{n\PYZus{}dry}\PY{l+s+s1}{\PYZsq{}}\PY{p}{,} \PY{l+s+s1}{\PYZsq{}}\PY{l+s+s1}{SUM}\PY{l+s+s1}{\PYZsq{}}\PY{p}{,} \PY{l+s+s1}{\PYZsq{}}\PY{l+s+s1}{MEAN}\PY{l+s+s1}{\PYZsq{}}\PY{p}{,} \PY{l+s+s1}{\PYZsq{}}\PY{l+s+s1}{STD}\PY{l+s+s1}{\PYZsq{}}\PY{p}{]}
\end{Verbatim}
\end{tcolorbox}

    \hypertarget{seluruh-kolom}{%
\subsubsection{Seluruh kolom}\label{seluruh-kolom}}

    \begin{tcolorbox}[breakable, size=fbox, boxrule=1pt, pad at break*=1mm,colback=cellbackground, colframe=cellborder]
\prompt{In}{incolor}{ }{\boxspacing}
\begin{Verbatim}[commandchars=\\\{\}]
\PY{n}{summary\PYZus{}all}\PY{p}{(}
    \PY{n}{dataset}\PY{o}{=}\PY{n}{dataset}\PY{p}{,}
    \PY{n}{ufunc}\PY{o}{=}\PY{n}{myfunc\PYZus{}all}\PY{p}{,} \PY{n}{ufunc\PYZus{}col}\PY{o}{=}\PY{n}{myfunc\PYZus{}all\PYZus{}col}\PY{p}{,}
    \PY{n}{n\PYZus{}days}\PY{o}{=}\PY{l+s+s1}{\PYZsq{}}\PY{l+s+s1}{7D}\PY{l+s+s1}{\PYZsq{}}\PY{p}{)}
\end{Verbatim}
\end{tcolorbox}

            \begin{tcolorbox}[breakable, size=fbox, boxrule=.5pt, pad at break*=1mm, opacityfill=0]
\prompt{Out}{outcolor}{ }{\boxspacing}
\begin{Verbatim}[commandchars=\\\{\}]
            STA\_A                                            STA\_B         \textbackslash{}
           n\_days n\_rain n\_dry    SUM       MEAN        STD n\_days n\_rain
2001-01-01    7.0    0.0   7.0    0.0   0.000000   0.000000    7.0    3.0
2001-01-08    7.0    0.0   7.0    0.0   0.000000   0.000000    7.0    4.0
2001-01-15    7.0    0.0   7.0    0.0   0.000000   0.000000    7.0    5.0
2001-01-22    7.0    0.0   7.0    0.0   0.000000   0.000000    7.0    4.0
2001-01-29    3.0    0.0   3.0    0.0   0.000000   0.000000    3.0    2.0
{\ldots}           {\ldots}    {\ldots}   {\ldots}    {\ldots}        {\ldots}        {\ldots}    {\ldots}    {\ldots}
2015-12-01    7.0    3.0   4.0   86.0  12.285714  25.408098    7.0    1.0
2015-12-08    7.0    5.0   2.0   55.0   7.857143  12.266874    7.0    3.0
2015-12-15    7.0    7.0   0.0  105.0  15.000000  11.503623    7.0    4.0
2015-12-22    7.0    7.0   0.0  136.0  19.428571  12.053452    7.0    3.0
2015-12-29    3.0    3.0   0.0   15.0   5.000000   2.000000    3.0    0.0

                                               STA\_C                       \textbackslash{}
           n\_dry    SUM       MEAN        STD n\_days n\_rain n\_dry     SUM
2001-01-01   4.0   58.0   8.285714  16.499639    7.0    5.0   2.0   15.43
2001-01-08   3.0   68.0   9.714286  13.300555    7.0    7.0   0.0  125.30
2001-01-15   2.0  224.0  32.000000  37.434387    7.0    7.0   0.0   93.38
2001-01-22   3.0   60.0   8.571429  10.906529    7.0    6.0   1.0   55.88
2001-01-29   1.0   44.0  14.666667  13.650397    3.0    3.0   0.0   37.99
{\ldots}          {\ldots}    {\ldots}        {\ldots}        {\ldots}    {\ldots}    {\ldots}   {\ldots}     {\ldots}
2015-12-01   6.0   43.0   6.142857  16.252472    7.0    0.0   7.0    0.00
2015-12-08   4.0   67.0   9.571429  13.513662    7.0    0.0   7.0    0.00
2015-12-15   3.0   55.0   7.857143   9.118271    7.0    5.0   2.0   44.00
2015-12-22   4.0   51.0   7.285714  11.954278    7.0    2.0   5.0   18.00
2015-12-29   3.0    0.0   0.000000   0.000000    3.0    2.0   1.0    7.00


                 MEAN        STD
2001-01-01   2.204286   3.384252
2001-01-08  17.900000  32.992996
2001-01-15  13.340000  12.806884
2001-01-22   7.982857   5.429001
2001-01-29  12.663333  11.376741
{\ldots}               {\ldots}        {\ldots}
2015-12-01   0.000000   0.000000
2015-12-08   0.000000   0.000000
2015-12-15   6.285714   8.220184
2015-12-22   2.571429   5.255383
2015-12-29   2.333333   3.214550

[888 rows x 18 columns]
\end{Verbatim}
\end{tcolorbox}
        
    \hypertarget{kolom-tertentu}{%
\subsubsection{Kolom tertentu}\label{kolom-tertentu}}

    \begin{tcolorbox}[breakable, size=fbox, boxrule=1pt, pad at break*=1mm,colback=cellbackground, colframe=cellborder]
\prompt{In}{incolor}{ }{\boxspacing}
\begin{Verbatim}[commandchars=\\\{\}]
\PY{n}{summary\PYZus{}all}\PY{p}{(}
    \PY{n}{dataset}\PY{o}{=}\PY{n}{dataset}\PY{p}{,} \PY{n}{columns}\PY{o}{=}\PY{p}{[}\PY{l+s+s1}{\PYZsq{}}\PY{l+s+s1}{STA\PYZus{}A}\PY{l+s+s1}{\PYZsq{}}\PY{p}{,} \PY{l+s+s1}{\PYZsq{}}\PY{l+s+s1}{STA\PYZus{}C}\PY{l+s+s1}{\PYZsq{}}\PY{p}{]}\PY{p}{,}
    \PY{n}{ufunc}\PY{o}{=}\PY{n}{myfunc\PYZus{}all}\PY{p}{,} \PY{n}{ufunc\PYZus{}col}\PY{o}{=}\PY{n}{myfunc\PYZus{}all\PYZus{}col}\PY{p}{,}
    \PY{n}{n\PYZus{}days}\PY{o}{=}\PY{l+s+s1}{\PYZsq{}}\PY{l+s+s1}{16D}\PY{l+s+s1}{\PYZsq{}}\PY{p}{)}
\end{Verbatim}
\end{tcolorbox}

            \begin{tcolorbox}[breakable, size=fbox, boxrule=.5pt, pad at break*=1mm, opacityfill=0]
\prompt{Out}{outcolor}{ }{\boxspacing}
\begin{Verbatim}[commandchars=\\\{\}]
            STA\_A                                            STA\_C         \textbackslash{}
           n\_days n\_rain n\_dry    SUM       MEAN        STD n\_days n\_rain
2001-01-01   16.0    0.0  16.0    0.0   0.000000   0.000000   16.0   14.0
2001-01-17   15.0    0.0  15.0    0.0   0.000000   0.000000   15.0   14.0
2001-02-01   16.0    0.0  16.0    0.0   0.000000   0.000000   16.0   13.0
2001-02-17   12.0    0.0  12.0    0.0   0.000000   0.000000   12.0   10.0
2001-03-01   16.0    0.0  16.0    0.0   0.000000   0.000000   16.0   14.0
{\ldots}           {\ldots}    {\ldots}   {\ldots}    {\ldots}        {\ldots}        {\ldots}    {\ldots}    {\ldots}
2015-10-17   15.0    9.0   6.0   97.0   6.466667   9.500877   15.0    1.0
2015-11-01   16.0   12.0   4.0  190.0  11.875000  14.655488   16.0    7.0
2015-11-17   14.0   13.0   1.0  384.0  27.428571  15.360861   14.0    4.0
2015-12-01   16.0   10.0   6.0  197.0  12.312500  19.269038   16.0    1.0
2015-12-17   15.0   15.0   0.0  200.0  13.333333  10.587504   15.0    8.0


           n\_dry     SUM       MEAN        STD
2001-01-01   2.0  185.80  11.612500  22.786906
2001-01-17   1.0  142.18   9.478667   9.163421
2001-02-01   3.0  157.81   9.863125  15.319503
2001-02-17   2.0   77.44   6.453333   7.835151
2001-03-01   2.0    9.01   0.563125   0.719228
{\ldots}          {\ldots}     {\ldots}        {\ldots}        {\ldots}
2015-10-17  14.0    6.00   0.400000   1.549193
2015-11-01   9.0  152.00   9.500000  19.721393
2015-11-17  10.0   76.00   5.428571  13.119066
2015-12-01  15.0   23.00   1.437500   5.750000
2015-12-17   7.0   46.00   3.066667   4.317186

[360 rows x 12 columns]
\end{Verbatim}
\end{tcolorbox}
        
    \hypertarget{penggunaan-verbose}{%
\subsubsection{Penggunaan Verbose}\label{penggunaan-verbose}}

    \begin{tcolorbox}[breakable, size=fbox, boxrule=1pt, pad at break*=1mm,colback=cellbackground, colframe=cellborder]
\prompt{In}{incolor}{ }{\boxspacing}
\begin{Verbatim}[commandchars=\\\{\}]
\PY{n}{summary\PYZus{}all}\PY{p}{(}
    \PY{n}{dataset}\PY{o}{=}\PY{n}{dataset}\PY{p}{,}
    \PY{n}{ufunc}\PY{o}{=}\PY{n}{myfunc\PYZus{}all}\PY{p}{,} \PY{n}{ufunc\PYZus{}col}\PY{o}{=}\PY{n}{myfunc\PYZus{}all\PYZus{}col}\PY{p}{,}
    \PY{n}{n\PYZus{}days}\PY{o}{=}\PY{l+s+s1}{\PYZsq{}}\PY{l+s+s1}{7D}\PY{l+s+s1}{\PYZsq{}}\PY{p}{,} \PY{n}{verbose}\PY{o}{=}\PY{k+kc}{True}\PY{p}{)}
\end{Verbatim}
\end{tcolorbox}

    \begin{Verbatim}[commandchars=\\\{\}]
PROCESSING: STA\_A
PROCESSING: STA\_B
PROCESSING: STA\_C
    \end{Verbatim}

            \begin{tcolorbox}[breakable, size=fbox, boxrule=.5pt, pad at break*=1mm, opacityfill=0]
\prompt{Out}{outcolor}{ }{\boxspacing}
\begin{Verbatim}[commandchars=\\\{\}]
            STA\_A                                            STA\_B         \textbackslash{}
           n\_days n\_rain n\_dry    SUM       MEAN        STD n\_days n\_rain
2001-01-01    7.0    0.0   7.0    0.0   0.000000   0.000000    7.0    3.0
2001-01-08    7.0    0.0   7.0    0.0   0.000000   0.000000    7.0    4.0
2001-01-15    7.0    0.0   7.0    0.0   0.000000   0.000000    7.0    5.0
2001-01-22    7.0    0.0   7.0    0.0   0.000000   0.000000    7.0    4.0
2001-01-29    3.0    0.0   3.0    0.0   0.000000   0.000000    3.0    2.0
{\ldots}           {\ldots}    {\ldots}   {\ldots}    {\ldots}        {\ldots}        {\ldots}    {\ldots}    {\ldots}
2015-12-01    7.0    3.0   4.0   86.0  12.285714  25.408098    7.0    1.0
2015-12-08    7.0    5.0   2.0   55.0   7.857143  12.266874    7.0    3.0
2015-12-15    7.0    7.0   0.0  105.0  15.000000  11.503623    7.0    4.0
2015-12-22    7.0    7.0   0.0  136.0  19.428571  12.053452    7.0    3.0
2015-12-29    3.0    3.0   0.0   15.0   5.000000   2.000000    3.0    0.0

                                               STA\_C                       \textbackslash{}
           n\_dry    SUM       MEAN        STD n\_days n\_rain n\_dry     SUM
2001-01-01   4.0   58.0   8.285714  16.499639    7.0    5.0   2.0   15.43
2001-01-08   3.0   68.0   9.714286  13.300555    7.0    7.0   0.0  125.30
2001-01-15   2.0  224.0  32.000000  37.434387    7.0    7.0   0.0   93.38
2001-01-22   3.0   60.0   8.571429  10.906529    7.0    6.0   1.0   55.88
2001-01-29   1.0   44.0  14.666667  13.650397    3.0    3.0   0.0   37.99
{\ldots}          {\ldots}    {\ldots}        {\ldots}        {\ldots}    {\ldots}    {\ldots}   {\ldots}     {\ldots}
2015-12-01   6.0   43.0   6.142857  16.252472    7.0    0.0   7.0    0.00
2015-12-08   4.0   67.0   9.571429  13.513662    7.0    0.0   7.0    0.00
2015-12-15   3.0   55.0   7.857143   9.118271    7.0    5.0   2.0   44.00
2015-12-22   4.0   51.0   7.285714  11.954278    7.0    2.0   5.0   18.00
2015-12-29   3.0    0.0   0.000000   0.000000    3.0    2.0   1.0    7.00


                 MEAN        STD
2001-01-01   2.204286   3.384252
2001-01-08  17.900000  32.992996
2001-01-15  13.340000  12.806884
2001-01-22   7.982857   5.429001
2001-01-29  12.663333  11.376741
{\ldots}               {\ldots}        {\ldots}
2015-12-01   0.000000   0.000000
2015-12-08   0.000000   0.000000
2015-12-15   6.285714   8.220184
2015-12-22   2.571429   5.255383
2015-12-29   2.333333   3.214550

[888 rows x 18 columns]
\end{Verbatim}
\end{tcolorbox}
        
    \hypertarget{changelog}{%
\section{Changelog}\label{changelog}}

\begin{verbatim}
- 20220311 - 1.0.1 - Add verbose (#115)
- 20191217 - 1.0.0 - Initial
\end{verbatim}

\hypertarget{copyright-2022-taruma-sakti-megariansyah}{%
\paragraph{\texorpdfstring{Copyright © 2022
\href{https://taruma.github.io}{Taruma Sakti
Megariansyah}}{Copyright © 2022 Taruma Sakti Megariansyah}}\label{copyright-2022-taruma-sakti-megariansyah}}

Source code in this notebook is licensed under a
\href{https://choosealicense.com/licenses/mit/}{MIT License}. Data in
this notebook is licensed under a
\href{https://creativecommons.org/licenses/by/4.0/}{Creative Common
Attribution 4.0 International}.


    % Add a bibliography block to the postdoc
    
    
    
\end{document}
