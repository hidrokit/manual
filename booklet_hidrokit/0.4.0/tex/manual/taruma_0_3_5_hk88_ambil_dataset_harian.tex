\documentclass[11pt]{article}

    \usepackage[breakable]{tcolorbox}
    \usepackage{parskip} % Stop auto-indenting (to mimic markdown behaviour)
    
    \usepackage{iftex}
    \ifPDFTeX
    	\usepackage[T1]{fontenc}
    	\usepackage{mathpazo}
    \else
    	\usepackage{fontspec}
    \fi

    % Basic figure setup, for now with no caption control since it's done
    % automatically by Pandoc (which extracts ![](path) syntax from Markdown).
    \usepackage{graphicx}
    % Maintain compatibility with old templates. Remove in nbconvert 6.0
    \let\Oldincludegraphics\includegraphics
    % Ensure that by default, figures have no caption (until we provide a
    % proper Figure object with a Caption API and a way to capture that
    % in the conversion process - todo).
    \usepackage{caption}
    \DeclareCaptionFormat{nocaption}{}
    \captionsetup{format=nocaption,aboveskip=0pt,belowskip=0pt}

    \usepackage{float}
    \floatplacement{figure}{H} % forces figures to be placed at the correct location
    \usepackage{xcolor} % Allow colors to be defined
    \usepackage{enumerate} % Needed for markdown enumerations to work
    \usepackage{geometry} % Used to adjust the document margins
    \usepackage{amsmath} % Equations
    \usepackage{amssymb} % Equations
    \usepackage{textcomp} % defines textquotesingle
    % Hack from http://tex.stackexchange.com/a/47451/13684:
    \AtBeginDocument{%
        \def\PYZsq{\textquotesingle}% Upright quotes in Pygmentized code
    }
    \usepackage{upquote} % Upright quotes for verbatim code
    \usepackage{eurosym} % defines \euro
    \usepackage[mathletters]{ucs} % Extended unicode (utf-8) support
    \usepackage{fancyvrb} % verbatim replacement that allows latex
    \usepackage{grffile} % extends the file name processing of package graphics 
                         % to support a larger range
    \makeatletter % fix for old versions of grffile with XeLaTeX
    \@ifpackagelater{grffile}{2019/11/01}
    {
      % Do nothing on new versions
    }
    {
      \def\Gread@@xetex#1{%
        \IfFileExists{"\Gin@base".bb}%
        {\Gread@eps{\Gin@base.bb}}%
        {\Gread@@xetex@aux#1}%
      }
    }
    \makeatother
    \usepackage[Export]{adjustbox} % Used to constrain images to a maximum size
    \adjustboxset{max size={0.9\linewidth}{0.9\paperheight}}

    % The hyperref package gives us a pdf with properly built
    % internal navigation ('pdf bookmarks' for the table of contents,
    % internal cross-reference links, web links for URLs, etc.)
    \usepackage{hyperref}
    % The default LaTeX title has an obnoxious amount of whitespace. By default,
    % titling removes some of it. It also provides customization options.
    \usepackage{titling}
    \usepackage{longtable} % longtable support required by pandoc >1.10
    \usepackage{booktabs}  % table support for pandoc > 1.12.2
    \usepackage[inline]{enumitem} % IRkernel/repr support (it uses the enumerate* environment)
    \usepackage[normalem]{ulem} % ulem is needed to support strikethroughs (\sout)
                                % normalem makes italics be italics, not underlines
    \usepackage{mathrsfs}
    

    
    % Colors for the hyperref package
    \definecolor{urlcolor}{rgb}{0,.145,.698}
    \definecolor{linkcolor}{rgb}{.71,0.21,0.01}
    \definecolor{citecolor}{rgb}{.12,.54,.11}

    % ANSI colors
    \definecolor{ansi-black}{HTML}{3E424D}
    \definecolor{ansi-black-intense}{HTML}{282C36}
    \definecolor{ansi-red}{HTML}{E75C58}
    \definecolor{ansi-red-intense}{HTML}{B22B31}
    \definecolor{ansi-green}{HTML}{00A250}
    \definecolor{ansi-green-intense}{HTML}{007427}
    \definecolor{ansi-yellow}{HTML}{DDB62B}
    \definecolor{ansi-yellow-intense}{HTML}{B27D12}
    \definecolor{ansi-blue}{HTML}{208FFB}
    \definecolor{ansi-blue-intense}{HTML}{0065CA}
    \definecolor{ansi-magenta}{HTML}{D160C4}
    \definecolor{ansi-magenta-intense}{HTML}{A03196}
    \definecolor{ansi-cyan}{HTML}{60C6C8}
    \definecolor{ansi-cyan-intense}{HTML}{258F8F}
    \definecolor{ansi-white}{HTML}{C5C1B4}
    \definecolor{ansi-white-intense}{HTML}{A1A6B2}
    \definecolor{ansi-default-inverse-fg}{HTML}{FFFFFF}
    \definecolor{ansi-default-inverse-bg}{HTML}{000000}

    % common color for the border for error outputs.
    \definecolor{outerrorbackground}{HTML}{FFDFDF}

    % commands and environments needed by pandoc snippets
    % extracted from the output of `pandoc -s`
    \providecommand{\tightlist}{%
      \setlength{\itemsep}{0pt}\setlength{\parskip}{0pt}}
    \DefineVerbatimEnvironment{Highlighting}{Verbatim}{commandchars=\\\{\}}
    % Add ',fontsize=\small' for more characters per line
    \newenvironment{Shaded}{}{}
    \newcommand{\KeywordTok}[1]{\textcolor[rgb]{0.00,0.44,0.13}{\textbf{{#1}}}}
    \newcommand{\DataTypeTok}[1]{\textcolor[rgb]{0.56,0.13,0.00}{{#1}}}
    \newcommand{\DecValTok}[1]{\textcolor[rgb]{0.25,0.63,0.44}{{#1}}}
    \newcommand{\BaseNTok}[1]{\textcolor[rgb]{0.25,0.63,0.44}{{#1}}}
    \newcommand{\FloatTok}[1]{\textcolor[rgb]{0.25,0.63,0.44}{{#1}}}
    \newcommand{\CharTok}[1]{\textcolor[rgb]{0.25,0.44,0.63}{{#1}}}
    \newcommand{\StringTok}[1]{\textcolor[rgb]{0.25,0.44,0.63}{{#1}}}
    \newcommand{\CommentTok}[1]{\textcolor[rgb]{0.38,0.63,0.69}{\textit{{#1}}}}
    \newcommand{\OtherTok}[1]{\textcolor[rgb]{0.00,0.44,0.13}{{#1}}}
    \newcommand{\AlertTok}[1]{\textcolor[rgb]{1.00,0.00,0.00}{\textbf{{#1}}}}
    \newcommand{\FunctionTok}[1]{\textcolor[rgb]{0.02,0.16,0.49}{{#1}}}
    \newcommand{\RegionMarkerTok}[1]{{#1}}
    \newcommand{\ErrorTok}[1]{\textcolor[rgb]{1.00,0.00,0.00}{\textbf{{#1}}}}
    \newcommand{\NormalTok}[1]{{#1}}
    
    % Additional commands for more recent versions of Pandoc
    \newcommand{\ConstantTok}[1]{\textcolor[rgb]{0.53,0.00,0.00}{{#1}}}
    \newcommand{\SpecialCharTok}[1]{\textcolor[rgb]{0.25,0.44,0.63}{{#1}}}
    \newcommand{\VerbatimStringTok}[1]{\textcolor[rgb]{0.25,0.44,0.63}{{#1}}}
    \newcommand{\SpecialStringTok}[1]{\textcolor[rgb]{0.73,0.40,0.53}{{#1}}}
    \newcommand{\ImportTok}[1]{{#1}}
    \newcommand{\DocumentationTok}[1]{\textcolor[rgb]{0.73,0.13,0.13}{\textit{{#1}}}}
    \newcommand{\AnnotationTok}[1]{\textcolor[rgb]{0.38,0.63,0.69}{\textbf{\textit{{#1}}}}}
    \newcommand{\CommentVarTok}[1]{\textcolor[rgb]{0.38,0.63,0.69}{\textbf{\textit{{#1}}}}}
    \newcommand{\VariableTok}[1]{\textcolor[rgb]{0.10,0.09,0.49}{{#1}}}
    \newcommand{\ControlFlowTok}[1]{\textcolor[rgb]{0.00,0.44,0.13}{\textbf{{#1}}}}
    \newcommand{\OperatorTok}[1]{\textcolor[rgb]{0.40,0.40,0.40}{{#1}}}
    \newcommand{\BuiltInTok}[1]{{#1}}
    \newcommand{\ExtensionTok}[1]{{#1}}
    \newcommand{\PreprocessorTok}[1]{\textcolor[rgb]{0.74,0.48,0.00}{{#1}}}
    \newcommand{\AttributeTok}[1]{\textcolor[rgb]{0.49,0.56,0.16}{{#1}}}
    \newcommand{\InformationTok}[1]{\textcolor[rgb]{0.38,0.63,0.69}{\textbf{\textit{{#1}}}}}
    \newcommand{\WarningTok}[1]{\textcolor[rgb]{0.38,0.63,0.69}{\textbf{\textit{{#1}}}}}
    
    
    % Define a nice break command that doesn't care if a line doesn't already
    % exist.
    \def\br{\hspace*{\fill} \\* }
    % Math Jax compatibility definitions
    \def\gt{>}
    \def\lt{<}
    \let\Oldtex\TeX
    \let\Oldlatex\LaTeX
    \renewcommand{\TeX}{\textrm{\Oldtex}}
    \renewcommand{\LaTeX}{\textrm{\Oldlatex}}
    % Document parameters
    % Document title
    \title{taruma\_0\_3\_5\_hk88\_ambil\_dataset\_harian}
    
    
    
    
    
% Pygments definitions
\makeatletter
\def\PY@reset{\let\PY@it=\relax \let\PY@bf=\relax%
    \let\PY@ul=\relax \let\PY@tc=\relax%
    \let\PY@bc=\relax \let\PY@ff=\relax}
\def\PY@tok#1{\csname PY@tok@#1\endcsname}
\def\PY@toks#1+{\ifx\relax#1\empty\else%
    \PY@tok{#1}\expandafter\PY@toks\fi}
\def\PY@do#1{\PY@bc{\PY@tc{\PY@ul{%
    \PY@it{\PY@bf{\PY@ff{#1}}}}}}}
\def\PY#1#2{\PY@reset\PY@toks#1+\relax+\PY@do{#2}}

\@namedef{PY@tok@w}{\def\PY@tc##1{\textcolor[rgb]{0.73,0.73,0.73}{##1}}}
\@namedef{PY@tok@c}{\let\PY@it=\textit\def\PY@tc##1{\textcolor[rgb]{0.24,0.48,0.48}{##1}}}
\@namedef{PY@tok@cp}{\def\PY@tc##1{\textcolor[rgb]{0.61,0.40,0.00}{##1}}}
\@namedef{PY@tok@k}{\let\PY@bf=\textbf\def\PY@tc##1{\textcolor[rgb]{0.00,0.50,0.00}{##1}}}
\@namedef{PY@tok@kp}{\def\PY@tc##1{\textcolor[rgb]{0.00,0.50,0.00}{##1}}}
\@namedef{PY@tok@kt}{\def\PY@tc##1{\textcolor[rgb]{0.69,0.00,0.25}{##1}}}
\@namedef{PY@tok@o}{\def\PY@tc##1{\textcolor[rgb]{0.40,0.40,0.40}{##1}}}
\@namedef{PY@tok@ow}{\let\PY@bf=\textbf\def\PY@tc##1{\textcolor[rgb]{0.67,0.13,1.00}{##1}}}
\@namedef{PY@tok@nb}{\def\PY@tc##1{\textcolor[rgb]{0.00,0.50,0.00}{##1}}}
\@namedef{PY@tok@nf}{\def\PY@tc##1{\textcolor[rgb]{0.00,0.00,1.00}{##1}}}
\@namedef{PY@tok@nc}{\let\PY@bf=\textbf\def\PY@tc##1{\textcolor[rgb]{0.00,0.00,1.00}{##1}}}
\@namedef{PY@tok@nn}{\let\PY@bf=\textbf\def\PY@tc##1{\textcolor[rgb]{0.00,0.00,1.00}{##1}}}
\@namedef{PY@tok@ne}{\let\PY@bf=\textbf\def\PY@tc##1{\textcolor[rgb]{0.80,0.25,0.22}{##1}}}
\@namedef{PY@tok@nv}{\def\PY@tc##1{\textcolor[rgb]{0.10,0.09,0.49}{##1}}}
\@namedef{PY@tok@no}{\def\PY@tc##1{\textcolor[rgb]{0.53,0.00,0.00}{##1}}}
\@namedef{PY@tok@nl}{\def\PY@tc##1{\textcolor[rgb]{0.46,0.46,0.00}{##1}}}
\@namedef{PY@tok@ni}{\let\PY@bf=\textbf\def\PY@tc##1{\textcolor[rgb]{0.44,0.44,0.44}{##1}}}
\@namedef{PY@tok@na}{\def\PY@tc##1{\textcolor[rgb]{0.41,0.47,0.13}{##1}}}
\@namedef{PY@tok@nt}{\let\PY@bf=\textbf\def\PY@tc##1{\textcolor[rgb]{0.00,0.50,0.00}{##1}}}
\@namedef{PY@tok@nd}{\def\PY@tc##1{\textcolor[rgb]{0.67,0.13,1.00}{##1}}}
\@namedef{PY@tok@s}{\def\PY@tc##1{\textcolor[rgb]{0.73,0.13,0.13}{##1}}}
\@namedef{PY@tok@sd}{\let\PY@it=\textit\def\PY@tc##1{\textcolor[rgb]{0.73,0.13,0.13}{##1}}}
\@namedef{PY@tok@si}{\let\PY@bf=\textbf\def\PY@tc##1{\textcolor[rgb]{0.64,0.35,0.47}{##1}}}
\@namedef{PY@tok@se}{\let\PY@bf=\textbf\def\PY@tc##1{\textcolor[rgb]{0.67,0.36,0.12}{##1}}}
\@namedef{PY@tok@sr}{\def\PY@tc##1{\textcolor[rgb]{0.64,0.35,0.47}{##1}}}
\@namedef{PY@tok@ss}{\def\PY@tc##1{\textcolor[rgb]{0.10,0.09,0.49}{##1}}}
\@namedef{PY@tok@sx}{\def\PY@tc##1{\textcolor[rgb]{0.00,0.50,0.00}{##1}}}
\@namedef{PY@tok@m}{\def\PY@tc##1{\textcolor[rgb]{0.40,0.40,0.40}{##1}}}
\@namedef{PY@tok@gh}{\let\PY@bf=\textbf\def\PY@tc##1{\textcolor[rgb]{0.00,0.00,0.50}{##1}}}
\@namedef{PY@tok@gu}{\let\PY@bf=\textbf\def\PY@tc##1{\textcolor[rgb]{0.50,0.00,0.50}{##1}}}
\@namedef{PY@tok@gd}{\def\PY@tc##1{\textcolor[rgb]{0.63,0.00,0.00}{##1}}}
\@namedef{PY@tok@gi}{\def\PY@tc##1{\textcolor[rgb]{0.00,0.52,0.00}{##1}}}
\@namedef{PY@tok@gr}{\def\PY@tc##1{\textcolor[rgb]{0.89,0.00,0.00}{##1}}}
\@namedef{PY@tok@ge}{\let\PY@it=\textit}
\@namedef{PY@tok@gs}{\let\PY@bf=\textbf}
\@namedef{PY@tok@gp}{\let\PY@bf=\textbf\def\PY@tc##1{\textcolor[rgb]{0.00,0.00,0.50}{##1}}}
\@namedef{PY@tok@go}{\def\PY@tc##1{\textcolor[rgb]{0.44,0.44,0.44}{##1}}}
\@namedef{PY@tok@gt}{\def\PY@tc##1{\textcolor[rgb]{0.00,0.27,0.87}{##1}}}
\@namedef{PY@tok@err}{\def\PY@bc##1{{\setlength{\fboxsep}{\string -\fboxrule}\fcolorbox[rgb]{1.00,0.00,0.00}{1,1,1}{\strut ##1}}}}
\@namedef{PY@tok@kc}{\let\PY@bf=\textbf\def\PY@tc##1{\textcolor[rgb]{0.00,0.50,0.00}{##1}}}
\@namedef{PY@tok@kd}{\let\PY@bf=\textbf\def\PY@tc##1{\textcolor[rgb]{0.00,0.50,0.00}{##1}}}
\@namedef{PY@tok@kn}{\let\PY@bf=\textbf\def\PY@tc##1{\textcolor[rgb]{0.00,0.50,0.00}{##1}}}
\@namedef{PY@tok@kr}{\let\PY@bf=\textbf\def\PY@tc##1{\textcolor[rgb]{0.00,0.50,0.00}{##1}}}
\@namedef{PY@tok@bp}{\def\PY@tc##1{\textcolor[rgb]{0.00,0.50,0.00}{##1}}}
\@namedef{PY@tok@fm}{\def\PY@tc##1{\textcolor[rgb]{0.00,0.00,1.00}{##1}}}
\@namedef{PY@tok@vc}{\def\PY@tc##1{\textcolor[rgb]{0.10,0.09,0.49}{##1}}}
\@namedef{PY@tok@vg}{\def\PY@tc##1{\textcolor[rgb]{0.10,0.09,0.49}{##1}}}
\@namedef{PY@tok@vi}{\def\PY@tc##1{\textcolor[rgb]{0.10,0.09,0.49}{##1}}}
\@namedef{PY@tok@vm}{\def\PY@tc##1{\textcolor[rgb]{0.10,0.09,0.49}{##1}}}
\@namedef{PY@tok@sa}{\def\PY@tc##1{\textcolor[rgb]{0.73,0.13,0.13}{##1}}}
\@namedef{PY@tok@sb}{\def\PY@tc##1{\textcolor[rgb]{0.73,0.13,0.13}{##1}}}
\@namedef{PY@tok@sc}{\def\PY@tc##1{\textcolor[rgb]{0.73,0.13,0.13}{##1}}}
\@namedef{PY@tok@dl}{\def\PY@tc##1{\textcolor[rgb]{0.73,0.13,0.13}{##1}}}
\@namedef{PY@tok@s2}{\def\PY@tc##1{\textcolor[rgb]{0.73,0.13,0.13}{##1}}}
\@namedef{PY@tok@sh}{\def\PY@tc##1{\textcolor[rgb]{0.73,0.13,0.13}{##1}}}
\@namedef{PY@tok@s1}{\def\PY@tc##1{\textcolor[rgb]{0.73,0.13,0.13}{##1}}}
\@namedef{PY@tok@mb}{\def\PY@tc##1{\textcolor[rgb]{0.40,0.40,0.40}{##1}}}
\@namedef{PY@tok@mf}{\def\PY@tc##1{\textcolor[rgb]{0.40,0.40,0.40}{##1}}}
\@namedef{PY@tok@mh}{\def\PY@tc##1{\textcolor[rgb]{0.40,0.40,0.40}{##1}}}
\@namedef{PY@tok@mi}{\def\PY@tc##1{\textcolor[rgb]{0.40,0.40,0.40}{##1}}}
\@namedef{PY@tok@il}{\def\PY@tc##1{\textcolor[rgb]{0.40,0.40,0.40}{##1}}}
\@namedef{PY@tok@mo}{\def\PY@tc##1{\textcolor[rgb]{0.40,0.40,0.40}{##1}}}
\@namedef{PY@tok@ch}{\let\PY@it=\textit\def\PY@tc##1{\textcolor[rgb]{0.24,0.48,0.48}{##1}}}
\@namedef{PY@tok@cm}{\let\PY@it=\textit\def\PY@tc##1{\textcolor[rgb]{0.24,0.48,0.48}{##1}}}
\@namedef{PY@tok@cpf}{\let\PY@it=\textit\def\PY@tc##1{\textcolor[rgb]{0.24,0.48,0.48}{##1}}}
\@namedef{PY@tok@c1}{\let\PY@it=\textit\def\PY@tc##1{\textcolor[rgb]{0.24,0.48,0.48}{##1}}}
\@namedef{PY@tok@cs}{\let\PY@it=\textit\def\PY@tc##1{\textcolor[rgb]{0.24,0.48,0.48}{##1}}}

\def\PYZbs{\char`\\}
\def\PYZus{\char`\_}
\def\PYZob{\char`\{}
\def\PYZcb{\char`\}}
\def\PYZca{\char`\^}
\def\PYZam{\char`\&}
\def\PYZlt{\char`\<}
\def\PYZgt{\char`\>}
\def\PYZsh{\char`\#}
\def\PYZpc{\char`\%}
\def\PYZdl{\char`\$}
\def\PYZhy{\char`\-}
\def\PYZsq{\char`\'}
\def\PYZdq{\char`\"}
\def\PYZti{\char`\~}
% for compatibility with earlier versions
\def\PYZat{@}
\def\PYZlb{[}
\def\PYZrb{]}
\makeatother


    % For linebreaks inside Verbatim environment from package fancyvrb. 
    \makeatletter
        \newbox\Wrappedcontinuationbox 
        \newbox\Wrappedvisiblespacebox 
        \newcommand*\Wrappedvisiblespace {\textcolor{red}{\textvisiblespace}} 
        \newcommand*\Wrappedcontinuationsymbol {\textcolor{red}{\llap{\tiny$\m@th\hookrightarrow$}}} 
        \newcommand*\Wrappedcontinuationindent {3ex } 
        \newcommand*\Wrappedafterbreak {\kern\Wrappedcontinuationindent\copy\Wrappedcontinuationbox} 
        % Take advantage of the already applied Pygments mark-up to insert 
        % potential linebreaks for TeX processing. 
        %        {, <, #, %, $, ' and ": go to next line. 
        %        _, }, ^, &, >, - and ~: stay at end of broken line. 
        % Use of \textquotesingle for straight quote. 
        \newcommand*\Wrappedbreaksatspecials {% 
            \def\PYGZus{\discretionary{\char`\_}{\Wrappedafterbreak}{\char`\_}}% 
            \def\PYGZob{\discretionary{}{\Wrappedafterbreak\char`\{}{\char`\{}}% 
            \def\PYGZcb{\discretionary{\char`\}}{\Wrappedafterbreak}{\char`\}}}% 
            \def\PYGZca{\discretionary{\char`\^}{\Wrappedafterbreak}{\char`\^}}% 
            \def\PYGZam{\discretionary{\char`\&}{\Wrappedafterbreak}{\char`\&}}% 
            \def\PYGZlt{\discretionary{}{\Wrappedafterbreak\char`\<}{\char`\<}}% 
            \def\PYGZgt{\discretionary{\char`\>}{\Wrappedafterbreak}{\char`\>}}% 
            \def\PYGZsh{\discretionary{}{\Wrappedafterbreak\char`\#}{\char`\#}}% 
            \def\PYGZpc{\discretionary{}{\Wrappedafterbreak\char`\%}{\char`\%}}% 
            \def\PYGZdl{\discretionary{}{\Wrappedafterbreak\char`\$}{\char`\$}}% 
            \def\PYGZhy{\discretionary{\char`\-}{\Wrappedafterbreak}{\char`\-}}% 
            \def\PYGZsq{\discretionary{}{\Wrappedafterbreak\textquotesingle}{\textquotesingle}}% 
            \def\PYGZdq{\discretionary{}{\Wrappedafterbreak\char`\"}{\char`\"}}% 
            \def\PYGZti{\discretionary{\char`\~}{\Wrappedafterbreak}{\char`\~}}% 
        } 
        % Some characters . , ; ? ! / are not pygmentized. 
        % This macro makes them "active" and they will insert potential linebreaks 
        \newcommand*\Wrappedbreaksatpunct {% 
            \lccode`\~`\.\lowercase{\def~}{\discretionary{\hbox{\char`\.}}{\Wrappedafterbreak}{\hbox{\char`\.}}}% 
            \lccode`\~`\,\lowercase{\def~}{\discretionary{\hbox{\char`\,}}{\Wrappedafterbreak}{\hbox{\char`\,}}}% 
            \lccode`\~`\;\lowercase{\def~}{\discretionary{\hbox{\char`\;}}{\Wrappedafterbreak}{\hbox{\char`\;}}}% 
            \lccode`\~`\:\lowercase{\def~}{\discretionary{\hbox{\char`\:}}{\Wrappedafterbreak}{\hbox{\char`\:}}}% 
            \lccode`\~`\?\lowercase{\def~}{\discretionary{\hbox{\char`\?}}{\Wrappedafterbreak}{\hbox{\char`\?}}}% 
            \lccode`\~`\!\lowercase{\def~}{\discretionary{\hbox{\char`\!}}{\Wrappedafterbreak}{\hbox{\char`\!}}}% 
            \lccode`\~`\/\lowercase{\def~}{\discretionary{\hbox{\char`\/}}{\Wrappedafterbreak}{\hbox{\char`\/}}}% 
            \catcode`\.\active
            \catcode`\,\active 
            \catcode`\;\active
            \catcode`\:\active
            \catcode`\?\active
            \catcode`\!\active
            \catcode`\/\active 
            \lccode`\~`\~ 	
        }
    \makeatother

    \let\OriginalVerbatim=\Verbatim
    \makeatletter
    \renewcommand{\Verbatim}[1][1]{%
        %\parskip\z@skip
        \sbox\Wrappedcontinuationbox {\Wrappedcontinuationsymbol}%
        \sbox\Wrappedvisiblespacebox {\FV@SetupFont\Wrappedvisiblespace}%
        \def\FancyVerbFormatLine ##1{\hsize\linewidth
            \vtop{\raggedright\hyphenpenalty\z@\exhyphenpenalty\z@
                \doublehyphendemerits\z@\finalhyphendemerits\z@
                \strut ##1\strut}%
        }%
        % If the linebreak is at a space, the latter will be displayed as visible
        % space at end of first line, and a continuation symbol starts next line.
        % Stretch/shrink are however usually zero for typewriter font.
        \def\FV@Space {%
            \nobreak\hskip\z@ plus\fontdimen3\font minus\fontdimen4\font
            \discretionary{\copy\Wrappedvisiblespacebox}{\Wrappedafterbreak}
            {\kern\fontdimen2\font}%
        }%
        
        % Allow breaks at special characters using \PYG... macros.
        \Wrappedbreaksatspecials
        % Breaks at punctuation characters . , ; ? ! and / need catcode=\active 	
        \OriginalVerbatim[#1,codes*=\Wrappedbreaksatpunct]%
    }
    \makeatother

    % Exact colors from NB
    \definecolor{incolor}{HTML}{303F9F}
    \definecolor{outcolor}{HTML}{D84315}
    \definecolor{cellborder}{HTML}{CFCFCF}
    \definecolor{cellbackground}{HTML}{F7F7F7}
    
    % prompt
    \makeatletter
    \newcommand{\boxspacing}{\kern\kvtcb@left@rule\kern\kvtcb@boxsep}
    \makeatother
    \newcommand{\prompt}[4]{
        {\ttfamily\llap{{\color{#2}[#3]:\hspace{3pt}#4}}\vspace{-\baselineskip}}
    }
    

    
    % Prevent overflowing lines due to hard-to-break entities
    \sloppy 
    % Setup hyperref package
    \hypersetup{
      breaklinks=true,  % so long urls are correctly broken across lines
      colorlinks=true,
      urlcolor=urlcolor,
      linkcolor=linkcolor,
      citecolor=citecolor,
      }
    % Slightly bigger margins than the latex defaults
    
    \geometry{verbose,tmargin=1in,bmargin=1in,lmargin=1in,rmargin=1in}
    
    

\begin{document}
    
    \maketitle
    
    

    
    Berdasarkan isu
\href{https://github.com/taruma/hidrokit/issues/88}{\#88}:
\textbf{request: ambil dataset hujan harian}

Referensi isu: - \texttt{hidrokit.contrib.taruma.hk79}
\href{https://github.com/taruma/hidrokit/issues/79}{\#79}.
(\href{https://nbviewer.jupyter.org/gist/taruma/05dab67fac8313a94134ac02d0398897}{lihat
notebook/manual}). \textbf{request: ambil dataset hujan jam-jaman dari
excel}

Deskripsi permasalahan: - Serupa dengan isu \#79, akan tetapi dataset
merupakan data harian. - Mengambil dataset harian dalam excel yang
berupa \emph{pivot table}. - Mengubah tabel tersebut ke dalam bentuk
\texttt{pandas.DataFrame}, dengan baris menunjukkan observasi/kejadian
dan kolom menunjukkan stasiun.

Change Log: - \textbf{\emph{New in version 0.4.0.}}
\href{https://github.com/hidrokit/hidrokit/issues/162}{Isu \#162}
Menambah fitur fungsi \texttt{read\_workbook(...)} untuk membaca berkas
tanpa perlu mengetahui nama \emph{sheet} dan membaca seluruh sheet
kecuali yang \emph{sheet} berawalan
\texttt{ignore\_str=\textquotesingle{}\_\textquotesingle{}}. -
\textbf{\emph{Breaking changes in version 0.4.0}} Bagi yang menggunakan
luaran dengan argumen \texttt{as\_df=False} akan memperoleh pesan error
karena luaran sebelum versi 0.4.0 berupa \texttt{list}, sedangkan untuk
versi 0.4.0 luaran berupa \texttt{dictionary}.

    \hypertarget{persiapan-dan-dataset}{%
\section{PERSIAPAN DAN DATASET}\label{persiapan-dan-dataset}}

    \begin{tcolorbox}[breakable, size=fbox, boxrule=1pt, pad at break*=1mm,colback=cellbackground, colframe=cellborder]
\prompt{In}{incolor}{ }{\boxspacing}
\begin{Verbatim}[commandchars=\\\{\}]
\PY{k}{try}\PY{p}{:}
    \PY{k+kn}{import} \PY{n+nn}{hidrokit}
\PY{k}{except} \PY{n+ne}{ModuleNotFoundError}\PY{p}{:}
    \PY{o}{!}pip install hidrokit \PYZhy{}q
    \PY{k+kn}{import} \PY{n+nn}{hidrokit}
\PY{n+nb}{print}\PY{p}{(}\PY{l+s+sa}{f}\PY{l+s+s1}{\PYZsq{}}\PY{l+s+s1}{hidrokit version: }\PY{l+s+si}{\PYZob{}}\PY{n}{hidrokit}\PY{o}{.}\PY{n}{\PYZus{}\PYZus{}version\PYZus{}\PYZus{}}\PY{l+s+si}{\PYZcb{}}\PY{l+s+s1}{\PYZsq{}}\PY{p}{)}
\end{Verbatim}
\end{tcolorbox}

    \begin{Verbatim}[commandchars=\\\{\}]
hidrokit version: 0.3.6
    \end{Verbatim}

    \begin{tcolorbox}[breakable, size=fbox, boxrule=1pt, pad at break*=1mm,colback=cellbackground, colframe=cellborder]
\prompt{In}{incolor}{ }{\boxspacing}
\begin{Verbatim}[commandchars=\\\{\}]
\PY{c+c1}{\PYZsh{} Unduh dataset}
\PY{o}{!}wget \PYZhy{}O sample.xlsx \PY{l+s+s2}{\PYZdq{}https://taruma.github.io/assets/hidrokit\PYZus{}dataset/hidrokit\PYZus{}daily\PYZus{}template.xlsx\PYZdq{}} \PYZhy{}q
\PY{n}{FILE} \PY{o}{=} \PY{l+s+s1}{\PYZsq{}}\PY{l+s+s1}{sample.xlsx}\PY{l+s+s1}{\PYZsq{}}
\end{Verbatim}
\end{tcolorbox}

    \begin{tcolorbox}[breakable, size=fbox, boxrule=1pt, pad at break*=1mm,colback=cellbackground, colframe=cellborder]
\prompt{In}{incolor}{ }{\boxspacing}
\begin{Verbatim}[commandchars=\\\{\}]
\PY{k+kn}{import} \PY{n+nn}{pandas} \PY{k}{as} \PY{n+nn}{pd}
\PY{k+kn}{import} \PY{n+nn}{numpy} \PY{k}{as} \PY{n+nn}{np}
\PY{k+kn}{import} \PY{n+nn}{matplotlib}\PY{n+nn}{.}\PY{n+nn}{pyplot} \PY{k}{as} \PY{n+nn}{plt}
\end{Verbatim}
\end{tcolorbox}

    \hypertarget{kode}{%
\section{KODE}\label{kode}}

    \begin{tcolorbox}[breakable, size=fbox, boxrule=1pt, pad at break*=1mm,colback=cellbackground, colframe=cellborder]
\prompt{In}{incolor}{ }{\boxspacing}
\begin{Verbatim}[commandchars=\\\{\}]
\PY{k+kn}{from} \PY{n+nn}{calendar} \PY{k+kn}{import} \PY{n}{isleap}
\PY{k+kn}{import} \PY{n+nn}{pandas} \PY{k}{as} \PY{n+nn}{pd}


\PY{k}{def} \PY{n+nf}{\PYZus{}melt\PYZus{}to\PYZus{}array}\PY{p}{(}\PY{n}{df}\PY{p}{,} \PY{n}{year}\PY{p}{)}\PY{p}{:}
    \PY{l+s+sd}{\PYZdq{}\PYZdq{}\PYZdq{}Melt dataframe to 1D array one year\PYZdq{}\PYZdq{}\PYZdq{}}
    \PY{c+c1}{\PYZsh{} ref: hidrokit.contrib.taruma.hk43}
    \PY{n}{\PYZus{}drop} \PY{o}{=} \PY{p}{[}\PY{l+m+mi}{59}\PY{p}{,} \PY{l+m+mi}{60}\PY{p}{,} \PY{l+m+mi}{61}\PY{p}{,} \PY{l+m+mi}{123}\PY{p}{,} \PY{l+m+mi}{185}\PY{p}{,} \PY{l+m+mi}{278}\PY{p}{,} \PY{l+m+mi}{340}\PY{p}{]}
    \PY{n}{\PYZus{}drop\PYZus{}leap} \PY{o}{=} \PY{p}{[}\PY{l+m+mi}{60}\PY{p}{,} \PY{l+m+mi}{61}\PY{p}{,} \PY{l+m+mi}{123}\PY{p}{,} \PY{l+m+mi}{185}\PY{p}{,} \PY{l+m+mi}{278}\PY{p}{,} \PY{l+m+mi}{340}\PY{p}{]}

    \PY{n}{data} \PY{o}{=} \PY{n}{df}\PY{o}{.}\PY{n}{melt}\PY{p}{(}\PY{p}{)}\PY{o}{.}\PY{n}{drop}\PY{p}{(}\PY{l+s+s1}{\PYZsq{}}\PY{l+s+s1}{variable}\PY{l+s+s1}{\PYZsq{}}\PY{p}{,} \PY{n}{axis}\PY{o}{=}\PY{l+m+mi}{1}\PY{p}{)}
    \PY{k}{if} \PY{n}{isleap}\PY{p}{(}\PY{n}{year}\PY{p}{)}\PY{p}{:}
        \PY{k}{return} \PY{n}{data}\PY{p}{[}\PY{l+s+s1}{\PYZsq{}}\PY{l+s+s1}{value}\PY{l+s+s1}{\PYZsq{}}\PY{p}{]}\PY{o}{.}\PY{n}{drop}\PY{p}{(}\PY{n}{\PYZus{}drop\PYZus{}leap}\PY{p}{)}\PY{o}{.}\PY{n}{values}
    \PY{k}{else}\PY{p}{:}
        \PY{k}{return} \PY{n}{data}\PY{p}{[}\PY{l+s+s1}{\PYZsq{}}\PY{l+s+s1}{value}\PY{l+s+s1}{\PYZsq{}}\PY{p}{]}\PY{o}{.}\PY{n}{drop}\PY{p}{(}\PY{n}{\PYZus{}drop}\PY{p}{)}\PY{o}{.}\PY{n}{values}


\PY{k}{def} \PY{n+nf}{\PYZus{}index\PYZus{}daily}\PY{p}{(}\PY{n}{year}\PY{p}{)}\PY{p}{:}
    \PY{l+s+sd}{\PYZdq{}\PYZdq{}\PYZdq{}Return DateTimeIndex object for one year\PYZdq{}\PYZdq{}\PYZdq{}}
    \PY{n}{year\PYZus{}range} \PY{o}{=} \PY{l+s+s1}{\PYZsq{}}\PY{l+s+si}{\PYZob{}\PYZcb{}}\PY{l+s+s1}{0101 }\PY{l+s+si}{\PYZob{}\PYZcb{}}\PY{l+s+s1}{0101}\PY{l+s+s1}{\PYZsq{}}\PY{o}{.}\PY{n}{format}\PY{p}{(}\PY{n}{year}\PY{p}{,} \PY{n}{year} \PY{o}{+} \PY{l+m+mi}{1}\PY{p}{)}\PY{o}{.}\PY{n}{split}\PY{p}{(}\PY{p}{)}
    \PY{k}{return} \PY{n}{pd}\PY{o}{.}\PY{n}{date\PYZus{}range}\PY{p}{(}\PY{o}{*}\PY{n}{year\PYZus{}range}\PY{p}{,} \PY{n}{closed}\PY{o}{=}\PY{l+s+s1}{\PYZsq{}}\PY{l+s+s1}{left}\PY{l+s+s1}{\PYZsq{}}\PY{p}{)}


\PY{k}{def} \PY{n+nf}{\PYZus{}yearly\PYZus{}df}\PY{p}{(}\PY{n}{df}\PY{p}{,} \PY{n}{year}\PY{p}{,} \PY{n}{station\PYZus{}name}\PY{p}{)}\PY{p}{:}
    \PY{l+s+sd}{\PYZdq{}\PYZdq{}\PYZdq{}Create dataframe for one year\PYZdq{}\PYZdq{}\PYZdq{}}
    \PY{k}{return} \PY{n}{pd}\PY{o}{.}\PY{n}{DataFrame}\PY{p}{(}
        \PY{n}{data}\PY{o}{=}\PY{n}{\PYZus{}melt\PYZus{}to\PYZus{}array}\PY{p}{(}\PY{n}{df}\PY{p}{,} \PY{n}{year}\PY{p}{)}\PY{p}{,}
        \PY{n}{index}\PY{o}{=}\PY{n}{\PYZus{}index\PYZus{}daily}\PY{p}{(}\PY{n}{year}\PY{p}{)}\PY{p}{,}
        \PY{n}{columns}\PY{o}{=}\PY{p}{[}\PY{n}{station\PYZus{}name}\PY{p}{]}
    \PY{p}{)}


\PY{k}{def} \PY{n+nf}{\PYZus{}data\PYZus{}from\PYZus{}sheet}\PY{p}{(}\PY{n}{df}\PY{p}{,} \PY{n}{station\PYZus{}name}\PY{p}{,} \PY{n}{as\PYZus{}df}\PY{o}{=}\PY{k+kc}{True}\PY{p}{)}\PY{p}{:}
    \PY{l+s+sd}{\PYZdq{}\PYZdq{}\PYZdq{}Read dataset from single sheet as dataframe (or list of dataframe)\PYZdq{}\PYZdq{}\PYZdq{}}
    \PY{n}{n\PYZus{}years} \PY{o}{=} \PY{n+nb}{int}\PY{p}{(}\PY{n}{df}\PY{o}{.}\PY{n}{iloc}\PY{p}{[}\PY{l+m+mi}{0}\PY{p}{,} \PY{l+m+mi}{1}\PY{p}{]}\PY{p}{)}

    \PY{n}{frames} \PY{o}{=} \PY{p}{[}\PY{p}{]}
    \PY{k}{for} \PY{n}{i} \PY{o+ow}{in} \PY{n+nb}{range}\PY{p}{(}\PY{l+m+mi}{2}\PY{p}{,} \PY{n}{n\PYZus{}years} \PY{o}{*} \PY{l+m+mi}{33}\PY{p}{,} \PY{l+m+mi}{33}\PY{p}{)}\PY{p}{:}
        \PY{n}{year} \PY{o}{=} \PY{n+nb}{int}\PY{p}{(}\PY{n}{df}\PY{o}{.}\PY{n}{iloc}\PY{p}{[}\PY{n}{i}\PY{p}{,} \PY{l+m+mi}{1}\PY{p}{]}\PY{p}{)}
        \PY{n}{pivot} \PY{o}{=} \PY{n}{df}\PY{o}{.}\PY{n}{iloc}\PY{p}{[}\PY{n}{i}\PY{p}{:}\PY{n}{i} \PY{o}{+} \PY{l+m+mi}{31}\PY{p}{,} \PY{l+m+mi}{4}\PY{p}{:}\PY{l+m+mi}{16}\PY{p}{]}
        \PY{n}{data} \PY{o}{=} \PY{n}{\PYZus{}yearly\PYZus{}df}\PY{p}{(}\PY{n}{pivot}\PY{p}{,} \PY{n}{year}\PY{p}{,} \PY{n}{station\PYZus{}name}\PY{p}{)}
        \PY{n}{frames}\PY{o}{.}\PY{n}{append}\PY{p}{(}\PY{n}{data}\PY{p}{)}

    \PY{k}{if} \PY{n}{as\PYZus{}df}\PY{p}{:}
        \PY{k}{return} \PY{n}{pd}\PY{o}{.}\PY{n}{concat}\PY{p}{(}\PY{n}{frames}\PY{p}{,} \PY{n}{sort}\PY{o}{=}\PY{k+kc}{True}\PY{p}{)}
    \PY{k}{else}\PY{p}{:}
        \PY{k}{return} \PY{n}{frames}

\PY{k}{def} \PY{n+nf}{read\PYZus{}workbook}\PY{p}{(}\PY{n}{io}\PY{p}{,} \PY{n}{stations}\PY{o}{=}\PY{k+kc}{None}\PY{p}{,} \PY{n}{ignore\PYZus{}str}\PY{o}{=}\PY{l+s+s1}{\PYZsq{}}\PY{l+s+s1}{\PYZus{}}\PY{l+s+s1}{\PYZsq{}}\PY{p}{,} \PY{n}{as\PYZus{}df}\PY{o}{=}\PY{k+kc}{True}\PY{p}{)}\PY{p}{:}
    \PY{l+s+sd}{\PYZdq{}\PYZdq{}\PYZdq{}Read dataset from workbook\PYZdq{}\PYZdq{}\PYZdq{}}
    \PY{n}{excel} \PY{o}{=} \PY{n}{pd}\PY{o}{.}\PY{n}{ExcelFile}\PY{p}{(}\PY{n}{io}\PY{p}{)}

    \PY{n}{data} \PY{o}{=} \PY{p}{\PYZob{}}\PY{p}{\PYZcb{}}
    \PY{n}{sheet\PYZus{}names} \PY{o}{=} \PY{n}{excel}\PY{o}{.}\PY{n}{sheet\PYZus{}names}
    \PY{k}{if} \PY{n}{stations} \PY{o+ow}{is} \PY{k+kc}{None}\PY{p}{:}
        \PY{n}{stations} \PY{o}{=} \PY{p}{[}\PY{p}{]}
        \PY{k}{for} \PY{n}{sheet} \PY{o+ow}{in} \PY{n}{sheet\PYZus{}names}\PY{p}{:}
            \PY{k}{if} \PY{o+ow}{not} \PY{n}{sheet}\PY{o}{.}\PY{n}{startswith}\PY{p}{(}\PY{n}{ignore\PYZus{}str}\PY{p}{)}\PY{p}{:}
                \PY{n}{stations}\PY{o}{.}\PY{n}{append}\PY{p}{(}\PY{n}{sheet}\PY{p}{)}
    \PY{k}{else}\PY{p}{:}
        \PY{n}{stations} \PY{o}{=} \PY{p}{[}\PY{n}{stations}\PY{p}{]} \PY{k}{if} \PY{n+nb}{isinstance}\PY{p}{(}\PY{n}{stations}\PY{p}{,} \PY{n+nb}{str}\PY{p}{)} \PY{k}{else} \PY{n}{stations}

    \PY{k}{for} \PY{n}{station} \PY{o+ow}{in} \PY{n}{stations}\PY{p}{:}
        \PY{n}{df} \PY{o}{=} \PY{n}{excel}\PY{o}{.}\PY{n}{parse}\PY{p}{(}\PY{n}{sheet\PYZus{}name}\PY{o}{=}\PY{n}{station}\PY{p}{,} \PY{n}{header}\PY{o}{=}\PY{k+kc}{None}\PY{p}{)}
        \PY{n}{data}\PY{p}{[}\PY{n}{station}\PY{p}{]} \PY{o}{=} \PY{n}{\PYZus{}data\PYZus{}from\PYZus{}sheet}\PY{p}{(}\PY{n}{df}\PY{p}{,} \PY{n}{station}\PY{p}{)}

    \PY{k}{if} \PY{n}{as\PYZus{}df}\PY{p}{:}
        \PY{k}{return} \PY{n}{pd}\PY{o}{.}\PY{n}{concat}\PY{p}{(}\PY{n}{data}\PY{o}{.}\PY{n}{values}\PY{p}{(}\PY{p}{)}\PY{p}{,} \PY{n}{sort}\PY{o}{=}\PY{k+kc}{True}\PY{p}{,} \PY{n}{axis}\PY{o}{=}\PY{l+m+mi}{1}\PY{p}{)}
    \PY{k}{else}\PY{p}{:}
        \PY{k}{return} \PY{n}{data}
\end{Verbatim}
\end{tcolorbox}

    \hypertarget{penerapan}{%
\section{PENERAPAN}\label{penerapan}}

    \begin{tcolorbox}[breakable, size=fbox, boxrule=1pt, pad at break*=1mm,colback=cellbackground, colframe=cellborder]
\prompt{In}{incolor}{ }{\boxspacing}
\begin{Verbatim}[commandchars=\\\{\}]
\PY{k+kn}{from} \PY{n+nn}{hidrokit}\PY{n+nn}{.}\PY{n+nn}{contrib}\PY{n+nn}{.}\PY{n+nn}{taruma} \PY{k+kn}{import} \PY{n}{hk79}

\PY{c+c1}{\PYZsh{} Ambil informasi excel menggunakan modul .hk79}
\PY{n}{data\PYZus{}info} \PY{o}{=} \PY{n}{hk79}\PY{o}{.}\PY{n}{\PYZus{}get\PYZus{}info}\PY{p}{(}\PY{n}{FILE}\PY{p}{,} \PY{n}{config\PYZus{}sheet}\PY{o}{=}\PY{l+s+s1}{\PYZsq{}}\PY{l+s+s1}{\PYZus{}INFO}\PY{l+s+s1}{\PYZsq{}}\PY{p}{)}
\PY{n+nb}{print}\PY{p}{(}\PY{l+s+s1}{\PYZsq{}}\PY{l+s+s1}{:: INFORMASI PADA BERKAS}\PY{l+s+s1}{\PYZsq{}}\PY{p}{)}
\PY{n+nb}{print}\PY{p}{(}\PY{n}{data\PYZus{}info}\PY{p}{)}
\end{Verbatim}
\end{tcolorbox}

    \begin{Verbatim}[commandchars=\\\{\}]
:: INFORMASI PADA BERKAS
\{'key': 'VALUE', 'n\_stations': 2, 'stations': 'AURENE, TYBALT', 'source': 'RATA
SUM', 'station\_1\_years': '2002, 2003, 2004', 'station\_2\_years': '2007, 2008'\}
    \end{Verbatim}

    \begin{tcolorbox}[breakable, size=fbox, boxrule=1pt, pad at break*=1mm,colback=cellbackground, colframe=cellborder]
\prompt{In}{incolor}{ }{\boxspacing}
\begin{Verbatim}[commandchars=\\\{\}]
\PY{n}{stations} \PY{o}{=} \PY{n}{data\PYZus{}info}\PY{p}{[}\PY{l+s+s1}{\PYZsq{}}\PY{l+s+s1}{stations}\PY{l+s+s1}{\PYZsq{}}\PY{p}{]}\PY{o}{.}\PY{n}{replace}\PY{p}{(}\PY{l+s+s1}{\PYZsq{}}\PY{l+s+s1}{ }\PY{l+s+s1}{\PYZsq{}}\PY{p}{,} \PY{l+s+s1}{\PYZsq{}}\PY{l+s+s1}{\PYZsq{}}\PY{p}{)}\PY{o}{.}\PY{n}{split}\PY{p}{(}\PY{l+s+s1}{\PYZsq{}}\PY{l+s+s1}{,}\PY{l+s+s1}{\PYZsq{}}\PY{p}{)}
\PY{n+nb}{print}\PY{p}{(}\PY{l+s+s1}{\PYZsq{}}\PY{l+s+s1}{nama stasiun dalam berkas:}\PY{l+s+s1}{\PYZsq{}}\PY{p}{,} \PY{n}{stations}\PY{p}{)}
\end{Verbatim}
\end{tcolorbox}

    \begin{Verbatim}[commandchars=\\\{\}]
nama stasiun dalam berkas: ['AURENE', 'TYBALT']
    \end{Verbatim}

    \hypertarget{baca-satu-stasiun}{%
\subsection{Baca satu stasiun}\label{baca-satu-stasiun}}

    \begin{tcolorbox}[breakable, size=fbox, boxrule=1pt, pad at break*=1mm,colback=cellbackground, colframe=cellborder]
\prompt{In}{incolor}{ }{\boxspacing}
\begin{Verbatim}[commandchars=\\\{\}]
\PY{n}{aurene} \PY{o}{=} \PY{n}{read\PYZus{}workbook}\PY{p}{(}\PY{n}{FILE}\PY{p}{,} \PY{p}{[}\PY{l+s+s1}{\PYZsq{}}\PY{l+s+s1}{AURENE}\PY{l+s+s1}{\PYZsq{}}\PY{p}{]}\PY{p}{)}
\PY{n}{aurene}\PY{o}{.}\PY{n}{info}\PY{p}{(}\PY{p}{)}
\PY{n}{aurene}\PY{o}{.}\PY{n}{head}\PY{p}{(}\PY{p}{)}
\end{Verbatim}
\end{tcolorbox}

    \begin{Verbatim}[commandchars=\\\{\}]
<class 'pandas.core.frame.DataFrame'>
DatetimeIndex: 1096 entries, 2002-01-01 to 2004-12-31
Freq: D
Data columns (total 1 columns):
 \#   Column  Non-Null Count  Dtype
---  ------  --------------  -----
 0   AURENE  1096 non-null   object
dtypes: object(1)
memory usage: 17.1+ KB
    \end{Verbatim}

            \begin{tcolorbox}[breakable, size=fbox, boxrule=.5pt, pad at break*=1mm, opacityfill=0]
\prompt{Out}{outcolor}{ }{\boxspacing}
\begin{Verbatim}[commandchars=\\\{\}]
           AURENE
2002-01-01  17.08
2002-01-02  16.28
2002-01-03  20.32
2002-01-04  18.34
2002-01-05  13.16
\end{Verbatim}
\end{tcolorbox}
        
    \hypertarget{baca-lebih-dari-satu-stasiun}{%
\subsection{Baca lebih dari satu
stasiun}\label{baca-lebih-dari-satu-stasiun}}

    \begin{tcolorbox}[breakable, size=fbox, boxrule=1pt, pad at break*=1mm,colback=cellbackground, colframe=cellborder]
\prompt{In}{incolor}{ }{\boxspacing}
\begin{Verbatim}[commandchars=\\\{\}]
\PY{n}{dataset} \PY{o}{=} \PY{n}{read\PYZus{}workbook}\PY{p}{(}\PY{n}{FILE}\PY{p}{,} \PY{p}{[}\PY{l+s+s1}{\PYZsq{}}\PY{l+s+s1}{TYBALT}\PY{l+s+s1}{\PYZsq{}}\PY{p}{,} \PY{l+s+s1}{\PYZsq{}}\PY{l+s+s1}{AURENE}\PY{l+s+s1}{\PYZsq{}}\PY{p}{]}\PY{p}{)}
\PY{n}{dataset}\PY{o}{.}\PY{n}{sort\PYZus{}index}\PY{p}{(}\PY{p}{)}
\end{Verbatim}
\end{tcolorbox}

            \begin{tcolorbox}[breakable, size=fbox, boxrule=.5pt, pad at break*=1mm, opacityfill=0]
\prompt{Out}{outcolor}{ }{\boxspacing}
\begin{Verbatim}[commandchars=\\\{\}]
            TYBALT AURENE
2002-01-01     NaN  17.08
2002-01-02     NaN  16.28
2002-01-03     NaN  20.32
2002-01-04     NaN  18.34
2002-01-05     NaN  13.16
{\ldots}            {\ldots}    {\ldots}
2008-12-27  134.83    NaN
2008-12-28   81.88    NaN
2008-12-29   20.14    NaN
2008-12-30  208.54    NaN
2008-12-31  208.14    NaN

[1827 rows x 2 columns]
\end{Verbatim}
\end{tcolorbox}
        
    \hypertarget{fungsi-read_workbookio-...}{%
\subsection{\texorpdfstring{FUNGSI
\texttt{read\_workbook(io,\ ...)}}{FUNGSI read\_workbook(io, ...)}}\label{fungsi-read_workbookio-...}}

Function:
\texttt{read\_workbook(io,\ stations=None,\ ignore\_str=\textquotesingle{}\_\textquotesingle{},\ as\_df=True)}

\begin{itemize}
\tightlist
\item
  Argumen Posisi:

  \begin{itemize}
  \tightlist
  \item
    \texttt{io}: \emph{Path} untuk berkas excel. Bisa berupa `string',
    objek \texttt{pandas.ExcelFile}, atau objek \texttt{pathlib.Path}.
  \end{itemize}
\item
  Argumen Opsional:

  \begin{itemize}
  \tightlist
  \item
    \texttt{stations}: nama stasiun/\emph{sheet}. \texttt{None}
    (default). Akan membaca seluruh sheet kecuali yang diawali
    \texttt{ignore\_str} jika \texttt{None}. Dapat berupa
    \texttt{string}, atau \texttt{list}/\texttt{tuple}.
    \textbf{\emph{Change in version 0.4.0}}
  \item
    \texttt{ignore\_str}: teks yang diawali \texttt{ignore\_str} akan
    diabaikan saat proses membaca buku.
    \texttt{\textquotesingle{}\_\textquotesingle{}} (default).
    \textbf{\emph{New in version 0.4.0}}
  \item
    \texttt{as\_df}: luaran dapet berupa \texttt{pandas.DataFrame} jika
    \texttt{True}. \texttt{True} (default). Jika \texttt{False}, luaran
    berupa \texttt{dictionary}. \textbf{\emph{Change in version 0.4.0}}
  \end{itemize}
\end{itemize}

    \begin{tcolorbox}[breakable, size=fbox, boxrule=1pt, pad at break*=1mm,colback=cellbackground, colframe=cellborder]
\prompt{In}{incolor}{ }{\boxspacing}
\begin{Verbatim}[commandchars=\\\{\}]
\PY{n}{read\PYZus{}workbook}\PY{p}{(}\PY{n}{FILE}\PY{p}{)}
\end{Verbatim}
\end{tcolorbox}

            \begin{tcolorbox}[breakable, size=fbox, boxrule=.5pt, pad at break*=1mm, opacityfill=0]
\prompt{Out}{outcolor}{ }{\boxspacing}
\begin{Verbatim}[commandchars=\\\{\}]
           AURENE  TYBALT
2002-01-01  17.08     NaN
2002-01-02  16.28     NaN
2002-01-03  20.32     NaN
2002-01-04  18.34     NaN
2002-01-05  13.16     NaN
{\ldots}           {\ldots}     {\ldots}
2008-12-27    NaN  134.83
2008-12-28    NaN   81.88
2008-12-29    NaN   20.14
2008-12-30    NaN  208.54
2008-12-31    NaN  208.14

[1827 rows x 2 columns]
\end{Verbatim}
\end{tcolorbox}
        
    \begin{tcolorbox}[breakable, size=fbox, boxrule=1pt, pad at break*=1mm,colback=cellbackground, colframe=cellborder]
\prompt{In}{incolor}{ }{\boxspacing}
\begin{Verbatim}[commandchars=\\\{\}]
\PY{n}{read\PYZus{}workbook}\PY{p}{(}\PY{n}{FILE}\PY{p}{,} \PY{n}{as\PYZus{}df}\PY{o}{=}\PY{k+kc}{False}\PY{p}{)}
\end{Verbatim}
\end{tcolorbox}

            \begin{tcolorbox}[breakable, size=fbox, boxrule=.5pt, pad at break*=1mm, opacityfill=0]
\prompt{Out}{outcolor}{ }{\boxspacing}
\begin{Verbatim}[commandchars=\\\{\}]
\{'AURENE':             AURENE
 2002-01-01   17.08
 2002-01-02   16.28
 2002-01-03   20.32
 2002-01-04   18.34
 2002-01-05   13.16
 {\ldots}            {\ldots}
 2004-12-27  454.75
 2004-12-28   69.75
 2004-12-29  189.39
 2004-12-30  166.39
 2004-12-31  133.99

 [1096 rows x 1 columns], 'TYBALT':             TYBALT
 2007-01-01  201.92
 2007-01-02  255.38
 2007-01-03  130.84
 2007-01-04   17.96
 2007-01-05   19.12
 {\ldots}            {\ldots}
 2008-12-27  134.83
 2008-12-28   81.88
 2008-12-29   20.14
 2008-12-30  208.54
 2008-12-31  208.14

 [731 rows x 1 columns]\}
\end{Verbatim}
\end{tcolorbox}
        
    \hypertarget{changelog}{%
\section{Changelog}\label{changelog}}

\begin{verbatim}
- 20220402 - 1.1.1 - Change 0.3.7 to 0.4.0
- 20220318 - 1.1.0 - Improve read_workbook(...) function Issue #162
- 20191217 - 1.0.1 - Fix read_workbook() issue#95
- 20191213 - 1.0.0 - Initial
\end{verbatim}

\hypertarget{copyright-2022-taruma-sakti-megariansyah}{%
\paragraph{\texorpdfstring{Copyright © 2022
\href{https://taruma.github.io}{Taruma Sakti
Megariansyah}}{Copyright © 2022 Taruma Sakti Megariansyah}}\label{copyright-2022-taruma-sakti-megariansyah}}

Source code in this notebook is licensed under a
\href{https://choosealicense.com/licenses/mit/}{MIT License}. Data in
this notebook is licensed under a
\href{https://creativecommons.org/licenses/by/4.0/}{Creative Common
Attribution 4.0 International}.


    % Add a bibliography block to the postdoc
    
    
    
\end{document}
