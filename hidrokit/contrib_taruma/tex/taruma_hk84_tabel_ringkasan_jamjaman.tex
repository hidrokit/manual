\documentclass[11pt]{article}

    \usepackage[breakable]{tcolorbox}
    \usepackage{parskip} % Stop auto-indenting (to mimic markdown behaviour)
    
    \usepackage{iftex}
    \ifPDFTeX
    	\usepackage[T1]{fontenc}
    	\usepackage{mathpazo}
    \else
    	\usepackage{fontspec}
    \fi

    % Basic figure setup, for now with no caption control since it's done
    % automatically by Pandoc (which extracts ![](path) syntax from Markdown).
    \usepackage{graphicx}
    % Maintain compatibility with old templates. Remove in nbconvert 6.0
    \let\Oldincludegraphics\includegraphics
    % Ensure that by default, figures have no caption (until we provide a
    % proper Figure object with a Caption API and a way to capture that
    % in the conversion process - todo).
    \usepackage{caption}
    \DeclareCaptionFormat{nocaption}{}
    \captionsetup{format=nocaption,aboveskip=0pt,belowskip=0pt}

    \usepackage[Export]{adjustbox} % Used to constrain images to a maximum size
    \adjustboxset{max size={0.9\linewidth}{0.9\paperheight}}
    \usepackage{float}
    \floatplacement{figure}{H} % forces figures to be placed at the correct location
    \usepackage{xcolor} % Allow colors to be defined
    \usepackage{enumerate} % Needed for markdown enumerations to work
    \usepackage{geometry} % Used to adjust the document margins
    \usepackage{amsmath} % Equations
    \usepackage{amssymb} % Equations
    \usepackage{textcomp} % defines textquotesingle
    % Hack from http://tex.stackexchange.com/a/47451/13684:
    \AtBeginDocument{%
        \def\PYZsq{\textquotesingle}% Upright quotes in Pygmentized code
    }
    \usepackage{upquote} % Upright quotes for verbatim code
    \usepackage{eurosym} % defines \euro
    \usepackage[mathletters]{ucs} % Extended unicode (utf-8) support
    \usepackage{fancyvrb} % verbatim replacement that allows latex
    \usepackage{grffile} % extends the file name processing of package graphics 
                         % to support a larger range
    \makeatletter % fix for grffile with XeLaTeX
    \def\Gread@@xetex#1{%
      \IfFileExists{"\Gin@base".bb}%
      {\Gread@eps{\Gin@base.bb}}%
      {\Gread@@xetex@aux#1}%
    }
    \makeatother

    % The hyperref package gives us a pdf with properly built
    % internal navigation ('pdf bookmarks' for the table of contents,
    % internal cross-reference links, web links for URLs, etc.)
    \usepackage{hyperref}
    % The default LaTeX title has an obnoxious amount of whitespace. By default,
    % titling removes some of it. It also provides customization options.
    \usepackage{titling}
    \usepackage{longtable} % longtable support required by pandoc >1.10
    \usepackage{booktabs}  % table support for pandoc > 1.12.2
    \usepackage[inline]{enumitem} % IRkernel/repr support (it uses the enumerate* environment)
    \usepackage[normalem]{ulem} % ulem is needed to support strikethroughs (\sout)
                                % normalem makes italics be italics, not underlines
    \usepackage{mathrsfs}
    

    
    % Colors for the hyperref package
    \definecolor{urlcolor}{rgb}{0,.145,.698}
    \definecolor{linkcolor}{rgb}{.71,0.21,0.01}
    \definecolor{citecolor}{rgb}{.12,.54,.11}

    % ANSI colors
    \definecolor{ansi-black}{HTML}{3E424D}
    \definecolor{ansi-black-intense}{HTML}{282C36}
    \definecolor{ansi-red}{HTML}{E75C58}
    \definecolor{ansi-red-intense}{HTML}{B22B31}
    \definecolor{ansi-green}{HTML}{00A250}
    \definecolor{ansi-green-intense}{HTML}{007427}
    \definecolor{ansi-yellow}{HTML}{DDB62B}
    \definecolor{ansi-yellow-intense}{HTML}{B27D12}
    \definecolor{ansi-blue}{HTML}{208FFB}
    \definecolor{ansi-blue-intense}{HTML}{0065CA}
    \definecolor{ansi-magenta}{HTML}{D160C4}
    \definecolor{ansi-magenta-intense}{HTML}{A03196}
    \definecolor{ansi-cyan}{HTML}{60C6C8}
    \definecolor{ansi-cyan-intense}{HTML}{258F8F}
    \definecolor{ansi-white}{HTML}{C5C1B4}
    \definecolor{ansi-white-intense}{HTML}{A1A6B2}
    \definecolor{ansi-default-inverse-fg}{HTML}{FFFFFF}
    \definecolor{ansi-default-inverse-bg}{HTML}{000000}

    % commands and environments needed by pandoc snippets
    % extracted from the output of `pandoc -s`
    \providecommand{\tightlist}{%
      \setlength{\itemsep}{0pt}\setlength{\parskip}{0pt}}
    \DefineVerbatimEnvironment{Highlighting}{Verbatim}{commandchars=\\\{\}}
    % Add ',fontsize=\small' for more characters per line
    \newenvironment{Shaded}{}{}
    \newcommand{\KeywordTok}[1]{\textcolor[rgb]{0.00,0.44,0.13}{\textbf{{#1}}}}
    \newcommand{\DataTypeTok}[1]{\textcolor[rgb]{0.56,0.13,0.00}{{#1}}}
    \newcommand{\DecValTok}[1]{\textcolor[rgb]{0.25,0.63,0.44}{{#1}}}
    \newcommand{\BaseNTok}[1]{\textcolor[rgb]{0.25,0.63,0.44}{{#1}}}
    \newcommand{\FloatTok}[1]{\textcolor[rgb]{0.25,0.63,0.44}{{#1}}}
    \newcommand{\CharTok}[1]{\textcolor[rgb]{0.25,0.44,0.63}{{#1}}}
    \newcommand{\StringTok}[1]{\textcolor[rgb]{0.25,0.44,0.63}{{#1}}}
    \newcommand{\CommentTok}[1]{\textcolor[rgb]{0.38,0.63,0.69}{\textit{{#1}}}}
    \newcommand{\OtherTok}[1]{\textcolor[rgb]{0.00,0.44,0.13}{{#1}}}
    \newcommand{\AlertTok}[1]{\textcolor[rgb]{1.00,0.00,0.00}{\textbf{{#1}}}}
    \newcommand{\FunctionTok}[1]{\textcolor[rgb]{0.02,0.16,0.49}{{#1}}}
    \newcommand{\RegionMarkerTok}[1]{{#1}}
    \newcommand{\ErrorTok}[1]{\textcolor[rgb]{1.00,0.00,0.00}{\textbf{{#1}}}}
    \newcommand{\NormalTok}[1]{{#1}}
    
    % Additional commands for more recent versions of Pandoc
    \newcommand{\ConstantTok}[1]{\textcolor[rgb]{0.53,0.00,0.00}{{#1}}}
    \newcommand{\SpecialCharTok}[1]{\textcolor[rgb]{0.25,0.44,0.63}{{#1}}}
    \newcommand{\VerbatimStringTok}[1]{\textcolor[rgb]{0.25,0.44,0.63}{{#1}}}
    \newcommand{\SpecialStringTok}[1]{\textcolor[rgb]{0.73,0.40,0.53}{{#1}}}
    \newcommand{\ImportTok}[1]{{#1}}
    \newcommand{\DocumentationTok}[1]{\textcolor[rgb]{0.73,0.13,0.13}{\textit{{#1}}}}
    \newcommand{\AnnotationTok}[1]{\textcolor[rgb]{0.38,0.63,0.69}{\textbf{\textit{{#1}}}}}
    \newcommand{\CommentVarTok}[1]{\textcolor[rgb]{0.38,0.63,0.69}{\textbf{\textit{{#1}}}}}
    \newcommand{\VariableTok}[1]{\textcolor[rgb]{0.10,0.09,0.49}{{#1}}}
    \newcommand{\ControlFlowTok}[1]{\textcolor[rgb]{0.00,0.44,0.13}{\textbf{{#1}}}}
    \newcommand{\OperatorTok}[1]{\textcolor[rgb]{0.40,0.40,0.40}{{#1}}}
    \newcommand{\BuiltInTok}[1]{{#1}}
    \newcommand{\ExtensionTok}[1]{{#1}}
    \newcommand{\PreprocessorTok}[1]{\textcolor[rgb]{0.74,0.48,0.00}{{#1}}}
    \newcommand{\AttributeTok}[1]{\textcolor[rgb]{0.49,0.56,0.16}{{#1}}}
    \newcommand{\InformationTok}[1]{\textcolor[rgb]{0.38,0.63,0.69}{\textbf{\textit{{#1}}}}}
    \newcommand{\WarningTok}[1]{\textcolor[rgb]{0.38,0.63,0.69}{\textbf{\textit{{#1}}}}}
    
    
    % Define a nice break command that doesn't care if a line doesn't already
    % exist.
    \def\br{\hspace*{\fill} \\* }
    % Math Jax compatibility definitions
    \def\gt{>}
    \def\lt{<}
    \let\Oldtex\TeX
    \let\Oldlatex\LaTeX
    \renewcommand{\TeX}{\textrm{\Oldtex}}
    \renewcommand{\LaTeX}{\textrm{\Oldlatex}}
    % Document parameters
    % Document title
    \title{taruma\_hk84\_tabel\_ringkasan\_jamjaman}
    
    
    
    
    
% Pygments definitions
\makeatletter
\def\PY@reset{\let\PY@it=\relax \let\PY@bf=\relax%
    \let\PY@ul=\relax \let\PY@tc=\relax%
    \let\PY@bc=\relax \let\PY@ff=\relax}
\def\PY@tok#1{\csname PY@tok@#1\endcsname}
\def\PY@toks#1+{\ifx\relax#1\empty\else%
    \PY@tok{#1}\expandafter\PY@toks\fi}
\def\PY@do#1{\PY@bc{\PY@tc{\PY@ul{%
    \PY@it{\PY@bf{\PY@ff{#1}}}}}}}
\def\PY#1#2{\PY@reset\PY@toks#1+\relax+\PY@do{#2}}

\expandafter\def\csname PY@tok@w\endcsname{\def\PY@tc##1{\textcolor[rgb]{0.73,0.73,0.73}{##1}}}
\expandafter\def\csname PY@tok@c\endcsname{\let\PY@it=\textit\def\PY@tc##1{\textcolor[rgb]{0.25,0.50,0.50}{##1}}}
\expandafter\def\csname PY@tok@cp\endcsname{\def\PY@tc##1{\textcolor[rgb]{0.74,0.48,0.00}{##1}}}
\expandafter\def\csname PY@tok@k\endcsname{\let\PY@bf=\textbf\def\PY@tc##1{\textcolor[rgb]{0.00,0.50,0.00}{##1}}}
\expandafter\def\csname PY@tok@kp\endcsname{\def\PY@tc##1{\textcolor[rgb]{0.00,0.50,0.00}{##1}}}
\expandafter\def\csname PY@tok@kt\endcsname{\def\PY@tc##1{\textcolor[rgb]{0.69,0.00,0.25}{##1}}}
\expandafter\def\csname PY@tok@o\endcsname{\def\PY@tc##1{\textcolor[rgb]{0.40,0.40,0.40}{##1}}}
\expandafter\def\csname PY@tok@ow\endcsname{\let\PY@bf=\textbf\def\PY@tc##1{\textcolor[rgb]{0.67,0.13,1.00}{##1}}}
\expandafter\def\csname PY@tok@nb\endcsname{\def\PY@tc##1{\textcolor[rgb]{0.00,0.50,0.00}{##1}}}
\expandafter\def\csname PY@tok@nf\endcsname{\def\PY@tc##1{\textcolor[rgb]{0.00,0.00,1.00}{##1}}}
\expandafter\def\csname PY@tok@nc\endcsname{\let\PY@bf=\textbf\def\PY@tc##1{\textcolor[rgb]{0.00,0.00,1.00}{##1}}}
\expandafter\def\csname PY@tok@nn\endcsname{\let\PY@bf=\textbf\def\PY@tc##1{\textcolor[rgb]{0.00,0.00,1.00}{##1}}}
\expandafter\def\csname PY@tok@ne\endcsname{\let\PY@bf=\textbf\def\PY@tc##1{\textcolor[rgb]{0.82,0.25,0.23}{##1}}}
\expandafter\def\csname PY@tok@nv\endcsname{\def\PY@tc##1{\textcolor[rgb]{0.10,0.09,0.49}{##1}}}
\expandafter\def\csname PY@tok@no\endcsname{\def\PY@tc##1{\textcolor[rgb]{0.53,0.00,0.00}{##1}}}
\expandafter\def\csname PY@tok@nl\endcsname{\def\PY@tc##1{\textcolor[rgb]{0.63,0.63,0.00}{##1}}}
\expandafter\def\csname PY@tok@ni\endcsname{\let\PY@bf=\textbf\def\PY@tc##1{\textcolor[rgb]{0.60,0.60,0.60}{##1}}}
\expandafter\def\csname PY@tok@na\endcsname{\def\PY@tc##1{\textcolor[rgb]{0.49,0.56,0.16}{##1}}}
\expandafter\def\csname PY@tok@nt\endcsname{\let\PY@bf=\textbf\def\PY@tc##1{\textcolor[rgb]{0.00,0.50,0.00}{##1}}}
\expandafter\def\csname PY@tok@nd\endcsname{\def\PY@tc##1{\textcolor[rgb]{0.67,0.13,1.00}{##1}}}
\expandafter\def\csname PY@tok@s\endcsname{\def\PY@tc##1{\textcolor[rgb]{0.73,0.13,0.13}{##1}}}
\expandafter\def\csname PY@tok@sd\endcsname{\let\PY@it=\textit\def\PY@tc##1{\textcolor[rgb]{0.73,0.13,0.13}{##1}}}
\expandafter\def\csname PY@tok@si\endcsname{\let\PY@bf=\textbf\def\PY@tc##1{\textcolor[rgb]{0.73,0.40,0.53}{##1}}}
\expandafter\def\csname PY@tok@se\endcsname{\let\PY@bf=\textbf\def\PY@tc##1{\textcolor[rgb]{0.73,0.40,0.13}{##1}}}
\expandafter\def\csname PY@tok@sr\endcsname{\def\PY@tc##1{\textcolor[rgb]{0.73,0.40,0.53}{##1}}}
\expandafter\def\csname PY@tok@ss\endcsname{\def\PY@tc##1{\textcolor[rgb]{0.10,0.09,0.49}{##1}}}
\expandafter\def\csname PY@tok@sx\endcsname{\def\PY@tc##1{\textcolor[rgb]{0.00,0.50,0.00}{##1}}}
\expandafter\def\csname PY@tok@m\endcsname{\def\PY@tc##1{\textcolor[rgb]{0.40,0.40,0.40}{##1}}}
\expandafter\def\csname PY@tok@gh\endcsname{\let\PY@bf=\textbf\def\PY@tc##1{\textcolor[rgb]{0.00,0.00,0.50}{##1}}}
\expandafter\def\csname PY@tok@gu\endcsname{\let\PY@bf=\textbf\def\PY@tc##1{\textcolor[rgb]{0.50,0.00,0.50}{##1}}}
\expandafter\def\csname PY@tok@gd\endcsname{\def\PY@tc##1{\textcolor[rgb]{0.63,0.00,0.00}{##1}}}
\expandafter\def\csname PY@tok@gi\endcsname{\def\PY@tc##1{\textcolor[rgb]{0.00,0.63,0.00}{##1}}}
\expandafter\def\csname PY@tok@gr\endcsname{\def\PY@tc##1{\textcolor[rgb]{1.00,0.00,0.00}{##1}}}
\expandafter\def\csname PY@tok@ge\endcsname{\let\PY@it=\textit}
\expandafter\def\csname PY@tok@gs\endcsname{\let\PY@bf=\textbf}
\expandafter\def\csname PY@tok@gp\endcsname{\let\PY@bf=\textbf\def\PY@tc##1{\textcolor[rgb]{0.00,0.00,0.50}{##1}}}
\expandafter\def\csname PY@tok@go\endcsname{\def\PY@tc##1{\textcolor[rgb]{0.53,0.53,0.53}{##1}}}
\expandafter\def\csname PY@tok@gt\endcsname{\def\PY@tc##1{\textcolor[rgb]{0.00,0.27,0.87}{##1}}}
\expandafter\def\csname PY@tok@err\endcsname{\def\PY@bc##1{\setlength{\fboxsep}{0pt}\fcolorbox[rgb]{1.00,0.00,0.00}{1,1,1}{\strut ##1}}}
\expandafter\def\csname PY@tok@kc\endcsname{\let\PY@bf=\textbf\def\PY@tc##1{\textcolor[rgb]{0.00,0.50,0.00}{##1}}}
\expandafter\def\csname PY@tok@kd\endcsname{\let\PY@bf=\textbf\def\PY@tc##1{\textcolor[rgb]{0.00,0.50,0.00}{##1}}}
\expandafter\def\csname PY@tok@kn\endcsname{\let\PY@bf=\textbf\def\PY@tc##1{\textcolor[rgb]{0.00,0.50,0.00}{##1}}}
\expandafter\def\csname PY@tok@kr\endcsname{\let\PY@bf=\textbf\def\PY@tc##1{\textcolor[rgb]{0.00,0.50,0.00}{##1}}}
\expandafter\def\csname PY@tok@bp\endcsname{\def\PY@tc##1{\textcolor[rgb]{0.00,0.50,0.00}{##1}}}
\expandafter\def\csname PY@tok@fm\endcsname{\def\PY@tc##1{\textcolor[rgb]{0.00,0.00,1.00}{##1}}}
\expandafter\def\csname PY@tok@vc\endcsname{\def\PY@tc##1{\textcolor[rgb]{0.10,0.09,0.49}{##1}}}
\expandafter\def\csname PY@tok@vg\endcsname{\def\PY@tc##1{\textcolor[rgb]{0.10,0.09,0.49}{##1}}}
\expandafter\def\csname PY@tok@vi\endcsname{\def\PY@tc##1{\textcolor[rgb]{0.10,0.09,0.49}{##1}}}
\expandafter\def\csname PY@tok@vm\endcsname{\def\PY@tc##1{\textcolor[rgb]{0.10,0.09,0.49}{##1}}}
\expandafter\def\csname PY@tok@sa\endcsname{\def\PY@tc##1{\textcolor[rgb]{0.73,0.13,0.13}{##1}}}
\expandafter\def\csname PY@tok@sb\endcsname{\def\PY@tc##1{\textcolor[rgb]{0.73,0.13,0.13}{##1}}}
\expandafter\def\csname PY@tok@sc\endcsname{\def\PY@tc##1{\textcolor[rgb]{0.73,0.13,0.13}{##1}}}
\expandafter\def\csname PY@tok@dl\endcsname{\def\PY@tc##1{\textcolor[rgb]{0.73,0.13,0.13}{##1}}}
\expandafter\def\csname PY@tok@s2\endcsname{\def\PY@tc##1{\textcolor[rgb]{0.73,0.13,0.13}{##1}}}
\expandafter\def\csname PY@tok@sh\endcsname{\def\PY@tc##1{\textcolor[rgb]{0.73,0.13,0.13}{##1}}}
\expandafter\def\csname PY@tok@s1\endcsname{\def\PY@tc##1{\textcolor[rgb]{0.73,0.13,0.13}{##1}}}
\expandafter\def\csname PY@tok@mb\endcsname{\def\PY@tc##1{\textcolor[rgb]{0.40,0.40,0.40}{##1}}}
\expandafter\def\csname PY@tok@mf\endcsname{\def\PY@tc##1{\textcolor[rgb]{0.40,0.40,0.40}{##1}}}
\expandafter\def\csname PY@tok@mh\endcsname{\def\PY@tc##1{\textcolor[rgb]{0.40,0.40,0.40}{##1}}}
\expandafter\def\csname PY@tok@mi\endcsname{\def\PY@tc##1{\textcolor[rgb]{0.40,0.40,0.40}{##1}}}
\expandafter\def\csname PY@tok@il\endcsname{\def\PY@tc##1{\textcolor[rgb]{0.40,0.40,0.40}{##1}}}
\expandafter\def\csname PY@tok@mo\endcsname{\def\PY@tc##1{\textcolor[rgb]{0.40,0.40,0.40}{##1}}}
\expandafter\def\csname PY@tok@ch\endcsname{\let\PY@it=\textit\def\PY@tc##1{\textcolor[rgb]{0.25,0.50,0.50}{##1}}}
\expandafter\def\csname PY@tok@cm\endcsname{\let\PY@it=\textit\def\PY@tc##1{\textcolor[rgb]{0.25,0.50,0.50}{##1}}}
\expandafter\def\csname PY@tok@cpf\endcsname{\let\PY@it=\textit\def\PY@tc##1{\textcolor[rgb]{0.25,0.50,0.50}{##1}}}
\expandafter\def\csname PY@tok@c1\endcsname{\let\PY@it=\textit\def\PY@tc##1{\textcolor[rgb]{0.25,0.50,0.50}{##1}}}
\expandafter\def\csname PY@tok@cs\endcsname{\let\PY@it=\textit\def\PY@tc##1{\textcolor[rgb]{0.25,0.50,0.50}{##1}}}

\def\PYZbs{\char`\\}
\def\PYZus{\char`\_}
\def\PYZob{\char`\{}
\def\PYZcb{\char`\}}
\def\PYZca{\char`\^}
\def\PYZam{\char`\&}
\def\PYZlt{\char`\<}
\def\PYZgt{\char`\>}
\def\PYZsh{\char`\#}
\def\PYZpc{\char`\%}
\def\PYZdl{\char`\$}
\def\PYZhy{\char`\-}
\def\PYZsq{\char`\'}
\def\PYZdq{\char`\"}
\def\PYZti{\char`\~}
% for compatibility with earlier versions
\def\PYZat{@}
\def\PYZlb{[}
\def\PYZrb{]}
\makeatother


    % For linebreaks inside Verbatim environment from package fancyvrb. 
    \makeatletter
        \newbox\Wrappedcontinuationbox 
        \newbox\Wrappedvisiblespacebox 
        \newcommand*\Wrappedvisiblespace {\textcolor{red}{\textvisiblespace}} 
        \newcommand*\Wrappedcontinuationsymbol {\textcolor{red}{\llap{\tiny$\m@th\hookrightarrow$}}} 
        \newcommand*\Wrappedcontinuationindent {3ex } 
        \newcommand*\Wrappedafterbreak {\kern\Wrappedcontinuationindent\copy\Wrappedcontinuationbox} 
        % Take advantage of the already applied Pygments mark-up to insert 
        % potential linebreaks for TeX processing. 
        %        {, <, #, %, $, ' and ": go to next line. 
        %        _, }, ^, &, >, - and ~: stay at end of broken line. 
        % Use of \textquotesingle for straight quote. 
        \newcommand*\Wrappedbreaksatspecials {% 
            \def\PYGZus{\discretionary{\char`\_}{\Wrappedafterbreak}{\char`\_}}% 
            \def\PYGZob{\discretionary{}{\Wrappedafterbreak\char`\{}{\char`\{}}% 
            \def\PYGZcb{\discretionary{\char`\}}{\Wrappedafterbreak}{\char`\}}}% 
            \def\PYGZca{\discretionary{\char`\^}{\Wrappedafterbreak}{\char`\^}}% 
            \def\PYGZam{\discretionary{\char`\&}{\Wrappedafterbreak}{\char`\&}}% 
            \def\PYGZlt{\discretionary{}{\Wrappedafterbreak\char`\<}{\char`\<}}% 
            \def\PYGZgt{\discretionary{\char`\>}{\Wrappedafterbreak}{\char`\>}}% 
            \def\PYGZsh{\discretionary{}{\Wrappedafterbreak\char`\#}{\char`\#}}% 
            \def\PYGZpc{\discretionary{}{\Wrappedafterbreak\char`\%}{\char`\%}}% 
            \def\PYGZdl{\discretionary{}{\Wrappedafterbreak\char`\$}{\char`\$}}% 
            \def\PYGZhy{\discretionary{\char`\-}{\Wrappedafterbreak}{\char`\-}}% 
            \def\PYGZsq{\discretionary{}{\Wrappedafterbreak\textquotesingle}{\textquotesingle}}% 
            \def\PYGZdq{\discretionary{}{\Wrappedafterbreak\char`\"}{\char`\"}}% 
            \def\PYGZti{\discretionary{\char`\~}{\Wrappedafterbreak}{\char`\~}}% 
        } 
        % Some characters . , ; ? ! / are not pygmentized. 
        % This macro makes them "active" and they will insert potential linebreaks 
        \newcommand*\Wrappedbreaksatpunct {% 
            \lccode`\~`\.\lowercase{\def~}{\discretionary{\hbox{\char`\.}}{\Wrappedafterbreak}{\hbox{\char`\.}}}% 
            \lccode`\~`\,\lowercase{\def~}{\discretionary{\hbox{\char`\,}}{\Wrappedafterbreak}{\hbox{\char`\,}}}% 
            \lccode`\~`\;\lowercase{\def~}{\discretionary{\hbox{\char`\;}}{\Wrappedafterbreak}{\hbox{\char`\;}}}% 
            \lccode`\~`\:\lowercase{\def~}{\discretionary{\hbox{\char`\:}}{\Wrappedafterbreak}{\hbox{\char`\:}}}% 
            \lccode`\~`\?\lowercase{\def~}{\discretionary{\hbox{\char`\?}}{\Wrappedafterbreak}{\hbox{\char`\?}}}% 
            \lccode`\~`\!\lowercase{\def~}{\discretionary{\hbox{\char`\!}}{\Wrappedafterbreak}{\hbox{\char`\!}}}% 
            \lccode`\~`\/\lowercase{\def~}{\discretionary{\hbox{\char`\/}}{\Wrappedafterbreak}{\hbox{\char`\/}}}% 
            \catcode`\.\active
            \catcode`\,\active 
            \catcode`\;\active
            \catcode`\:\active
            \catcode`\?\active
            \catcode`\!\active
            \catcode`\/\active 
            \lccode`\~`\~ 	
        }
    \makeatother

    \let\OriginalVerbatim=\Verbatim
    \makeatletter
    \renewcommand{\Verbatim}[1][1]{%
        %\parskip\z@skip
        \sbox\Wrappedcontinuationbox {\Wrappedcontinuationsymbol}%
        \sbox\Wrappedvisiblespacebox {\FV@SetupFont\Wrappedvisiblespace}%
        \def\FancyVerbFormatLine ##1{\hsize\linewidth
            \vtop{\raggedright\hyphenpenalty\z@\exhyphenpenalty\z@
                \doublehyphendemerits\z@\finalhyphendemerits\z@
                \strut ##1\strut}%
        }%
        % If the linebreak is at a space, the latter will be displayed as visible
        % space at end of first line, and a continuation symbol starts next line.
        % Stretch/shrink are however usually zero for typewriter font.
        \def\FV@Space {%
            \nobreak\hskip\z@ plus\fontdimen3\font minus\fontdimen4\font
            \discretionary{\copy\Wrappedvisiblespacebox}{\Wrappedafterbreak}
            {\kern\fontdimen2\font}%
        }%
        
        % Allow breaks at special characters using \PYG... macros.
        \Wrappedbreaksatspecials
        % Breaks at punctuation characters . , ; ? ! and / need catcode=\active 	
        \OriginalVerbatim[#1,codes*=\Wrappedbreaksatpunct]%
    }
    \makeatother

    % Exact colors from NB
    \definecolor{incolor}{HTML}{303F9F}
    \definecolor{outcolor}{HTML}{D84315}
    \definecolor{cellborder}{HTML}{CFCFCF}
    \definecolor{cellbackground}{HTML}{F7F7F7}
    
    % prompt
    \makeatletter
    \newcommand{\boxspacing}{\kern\kvtcb@left@rule\kern\kvtcb@boxsep}
    \makeatother
    \newcommand{\prompt}[4]{
        \ttfamily\llap{{\color{#2}[#3]:\hspace{3pt}#4}}\vspace{-\baselineskip}
    }
    

    
    % Prevent overflowing lines due to hard-to-break entities
    \sloppy 
    % Setup hyperref package
    \hypersetup{
      breaklinks=true,  % so long urls are correctly broken across lines
      colorlinks=true,
      urlcolor=urlcolor,
      linkcolor=linkcolor,
      citecolor=citecolor,
      }
    % Slightly bigger margins than the latex defaults
    
    \geometry{verbose,tmargin=1in,bmargin=1in,lmargin=1in,rmargin=1in}
    
    

\begin{document}
  
%    \maketitle
%	ref: https://stackoverflow.com/questions/3141702/
	\begin{titlepage}
		\vspace*{\fill}
		\begin{center}
 		\normalsize Manual / Referensi Modul \texttt{hidrokit.contrib.taruma}\\
		\huge \texttt{.hk84}: Ringkasan Jam-Jaman\\ 
		\small telah tersedia pada versi hidrokit 0.3.4 \\[0.2cm]
      	\small Berdasarkan \emph{Jupyter Notebook}: \texttt{github-taruma\_hk84\_tabel\_ringkasan\_jamjaman.ipynb} \\[0.5cm]
      	\small Modul ini digunakan untuk membuat ringkasan tabel dari dataset jam-jaman yang memisahkan kejadian hujan beserta durasinya. \\[0.5cm]
		\normalsize oleh Taruma Sakti Megariansyah\\[0.5cm]
      	\normalsize 9 Desember 2019 (1.0.0)\\[1cm]
    	\adjustimage{max size={0.9\linewidth}{1cm}}{hidrokit_logo.jpg}\\
      	\normalsize github.com/taruma/hidrokit
		\end{center}
    	\vspace*{\fill}
	\end{titlepage}
    
    

    
    

    Berdasarkan isu
\href{https://github.com/taruma/hidrokit/issues/84}{\#84}:
\textbf{request: buat ringkasan tabel jam-jaman dengan durasinya}

Referensi isu:

\begin{itemize}
\tightlist
\item
  \texttt{hidrokit.contrib.taruma.hk79}
  \href{https://github.com/taruma/hidrokit/issues/79}{\#79}.
  (\href{https://nbviewer.jupyter.org/gist/taruma/05dab67fac8313a94134ac02d0398897}{lihat
  notebook / manual}). \textbf{request: ambil dataset hujan jam-jaman
  dari excel}
\item
  \texttt{hidrokit.contrib.taruma.hk73}
  \href{https://github.com/taruma/hidrokit/issues/73}{\#73}.
  (\href{https://nbviewer.jupyter.org/gist/taruma/b00880905f297013f046dad95dc2e284}{lihat
  notebook / manual}). \textbf{request: mengolah berkas dari data bmkg}
\end{itemize}

Deskripsi Permasalahan:

\begin{itemize}
\tightlist
\item
  Setelah memperoleh dataset menggunakan \texttt{.hk79}, maka data
  jam-jaman akan diproses lebih lanjut lagi.
\item
  Dalam isu ini, akan dibuat tabel ringkasan mengenai hujan jam-jaman
  beserta durasinya.
\item
  Tabel ringkasan dapat disimpan dalam berbagai format (excel atau csv)
  dengan menggunakan \texttt{pandas}.
\end{itemize}

Strategi Penyelesaian:

\begin{itemize}
\tightlist
\item
  Ditentukan/diasumsikan bahwa observasi hanya dilakukan per-24 jam,
  maka durasi terlama dalam tabel sebesar 24 jam.
\item
  Mempersiapkan DF (\texttt{DataFrame}) yang akan digunakan dan
  menentukan kolom mana yang akan dibuat ringkasannya.
\item
  Ambil sub-DF dari DF setiap 24 jam, kemudian diambil informasi
  kejadian hujan jam-jaman.

  \begin{itemize}
  \tightlist
  \item
    Membuat \texttt{index\_grouped} yang merupakan
    \texttt{list\ of\ list} index kejadian hujan yang berturut-turut.
    Fungsi ini telah dikembangkan pada modul \texttt{.hk73}.
  \item
    Membuat tiga \texttt{list} yang merupakan \texttt{list} untuk
    tanggal observasi, \texttt{list} untuk jam observasi, dan
    \texttt{list} untuk nilai hujan observasi.
  \item
    Mengubah tiga \texttt{list} tersebut ke dalam bentuk
    \texttt{dictionary}.
  \end{itemize}
\item
  Ulangi tahap sebelumnya untuk setiap hari (24 jam). Dan menggabungkan
  seluruh \texttt{dictionary} dalam satu peubah.
\item
  Mengubah \texttt{dictionary} menjadi \texttt{pandas.DataFrame}.
\end{itemize}

    \hypertarget{persiapan-dan-dataset}{%
\section{PERSIAPAN DAN DATASET}\label{persiapan-dan-dataset}}

    \begin{tcolorbox}[breakable, size=fbox, boxrule=1pt, pad at break*=1mm,colback=cellbackground, colframe=cellborder]
\prompt{In}{incolor}{1}{\boxspacing}
\begin{Verbatim}[commandchars=\\\{\}]
\PY{k}{try}\PY{p}{:}
    \PY{k+kn}{import} \PY{n+nn}{hidrokit}
\PY{k}{except}\PY{p}{:}
    \PY{o}{!}pip install hidrokit \PYZhy{}q
\PY{n+nb}{print}\PY{p}{(}\PY{l+s+sa}{f}\PY{l+s+s1}{\PYZsq{}}\PY{l+s+s1}{hidrokit version: }\PY{l+s+si}{\PYZob{}hidrokit.\PYZus{}\PYZus{}version\PYZus{}\PYZus{}\PYZcb{}}\PY{l+s+s1}{\PYZsq{}}\PY{p}{)}
\end{Verbatim}
\end{tcolorbox}

    \begin{Verbatim}[commandchars=\\\{\}]
hidrokit version: 0.3.3
    \end{Verbatim}

    \begin{tcolorbox}[breakable, size=fbox, boxrule=1pt, pad at break*=1mm,colback=cellbackground, colframe=cellborder]
\prompt{In}{incolor}{0}{\boxspacing}
\begin{Verbatim}[commandchars=\\\{\}]
\PY{o}{!}wget \PYZhy{}O aurene\PYZus{}clean.csv \PY{l+s+s2}{\PYZdq{}https://taruma.github.io/assets/hidrokit\PYZus{}dataset/AURENE\PYZus{}STATION\PYZus{}clean.csv\PYZdq{}} \PYZhy{}q
\end{Verbatim}
\end{tcolorbox}

    \begin{tcolorbox}[breakable, size=fbox, boxrule=1pt, pad at break*=1mm,colback=cellbackground, colframe=cellborder]
\prompt{In}{incolor}{0}{\boxspacing}
\begin{Verbatim}[commandchars=\\\{\}]
\PY{k+kn}{import} \PY{n+nn}{pandas} \PY{k}{as} \PY{n+nn}{pd}
\PY{k+kn}{import} \PY{n+nn}{numpy} \PY{k}{as} \PY{n+nn}{np}
\end{Verbatim}
\end{tcolorbox}

    \begin{tcolorbox}[breakable, size=fbox, boxrule=1pt, pad at break*=1mm,colback=cellbackground, colframe=cellborder]
\prompt{In}{incolor}{4}{\boxspacing}
\begin{Verbatim}[commandchars=\\\{\}]
\PY{c+c1}{\PYZsh{} Load Dataset}
\PY{n}{dataset} \PY{o}{=} \PY{n}{pd}\PY{o}{.}\PY{n}{read\PYZus{}csv}\PY{p}{(}\PY{l+s+s1}{\PYZsq{}}\PY{l+s+s1}{aurene\PYZus{}clean.csv}\PY{l+s+s1}{\PYZsq{}}\PY{p}{,} \PY{n}{index\PYZus{}col}\PY{o}{=}\PY{l+m+mi}{0}\PY{p}{,} \PY{n}{parse\PYZus{}dates}\PY{o}{=}\PY{k+kc}{True}\PY{p}{)}
\PY{n}{dataset}\PY{o}{.}\PY{n}{info}\PY{p}{(}\PY{p}{)}
\end{Verbatim}
\end{tcolorbox}

    \begin{Verbatim}[commandchars=\\\{\}]
<class 'pandas.core.frame.DataFrame'>
DatetimeIndex: 26304 entries, 2000-01-01 00:00:00 to 2002-12-31 23:00:00
Data columns (total 1 columns):
AURENE    2759 non-null float64
dtypes: float64(1)
memory usage: 411.0 KB
    \end{Verbatim}

    \hypertarget{kode}{%
\section{KODE}\label{kode}}

    \begin{tcolorbox}[breakable, size=fbox, boxrule=1pt, pad at break*=1mm,colback=cellbackground, colframe=cellborder]
\prompt{In}{incolor}{0}{\boxspacing}
\begin{Verbatim}[commandchars=\\\{\}]
\PY{k+kn}{from} \PY{n+nn}{hidrokit}\PY{n+nn}{.}\PY{n+nn}{contrib}\PY{n+nn}{.}\PY{n+nn}{taruma} \PY{k+kn}{import} \PY{n}{hk73}

\PY{k}{def} \PY{n+nf}{\PYZus{}time\PYZus{}grouped}\PY{p}{(}\PY{n}{df}\PY{p}{,} \PY{n}{index\PYZus{}grouped}\PY{p}{,} \PY{n}{col}\PY{p}{,} \PY{n}{date\PYZus{}fmt}\PY{o}{=}\PY{l+s+s1}{\PYZsq{}}\PY{l+s+s1}{\PYZpc{}}\PY{l+s+s1}{Y\PYZhy{}}\PY{l+s+s1}{\PYZpc{}}\PY{l+s+s1}{m\PYZhy{}}\PY{l+s+si}{\PYZpc{}d}\PY{l+s+s1}{\PYZsq{}}\PY{p}{,} \PY{n}{hour\PYZus{}fmt}\PY{o}{=}\PY{l+s+s1}{\PYZsq{}}\PY{l+s+s1}{\PYZpc{}}\PY{l+s+s1}{H:}\PY{l+s+s1}{\PYZpc{}}\PY{l+s+s1}{M}\PY{l+s+s1}{\PYZsq{}}\PY{p}{)}\PY{p}{:}
    \PY{l+s+sd}{\PYZdq{}\PYZdq{}\PYZdq{}Return index\PYZus{}grouped as (list of date, list of hour)\PYZdq{}\PYZdq{}\PYZdq{}}
    \PY{n}{date} \PY{o}{=} \PY{p}{[}\PY{p}{]}
    \PY{n}{hour} \PY{o}{=} \PY{p}{[}\PY{p}{]}
    \PY{k}{for} \PY{n}{item} \PY{o+ow}{in} \PY{n}{index\PYZus{}grouped}\PY{p}{:}
        \PY{n}{date\PYZus{}val} \PY{o}{=} \PY{n}{df}\PY{o}{.}\PY{n}{iloc}\PY{p}{[}\PY{p}{[}\PY{n}{item}\PY{p}{[}\PY{l+m+mi}{0}\PY{p}{]}\PY{p}{]}\PY{p}{,} \PY{n}{col}\PY{p}{]}\PY{o}{.}\PY{n}{index}\PY{o}{.}\PY{n}{strftime}\PY{p}{(}\PY{n}{date\PYZus{}fmt}\PY{p}{)}\PY{o}{.}\PY{n}{to\PYZus{}list}\PY{p}{(}\PY{p}{)}
        \PY{n}{hour\PYZus{}val} \PY{o}{=} \PY{n}{df}\PY{o}{.}\PY{n}{iloc}\PY{p}{[}\PY{p}{[}\PY{n}{item}\PY{p}{[}\PY{l+m+mi}{0}\PY{p}{]}\PY{p}{]}\PY{p}{,} \PY{n}{col}\PY{p}{]}\PY{o}{.}\PY{n}{index}\PY{o}{.}\PY{n}{strftime}\PY{p}{(}\PY{n}{hour\PYZus{}fmt}\PY{p}{)}\PY{o}{.}\PY{n}{to\PYZus{}list}\PY{p}{(}\PY{p}{)}
        \PY{n}{date}\PY{o}{.}\PY{n}{append}\PY{p}{(}\PY{n}{date\PYZus{}val}\PY{p}{)}
        \PY{n}{hour}\PY{o}{.}\PY{n}{append}\PY{p}{(}\PY{n}{hour\PYZus{}val}\PY{p}{)}
    \PY{k}{return} \PY{n}{date}\PY{p}{,} \PY{n}{hour}

\PY{k}{def} \PY{n+nf}{\PYZus{}value\PYZus{}grouped}\PY{p}{(}\PY{n}{df}\PY{p}{,} \PY{n}{index\PYZus{}grouped}\PY{p}{,} \PY{n}{col}\PY{p}{)}\PY{p}{:}
    \PY{l+s+sd}{\PYZdq{}\PYZdq{}\PYZdq{}Return index\PYZus{}grouped as list of value list\PYZdq{}\PYZdq{}\PYZdq{}}
    \PY{n}{value} \PY{o}{=} \PY{p}{[}\PY{p}{]}
    \PY{k}{for} \PY{n}{item} \PY{o+ow}{in} \PY{n}{index\PYZus{}grouped}\PY{p}{:}
        \PY{n}{value\PYZus{}val} \PY{o}{=} \PY{n}{df}\PY{o}{.}\PY{n}{iloc}\PY{p}{[}\PY{n}{item}\PY{p}{,} \PY{n}{col}\PY{p}{]}\PY{o}{.}\PY{n}{to\PYZus{}list}\PY{p}{(}\PY{p}{)}
        \PY{n}{value}\PY{o}{.}\PY{n}{append}\PY{p}{(}\PY{n}{value\PYZus{}val}\PY{p}{)}
    \PY{k}{return} \PY{n}{value}

\PY{k}{def} \PY{n+nf}{\PYZus{}dict\PYZus{}grouped}\PY{p}{(}\PY{n}{date\PYZus{}list}\PY{p}{,} \PY{n}{hour\PYZus{}list}\PY{p}{,} \PY{n}{value\PYZus{}list}\PY{p}{,} \PY{n}{start}\PY{o}{=}\PY{l+m+mi}{0}\PY{p}{)}\PY{p}{:}
    \PY{l+s+sd}{\PYZdq{}\PYZdq{}\PYZdq{}Join three list and return as dictionary\PYZdq{}\PYZdq{}\PYZdq{}}
    \PY{n}{item\PYZus{}list} \PY{o}{=} \PY{n+nb}{enumerate}\PY{p}{(}\PY{n+nb}{zip}\PY{p}{(}\PY{n}{date\PYZus{}list}\PY{p}{,} \PY{n}{hour\PYZus{}list}\PY{p}{,} \PY{n}{value\PYZus{}list}\PY{p}{)}\PY{p}{,} \PY{n}{start}\PY{o}{=}\PY{n}{start}\PY{p}{)}
    \PY{k}{return} \PY{p}{\PYZob{}}
        \PY{n}{i}\PY{p}{:} \PY{n}{date}\PY{o}{+}\PY{n}{hour}\PY{o}{+}\PY{n}{value} \PY{k}{for} \PY{n}{i}\PY{p}{,} \PY{p}{(}\PY{n}{date}\PY{p}{,} \PY{n}{hour}\PY{p}{,} \PY{n}{value}\PY{p}{)} \PY{o+ow}{in} \PY{n}{item\PYZus{}list}
    \PY{p}{\PYZcb{}}

\PY{k}{def} \PY{n+nf}{summary\PYZus{}hourly}\PY{p}{(}\PY{n}{df}\PY{p}{,} \PY{n}{column}\PY{p}{,} \PY{n}{n\PYZus{}hours}\PY{o}{=}\PY{l+m+mi}{24}\PY{p}{,} 
                   \PY{n}{text\PYZus{}date}\PY{o}{=}\PY{p}{[}\PY{l+s+s1}{\PYZsq{}}\PY{l+s+s1}{date}\PY{l+s+s1}{\PYZsq{}}\PY{p}{,} \PY{l+s+s1}{\PYZsq{}}\PY{l+s+s1}{hour}\PY{l+s+s1}{\PYZsq{}}\PY{p}{]}\PY{p}{,} \PY{n}{as\PYZus{}df}\PY{o}{=}\PY{k+kc}{True}\PY{p}{,}
                   \PY{n}{date\PYZus{}fmt}\PY{o}{=}\PY{l+s+s1}{\PYZsq{}}\PY{l+s+s1}{\PYZpc{}}\PY{l+s+s1}{Y\PYZhy{}}\PY{l+s+s1}{\PYZpc{}}\PY{l+s+s1}{m\PYZhy{}}\PY{l+s+si}{\PYZpc{}d}\PY{l+s+s1}{\PYZsq{}}\PY{p}{,} \PY{n}{hour\PYZus{}fmt}\PY{o}{=}\PY{l+s+s1}{\PYZsq{}}\PY{l+s+s1}{\PYZpc{}}\PY{l+s+s1}{H:}\PY{l+s+s1}{\PYZpc{}}\PY{l+s+s1}{M}\PY{l+s+s1}{\PYZsq{}}\PY{p}{)}\PY{p}{:}
    \PY{n}{col} \PY{o}{=} \PY{n}{df}\PY{o}{.}\PY{n}{columns}\PY{o}{.}\PY{n}{get\PYZus{}loc}\PY{p}{(}\PY{n}{column}\PY{p}{)}
    \PY{n}{nrows}\PY{p}{,} \PY{n}{\PYZus{}} \PY{o}{=} \PY{n}{df}\PY{o}{.}\PY{n}{shape}
    \PY{n}{results} \PY{o}{=} \PY{p}{\PYZob{}}\PY{p}{\PYZcb{}}

    \PY{k}{for} \PY{n}{i} \PY{o+ow}{in} \PY{n+nb}{range}\PY{p}{(}\PY{l+m+mi}{0}\PY{p}{,} \PY{n}{nrows}\PY{p}{,} \PY{n}{n\PYZus{}hours}\PY{p}{)}\PY{p}{:}
        \PY{n}{sub\PYZus{}df} \PY{o}{=} \PY{n}{df}\PY{o}{.}\PY{n}{iloc}\PY{p}{[}\PY{n}{i}\PY{p}{:}\PY{n}{i}\PY{o}{+}\PY{n}{n\PYZus{}hours}\PY{p}{]}
        \PY{n}{ix\PYZus{}array} \PY{o}{=} \PY{n}{hk73}\PY{o}{.}\PY{n}{\PYZus{}get\PYZus{}index1D}\PY{p}{(}\PY{o}{\PYZti{}}\PY{n}{sub\PYZus{}df}\PY{o}{.}\PY{n}{iloc}\PY{p}{[}\PY{p}{:}\PY{p}{,} \PY{n}{col}\PY{p}{]}\PY{o}{.}\PY{n}{isna}\PY{p}{(}\PY{p}{)}\PY{o}{.}\PY{n}{values}\PY{p}{)}
        \PY{n}{ix\PYZus{}grouped} \PY{o}{=} \PY{n}{hk73}\PY{o}{.}\PY{n}{\PYZus{}group\PYZus{}as\PYZus{}list}\PY{p}{(}\PY{n}{ix\PYZus{}array}\PY{p}{)}
        \PY{n}{date}\PY{p}{,} \PY{n}{hour} \PY{o}{=} \PY{n}{\PYZus{}time\PYZus{}grouped}\PY{p}{(}\PY{n}{sub\PYZus{}df}\PY{p}{,} \PY{n}{ix\PYZus{}grouped}\PY{p}{,} 
                                   \PY{n}{col}\PY{p}{,} \PY{n}{date\PYZus{}fmt}\PY{o}{=}\PY{n}{date\PYZus{}fmt}\PY{p}{,} \PY{n}{hour\PYZus{}fmt}\PY{o}{=}\PY{n}{hour\PYZus{}fmt}\PY{p}{)}
        \PY{n}{value} \PY{o}{=} \PY{n}{\PYZus{}value\PYZus{}grouped}\PY{p}{(}\PY{n}{sub\PYZus{}df}\PY{p}{,} \PY{n}{ix\PYZus{}grouped}\PY{p}{,} \PY{n}{col}\PY{p}{)}
        \PY{n}{each\PYZus{}hours} \PY{o}{=} \PY{n}{\PYZus{}dict\PYZus{}grouped}\PY{p}{(}\PY{n}{date}\PY{p}{,} \PY{n}{hour}\PY{p}{,} \PY{n}{value}\PY{p}{,} \PY{n}{start}\PY{o}{=}\PY{n}{i}\PY{p}{)}
        \PY{n}{results}\PY{o}{.}\PY{n}{update}\PY{p}{(}\PY{n}{each\PYZus{}hours}\PY{p}{)}

    \PY{k}{if} \PY{n}{as\PYZus{}df}\PY{p}{:}
        \PY{n}{columns\PYZus{}name} \PY{o}{=} \PY{n}{text\PYZus{}date} \PY{o}{+} \PY{p}{[}\PY{n}{i} \PY{k}{for} \PY{n}{i} \PY{o+ow}{in} \PY{n+nb}{range}\PY{p}{(}\PY{l+m+mi}{1}\PY{p}{,} \PY{n}{n\PYZus{}hours}\PY{o}{+}\PY{l+m+mi}{1}\PY{p}{)}\PY{p}{]}
        \PY{n}{df\PYZus{}results} \PY{o}{=} \PY{n}{pd}\PY{o}{.}\PY{n}{DataFrame}\PY{o}{.}\PY{n}{from\PYZus{}dict}\PY{p}{(}
            \PY{n}{results}\PY{p}{,} \PY{n}{orient}\PY{o}{=}\PY{l+s+s1}{\PYZsq{}}\PY{l+s+s1}{index}\PY{l+s+s1}{\PYZsq{}}\PY{p}{,} \PY{n}{columns}\PY{o}{=}\PY{n}{columns\PYZus{}name}
        \PY{p}{)} 
        \PY{k}{return} \PY{n}{df\PYZus{}results}
    \PY{k}{else}\PY{p}{:}
        \PY{k}{return} \PY{n}{results}
\end{Verbatim}
\end{tcolorbox}

    \hypertarget{penerapan}{%
\section{PENERAPAN}\label{penerapan}}

    \begin{tcolorbox}[breakable, size=fbox, boxrule=1pt, pad at break*=1mm,colback=cellbackground, colframe=cellborder]
\prompt{In}{incolor}{6}{\boxspacing}
\begin{Verbatim}[commandchars=\\\{\}]
\PY{n}{summary} \PY{o}{=} \PY{n}{summary\PYZus{}hourly}\PY{p}{(}\PY{n}{dataset}\PY{p}{,} \PY{l+s+s1}{\PYZsq{}}\PY{l+s+s1}{AURENE}\PY{l+s+s1}{\PYZsq{}}\PY{p}{)}
\PY{n}{summary}\PY{o}{.}\PY{n}{head}\PY{p}{(}\PY{p}{)}
\end{Verbatim}
\end{tcolorbox}

            \begin{tcolorbox}[breakable, size=fbox, boxrule=.5pt, pad at break*=1mm, opacityfill=0]
\prompt{Out}{outcolor}{6}{\boxspacing}
\begin{Verbatim}[commandchars=\\\{\}]
          date   hour    1    2    3   4   5  {\ldots}  18  19  20  21  22  23  24
0   2000-01-01  08:00  0.3  NaN  NaN NaN NaN  {\ldots} NaN NaN NaN NaN NaN NaN NaN
24  2000-01-02  08:00  7.0  0.5  0.5 NaN NaN  {\ldots} NaN NaN NaN NaN NaN NaN NaN
48  2000-01-03  00:00  0.6  NaN  NaN NaN NaN  {\ldots} NaN NaN NaN NaN NaN NaN NaN
49  2000-01-03  05:00  0.6  NaN  NaN NaN NaN  {\ldots} NaN NaN NaN NaN NaN NaN NaN
50  2000-01-03  09:00  2.2  NaN  NaN NaN NaN  {\ldots} NaN NaN NaN NaN NaN NaN NaN

[5 rows x 26 columns]
\end{Verbatim}
\end{tcolorbox}
        
    \begin{tcolorbox}[breakable, size=fbox, boxrule=1pt, pad at break*=1mm,colback=cellbackground, colframe=cellborder]
\prompt{In}{incolor}{7}{\boxspacing}
\begin{Verbatim}[commandchars=\\\{\}]
\PY{n}{summary}\PY{o}{.}\PY{n}{sample}\PY{p}{(}\PY{n}{n}\PY{o}{=}\PY{l+m+mi}{15}\PY{p}{)}
\end{Verbatim}
\end{tcolorbox}

            \begin{tcolorbox}[breakable, size=fbox, boxrule=.5pt, pad at break*=1mm, opacityfill=0]
\prompt{Out}{outcolor}{7}{\boxspacing}
\begin{Verbatim}[commandchars=\\\{\}]
             date   hour     1    2    3    4    5  {\ldots}  18  19  20  21  22  23
24
480    2000-01-21  00:00   2.8  1.2  NaN  NaN  NaN  {\ldots} NaN NaN NaN NaN NaN NaN
NaN
15312  2001-09-30  16:00   0.2  NaN  NaN  NaN  NaN  {\ldots} NaN NaN NaN NaN NaN NaN
NaN
384    2000-01-17  06:00   0.5  NaN  NaN  NaN  NaN  {\ldots} NaN NaN NaN NaN NaN NaN
NaN
937    2000-02-09  04:00   6.0  3.3  3.7  2.9  NaN  {\ldots} NaN NaN NaN NaN NaN NaN
NaN
25032  2002-11-09  08:00   1.0  NaN  NaN  NaN  NaN  {\ldots} NaN NaN NaN NaN NaN NaN
NaN
17593  2002-01-03  06:00   0.5  0.8  NaN  NaN  NaN  {\ldots} NaN NaN NaN NaN NaN NaN
NaN
2856   2000-04-29  08:00  11.8  NaN  NaN  NaN  NaN  {\ldots} NaN NaN NaN NaN NaN NaN
NaN
2496   2000-04-14  11:00   0.2  0.2  NaN  NaN  NaN  {\ldots} NaN NaN NaN NaN NaN NaN
NaN
19634  2002-03-29  13:00   0.6  NaN  NaN  NaN  NaN  {\ldots} NaN NaN NaN NaN NaN NaN
NaN
19778  2002-04-04  16:00   0.1  NaN  NaN  NaN  NaN  {\ldots} NaN NaN NaN NaN NaN NaN
NaN
16512  2001-11-19  06:00   6.6  NaN  NaN  NaN  NaN  {\ldots} NaN NaN NaN NaN NaN NaN
NaN
1657   2000-03-10  08:00   4.5  1.5  NaN  NaN  NaN  {\ldots} NaN NaN NaN NaN NaN NaN
NaN
20808  2002-05-17  13:00   1.4  1.3  NaN  NaN  NaN  {\ldots} NaN NaN NaN NaN NaN NaN
NaN
15361  2001-10-02  13:00   0.2  NaN  NaN  NaN  NaN  {\ldots} NaN NaN NaN NaN NaN NaN
NaN
22128  2002-07-11  12:00  19.0  6.5  5.0  2.0  2.2  {\ldots} NaN NaN NaN NaN NaN NaN
NaN

[15 rows x 26 columns]
\end{Verbatim}
\end{tcolorbox}
        
    \begin{tcolorbox}[breakable, size=fbox, boxrule=1pt, pad at break*=1mm,colback=cellbackground, colframe=cellborder]
\prompt{In}{incolor}{8}{\boxspacing}
\begin{Verbatim}[commandchars=\\\{\}]
\PY{n}{summary2} \PY{o}{=} \PY{n}{summary\PYZus{}hourly}\PY{p}{(}\PY{n}{dataset}\PY{p}{,} \PY{l+s+s1}{\PYZsq{}}\PY{l+s+s1}{AURENE}\PY{l+s+s1}{\PYZsq{}}\PY{p}{,} \PY{n}{n\PYZus{}hours}\PY{o}{=}\PY{l+m+mi}{10}\PY{p}{,} \PY{n}{text\PYZus{}date}\PY{o}{=}\PY{p}{[}\PY{l+s+s1}{\PYZsq{}}\PY{l+s+s1}{Tanggal}\PY{l+s+s1}{\PYZsq{}}\PY{p}{,} \PY{l+s+s1}{\PYZsq{}}\PY{l+s+s1}{Jam}\PY{l+s+s1}{\PYZsq{}}\PY{p}{]}\PY{p}{,} \PY{n}{date\PYZus{}fmt}\PY{o}{=}\PY{l+s+s1}{\PYZsq{}}\PY{l+s+si}{\PYZpc{}d}\PY{l+s+s1}{ }\PY{l+s+s1}{\PYZpc{}}\PY{l+s+s1}{b }\PY{l+s+s1}{\PYZpc{}}\PY{l+s+s1}{Y}\PY{l+s+s1}{\PYZsq{}}\PY{p}{,} \PY{n}{hour\PYZus{}fmt}\PY{o}{=}\PY{l+s+s1}{\PYZsq{}}\PY{l+s+s1}{\PYZpc{}}\PY{l+s+s1}{I:}\PY{l+s+s1}{\PYZpc{}}\PY{l+s+s1}{M }\PY{l+s+s1}{\PYZpc{}}\PY{l+s+s1}{p}\PY{l+s+s1}{\PYZsq{}}\PY{p}{)}
\PY{n}{summary2}\PY{o}{.}\PY{n}{sample}\PY{p}{(}\PY{n}{n}\PY{o}{=}\PY{l+m+mi}{10}\PY{p}{)}
\end{Verbatim}
\end{tcolorbox}

            \begin{tcolorbox}[breakable, size=fbox, boxrule=.5pt, pad at break*=1mm, opacityfill=0]
\prompt{Out}{outcolor}{8}{\boxspacing}
\begin{Verbatim}[commandchars=\\\{\}]
           Tanggal       Jam     1     2     3     4     5   6   7   8   9  10
4420   03 Jul 2000  06:00 AM   5.5  10.0  41.5  23.0  14.5 NaN NaN NaN NaN NaN
22020  06 Jul 2002  03:00 PM   0.1   NaN   NaN   NaN   NaN NaN NaN NaN NaN NaN
19920  10 Apr 2002  02:00 AM   1.0   2.6   NaN   NaN   NaN NaN NaN NaN NaN NaN
25951  17 Dec 2002  09:00 AM  31.8   0.1   2.4   3.1   NaN NaN NaN NaN NaN NaN
12390  31 May 2001  07:00 AM   1.8   NaN   NaN   NaN   NaN NaN NaN NaN NaN NaN
4590   10 Jul 2000  12:00 PM   3.2   NaN   NaN   NaN   NaN NaN NaN NaN NaN NaN
11750  04 May 2001  02:00 PM   1.2   NaN   NaN   NaN   NaN NaN NaN NaN NaN NaN
18340  03 Feb 2002  07:00 AM   0.4   NaN   NaN   NaN   NaN NaN NaN NaN NaN NaN
18330  02 Feb 2002  07:00 PM   0.4   0.2   NaN   NaN   NaN NaN NaN NaN NaN NaN
10950  01 Apr 2001  10:00 AM   1.2   1.0   0.7   0.2   NaN NaN NaN NaN NaN NaN
\end{Verbatim}
\end{tcolorbox}
        
    \begin{tcolorbox}[breakable, size=fbox, boxrule=1pt, pad at break*=1mm,colback=cellbackground, colframe=cellborder]
\prompt{In}{incolor}{9}{\boxspacing}
\begin{Verbatim}[commandchars=\\\{\}]
\PY{n}{summary3} \PY{o}{=} \PY{n}{summary\PYZus{}hourly}\PY{p}{(}\PY{n}{dataset}\PY{p}{,} \PY{l+s+s1}{\PYZsq{}}\PY{l+s+s1}{AURENE}\PY{l+s+s1}{\PYZsq{}}\PY{p}{,} \PY{n}{n\PYZus{}hours}\PY{o}{=}\PY{l+m+mi}{24}\PY{p}{,} \PY{n}{as\PYZus{}df}\PY{o}{=}\PY{k+kc}{False}\PY{p}{)}
\PY{n+nb}{print}\PY{p}{(}\PY{n+nb}{type}\PY{p}{(}\PY{n}{summary3}\PY{p}{)}\PY{p}{)}
\PY{n+nb}{print}\PY{p}{(}\PY{n+nb}{len}\PY{p}{(}\PY{n}{summary3}\PY{o}{.}\PY{n}{keys}\PY{p}{(}\PY{p}{)}\PY{p}{)}\PY{p}{)}
\PY{n+nb}{print}\PY{p}{(}\PY{n+nb}{list}\PY{p}{(}\PY{n}{summary3}\PY{o}{.}\PY{n}{keys}\PY{p}{(}\PY{p}{)}\PY{p}{)}\PY{p}{[}\PY{p}{:}\PY{l+m+mi}{10}\PY{p}{]}\PY{p}{)}
\end{Verbatim}
\end{tcolorbox}

    \begin{Verbatim}[commandchars=\\\{\}]
<class 'dict'>
1093
[0, 24, 48, 49, 50, 72, 96, 97, 120, 144]
    \end{Verbatim}

    \hypertarget{changelog}{%
\section{Changelog}\label{changelog}}

\begin{verbatim}
- 20191209 - 1.0.0 - Initial
\end{verbatim}

\hypertarget{copyright-2019-taruma-sakti-megariansyah}{%
\paragraph{\texorpdfstring{Copyright © 2019
\href{https://taruma.github.io}{Taruma Sakti
Megariansyah}}{Copyright © 2019 Taruma Sakti Megariansyah}}\label{copyright-2019-taruma-sakti-megariansyah}}

Source code in this notebook is licensed under a
\href{https://choosealicense.com/licenses/mit/}{MIT License}. Data in
this notebook is licensed under a
\href{https://creativecommons.org/licenses/by/4.0/}{Creative Common
Attribution 4.0 International}.


    % Add a bibliography block to the postdoc
    
    
    
\end{document}
