\documentclass[11pt]{article}

    \usepackage[breakable]{tcolorbox}
    \usepackage{parskip} % Stop auto-indenting (to mimic markdown behaviour)
    
    \usepackage{iftex}
    \ifPDFTeX
    	\usepackage[T1]{fontenc}
    	\usepackage{mathpazo}
    \else
    	\usepackage{fontspec}
    \fi

    % Basic figure setup, for now with no caption control since it's done
    % automatically by Pandoc (which extracts ![](path) syntax from Markdown).
    \usepackage{graphicx}
    % Maintain compatibility with old templates. Remove in nbconvert 6.0
    \let\Oldincludegraphics\includegraphics
    % Ensure that by default, figures have no caption (until we provide a
    % proper Figure object with a Caption API and a way to capture that
    % in the conversion process - todo).
    \usepackage{caption}
    \DeclareCaptionFormat{nocaption}{}
    \captionsetup{format=nocaption,aboveskip=0pt,belowskip=0pt}

    \usepackage[Export]{adjustbox} % Used to constrain images to a maximum size
    \adjustboxset{max size={0.9\linewidth}{0.9\paperheight}}
    \usepackage{float}
    \floatplacement{figure}{H} % forces figures to be placed at the correct location
    \usepackage{xcolor} % Allow colors to be defined
    \usepackage{enumerate} % Needed for markdown enumerations to work
    \usepackage{geometry} % Used to adjust the document margins
    \usepackage{amsmath} % Equations
    \usepackage{amssymb} % Equations
    \usepackage{textcomp} % defines textquotesingle
    % Hack from http://tex.stackexchange.com/a/47451/13684:
    \AtBeginDocument{%
        \def\PYZsq{\textquotesingle}% Upright quotes in Pygmentized code
    }
    \usepackage{upquote} % Upright quotes for verbatim code
    \usepackage{eurosym} % defines \euro
    \usepackage[mathletters]{ucs} % Extended unicode (utf-8) support
    \usepackage{fancyvrb} % verbatim replacement that allows latex
    \usepackage{grffile} % extends the file name processing of package graphics 
                         % to support a larger range
    \makeatletter % fix for grffile with XeLaTeX
    \def\Gread@@xetex#1{%
      \IfFileExists{"\Gin@base".bb}%
      {\Gread@eps{\Gin@base.bb}}%
      {\Gread@@xetex@aux#1}%
    }
    \makeatother

    % The hyperref package gives us a pdf with properly built
    % internal navigation ('pdf bookmarks' for the table of contents,
    % internal cross-reference links, web links for URLs, etc.)
    \usepackage{hyperref}
    % The default LaTeX title has an obnoxious amount of whitespace. By default,
    % titling removes some of it. It also provides customization options.
    \usepackage{titling}
    \usepackage{longtable} % longtable support required by pandoc >1.10
    \usepackage{booktabs}  % table support for pandoc > 1.12.2
    \usepackage[inline]{enumitem} % IRkernel/repr support (it uses the enumerate* environment)
    \usepackage[normalem]{ulem} % ulem is needed to support strikethroughs (\sout)
                                % normalem makes italics be italics, not underlines
    \usepackage{mathrsfs}
    

    
    % Colors for the hyperref package
    \definecolor{urlcolor}{rgb}{0,.145,.698}
    \definecolor{linkcolor}{rgb}{.71,0.21,0.01}
    \definecolor{citecolor}{rgb}{.12,.54,.11}

    % ANSI colors
    \definecolor{ansi-black}{HTML}{3E424D}
    \definecolor{ansi-black-intense}{HTML}{282C36}
    \definecolor{ansi-red}{HTML}{E75C58}
    \definecolor{ansi-red-intense}{HTML}{B22B31}
    \definecolor{ansi-green}{HTML}{00A250}
    \definecolor{ansi-green-intense}{HTML}{007427}
    \definecolor{ansi-yellow}{HTML}{DDB62B}
    \definecolor{ansi-yellow-intense}{HTML}{B27D12}
    \definecolor{ansi-blue}{HTML}{208FFB}
    \definecolor{ansi-blue-intense}{HTML}{0065CA}
    \definecolor{ansi-magenta}{HTML}{D160C4}
    \definecolor{ansi-magenta-intense}{HTML}{A03196}
    \definecolor{ansi-cyan}{HTML}{60C6C8}
    \definecolor{ansi-cyan-intense}{HTML}{258F8F}
    \definecolor{ansi-white}{HTML}{C5C1B4}
    \definecolor{ansi-white-intense}{HTML}{A1A6B2}
    \definecolor{ansi-default-inverse-fg}{HTML}{FFFFFF}
    \definecolor{ansi-default-inverse-bg}{HTML}{000000}

    % commands and environments needed by pandoc snippets
    % extracted from the output of `pandoc -s`
    \providecommand{\tightlist}{%
      \setlength{\itemsep}{0pt}\setlength{\parskip}{0pt}}
    \DefineVerbatimEnvironment{Highlighting}{Verbatim}{commandchars=\\\{\}}
    % Add ',fontsize=\small' for more characters per line
    \newenvironment{Shaded}{}{}
    \newcommand{\KeywordTok}[1]{\textcolor[rgb]{0.00,0.44,0.13}{\textbf{{#1}}}}
    \newcommand{\DataTypeTok}[1]{\textcolor[rgb]{0.56,0.13,0.00}{{#1}}}
    \newcommand{\DecValTok}[1]{\textcolor[rgb]{0.25,0.63,0.44}{{#1}}}
    \newcommand{\BaseNTok}[1]{\textcolor[rgb]{0.25,0.63,0.44}{{#1}}}
    \newcommand{\FloatTok}[1]{\textcolor[rgb]{0.25,0.63,0.44}{{#1}}}
    \newcommand{\CharTok}[1]{\textcolor[rgb]{0.25,0.44,0.63}{{#1}}}
    \newcommand{\StringTok}[1]{\textcolor[rgb]{0.25,0.44,0.63}{{#1}}}
    \newcommand{\CommentTok}[1]{\textcolor[rgb]{0.38,0.63,0.69}{\textit{{#1}}}}
    \newcommand{\OtherTok}[1]{\textcolor[rgb]{0.00,0.44,0.13}{{#1}}}
    \newcommand{\AlertTok}[1]{\textcolor[rgb]{1.00,0.00,0.00}{\textbf{{#1}}}}
    \newcommand{\FunctionTok}[1]{\textcolor[rgb]{0.02,0.16,0.49}{{#1}}}
    \newcommand{\RegionMarkerTok}[1]{{#1}}
    \newcommand{\ErrorTok}[1]{\textcolor[rgb]{1.00,0.00,0.00}{\textbf{{#1}}}}
    \newcommand{\NormalTok}[1]{{#1}}
    
    % Additional commands for more recent versions of Pandoc
    \newcommand{\ConstantTok}[1]{\textcolor[rgb]{0.53,0.00,0.00}{{#1}}}
    \newcommand{\SpecialCharTok}[1]{\textcolor[rgb]{0.25,0.44,0.63}{{#1}}}
    \newcommand{\VerbatimStringTok}[1]{\textcolor[rgb]{0.25,0.44,0.63}{{#1}}}
    \newcommand{\SpecialStringTok}[1]{\textcolor[rgb]{0.73,0.40,0.53}{{#1}}}
    \newcommand{\ImportTok}[1]{{#1}}
    \newcommand{\DocumentationTok}[1]{\textcolor[rgb]{0.73,0.13,0.13}{\textit{{#1}}}}
    \newcommand{\AnnotationTok}[1]{\textcolor[rgb]{0.38,0.63,0.69}{\textbf{\textit{{#1}}}}}
    \newcommand{\CommentVarTok}[1]{\textcolor[rgb]{0.38,0.63,0.69}{\textbf{\textit{{#1}}}}}
    \newcommand{\VariableTok}[1]{\textcolor[rgb]{0.10,0.09,0.49}{{#1}}}
    \newcommand{\ControlFlowTok}[1]{\textcolor[rgb]{0.00,0.44,0.13}{\textbf{{#1}}}}
    \newcommand{\OperatorTok}[1]{\textcolor[rgb]{0.40,0.40,0.40}{{#1}}}
    \newcommand{\BuiltInTok}[1]{{#1}}
    \newcommand{\ExtensionTok}[1]{{#1}}
    \newcommand{\PreprocessorTok}[1]{\textcolor[rgb]{0.74,0.48,0.00}{{#1}}}
    \newcommand{\AttributeTok}[1]{\textcolor[rgb]{0.49,0.56,0.16}{{#1}}}
    \newcommand{\InformationTok}[1]{\textcolor[rgb]{0.38,0.63,0.69}{\textbf{\textit{{#1}}}}}
    \newcommand{\WarningTok}[1]{\textcolor[rgb]{0.38,0.63,0.69}{\textbf{\textit{{#1}}}}}
    
    
    % Define a nice break command that doesn't care if a line doesn't already
    % exist.
    \def\br{\hspace*{\fill} \\* }
    % Math Jax compatibility definitions
    \def\gt{>}
    \def\lt{<}
    \let\Oldtex\TeX
    \let\Oldlatex\LaTeX
    \renewcommand{\TeX}{\textrm{\Oldtex}}
    \renewcommand{\LaTeX}{\textrm{\Oldlatex}}
    % Document parameters
    % Document title
    \title{taruma\_hk87\_debit\_andal\_kurva\_durasi}
    
    
    
    
    
% Pygments definitions
\makeatletter
\def\PY@reset{\let\PY@it=\relax \let\PY@bf=\relax%
    \let\PY@ul=\relax \let\PY@tc=\relax%
    \let\PY@bc=\relax \let\PY@ff=\relax}
\def\PY@tok#1{\csname PY@tok@#1\endcsname}
\def\PY@toks#1+{\ifx\relax#1\empty\else%
    \PY@tok{#1}\expandafter\PY@toks\fi}
\def\PY@do#1{\PY@bc{\PY@tc{\PY@ul{%
    \PY@it{\PY@bf{\PY@ff{#1}}}}}}}
\def\PY#1#2{\PY@reset\PY@toks#1+\relax+\PY@do{#2}}

\expandafter\def\csname PY@tok@w\endcsname{\def\PY@tc##1{\textcolor[rgb]{0.73,0.73,0.73}{##1}}}
\expandafter\def\csname PY@tok@c\endcsname{\let\PY@it=\textit\def\PY@tc##1{\textcolor[rgb]{0.25,0.50,0.50}{##1}}}
\expandafter\def\csname PY@tok@cp\endcsname{\def\PY@tc##1{\textcolor[rgb]{0.74,0.48,0.00}{##1}}}
\expandafter\def\csname PY@tok@k\endcsname{\let\PY@bf=\textbf\def\PY@tc##1{\textcolor[rgb]{0.00,0.50,0.00}{##1}}}
\expandafter\def\csname PY@tok@kp\endcsname{\def\PY@tc##1{\textcolor[rgb]{0.00,0.50,0.00}{##1}}}
\expandafter\def\csname PY@tok@kt\endcsname{\def\PY@tc##1{\textcolor[rgb]{0.69,0.00,0.25}{##1}}}
\expandafter\def\csname PY@tok@o\endcsname{\def\PY@tc##1{\textcolor[rgb]{0.40,0.40,0.40}{##1}}}
\expandafter\def\csname PY@tok@ow\endcsname{\let\PY@bf=\textbf\def\PY@tc##1{\textcolor[rgb]{0.67,0.13,1.00}{##1}}}
\expandafter\def\csname PY@tok@nb\endcsname{\def\PY@tc##1{\textcolor[rgb]{0.00,0.50,0.00}{##1}}}
\expandafter\def\csname PY@tok@nf\endcsname{\def\PY@tc##1{\textcolor[rgb]{0.00,0.00,1.00}{##1}}}
\expandafter\def\csname PY@tok@nc\endcsname{\let\PY@bf=\textbf\def\PY@tc##1{\textcolor[rgb]{0.00,0.00,1.00}{##1}}}
\expandafter\def\csname PY@tok@nn\endcsname{\let\PY@bf=\textbf\def\PY@tc##1{\textcolor[rgb]{0.00,0.00,1.00}{##1}}}
\expandafter\def\csname PY@tok@ne\endcsname{\let\PY@bf=\textbf\def\PY@tc##1{\textcolor[rgb]{0.82,0.25,0.23}{##1}}}
\expandafter\def\csname PY@tok@nv\endcsname{\def\PY@tc##1{\textcolor[rgb]{0.10,0.09,0.49}{##1}}}
\expandafter\def\csname PY@tok@no\endcsname{\def\PY@tc##1{\textcolor[rgb]{0.53,0.00,0.00}{##1}}}
\expandafter\def\csname PY@tok@nl\endcsname{\def\PY@tc##1{\textcolor[rgb]{0.63,0.63,0.00}{##1}}}
\expandafter\def\csname PY@tok@ni\endcsname{\let\PY@bf=\textbf\def\PY@tc##1{\textcolor[rgb]{0.60,0.60,0.60}{##1}}}
\expandafter\def\csname PY@tok@na\endcsname{\def\PY@tc##1{\textcolor[rgb]{0.49,0.56,0.16}{##1}}}
\expandafter\def\csname PY@tok@nt\endcsname{\let\PY@bf=\textbf\def\PY@tc##1{\textcolor[rgb]{0.00,0.50,0.00}{##1}}}
\expandafter\def\csname PY@tok@nd\endcsname{\def\PY@tc##1{\textcolor[rgb]{0.67,0.13,1.00}{##1}}}
\expandafter\def\csname PY@tok@s\endcsname{\def\PY@tc##1{\textcolor[rgb]{0.73,0.13,0.13}{##1}}}
\expandafter\def\csname PY@tok@sd\endcsname{\let\PY@it=\textit\def\PY@tc##1{\textcolor[rgb]{0.73,0.13,0.13}{##1}}}
\expandafter\def\csname PY@tok@si\endcsname{\let\PY@bf=\textbf\def\PY@tc##1{\textcolor[rgb]{0.73,0.40,0.53}{##1}}}
\expandafter\def\csname PY@tok@se\endcsname{\let\PY@bf=\textbf\def\PY@tc##1{\textcolor[rgb]{0.73,0.40,0.13}{##1}}}
\expandafter\def\csname PY@tok@sr\endcsname{\def\PY@tc##1{\textcolor[rgb]{0.73,0.40,0.53}{##1}}}
\expandafter\def\csname PY@tok@ss\endcsname{\def\PY@tc##1{\textcolor[rgb]{0.10,0.09,0.49}{##1}}}
\expandafter\def\csname PY@tok@sx\endcsname{\def\PY@tc##1{\textcolor[rgb]{0.00,0.50,0.00}{##1}}}
\expandafter\def\csname PY@tok@m\endcsname{\def\PY@tc##1{\textcolor[rgb]{0.40,0.40,0.40}{##1}}}
\expandafter\def\csname PY@tok@gh\endcsname{\let\PY@bf=\textbf\def\PY@tc##1{\textcolor[rgb]{0.00,0.00,0.50}{##1}}}
\expandafter\def\csname PY@tok@gu\endcsname{\let\PY@bf=\textbf\def\PY@tc##1{\textcolor[rgb]{0.50,0.00,0.50}{##1}}}
\expandafter\def\csname PY@tok@gd\endcsname{\def\PY@tc##1{\textcolor[rgb]{0.63,0.00,0.00}{##1}}}
\expandafter\def\csname PY@tok@gi\endcsname{\def\PY@tc##1{\textcolor[rgb]{0.00,0.63,0.00}{##1}}}
\expandafter\def\csname PY@tok@gr\endcsname{\def\PY@tc##1{\textcolor[rgb]{1.00,0.00,0.00}{##1}}}
\expandafter\def\csname PY@tok@ge\endcsname{\let\PY@it=\textit}
\expandafter\def\csname PY@tok@gs\endcsname{\let\PY@bf=\textbf}
\expandafter\def\csname PY@tok@gp\endcsname{\let\PY@bf=\textbf\def\PY@tc##1{\textcolor[rgb]{0.00,0.00,0.50}{##1}}}
\expandafter\def\csname PY@tok@go\endcsname{\def\PY@tc##1{\textcolor[rgb]{0.53,0.53,0.53}{##1}}}
\expandafter\def\csname PY@tok@gt\endcsname{\def\PY@tc##1{\textcolor[rgb]{0.00,0.27,0.87}{##1}}}
\expandafter\def\csname PY@tok@err\endcsname{\def\PY@bc##1{\setlength{\fboxsep}{0pt}\fcolorbox[rgb]{1.00,0.00,0.00}{1,1,1}{\strut ##1}}}
\expandafter\def\csname PY@tok@kc\endcsname{\let\PY@bf=\textbf\def\PY@tc##1{\textcolor[rgb]{0.00,0.50,0.00}{##1}}}
\expandafter\def\csname PY@tok@kd\endcsname{\let\PY@bf=\textbf\def\PY@tc##1{\textcolor[rgb]{0.00,0.50,0.00}{##1}}}
\expandafter\def\csname PY@tok@kn\endcsname{\let\PY@bf=\textbf\def\PY@tc##1{\textcolor[rgb]{0.00,0.50,0.00}{##1}}}
\expandafter\def\csname PY@tok@kr\endcsname{\let\PY@bf=\textbf\def\PY@tc##1{\textcolor[rgb]{0.00,0.50,0.00}{##1}}}
\expandafter\def\csname PY@tok@bp\endcsname{\def\PY@tc##1{\textcolor[rgb]{0.00,0.50,0.00}{##1}}}
\expandafter\def\csname PY@tok@fm\endcsname{\def\PY@tc##1{\textcolor[rgb]{0.00,0.00,1.00}{##1}}}
\expandafter\def\csname PY@tok@vc\endcsname{\def\PY@tc##1{\textcolor[rgb]{0.10,0.09,0.49}{##1}}}
\expandafter\def\csname PY@tok@vg\endcsname{\def\PY@tc##1{\textcolor[rgb]{0.10,0.09,0.49}{##1}}}
\expandafter\def\csname PY@tok@vi\endcsname{\def\PY@tc##1{\textcolor[rgb]{0.10,0.09,0.49}{##1}}}
\expandafter\def\csname PY@tok@vm\endcsname{\def\PY@tc##1{\textcolor[rgb]{0.10,0.09,0.49}{##1}}}
\expandafter\def\csname PY@tok@sa\endcsname{\def\PY@tc##1{\textcolor[rgb]{0.73,0.13,0.13}{##1}}}
\expandafter\def\csname PY@tok@sb\endcsname{\def\PY@tc##1{\textcolor[rgb]{0.73,0.13,0.13}{##1}}}
\expandafter\def\csname PY@tok@sc\endcsname{\def\PY@tc##1{\textcolor[rgb]{0.73,0.13,0.13}{##1}}}
\expandafter\def\csname PY@tok@dl\endcsname{\def\PY@tc##1{\textcolor[rgb]{0.73,0.13,0.13}{##1}}}
\expandafter\def\csname PY@tok@s2\endcsname{\def\PY@tc##1{\textcolor[rgb]{0.73,0.13,0.13}{##1}}}
\expandafter\def\csname PY@tok@sh\endcsname{\def\PY@tc##1{\textcolor[rgb]{0.73,0.13,0.13}{##1}}}
\expandafter\def\csname PY@tok@s1\endcsname{\def\PY@tc##1{\textcolor[rgb]{0.73,0.13,0.13}{##1}}}
\expandafter\def\csname PY@tok@mb\endcsname{\def\PY@tc##1{\textcolor[rgb]{0.40,0.40,0.40}{##1}}}
\expandafter\def\csname PY@tok@mf\endcsname{\def\PY@tc##1{\textcolor[rgb]{0.40,0.40,0.40}{##1}}}
\expandafter\def\csname PY@tok@mh\endcsname{\def\PY@tc##1{\textcolor[rgb]{0.40,0.40,0.40}{##1}}}
\expandafter\def\csname PY@tok@mi\endcsname{\def\PY@tc##1{\textcolor[rgb]{0.40,0.40,0.40}{##1}}}
\expandafter\def\csname PY@tok@il\endcsname{\def\PY@tc##1{\textcolor[rgb]{0.40,0.40,0.40}{##1}}}
\expandafter\def\csname PY@tok@mo\endcsname{\def\PY@tc##1{\textcolor[rgb]{0.40,0.40,0.40}{##1}}}
\expandafter\def\csname PY@tok@ch\endcsname{\let\PY@it=\textit\def\PY@tc##1{\textcolor[rgb]{0.25,0.50,0.50}{##1}}}
\expandafter\def\csname PY@tok@cm\endcsname{\let\PY@it=\textit\def\PY@tc##1{\textcolor[rgb]{0.25,0.50,0.50}{##1}}}
\expandafter\def\csname PY@tok@cpf\endcsname{\let\PY@it=\textit\def\PY@tc##1{\textcolor[rgb]{0.25,0.50,0.50}{##1}}}
\expandafter\def\csname PY@tok@c1\endcsname{\let\PY@it=\textit\def\PY@tc##1{\textcolor[rgb]{0.25,0.50,0.50}{##1}}}
\expandafter\def\csname PY@tok@cs\endcsname{\let\PY@it=\textit\def\PY@tc##1{\textcolor[rgb]{0.25,0.50,0.50}{##1}}}

\def\PYZbs{\char`\\}
\def\PYZus{\char`\_}
\def\PYZob{\char`\{}
\def\PYZcb{\char`\}}
\def\PYZca{\char`\^}
\def\PYZam{\char`\&}
\def\PYZlt{\char`\<}
\def\PYZgt{\char`\>}
\def\PYZsh{\char`\#}
\def\PYZpc{\char`\%}
\def\PYZdl{\char`\$}
\def\PYZhy{\char`\-}
\def\PYZsq{\char`\'}
\def\PYZdq{\char`\"}
\def\PYZti{\char`\~}
% for compatibility with earlier versions
\def\PYZat{@}
\def\PYZlb{[}
\def\PYZrb{]}
\makeatother


    % For linebreaks inside Verbatim environment from package fancyvrb. 
    \makeatletter
        \newbox\Wrappedcontinuationbox 
        \newbox\Wrappedvisiblespacebox 
        \newcommand*\Wrappedvisiblespace {\textcolor{red}{\textvisiblespace}} 
        \newcommand*\Wrappedcontinuationsymbol {\textcolor{red}{\llap{\tiny$\m@th\hookrightarrow$}}} 
        \newcommand*\Wrappedcontinuationindent {3ex } 
        \newcommand*\Wrappedafterbreak {\kern\Wrappedcontinuationindent\copy\Wrappedcontinuationbox} 
        % Take advantage of the already applied Pygments mark-up to insert 
        % potential linebreaks for TeX processing. 
        %        {, <, #, %, $, ' and ": go to next line. 
        %        _, }, ^, &, >, - and ~: stay at end of broken line. 
        % Use of \textquotesingle for straight quote. 
        \newcommand*\Wrappedbreaksatspecials {% 
            \def\PYGZus{\discretionary{\char`\_}{\Wrappedafterbreak}{\char`\_}}% 
            \def\PYGZob{\discretionary{}{\Wrappedafterbreak\char`\{}{\char`\{}}% 
            \def\PYGZcb{\discretionary{\char`\}}{\Wrappedafterbreak}{\char`\}}}% 
            \def\PYGZca{\discretionary{\char`\^}{\Wrappedafterbreak}{\char`\^}}% 
            \def\PYGZam{\discretionary{\char`\&}{\Wrappedafterbreak}{\char`\&}}% 
            \def\PYGZlt{\discretionary{}{\Wrappedafterbreak\char`\<}{\char`\<}}% 
            \def\PYGZgt{\discretionary{\char`\>}{\Wrappedafterbreak}{\char`\>}}% 
            \def\PYGZsh{\discretionary{}{\Wrappedafterbreak\char`\#}{\char`\#}}% 
            \def\PYGZpc{\discretionary{}{\Wrappedafterbreak\char`\%}{\char`\%}}% 
            \def\PYGZdl{\discretionary{}{\Wrappedafterbreak\char`\$}{\char`\$}}% 
            \def\PYGZhy{\discretionary{\char`\-}{\Wrappedafterbreak}{\char`\-}}% 
            \def\PYGZsq{\discretionary{}{\Wrappedafterbreak\textquotesingle}{\textquotesingle}}% 
            \def\PYGZdq{\discretionary{}{\Wrappedafterbreak\char`\"}{\char`\"}}% 
            \def\PYGZti{\discretionary{\char`\~}{\Wrappedafterbreak}{\char`\~}}% 
        } 
        % Some characters . , ; ? ! / are not pygmentized. 
        % This macro makes them "active" and they will insert potential linebreaks 
        \newcommand*\Wrappedbreaksatpunct {% 
            \lccode`\~`\.\lowercase{\def~}{\discretionary{\hbox{\char`\.}}{\Wrappedafterbreak}{\hbox{\char`\.}}}% 
            \lccode`\~`\,\lowercase{\def~}{\discretionary{\hbox{\char`\,}}{\Wrappedafterbreak}{\hbox{\char`\,}}}% 
            \lccode`\~`\;\lowercase{\def~}{\discretionary{\hbox{\char`\;}}{\Wrappedafterbreak}{\hbox{\char`\;}}}% 
            \lccode`\~`\:\lowercase{\def~}{\discretionary{\hbox{\char`\:}}{\Wrappedafterbreak}{\hbox{\char`\:}}}% 
            \lccode`\~`\?\lowercase{\def~}{\discretionary{\hbox{\char`\?}}{\Wrappedafterbreak}{\hbox{\char`\?}}}% 
            \lccode`\~`\!\lowercase{\def~}{\discretionary{\hbox{\char`\!}}{\Wrappedafterbreak}{\hbox{\char`\!}}}% 
            \lccode`\~`\/\lowercase{\def~}{\discretionary{\hbox{\char`\/}}{\Wrappedafterbreak}{\hbox{\char`\/}}}% 
            \catcode`\.\active
            \catcode`\,\active 
            \catcode`\;\active
            \catcode`\:\active
            \catcode`\?\active
            \catcode`\!\active
            \catcode`\/\active 
            \lccode`\~`\~ 	
        }
    \makeatother

    \let\OriginalVerbatim=\Verbatim
    \makeatletter
    \renewcommand{\Verbatim}[1][1]{%
        %\parskip\z@skip
        \sbox\Wrappedcontinuationbox {\Wrappedcontinuationsymbol}%
        \sbox\Wrappedvisiblespacebox {\FV@SetupFont\Wrappedvisiblespace}%
        \def\FancyVerbFormatLine ##1{\hsize\linewidth
            \vtop{\raggedright\hyphenpenalty\z@\exhyphenpenalty\z@
                \doublehyphendemerits\z@\finalhyphendemerits\z@
                \strut ##1\strut}%
        }%
        % If the linebreak is at a space, the latter will be displayed as visible
        % space at end of first line, and a continuation symbol starts next line.
        % Stretch/shrink are however usually zero for typewriter font.
        \def\FV@Space {%
            \nobreak\hskip\z@ plus\fontdimen3\font minus\fontdimen4\font
            \discretionary{\copy\Wrappedvisiblespacebox}{\Wrappedafterbreak}
            {\kern\fontdimen2\font}%
        }%
        
        % Allow breaks at special characters using \PYG... macros.
        \Wrappedbreaksatspecials
        % Breaks at punctuation characters . , ; ? ! and / need catcode=\active 	
        \OriginalVerbatim[#1,codes*=\Wrappedbreaksatpunct]%
    }
    \makeatother

    % Exact colors from NB
    \definecolor{incolor}{HTML}{303F9F}
    \definecolor{outcolor}{HTML}{D84315}
    \definecolor{cellborder}{HTML}{CFCFCF}
    \definecolor{cellbackground}{HTML}{F7F7F7}
    
    % prompt
    \makeatletter
    \newcommand{\boxspacing}{\kern\kvtcb@left@rule\kern\kvtcb@boxsep}
    \makeatother
    \newcommand{\prompt}[4]{
        \ttfamily\llap{{\color{#2}[#3]:\hspace{3pt}#4}}\vspace{-\baselineskip}
    }
    

    
    % Prevent overflowing lines due to hard-to-break entities
    \sloppy 
    % Setup hyperref package
    \hypersetup{
      breaklinks=true,  % so long urls are correctly broken across lines
      colorlinks=true,
      urlcolor=urlcolor,
      linkcolor=linkcolor,
      citecolor=citecolor,
      }
    % Slightly bigger margins than the latex defaults
    
    \geometry{verbose,tmargin=1in,bmargin=1in,lmargin=1in,rmargin=1in}
    
    

\begin{document}
  
%    \maketitle
%	ref: https://stackoverflow.com/questions/3141702/
	\begin{titlepage}
		\vspace*{\fill}
		\begin{center}
 		\normalsize Manual / Referensi Modul \texttt{hidrokit.contrib.taruma}\\
		\huge \texttt{.hk87}: Debit Andal Menggunakan Kurva Durasi\\ 
		\small telah tersedia pada versi hidrokit 0.3.5 \\[0.2cm]
      	\small Berdasarkan \emph{Jupyter Notebook}: \texttt{github-taruma\_hk87\_debit\_andal\_kurva\_durasi.ipynb} \\[0.5cm]
      	\small Modul ini digunakan untuk memperoleh debit andal $Q_{prob}$. Modul ini berdasarkan SNI 6738:2015: Perhitungan Debit Andalan Sungai dengan Kurva Durasi Debit. \\[0.5cm]
		\normalsize oleh Taruma Sakti Megariansyah\\[0.5cm]
      	\normalsize 14 Desember 2019 (1.0.0)\\[1cm]
    	\adjustimage{max size={0.9\linewidth}{1cm}}{hidrokit_logo.jpg}\\
      	\normalsize github.com/taruma/hidrokit
		\end{center}
    	\vspace*{\fill}
	\end{titlepage}
    
    

    
    

    Berdasarkan isu
\href{https://github.com/taruma/hidrokit/issues/87}{\#87}:
\textbf{perhitungan debit andalan sungai dengan kurva durasi debit}

Referensi isu:

\begin{itemize}
\tightlist
\item
  SNI 6738:2015: \textbf{Perhitungan debit andalan sungai dengan kurva
  durasi debit}. Pranala:
  \href{http://sni.litbang.pu.go.id/index.php?r=/sni/new/sni/detail/id/458}{sni.litbang.pu.go.id}
  diakses 10 desember 2019.
\end{itemize}

Deskripsi Permasalahan:

\begin{itemize}
\tightlist
\item
  Mencari nilai debit andal pada \(Q_{80}\), \(Q_{90}\), dan \(Q_{95}\)
  pada periode tertentu (dalam satu tahun, atau setiap bulan).
\end{itemize}

Strategi Penyelesaian Masalah:

\begin{itemize}
\tightlist
\item
  Mengikuti panduan SNI 6738:2015.
\item
  Membuat fungsi dengan input \texttt{pandas.DataFrame} dengan jumlah
  observasi sembarang. Sehingga, penggunaan fungsi akan ditentukan.

  \begin{itemize}
  \tightlist
  \item
    Menyusun berurutan nilai dalam dataframe.
  \item
    Membuat \texttt{list/array} yang merupakan nilai (kumulatif)
    probabilitas.
  \item
    Mengembalikan d
  \end{itemize}
\item
  Mencari nilai \(Q_{80}\), \(Q_{90}\), dan \(Q_{95}\)
\end{itemize}

Catatan:

\begin{itemize}
\tightlist
\item
  Dataset yang digunakan serupa dengan lampiran A dalam SNI 6738:2015.
\end{itemize}

    \hypertarget{persiapan-dan-dataset}{%
\section{PERSIAPAN DAN DATASET}\label{persiapan-dan-dataset}}

    \begin{tcolorbox}[breakable, size=fbox, boxrule=1pt, pad at break*=1mm,colback=cellbackground, colframe=cellborder]
\prompt{In}{incolor}{0}{\boxspacing}
\begin{Verbatim}[commandchars=\\\{\}]
\PY{c+c1}{\PYZsh{} Unduh dataset}
\PY{o}{!}wget \PYZhy{}O data.csv \PY{l+s+s2}{\PYZdq{}https://taruma.github.io/assets/hidrokit\PYZus{}dataset/data\PYZus{}sni\PYZus{}67382015.csv\PYZdq{}} \PYZhy{}q
\PY{n}{FILE} \PY{o}{=} \PY{l+s+s1}{\PYZsq{}}\PY{l+s+s1}{data.csv}\PY{l+s+s1}{\PYZsq{}}
\end{Verbatim}
\end{tcolorbox}

    \begin{tcolorbox}[breakable, size=fbox, boxrule=1pt, pad at break*=1mm,colback=cellbackground, colframe=cellborder]
\prompt{In}{incolor}{0}{\boxspacing}
\begin{Verbatim}[commandchars=\\\{\}]
\PY{k+kn}{import} \PY{n+nn}{numpy} \PY{k}{as} \PY{n+nn}{np}
\PY{k+kn}{import} \PY{n+nn}{pandas} \PY{k}{as} \PY{n+nn}{pd}
\PY{k+kn}{import} \PY{n+nn}{matplotlib}\PY{n+nn}{.}\PY{n+nn}{pyplot} \PY{k}{as} \PY{n+nn}{plt}
\end{Verbatim}
\end{tcolorbox}

    \begin{tcolorbox}[breakable, size=fbox, boxrule=1pt, pad at break*=1mm,colback=cellbackground, colframe=cellborder]
\prompt{In}{incolor}{3}{\boxspacing}
\begin{Verbatim}[commandchars=\\\{\}]
\PY{c+c1}{\PYZsh{} Import dataset}
\PY{n}{dataset} \PY{o}{=} \PY{n}{pd}\PY{o}{.}\PY{n}{read\PYZus{}csv}\PY{p}{(}\PY{n}{FILE}\PY{p}{,} \PY{n}{index\PYZus{}col}\PY{o}{=}\PY{l+m+mi}{0}\PY{p}{,} \PY{n}{header}\PY{o}{=}\PY{l+m+mi}{0}\PY{p}{,} \PY{n}{parse\PYZus{}dates}\PY{o}{=}\PY{k+kc}{True}\PY{p}{)}
\PY{n}{dataset}\PY{o}{.}\PY{n}{info}\PY{p}{(}\PY{p}{)}
\PY{n}{dataset}\PY{o}{.}\PY{n}{head}\PY{p}{(}\PY{p}{)}
\end{Verbatim}
\end{tcolorbox}

    \begin{Verbatim}[commandchars=\\\{\}]
<class 'pandas.core.frame.DataFrame'>
DatetimeIndex: 240 entries, 1982-01-31 to 2001-12-31
Data columns (total 1 columns):
debit    240 non-null float64
dtypes: float64(1)
memory usage: 3.8 KB
    \end{Verbatim}

            \begin{tcolorbox}[breakable, size=fbox, boxrule=.5pt, pad at break*=1mm, opacityfill=0]
\prompt{Out}{outcolor}{3}{\boxspacing}
\begin{Verbatim}[commandchars=\\\{\}]
            debit
1982-01-31  118.0
1982-02-28   63.9
1982-03-31   77.2
1982-04-30  155.0
1982-05-31   39.6
\end{Verbatim}
\end{tcolorbox}
        
    \hypertarget{kode}{%
\section{KODE}\label{kode}}

    \begin{tcolorbox}[breakable, size=fbox, boxrule=1pt, pad at break*=1mm,colback=cellbackground, colframe=cellborder]
\prompt{In}{incolor}{0}{\boxspacing}
\begin{Verbatim}[commandchars=\\\{\}]
\PY{k+kn}{import} \PY{n+nn}{numpy} \PY{k}{as} \PY{n+nn}{np}
\PY{k+kn}{import} \PY{n+nn}{pandas} \PY{k}{as} \PY{n+nn}{pd}

\PY{k}{def} \PY{n+nf}{prob\PYZus{}weibull}\PY{p}{(}\PY{n}{m}\PY{p}{,} \PY{n}{n}\PY{p}{)}\PY{p}{:}
    \PY{k}{return} \PY{n}{m} \PY{o}{/} \PY{p}{(}\PY{n}{n} \PY{o}{+} \PY{l+m+mi}{1}\PY{p}{)} \PY{o}{*} \PY{l+m+mi}{100}

\PY{k}{def} \PY{n+nf}{\PYZus{}array\PYZus{}weibull}\PY{p}{(}\PY{n}{n}\PY{p}{)}\PY{p}{:}
    \PY{k}{return} \PY{n}{np}\PY{o}{.}\PY{n}{array}\PY{p}{(}\PY{p}{[}\PY{n}{prob\PYZus{}weibull}\PY{p}{(}\PY{n}{i}\PY{p}{,} \PY{n}{n}\PY{p}{)} \PY{k}{for} \PY{n}{i} \PY{o+ow}{in} \PY{n+nb}{range}\PY{p}{(}\PY{l+m+mi}{1}\PY{p}{,} \PY{n}{n}\PY{o}{+}\PY{l+m+mi}{1}\PY{p}{)}\PY{p}{]}\PY{p}{)}

\PY{k}{def} \PY{n+nf}{\PYZus{}fdc\PYZus{}xy}\PY{p}{(}\PY{n}{df}\PY{p}{)}\PY{p}{:}
    \PY{n}{n} \PY{o}{=} \PY{n+nb}{len}\PY{p}{(}\PY{n}{df}\PY{o}{.}\PY{n}{index}\PY{p}{)}
    \PY{n}{x} \PY{o}{=} \PY{n}{\PYZus{}array\PYZus{}weibull}\PY{p}{(}\PY{n}{n}\PY{p}{)}
    \PY{n}{y} \PY{o}{=} \PY{n}{df}\PY{o}{.}\PY{n}{sort\PYZus{}values}\PY{p}{(}\PY{n}{ascending}\PY{o}{=}\PY{k+kc}{False}\PY{p}{)}\PY{o}{.}\PY{n}{values}
    \PY{k}{return} \PY{n}{x}\PY{p}{,} \PY{n}{y}

\PY{k}{def} \PY{n+nf}{\PYZus{}interpolate}\PY{p}{(}\PY{n}{probability}\PY{p}{,} \PY{n}{x}\PY{p}{,} \PY{n}{y}\PY{p}{)}\PY{p}{:}
    \PY{k}{return} \PY{p}{\PYZob{}}\PY{n}{p}\PY{p}{:} \PY{n}{np}\PY{o}{.}\PY{n}{interp}\PY{p}{(}\PY{n}{p}\PY{p}{,} \PY{n}{x}\PY{p}{,} \PY{n}{y}\PY{p}{)} \PY{k}{for} \PY{n}{p} \PY{o+ow}{in} \PY{n}{probability}\PY{p}{\PYZcb{}}

\PY{k}{def} \PY{n+nf}{debit\PYZus{}andal}\PY{p}{(}\PY{n}{df}\PY{p}{,} \PY{n}{column}\PY{p}{,} \PY{n}{kind}\PY{o}{=}\PY{l+s+s1}{\PYZsq{}}\PY{l+s+s1}{table}\PY{l+s+s1}{\PYZsq{}}\PY{p}{,} \PY{n}{prob}\PY{o}{=}\PY{p}{[}\PY{l+m+mi}{80}\PY{p}{,} \PY{l+m+mi}{90}\PY{p}{,} \PY{l+m+mi}{95}\PY{p}{]}\PY{p}{)}\PY{p}{:}
    \PY{n}{x}\PY{p}{,} \PY{n}{y} \PY{o}{=} \PY{n}{\PYZus{}fdc\PYZus{}xy}\PY{p}{(}\PY{n}{df}\PY{o}{.}\PY{n}{loc}\PY{p}{[}\PY{p}{:}\PY{p}{,} \PY{n}{column}\PY{p}{]}\PY{p}{)}

    \PY{k}{if} \PY{n}{kind}\PY{o}{.}\PY{n}{lower}\PY{p}{(}\PY{p}{)} \PY{o}{==} \PY{l+s+s1}{\PYZsq{}}\PY{l+s+s1}{array}\PY{l+s+s1}{\PYZsq{}}\PY{p}{:}
        \PY{k}{return} \PY{n}{x}\PY{p}{,} \PY{n}{y}

    \PY{k}{if} \PY{n}{kind}\PY{o}{.}\PY{n}{lower}\PY{p}{(}\PY{p}{)} \PY{o}{==} \PY{l+s+s1}{\PYZsq{}}\PY{l+s+s1}{prob}\PY{l+s+s1}{\PYZsq{}}\PY{p}{:}
        \PY{k}{return} \PY{n}{\PYZus{}interpolate}\PY{p}{(}\PY{n}{prob}\PY{p}{,} \PY{n}{x}\PY{p}{,} \PY{n}{y}\PY{p}{)}

    \PY{k}{if} \PY{n}{kind}\PY{o}{.}\PY{n}{lower}\PY{p}{(}\PY{p}{)} \PY{o}{==} \PY{l+s+s1}{\PYZsq{}}\PY{l+s+s1}{table}\PY{l+s+s1}{\PYZsq{}}\PY{p}{:}
        \PY{n}{data} \PY{o}{=} \PY{p}{\PYZob{}}
            \PY{l+s+s1}{\PYZsq{}}\PY{l+s+s1}{idx}\PY{l+s+s1}{\PYZsq{}}\PY{p}{:} \PY{n}{df}\PY{o}{.}\PY{n}{loc}\PY{p}{[}\PY{p}{:}\PY{p}{,} \PY{n}{column}\PY{p}{]}\PY{o}{.}\PY{n}{sort\PYZus{}values}\PY{p}{(}\PY{n}{ascending}\PY{o}{=}\PY{k+kc}{False}\PY{p}{)}\PY{o}{.}\PY{n}{index}\PY{p}{,}
            \PY{l+s+s1}{\PYZsq{}}\PY{l+s+s1}{rank}\PY{l+s+s1}{\PYZsq{}}\PY{p}{:} \PY{n+nb}{list}\PY{p}{(}\PY{n+nb}{range}\PY{p}{(}\PY{l+m+mi}{1}\PY{p}{,} \PY{n+nb}{len}\PY{p}{(}\PY{n}{df}\PY{o}{.}\PY{n}{index}\PY{p}{)}\PY{o}{+}\PY{l+m+mi}{1}\PY{p}{)}\PY{p}{)}\PY{p}{,}
            \PY{l+s+s1}{\PYZsq{}}\PY{l+s+s1}{prob}\PY{l+s+s1}{\PYZsq{}}\PY{p}{:} \PY{n}{x}\PY{p}{,}
            \PY{l+s+s1}{\PYZsq{}}\PY{l+s+s1}{data}\PY{l+s+s1}{\PYZsq{}}\PY{p}{:} \PY{n}{y}\PY{p}{,}            
        \PY{p}{\PYZcb{}}
        \PY{k}{return} \PY{n}{pd}\PY{o}{.}\PY{n}{DataFrame}\PY{p}{(}\PY{n}{data}\PY{p}{)}

\PY{k}{def} \PY{n+nf}{debit\PYZus{}andal\PYZus{}bulanan}\PY{p}{(}\PY{n}{df}\PY{p}{,} \PY{n}{column}\PY{p}{,} \PY{o}{*}\PY{o}{*}\PY{n}{kwargs}\PY{p}{)}\PY{p}{:}
    \PY{k}{return} \PY{p}{\PYZob{}}
        \PY{n}{m}\PY{p}{:} \PY{n}{debit\PYZus{}andal}\PY{p}{(}\PY{n}{df}\PY{p}{[}\PY{n}{df}\PY{o}{.}\PY{n}{index}\PY{o}{.}\PY{n}{month} \PY{o}{==} \PY{n}{m}\PY{p}{]}\PY{p}{,} \PY{n}{column}\PY{p}{,} \PY{o}{*}\PY{o}{*}\PY{n}{kwargs}\PY{p}{)} 
        \PY{k}{for} \PY{n}{m} \PY{o+ow}{in} \PY{n+nb}{range}\PY{p}{(}\PY{l+m+mi}{1}\PY{p}{,} \PY{l+m+mi}{13}\PY{p}{)}
    \PY{p}{\PYZcb{}}
\end{Verbatim}
\end{tcolorbox}

    \hypertarget{penggunaan}{%
\section{PENGGUNAAN}\label{penggunaan}}

    \hypertarget{fungsi-.debit_andal}{%
\subsection{\texorpdfstring{Fungsi
\texttt{.debit\_andal()}}{Fungsi .debit\_andal()}}\label{fungsi-.debit_andal}}

Pada fungsi ini terdapat parameter yang perlu diperhatikan selain
\texttt{df} dan \texttt{column} yaitu \texttt{kind}. Parameter
\texttt{kind} menentukan hasil keluaran dari fungsi. Berikut nilai yang
diterima oleh parameter \texttt{kind}:

\begin{itemize}
\tightlist
\item
  \texttt{\textquotesingle{}array\textquotesingle{}}: keluaran berupa
  \emph{tuple} berisi dua \texttt{np.array} yaitu \texttt{x} (untuk
  sumbu x, probabilitas weibull) dan \texttt{y} (untuk sumbu y, nilai
  debit yang telah diurutkan).
\item
  \texttt{\textquotesingle{}table\textquotesingle{}} (\textbf{default})
  : keluaran berupa \texttt{pandas.DataFrame} tabelaris yang berisikan
  kolom \texttt{idx} (indeks/tanggal kejadian), \texttt{rank} (ranking),
  \texttt{prob} (probabilitas weibull), \texttt{data} (nilai yang telah
  diurutkan).
\item
  \texttt{\textquotesingle{}prob\textquotesingle{}}: keluaran berupa
  \emph{dictionary} dengan \emph{key} sebagai nilai probabilitas dan
  \emph{value} sebagai nilai data yang dicari. Nilai tersebut diperoleh
  menggunakan fungsi interpolasi dari \texttt{numpy} yaitu
  \texttt{np.interp()}.
\end{itemize}

    \hypertarget{kindarray}{%
\subsubsection{\texorpdfstring{\texttt{kind=\textquotesingle{}array\textquotesingle{}}}{kind='array'}}\label{kindarray}}

    \begin{tcolorbox}[breakable, size=fbox, boxrule=1pt, pad at break*=1mm,colback=cellbackground, colframe=cellborder]
\prompt{In}{incolor}{5}{\boxspacing}
\begin{Verbatim}[commandchars=\\\{\}]
\PY{n}{x}\PY{p}{,} \PY{n}{y} \PY{o}{=} \PY{n}{debit\PYZus{}andal}\PY{p}{(}\PY{n}{dataset}\PY{p}{,} \PY{l+s+s1}{\PYZsq{}}\PY{l+s+s1}{debit}\PY{l+s+s1}{\PYZsq{}}\PY{p}{,} \PY{n}{kind}\PY{o}{=}\PY{l+s+s1}{\PYZsq{}}\PY{l+s+s1}{array}\PY{l+s+s1}{\PYZsq{}}\PY{p}{)}
\PY{n+nb}{print}\PY{p}{(}\PY{l+s+sa}{f}\PY{l+s+s1}{\PYZsq{}}\PY{l+s+s1}{len(x) = }\PY{l+s+s1}{\PYZob{}}\PY{l+s+s1}{len(x)\PYZcb{}}\PY{l+s+se}{\PYZbs{}t}\PY{l+s+s1}{x[:5] = }\PY{l+s+si}{\PYZob{}x[:5]\PYZcb{}}\PY{l+s+s1}{\PYZsq{}}\PY{p}{)}
\PY{n+nb}{print}\PY{p}{(}\PY{l+s+sa}{f}\PY{l+s+s1}{\PYZsq{}}\PY{l+s+s1}{len(y) = }\PY{l+s+s1}{\PYZob{}}\PY{l+s+s1}{len(y)\PYZcb{}}\PY{l+s+se}{\PYZbs{}t}\PY{l+s+s1}{y[:5] = }\PY{l+s+si}{\PYZob{}y[:5]\PYZcb{}}\PY{l+s+s1}{\PYZsq{}}\PY{p}{)}
\end{Verbatim}
\end{tcolorbox}

    \begin{Verbatim}[commandchars=\\\{\}]
len(x) = 240    x[:5] = [0.41493776 0.82987552 1.24481328 1.65975104 2.0746888 ]
len(y) = 240    y[:5] = [226. 210. 194. 184. 184.]
    \end{Verbatim}

    \hypertarget{kindtable-default}{%
\subsubsection{\texorpdfstring{\texttt{kind=\textquotesingle{}table\textquotesingle{}}
(\textbf{default})}{kind='table' (default)}}\label{kindtable-default}}

    \begin{tcolorbox}[breakable, size=fbox, boxrule=1pt, pad at break*=1mm,colback=cellbackground, colframe=cellborder]
\prompt{In}{incolor}{6}{\boxspacing}
\begin{Verbatim}[commandchars=\\\{\}]
\PY{n}{debit\PYZus{}andal}\PY{p}{(}\PY{n}{dataset}\PY{p}{,} \PY{l+s+s1}{\PYZsq{}}\PY{l+s+s1}{debit}\PY{l+s+s1}{\PYZsq{}}\PY{p}{)} \PY{c+c1}{\PYZsh{} atau debit\PYZus{}andal(dataset, \PYZsq{}debit\PYZsq{}, kind=\PYZsq{}table\PYZsq{})}
\end{Verbatim}
\end{tcolorbox}

            \begin{tcolorbox}[breakable, size=fbox, boxrule=.5pt, pad at break*=1mm, opacityfill=0]
\prompt{Out}{outcolor}{6}{\boxspacing}
\begin{Verbatim}[commandchars=\\\{\}]
           idx  rank       prob   data
0   2001-11-30     1   0.414938  226.0
1   1986-03-31     2   0.829876  210.0
2   2001-04-30     3   1.244813  194.0
3   1996-11-30     4   1.659751  184.0
4   1988-01-31     5   2.074689  184.0
..         {\ldots}   {\ldots}        {\ldots}    {\ldots}
235 1982-10-31   236  97.925311    6.2
236 1994-09-30   237  98.340249    6.0
237 1991-08-31   238  98.755187    5.7
238 1994-08-31   239  99.170124    5.3
239 1983-09-30   240  99.585062    2.2

[240 rows x 4 columns]
\end{Verbatim}
\end{tcolorbox}
        
    \hypertarget{kindprob}{%
\subsubsection{\texorpdfstring{\texttt{kind=\textquotesingle{}prob\textquotesingle{}}}{kind='prob'}}\label{kindprob}}

Nilai probabilitas yang digunakan yaitu \(Q_{80}\), \(Q_{90}\),
\(Q_{95}\) atau \texttt{{[}80,\ 90,\ 95}{]}

    \begin{tcolorbox}[breakable, size=fbox, boxrule=1pt, pad at break*=1mm,colback=cellbackground, colframe=cellborder]
\prompt{In}{incolor}{7}{\boxspacing}
\begin{Verbatim}[commandchars=\\\{\}]
\PY{n}{debit\PYZus{}andal}\PY{p}{(}\PY{n}{dataset}\PY{p}{,} \PY{l+s+s1}{\PYZsq{}}\PY{l+s+s1}{debit}\PY{l+s+s1}{\PYZsq{}}\PY{p}{,} \PY{n}{kind}\PY{o}{=}\PY{l+s+s1}{\PYZsq{}}\PY{l+s+s1}{prob}\PY{l+s+s1}{\PYZsq{}}\PY{p}{)}
\PY{c+c1}{\PYZsh{} atau debit\PYZus{}andal(dataset, \PYZsq{}debit\PYZsq{}, kind=\PYZsq{}prob\PYZsq{}, prob=[80, 90, 95])}
\end{Verbatim}
\end{tcolorbox}

            \begin{tcolorbox}[breakable, size=fbox, boxrule=.5pt, pad at break*=1mm, opacityfill=0]
\prompt{Out}{outcolor}{7}{\boxspacing}
\begin{Verbatim}[commandchars=\\\{\}]
\{80: 27.12, 90: 13.330000000000005, 95: 7.249999999999974\}
\end{Verbatim}
\end{tcolorbox}
        
    Contoh menggunakan nilai probabilitas yang berbeda
\texttt{{[}30,\ 35,\ 70,\ 85,\ 95{]}}.

    \begin{tcolorbox}[breakable, size=fbox, boxrule=1pt, pad at break*=1mm,colback=cellbackground, colframe=cellborder]
\prompt{In}{incolor}{8}{\boxspacing}
\begin{Verbatim}[commandchars=\\\{\}]
\PY{n}{debit\PYZus{}andal}\PY{p}{(}\PY{n}{dataset}\PY{p}{,} \PY{l+s+s1}{\PYZsq{}}\PY{l+s+s1}{debit}\PY{l+s+s1}{\PYZsq{}}\PY{p}{,} \PY{n}{kind}\PY{o}{=}\PY{l+s+s1}{\PYZsq{}}\PY{l+s+s1}{prob}\PY{l+s+s1}{\PYZsq{}}\PY{p}{,} \PY{n}{prob}\PY{o}{=}\PY{p}{[}\PY{l+m+mi}{30}\PY{p}{,} \PY{l+m+mi}{35}\PY{p}{,} \PY{l+m+mi}{70}\PY{p}{,} \PY{l+m+mi}{85}\PY{p}{,} \PY{l+m+mi}{95}\PY{p}{]}\PY{p}{)}
\end{Verbatim}
\end{tcolorbox}

            \begin{tcolorbox}[breakable, size=fbox, boxrule=.5pt, pad at break*=1mm, opacityfill=0]
\prompt{Out}{outcolor}{8}{\boxspacing}
\begin{Verbatim}[commandchars=\\\{\}]
\{30: 103.7,
 35: 96.165,
 70: 40.469999999999985,
 85: 19.1,
 95: 7.249999999999974\}
\end{Verbatim}
\end{tcolorbox}
        
    \hypertarget{fungsi-.debit_andal_bulanan}{%
\subsection{\texorpdfstring{Fungsi
\texttt{.debit\_andal\_bulanan()}}{Fungsi .debit\_andal\_bulanan()}}\label{fungsi-.debit_andal_bulanan}}

Fungsi ini merupakan pengembangan lebih lanjut dari
\texttt{.debit\_andal()} yang dapat digunakan untuk membuat kurva durasi
debit per bulan. Fungsi \texttt{.debit\_andal\_bulanan()} dapat menerima
parameter yang sama dengan \texttt{.debit\_andal()} seperti
\texttt{kind} dan \texttt{prob}.

\emph{Key} pada hasil keluaran fungsi ini menunjukkan bulan, contoh:
\texttt{{[}1{]}} mengartikan bulan ke-1 (Januari).

    \begin{tcolorbox}[breakable, size=fbox, boxrule=1pt, pad at break*=1mm,colback=cellbackground, colframe=cellborder]
\prompt{In}{incolor}{17}{\boxspacing}
\begin{Verbatim}[commandchars=\\\{\}]
\PY{n}{bulanan} \PY{o}{=} \PY{n}{debit\PYZus{}andal\PYZus{}bulanan}\PY{p}{(}\PY{n}{dataset}\PY{p}{,} \PY{l+s+s1}{\PYZsq{}}\PY{l+s+s1}{debit}\PY{l+s+s1}{\PYZsq{}}\PY{p}{)}
\PY{n+nb}{print}\PY{p}{(}\PY{l+s+sa}{f}\PY{l+s+s1}{\PYZsq{}}\PY{l+s+s1}{keys = }\PY{l+s+s1}{\PYZob{}}\PY{l+s+s1}{bulanan.keys()\PYZcb{}}\PY{l+s+s1}{\PYZsq{}}\PY{p}{)}
\PY{n+nb}{print}\PY{p}{(}\PY{l+s+sa}{f}\PY{l+s+s1}{\PYZsq{}}\PY{l+s+s1}{values = }\PY{l+s+s1}{\PYZob{}}\PY{l+s+s1}{type(bulanan[1])\PYZcb{}}\PY{l+s+s1}{\PYZsq{}}\PY{p}{)}
\PY{c+c1}{\PYZsh{} out: berupa dataframe karena nilai kind=\PYZsq{}table\PYZsq{} (default) fungsi debit\PYZus{}andal()}
\end{Verbatim}
\end{tcolorbox}

    \begin{Verbatim}[commandchars=\\\{\}]
keys = dict\_keys([1, 2, 3, 4, 5, 6, 7, 8, 9, 10, 11, 12])
values = <class 'pandas.core.frame.DataFrame'>
    \end{Verbatim}

    Menampilkan tabel untuk bulan Maret (ke-3)

    \begin{tcolorbox}[breakable, size=fbox, boxrule=1pt, pad at break*=1mm,colback=cellbackground, colframe=cellborder]
\prompt{In}{incolor}{25}{\boxspacing}
\begin{Verbatim}[commandchars=\\\{\}]
\PY{n}{bulanan}\PY{p}{[}\PY{l+m+mi}{3}\PY{p}{]}\PY{o}{.}\PY{n}{head}\PY{p}{(}\PY{p}{)}
\end{Verbatim}
\end{tcolorbox}

            \begin{tcolorbox}[breakable, size=fbox, boxrule=.5pt, pad at break*=1mm, opacityfill=0]
\prompt{Out}{outcolor}{25}{\boxspacing}
\begin{Verbatim}[commandchars=\\\{\}]
         idx  rank       prob   data
0 1986-03-31     1   4.761905  210.0
1 1998-03-31     2   9.523810  174.0
2 1992-03-31     3  14.285714  173.0
3 1993-03-31     4  19.047619  164.0
4 1991-03-31     5  23.809524  155.0
\end{Verbatim}
\end{tcolorbox}
        
    Contoh menampilkan nilai \(Q_{80}, Q_{85}, Q_{90}, Q_{95}\) untuk setiap
bulan

    \begin{tcolorbox}[breakable, size=fbox, boxrule=1pt, pad at break*=1mm,colback=cellbackground, colframe=cellborder]
\prompt{In}{incolor}{23}{\boxspacing}
\begin{Verbatim}[commandchars=\\\{\}]
\PY{n}{bulanan\PYZus{}prob} \PY{o}{=} \PY{n}{debit\PYZus{}andal\PYZus{}bulanan}\PY{p}{(}
    \PY{n}{dataset}\PY{p}{,} \PY{l+s+s1}{\PYZsq{}}\PY{l+s+s1}{debit}\PY{l+s+s1}{\PYZsq{}}\PY{p}{,} \PY{n}{kind}\PY{o}{=}\PY{l+s+s1}{\PYZsq{}}\PY{l+s+s1}{prob}\PY{l+s+s1}{\PYZsq{}}\PY{p}{,} \PY{n}{prob}\PY{o}{=}\PY{p}{[}\PY{l+m+mi}{80}\PY{p}{,} \PY{l+m+mi}{85}\PY{p}{,} \PY{l+m+mi}{90}\PY{p}{,} \PY{l+m+mi}{95}\PY{p}{]}
\PY{p}{)}
\PY{k}{for} \PY{n}{key}\PY{p}{,} \PY{n}{value} \PY{o+ow}{in} \PY{n}{bulanan\PYZus{}prob}\PY{o}{.}\PY{n}{items}\PY{p}{(}\PY{p}{)}\PY{p}{:}
    \PY{n+nb}{print}\PY{p}{(}\PY{l+s+s1}{\PYZsq{}}\PY{l+s+s1}{Bulan ke\PYZhy{}}\PY{l+s+s1}{\PYZsq{}}\PY{p}{,} \PY{n}{key}\PY{p}{,} \PY{l+s+s1}{\PYZsq{}}\PY{l+s+s1}{:}\PY{l+s+se}{\PYZbs{}t}\PY{l+s+s1}{\PYZsq{}}\PY{p}{,} \PY{n}{value}\PY{p}{,} \PY{n}{sep}\PY{o}{=}\PY{l+s+s1}{\PYZsq{}}\PY{l+s+s1}{\PYZsq{}}\PY{p}{)}
\end{Verbatim}
\end{tcolorbox}

    \begin{Verbatim}[commandchars=\\\{\}]
Bulan ke-1:     \{80: 74.0, 85: 72.05499999999999, 90: 68.11, 95:
56.67999999999998\}
Bulan ke-2:     \{80: 66.16, 85: 63.93, 90: 57.42000000000001, 95:
52.04499999999999\}
Bulan ke-3:     \{80: 78.67999999999999, 85: 70.82499999999999, 90:
57.01000000000001, 95: 36.219999999999956\}
Bulan ke-4:     \{80: 102.0, 85: 93.945, 90: 74.61000000000001, 95:
44.47999999999993\}
Bulan ke-5:     \{80: 55.599999999999994, 85: 45.329999999999984, 90: 40.02, 95:
39.03\}
Bulan ke-6:     \{80: 33.339999999999996, 85: 28.319999999999993, 90:
24.270000000000003, 95: 14.589999999999979\}
Bulan ke-7:     \{80: 14.419999999999998, 85: 13.86, 90: 10.020000000000003, 95:
7.319999999999995\}
Bulan ke-8:     \{80: 7.039999999999999, 85: 6.359999999999999, 90:
5.760000000000001, 95: 5.3199999999999985\}
Bulan ke-9:     \{80: 6.92, 85: 6.475, 90: 6.04, 95: 2.3899999999999917\}
Bulan ke-10:    \{80: 10.019999999999998, 85: 8.86, 90: 6.5500000000000025, 95:
6.205\}
Bulan ke-11:    \{80: 36.279999999999994, 85: 24.249999999999986, 90:
14.920000000000007, 95: 10.01499999999999\}
Bulan ke-12:    \{80: 65.16, 85: 49.49999999999998, 90: 42.480000000000004, 95:
37.249999999999986\}
    \end{Verbatim}

    \hypertarget{changelog}{%
\section{Changelog}\label{changelog}}

\begin{verbatim}
- 20191214 - 1.0.0 - Initial
\end{verbatim}

\hypertarget{copyright-2019-taruma-sakti-megariansyah}{%
\paragraph{\texorpdfstring{Copyright © 2019
\href{https://taruma.github.io}{Taruma Sakti
Megariansyah}}{Copyright © 2019 Taruma Sakti Megariansyah}}\label{copyright-2019-taruma-sakti-megariansyah}}

Source code in this notebook is licensed under a
\href{https://choosealicense.com/licenses/mit/}{MIT License}. Data in
this notebook is licensed under a
\href{https://creativecommons.org/licenses/by/4.0/}{Creative Common
Attribution 4.0 International}.


    % Add a bibliography block to the postdoc
    
    
    
\end{document}
