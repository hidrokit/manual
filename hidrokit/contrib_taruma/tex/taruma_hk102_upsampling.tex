\documentclass[11pt]{article}

    \usepackage[breakable]{tcolorbox}
    \usepackage{parskip} % Stop auto-indenting (to mimic markdown behaviour)
    
    \usepackage{iftex}
    \ifPDFTeX
    	\usepackage[T1]{fontenc}
    	\usepackage{mathpazo}
    \else
    	\usepackage{fontspec}
    \fi

    % Basic figure setup, for now with no caption control since it's done
    % automatically by Pandoc (which extracts ![](path) syntax from Markdown).
    \usepackage{graphicx}
    % Maintain compatibility with old templates. Remove in nbconvert 6.0
    \let\Oldincludegraphics\includegraphics
    % Ensure that by default, figures have no caption (until we provide a
    % proper Figure object with a Caption API and a way to capture that
    % in the conversion process - todo).
    \usepackage{caption}
    \DeclareCaptionFormat{nocaption}{}
    \captionsetup{format=nocaption,aboveskip=0pt,belowskip=0pt}

    \usepackage[Export]{adjustbox} % Used to constrain images to a maximum size
    \adjustboxset{max size={0.9\linewidth}{0.9\paperheight}}
    \usepackage{float}
    \floatplacement{figure}{H} % forces figures to be placed at the correct location
    \usepackage{xcolor} % Allow colors to be defined
    \usepackage{enumerate} % Needed for markdown enumerations to work
    \usepackage{geometry} % Used to adjust the document margins
    \usepackage{amsmath} % Equations
    \usepackage{amssymb} % Equations
    \usepackage{textcomp} % defines textquotesingle
    % Hack from http://tex.stackexchange.com/a/47451/13684:
    \AtBeginDocument{%
        \def\PYZsq{\textquotesingle}% Upright quotes in Pygmentized code
    }
    \usepackage{upquote} % Upright quotes for verbatim code
    \usepackage{eurosym} % defines \euro
    \usepackage[mathletters]{ucs} % Extended unicode (utf-8) support
    \usepackage{fancyvrb} % verbatim replacement that allows latex
    \usepackage{grffile} % extends the file name processing of package graphics 
                         % to support a larger range
    \makeatletter % fix for grffile with XeLaTeX
    \def\Gread@@xetex#1{%
      \IfFileExists{"\Gin@base".bb}%
      {\Gread@eps{\Gin@base.bb}}%
      {\Gread@@xetex@aux#1}%
    }
    \makeatother

    % The hyperref package gives us a pdf with properly built
    % internal navigation ('pdf bookmarks' for the table of contents,
    % internal cross-reference links, web links for URLs, etc.)
    \usepackage{hyperref}
    % The default LaTeX title has an obnoxious amount of whitespace. By default,
    % titling removes some of it. It also provides customization options.
    \usepackage{titling}
    \usepackage{longtable} % longtable support required by pandoc >1.10
    \usepackage{booktabs}  % table support for pandoc > 1.12.2
    \usepackage[inline]{enumitem} % IRkernel/repr support (it uses the enumerate* environment)
    \usepackage[normalem]{ulem} % ulem is needed to support strikethroughs (\sout)
                                % normalem makes italics be italics, not underlines
    \usepackage{mathrsfs}
    

    
    % Colors for the hyperref package
    \definecolor{urlcolor}{rgb}{0,.145,.698}
    \definecolor{linkcolor}{rgb}{.71,0.21,0.01}
    \definecolor{citecolor}{rgb}{.12,.54,.11}

    % ANSI colors
    \definecolor{ansi-black}{HTML}{3E424D}
    \definecolor{ansi-black-intense}{HTML}{282C36}
    \definecolor{ansi-red}{HTML}{E75C58}
    \definecolor{ansi-red-intense}{HTML}{B22B31}
    \definecolor{ansi-green}{HTML}{00A250}
    \definecolor{ansi-green-intense}{HTML}{007427}
    \definecolor{ansi-yellow}{HTML}{DDB62B}
    \definecolor{ansi-yellow-intense}{HTML}{B27D12}
    \definecolor{ansi-blue}{HTML}{208FFB}
    \definecolor{ansi-blue-intense}{HTML}{0065CA}
    \definecolor{ansi-magenta}{HTML}{D160C4}
    \definecolor{ansi-magenta-intense}{HTML}{A03196}
    \definecolor{ansi-cyan}{HTML}{60C6C8}
    \definecolor{ansi-cyan-intense}{HTML}{258F8F}
    \definecolor{ansi-white}{HTML}{C5C1B4}
    \definecolor{ansi-white-intense}{HTML}{A1A6B2}
    \definecolor{ansi-default-inverse-fg}{HTML}{FFFFFF}
    \definecolor{ansi-default-inverse-bg}{HTML}{000000}

    % commands and environments needed by pandoc snippets
    % extracted from the output of `pandoc -s`
    \providecommand{\tightlist}{%
      \setlength{\itemsep}{0pt}\setlength{\parskip}{0pt}}
    \DefineVerbatimEnvironment{Highlighting}{Verbatim}{commandchars=\\\{\}}
    % Add ',fontsize=\small' for more characters per line
    \newenvironment{Shaded}{}{}
    \newcommand{\KeywordTok}[1]{\textcolor[rgb]{0.00,0.44,0.13}{\textbf{{#1}}}}
    \newcommand{\DataTypeTok}[1]{\textcolor[rgb]{0.56,0.13,0.00}{{#1}}}
    \newcommand{\DecValTok}[1]{\textcolor[rgb]{0.25,0.63,0.44}{{#1}}}
    \newcommand{\BaseNTok}[1]{\textcolor[rgb]{0.25,0.63,0.44}{{#1}}}
    \newcommand{\FloatTok}[1]{\textcolor[rgb]{0.25,0.63,0.44}{{#1}}}
    \newcommand{\CharTok}[1]{\textcolor[rgb]{0.25,0.44,0.63}{{#1}}}
    \newcommand{\StringTok}[1]{\textcolor[rgb]{0.25,0.44,0.63}{{#1}}}
    \newcommand{\CommentTok}[1]{\textcolor[rgb]{0.38,0.63,0.69}{\textit{{#1}}}}
    \newcommand{\OtherTok}[1]{\textcolor[rgb]{0.00,0.44,0.13}{{#1}}}
    \newcommand{\AlertTok}[1]{\textcolor[rgb]{1.00,0.00,0.00}{\textbf{{#1}}}}
    \newcommand{\FunctionTok}[1]{\textcolor[rgb]{0.02,0.16,0.49}{{#1}}}
    \newcommand{\RegionMarkerTok}[1]{{#1}}
    \newcommand{\ErrorTok}[1]{\textcolor[rgb]{1.00,0.00,0.00}{\textbf{{#1}}}}
    \newcommand{\NormalTok}[1]{{#1}}
    
    % Additional commands for more recent versions of Pandoc
    \newcommand{\ConstantTok}[1]{\textcolor[rgb]{0.53,0.00,0.00}{{#1}}}
    \newcommand{\SpecialCharTok}[1]{\textcolor[rgb]{0.25,0.44,0.63}{{#1}}}
    \newcommand{\VerbatimStringTok}[1]{\textcolor[rgb]{0.25,0.44,0.63}{{#1}}}
    \newcommand{\SpecialStringTok}[1]{\textcolor[rgb]{0.73,0.40,0.53}{{#1}}}
    \newcommand{\ImportTok}[1]{{#1}}
    \newcommand{\DocumentationTok}[1]{\textcolor[rgb]{0.73,0.13,0.13}{\textit{{#1}}}}
    \newcommand{\AnnotationTok}[1]{\textcolor[rgb]{0.38,0.63,0.69}{\textbf{\textit{{#1}}}}}
    \newcommand{\CommentVarTok}[1]{\textcolor[rgb]{0.38,0.63,0.69}{\textbf{\textit{{#1}}}}}
    \newcommand{\VariableTok}[1]{\textcolor[rgb]{0.10,0.09,0.49}{{#1}}}
    \newcommand{\ControlFlowTok}[1]{\textcolor[rgb]{0.00,0.44,0.13}{\textbf{{#1}}}}
    \newcommand{\OperatorTok}[1]{\textcolor[rgb]{0.40,0.40,0.40}{{#1}}}
    \newcommand{\BuiltInTok}[1]{{#1}}
    \newcommand{\ExtensionTok}[1]{{#1}}
    \newcommand{\PreprocessorTok}[1]{\textcolor[rgb]{0.74,0.48,0.00}{{#1}}}
    \newcommand{\AttributeTok}[1]{\textcolor[rgb]{0.49,0.56,0.16}{{#1}}}
    \newcommand{\InformationTok}[1]{\textcolor[rgb]{0.38,0.63,0.69}{\textbf{\textit{{#1}}}}}
    \newcommand{\WarningTok}[1]{\textcolor[rgb]{0.38,0.63,0.69}{\textbf{\textit{{#1}}}}}
    
    
    % Define a nice break command that doesn't care if a line doesn't already
    % exist.
    \def\br{\hspace*{\fill} \\* }
    % Math Jax compatibility definitions
    \def\gt{>}
    \def\lt{<}
    \let\Oldtex\TeX
    \let\Oldlatex\LaTeX
    \renewcommand{\TeX}{\textrm{\Oldtex}}
    \renewcommand{\LaTeX}{\textrm{\Oldlatex}}
    % Document parameters
    % Document title
    \title{taruma\_hk102\_upsampling}
    
    
    
    
    
% Pygments definitions
\makeatletter
\def\PY@reset{\let\PY@it=\relax \let\PY@bf=\relax%
    \let\PY@ul=\relax \let\PY@tc=\relax%
    \let\PY@bc=\relax \let\PY@ff=\relax}
\def\PY@tok#1{\csname PY@tok@#1\endcsname}
\def\PY@toks#1+{\ifx\relax#1\empty\else%
    \PY@tok{#1}\expandafter\PY@toks\fi}
\def\PY@do#1{\PY@bc{\PY@tc{\PY@ul{%
    \PY@it{\PY@bf{\PY@ff{#1}}}}}}}
\def\PY#1#2{\PY@reset\PY@toks#1+\relax+\PY@do{#2}}

\expandafter\def\csname PY@tok@w\endcsname{\def\PY@tc##1{\textcolor[rgb]{0.73,0.73,0.73}{##1}}}
\expandafter\def\csname PY@tok@c\endcsname{\let\PY@it=\textit\def\PY@tc##1{\textcolor[rgb]{0.25,0.50,0.50}{##1}}}
\expandafter\def\csname PY@tok@cp\endcsname{\def\PY@tc##1{\textcolor[rgb]{0.74,0.48,0.00}{##1}}}
\expandafter\def\csname PY@tok@k\endcsname{\let\PY@bf=\textbf\def\PY@tc##1{\textcolor[rgb]{0.00,0.50,0.00}{##1}}}
\expandafter\def\csname PY@tok@kp\endcsname{\def\PY@tc##1{\textcolor[rgb]{0.00,0.50,0.00}{##1}}}
\expandafter\def\csname PY@tok@kt\endcsname{\def\PY@tc##1{\textcolor[rgb]{0.69,0.00,0.25}{##1}}}
\expandafter\def\csname PY@tok@o\endcsname{\def\PY@tc##1{\textcolor[rgb]{0.40,0.40,0.40}{##1}}}
\expandafter\def\csname PY@tok@ow\endcsname{\let\PY@bf=\textbf\def\PY@tc##1{\textcolor[rgb]{0.67,0.13,1.00}{##1}}}
\expandafter\def\csname PY@tok@nb\endcsname{\def\PY@tc##1{\textcolor[rgb]{0.00,0.50,0.00}{##1}}}
\expandafter\def\csname PY@tok@nf\endcsname{\def\PY@tc##1{\textcolor[rgb]{0.00,0.00,1.00}{##1}}}
\expandafter\def\csname PY@tok@nc\endcsname{\let\PY@bf=\textbf\def\PY@tc##1{\textcolor[rgb]{0.00,0.00,1.00}{##1}}}
\expandafter\def\csname PY@tok@nn\endcsname{\let\PY@bf=\textbf\def\PY@tc##1{\textcolor[rgb]{0.00,0.00,1.00}{##1}}}
\expandafter\def\csname PY@tok@ne\endcsname{\let\PY@bf=\textbf\def\PY@tc##1{\textcolor[rgb]{0.82,0.25,0.23}{##1}}}
\expandafter\def\csname PY@tok@nv\endcsname{\def\PY@tc##1{\textcolor[rgb]{0.10,0.09,0.49}{##1}}}
\expandafter\def\csname PY@tok@no\endcsname{\def\PY@tc##1{\textcolor[rgb]{0.53,0.00,0.00}{##1}}}
\expandafter\def\csname PY@tok@nl\endcsname{\def\PY@tc##1{\textcolor[rgb]{0.63,0.63,0.00}{##1}}}
\expandafter\def\csname PY@tok@ni\endcsname{\let\PY@bf=\textbf\def\PY@tc##1{\textcolor[rgb]{0.60,0.60,0.60}{##1}}}
\expandafter\def\csname PY@tok@na\endcsname{\def\PY@tc##1{\textcolor[rgb]{0.49,0.56,0.16}{##1}}}
\expandafter\def\csname PY@tok@nt\endcsname{\let\PY@bf=\textbf\def\PY@tc##1{\textcolor[rgb]{0.00,0.50,0.00}{##1}}}
\expandafter\def\csname PY@tok@nd\endcsname{\def\PY@tc##1{\textcolor[rgb]{0.67,0.13,1.00}{##1}}}
\expandafter\def\csname PY@tok@s\endcsname{\def\PY@tc##1{\textcolor[rgb]{0.73,0.13,0.13}{##1}}}
\expandafter\def\csname PY@tok@sd\endcsname{\let\PY@it=\textit\def\PY@tc##1{\textcolor[rgb]{0.73,0.13,0.13}{##1}}}
\expandafter\def\csname PY@tok@si\endcsname{\let\PY@bf=\textbf\def\PY@tc##1{\textcolor[rgb]{0.73,0.40,0.53}{##1}}}
\expandafter\def\csname PY@tok@se\endcsname{\let\PY@bf=\textbf\def\PY@tc##1{\textcolor[rgb]{0.73,0.40,0.13}{##1}}}
\expandafter\def\csname PY@tok@sr\endcsname{\def\PY@tc##1{\textcolor[rgb]{0.73,0.40,0.53}{##1}}}
\expandafter\def\csname PY@tok@ss\endcsname{\def\PY@tc##1{\textcolor[rgb]{0.10,0.09,0.49}{##1}}}
\expandafter\def\csname PY@tok@sx\endcsname{\def\PY@tc##1{\textcolor[rgb]{0.00,0.50,0.00}{##1}}}
\expandafter\def\csname PY@tok@m\endcsname{\def\PY@tc##1{\textcolor[rgb]{0.40,0.40,0.40}{##1}}}
\expandafter\def\csname PY@tok@gh\endcsname{\let\PY@bf=\textbf\def\PY@tc##1{\textcolor[rgb]{0.00,0.00,0.50}{##1}}}
\expandafter\def\csname PY@tok@gu\endcsname{\let\PY@bf=\textbf\def\PY@tc##1{\textcolor[rgb]{0.50,0.00,0.50}{##1}}}
\expandafter\def\csname PY@tok@gd\endcsname{\def\PY@tc##1{\textcolor[rgb]{0.63,0.00,0.00}{##1}}}
\expandafter\def\csname PY@tok@gi\endcsname{\def\PY@tc##1{\textcolor[rgb]{0.00,0.63,0.00}{##1}}}
\expandafter\def\csname PY@tok@gr\endcsname{\def\PY@tc##1{\textcolor[rgb]{1.00,0.00,0.00}{##1}}}
\expandafter\def\csname PY@tok@ge\endcsname{\let\PY@it=\textit}
\expandafter\def\csname PY@tok@gs\endcsname{\let\PY@bf=\textbf}
\expandafter\def\csname PY@tok@gp\endcsname{\let\PY@bf=\textbf\def\PY@tc##1{\textcolor[rgb]{0.00,0.00,0.50}{##1}}}
\expandafter\def\csname PY@tok@go\endcsname{\def\PY@tc##1{\textcolor[rgb]{0.53,0.53,0.53}{##1}}}
\expandafter\def\csname PY@tok@gt\endcsname{\def\PY@tc##1{\textcolor[rgb]{0.00,0.27,0.87}{##1}}}
\expandafter\def\csname PY@tok@err\endcsname{\def\PY@bc##1{\setlength{\fboxsep}{0pt}\fcolorbox[rgb]{1.00,0.00,0.00}{1,1,1}{\strut ##1}}}
\expandafter\def\csname PY@tok@kc\endcsname{\let\PY@bf=\textbf\def\PY@tc##1{\textcolor[rgb]{0.00,0.50,0.00}{##1}}}
\expandafter\def\csname PY@tok@kd\endcsname{\let\PY@bf=\textbf\def\PY@tc##1{\textcolor[rgb]{0.00,0.50,0.00}{##1}}}
\expandafter\def\csname PY@tok@kn\endcsname{\let\PY@bf=\textbf\def\PY@tc##1{\textcolor[rgb]{0.00,0.50,0.00}{##1}}}
\expandafter\def\csname PY@tok@kr\endcsname{\let\PY@bf=\textbf\def\PY@tc##1{\textcolor[rgb]{0.00,0.50,0.00}{##1}}}
\expandafter\def\csname PY@tok@bp\endcsname{\def\PY@tc##1{\textcolor[rgb]{0.00,0.50,0.00}{##1}}}
\expandafter\def\csname PY@tok@fm\endcsname{\def\PY@tc##1{\textcolor[rgb]{0.00,0.00,1.00}{##1}}}
\expandafter\def\csname PY@tok@vc\endcsname{\def\PY@tc##1{\textcolor[rgb]{0.10,0.09,0.49}{##1}}}
\expandafter\def\csname PY@tok@vg\endcsname{\def\PY@tc##1{\textcolor[rgb]{0.10,0.09,0.49}{##1}}}
\expandafter\def\csname PY@tok@vi\endcsname{\def\PY@tc##1{\textcolor[rgb]{0.10,0.09,0.49}{##1}}}
\expandafter\def\csname PY@tok@vm\endcsname{\def\PY@tc##1{\textcolor[rgb]{0.10,0.09,0.49}{##1}}}
\expandafter\def\csname PY@tok@sa\endcsname{\def\PY@tc##1{\textcolor[rgb]{0.73,0.13,0.13}{##1}}}
\expandafter\def\csname PY@tok@sb\endcsname{\def\PY@tc##1{\textcolor[rgb]{0.73,0.13,0.13}{##1}}}
\expandafter\def\csname PY@tok@sc\endcsname{\def\PY@tc##1{\textcolor[rgb]{0.73,0.13,0.13}{##1}}}
\expandafter\def\csname PY@tok@dl\endcsname{\def\PY@tc##1{\textcolor[rgb]{0.73,0.13,0.13}{##1}}}
\expandafter\def\csname PY@tok@s2\endcsname{\def\PY@tc##1{\textcolor[rgb]{0.73,0.13,0.13}{##1}}}
\expandafter\def\csname PY@tok@sh\endcsname{\def\PY@tc##1{\textcolor[rgb]{0.73,0.13,0.13}{##1}}}
\expandafter\def\csname PY@tok@s1\endcsname{\def\PY@tc##1{\textcolor[rgb]{0.73,0.13,0.13}{##1}}}
\expandafter\def\csname PY@tok@mb\endcsname{\def\PY@tc##1{\textcolor[rgb]{0.40,0.40,0.40}{##1}}}
\expandafter\def\csname PY@tok@mf\endcsname{\def\PY@tc##1{\textcolor[rgb]{0.40,0.40,0.40}{##1}}}
\expandafter\def\csname PY@tok@mh\endcsname{\def\PY@tc##1{\textcolor[rgb]{0.40,0.40,0.40}{##1}}}
\expandafter\def\csname PY@tok@mi\endcsname{\def\PY@tc##1{\textcolor[rgb]{0.40,0.40,0.40}{##1}}}
\expandafter\def\csname PY@tok@il\endcsname{\def\PY@tc##1{\textcolor[rgb]{0.40,0.40,0.40}{##1}}}
\expandafter\def\csname PY@tok@mo\endcsname{\def\PY@tc##1{\textcolor[rgb]{0.40,0.40,0.40}{##1}}}
\expandafter\def\csname PY@tok@ch\endcsname{\let\PY@it=\textit\def\PY@tc##1{\textcolor[rgb]{0.25,0.50,0.50}{##1}}}
\expandafter\def\csname PY@tok@cm\endcsname{\let\PY@it=\textit\def\PY@tc##1{\textcolor[rgb]{0.25,0.50,0.50}{##1}}}
\expandafter\def\csname PY@tok@cpf\endcsname{\let\PY@it=\textit\def\PY@tc##1{\textcolor[rgb]{0.25,0.50,0.50}{##1}}}
\expandafter\def\csname PY@tok@c1\endcsname{\let\PY@it=\textit\def\PY@tc##1{\textcolor[rgb]{0.25,0.50,0.50}{##1}}}
\expandafter\def\csname PY@tok@cs\endcsname{\let\PY@it=\textit\def\PY@tc##1{\textcolor[rgb]{0.25,0.50,0.50}{##1}}}

\def\PYZbs{\char`\\}
\def\PYZus{\char`\_}
\def\PYZob{\char`\{}
\def\PYZcb{\char`\}}
\def\PYZca{\char`\^}
\def\PYZam{\char`\&}
\def\PYZlt{\char`\<}
\def\PYZgt{\char`\>}
\def\PYZsh{\char`\#}
\def\PYZpc{\char`\%}
\def\PYZdl{\char`\$}
\def\PYZhy{\char`\-}
\def\PYZsq{\char`\'}
\def\PYZdq{\char`\"}
\def\PYZti{\char`\~}
% for compatibility with earlier versions
\def\PYZat{@}
\def\PYZlb{[}
\def\PYZrb{]}
\makeatother


    % For linebreaks inside Verbatim environment from package fancyvrb. 
    \makeatletter
        \newbox\Wrappedcontinuationbox 
        \newbox\Wrappedvisiblespacebox 
        \newcommand*\Wrappedvisiblespace {\textcolor{red}{\textvisiblespace}} 
        \newcommand*\Wrappedcontinuationsymbol {\textcolor{red}{\llap{\tiny$\m@th\hookrightarrow$}}} 
        \newcommand*\Wrappedcontinuationindent {3ex } 
        \newcommand*\Wrappedafterbreak {\kern\Wrappedcontinuationindent\copy\Wrappedcontinuationbox} 
        % Take advantage of the already applied Pygments mark-up to insert 
        % potential linebreaks for TeX processing. 
        %        {, <, #, %, $, ' and ": go to next line. 
        %        _, }, ^, &, >, - and ~: stay at end of broken line. 
        % Use of \textquotesingle for straight quote. 
        \newcommand*\Wrappedbreaksatspecials {% 
            \def\PYGZus{\discretionary{\char`\_}{\Wrappedafterbreak}{\char`\_}}% 
            \def\PYGZob{\discretionary{}{\Wrappedafterbreak\char`\{}{\char`\{}}% 
            \def\PYGZcb{\discretionary{\char`\}}{\Wrappedafterbreak}{\char`\}}}% 
            \def\PYGZca{\discretionary{\char`\^}{\Wrappedafterbreak}{\char`\^}}% 
            \def\PYGZam{\discretionary{\char`\&}{\Wrappedafterbreak}{\char`\&}}% 
            \def\PYGZlt{\discretionary{}{\Wrappedafterbreak\char`\<}{\char`\<}}% 
            \def\PYGZgt{\discretionary{\char`\>}{\Wrappedafterbreak}{\char`\>}}% 
            \def\PYGZsh{\discretionary{}{\Wrappedafterbreak\char`\#}{\char`\#}}% 
            \def\PYGZpc{\discretionary{}{\Wrappedafterbreak\char`\%}{\char`\%}}% 
            \def\PYGZdl{\discretionary{}{\Wrappedafterbreak\char`\$}{\char`\$}}% 
            \def\PYGZhy{\discretionary{\char`\-}{\Wrappedafterbreak}{\char`\-}}% 
            \def\PYGZsq{\discretionary{}{\Wrappedafterbreak\textquotesingle}{\textquotesingle}}% 
            \def\PYGZdq{\discretionary{}{\Wrappedafterbreak\char`\"}{\char`\"}}% 
            \def\PYGZti{\discretionary{\char`\~}{\Wrappedafterbreak}{\char`\~}}% 
        } 
        % Some characters . , ; ? ! / are not pygmentized. 
        % This macro makes them "active" and they will insert potential linebreaks 
        \newcommand*\Wrappedbreaksatpunct {% 
            \lccode`\~`\.\lowercase{\def~}{\discretionary{\hbox{\char`\.}}{\Wrappedafterbreak}{\hbox{\char`\.}}}% 
            \lccode`\~`\,\lowercase{\def~}{\discretionary{\hbox{\char`\,}}{\Wrappedafterbreak}{\hbox{\char`\,}}}% 
            \lccode`\~`\;\lowercase{\def~}{\discretionary{\hbox{\char`\;}}{\Wrappedafterbreak}{\hbox{\char`\;}}}% 
            \lccode`\~`\:\lowercase{\def~}{\discretionary{\hbox{\char`\:}}{\Wrappedafterbreak}{\hbox{\char`\:}}}% 
            \lccode`\~`\?\lowercase{\def~}{\discretionary{\hbox{\char`\?}}{\Wrappedafterbreak}{\hbox{\char`\?}}}% 
            \lccode`\~`\!\lowercase{\def~}{\discretionary{\hbox{\char`\!}}{\Wrappedafterbreak}{\hbox{\char`\!}}}% 
            \lccode`\~`\/\lowercase{\def~}{\discretionary{\hbox{\char`\/}}{\Wrappedafterbreak}{\hbox{\char`\/}}}% 
            \catcode`\.\active
            \catcode`\,\active 
            \catcode`\;\active
            \catcode`\:\active
            \catcode`\?\active
            \catcode`\!\active
            \catcode`\/\active 
            \lccode`\~`\~ 	
        }
    \makeatother

    \let\OriginalVerbatim=\Verbatim
    \makeatletter
    \renewcommand{\Verbatim}[1][1]{%
        %\parskip\z@skip
        \sbox\Wrappedcontinuationbox {\Wrappedcontinuationsymbol}%
        \sbox\Wrappedvisiblespacebox {\FV@SetupFont\Wrappedvisiblespace}%
        \def\FancyVerbFormatLine ##1{\hsize\linewidth
            \vtop{\raggedright\hyphenpenalty\z@\exhyphenpenalty\z@
                \doublehyphendemerits\z@\finalhyphendemerits\z@
                \strut ##1\strut}%
        }%
        % If the linebreak is at a space, the latter will be displayed as visible
        % space at end of first line, and a continuation symbol starts next line.
        % Stretch/shrink are however usually zero for typewriter font.
        \def\FV@Space {%
            \nobreak\hskip\z@ plus\fontdimen3\font minus\fontdimen4\font
            \discretionary{\copy\Wrappedvisiblespacebox}{\Wrappedafterbreak}
            {\kern\fontdimen2\font}%
        }%
        
        % Allow breaks at special characters using \PYG... macros.
        \Wrappedbreaksatspecials
        % Breaks at punctuation characters . , ; ? ! and / need catcode=\active 	
        \OriginalVerbatim[#1,codes*=\Wrappedbreaksatpunct]%
    }
    \makeatother

    % Exact colors from NB
    \definecolor{incolor}{HTML}{303F9F}
    \definecolor{outcolor}{HTML}{D84315}
    \definecolor{cellborder}{HTML}{CFCFCF}
    \definecolor{cellbackground}{HTML}{F7F7F7}
    
    % prompt
    \makeatletter
    \newcommand{\boxspacing}{\kern\kvtcb@left@rule\kern\kvtcb@boxsep}
    \makeatother
    \newcommand{\prompt}[4]{
        \ttfamily\llap{{\color{#2}[#3]:\hspace{3pt}#4}}\vspace{-\baselineskip}
    }
    

    
    % Prevent overflowing lines due to hard-to-break entities
    \sloppy 
    % Setup hyperref package
    \hypersetup{
      breaklinks=true,  % so long urls are correctly broken across lines
      colorlinks=true,
      urlcolor=urlcolor,
      linkcolor=linkcolor,
      citecolor=citecolor,
      }
    % Slightly bigger margins than the latex defaults
    
    \geometry{verbose,tmargin=1in,bmargin=1in,lmargin=1in,rmargin=1in}
    
    

\begin{document}
  
%    \maketitle
%	ref: https://stackoverflow.com/questions/3141702/
	\begin{titlepage}
		\vspace*{\fill}
		\begin{center}
 		\normalsize Manual / Referensi Modul \texttt{hidrokit.contrib.taruma}\\
		\huge \texttt{.hk102}: Upsampling\\ 
		\small telah tersedia pada versi hidrokit 0.3.5 \\[0.2cm]
      	\small Berdasarkan \emph{Jupyter Notebook}: \texttt{github-taruma\_hk.ipynb} \\[0.5cm]
      	\small Modul ini digunakan untuk melakukan \emph{upsampling} dataset. Contoh: data bulanan menjadi data harian. \\[0.5cm]
		\normalsize oleh Taruma Sakti Megariansyah\\[0.5cm]
      	\normalsize 19 Desember 2019 (1.0.0)\\[1cm]
    	\adjustimage{max size={0.9\linewidth}{1cm}}{hidrokit_logo.jpg}\\
      	\normalsize github.com/taruma/hidrokit
		\end{center}
    	\vspace*{\fill}
	\end{titlepage}
    
    

    
    

    Berdasarkan \href{https://github.com/taruma/hidrokit/issues/102}{\#102}:
\textbf{upsampling data}

Referensi Isu:

\begin{itemize}
\tightlist
\item
  \texttt{hidrokit.contrib.taruma.hk98}.
  (\href{https://gist.github.com/taruma/aca7f90c8fbb0034587809883d0d9e92}{manual/notebook}).
  \textbf{buat ringkasan/rekap data deret waktu}
\end{itemize}

Deskripsi permasalahan:

\begin{itemize}
\tightlist
\item
  Mengubah data bulanan menjadi data harian sehingga dapat di rekap
  menggunakan fungsi yang tersedia di
  \texttt{hidrokit.contrib.taruma.hk98}.
\end{itemize}

Strategi penyelesaian:

\begin{itemize}
\tightlist
\item
  Pada melakukan upsampling, nilai yang kosong bisa diisi menggunakan
  metode interpolasi atau diisi dengan nilai pada awal bulan
  (\emph{default}).
\end{itemize}

    \hypertarget{persiapan-dan-dataset}{%
\section{PERSIAPAN DAN DATASET}\label{persiapan-dan-dataset}}

    \begin{tcolorbox}[breakable, size=fbox, boxrule=1pt, pad at break*=1mm,colback=cellbackground, colframe=cellborder]
\prompt{In}{incolor}{0}{\boxspacing}
\begin{Verbatim}[commandchars=\\\{\}]
\PY{k+kn}{import} \PY{n+nn}{numpy} \PY{k}{as} \PY{n+nn}{np}
\PY{k+kn}{import} \PY{n+nn}{pandas} \PY{k}{as} \PY{n+nn}{pd}
\end{Verbatim}
\end{tcolorbox}

    \begin{tcolorbox}[breakable, size=fbox, boxrule=1pt, pad at break*=1mm,colback=cellbackground, colframe=cellborder]
\prompt{In}{incolor}{0}{\boxspacing}
\begin{Verbatim}[commandchars=\\\{\}]
\PY{k+kn}{import} \PY{n+nn}{matplotlib}\PY{n+nn}{.}\PY{n+nn}{pyplot} \PY{k}{as} \PY{n+nn}{plt}
\PY{n}{plt}\PY{o}{.}\PY{n}{rcParams}\PY{p}{[}\PY{l+s+s2}{\PYZdq{}}\PY{l+s+s2}{figure.figsize}\PY{l+s+s2}{\PYZdq{}}\PY{p}{]} \PY{o}{=} \PY{p}{[}\PY{l+m+mi}{15}\PY{p}{,} \PY{l+m+mi}{5}\PY{p}{]}
\end{Verbatim}
\end{tcolorbox}

    \begin{tcolorbox}[breakable, size=fbox, boxrule=1pt, pad at break*=1mm,colback=cellbackground, colframe=cellborder]
\prompt{In}{incolor}{3}{\boxspacing}
\begin{Verbatim}[commandchars=\\\{\}]
\PY{k}{try}\PY{p}{:}
    \PY{k+kn}{import} \PY{n+nn}{hidrokit}
\PY{k}{except} \PY{n}{ModuleNotFoundError}\PY{p}{:}
    \PY{o}{!}pip install git+https://github.com/taruma/hidrokit.git@latest \PYZhy{}q
    \PY{k+kn}{import} \PY{n+nn}{hidrokit}
\PY{n+nb}{print}\PY{p}{(}\PY{l+s+sa}{f}\PY{l+s+s1}{\PYZsq{}}\PY{l+s+s1}{hidrokit version: }\PY{l+s+si}{\PYZob{}hidrokit.\PYZus{}\PYZus{}version\PYZus{}\PYZus{}\PYZcb{}}\PY{l+s+s1}{\PYZsq{}}\PY{p}{)}
\end{Verbatim}
\end{tcolorbox}

    \begin{Verbatim}[commandchars=\\\{\}]
hidrokit version: 0.3.5-beta.6
    \end{Verbatim}

    \begin{tcolorbox}[breakable, size=fbox, boxrule=1pt, pad at break*=1mm,colback=cellbackground, colframe=cellborder]
\prompt{In}{incolor}{0}{\boxspacing}
\begin{Verbatim}[commandchars=\\\{\}]
\PY{o}{!}wget \PYZhy{}O pet.xlsx \PY{l+s+s2}{\PYZdq{}https://taruma.github.io/assets/hidrokit\PYZus{}dataset/fjmock\PYZus{}sample.xlsx\PYZdq{}} \PYZhy{}q
\PY{o}{!}wget \PYZhy{}O sta.xlsx \PY{l+s+s2}{\PYZdq{}https://taruma.github.io/assets/hidrokit\PYZus{}dataset/data\PYZus{}daily\PYZus{}sample.xlsx\PYZdq{}} \PYZhy{}q
\PY{n}{sta\PYZus{}path} \PY{o}{=} \PY{l+s+s1}{\PYZsq{}}\PY{l+s+s1}{sta.xlsx}\PY{l+s+s1}{\PYZsq{}}
\PY{n}{pet\PYZus{}path} \PY{o}{=} \PY{l+s+s1}{\PYZsq{}}\PY{l+s+s1}{sample.xlsx}\PY{l+s+s1}{\PYZsq{}}
\end{Verbatim}
\end{tcolorbox}

    \begin{tcolorbox}[breakable, size=fbox, boxrule=1pt, pad at break*=1mm,colback=cellbackground, colframe=cellborder]
\prompt{In}{incolor}{5}{\boxspacing}
\begin{Verbatim}[commandchars=\\\{\}]
\PY{n}{df\PYZus{}pet} \PY{o}{=} \PY{p}{(}\PY{n}{pd}\PY{o}{.}\PY{n}{read\PYZus{}excel}\PY{p}{(}\PY{n}{pet\PYZus{}path}\PY{p}{,} \PY{n}{usecols}\PY{o}{=}\PY{p}{[}\PY{l+m+mi}{3}\PY{p}{]}\PY{p}{)}
            \PY{o}{.}\PY{n}{set\PYZus{}index}\PY{p}{(}
                \PY{n}{pd}\PY{o}{.}\PY{n}{date\PYZus{}range}\PY{p}{(}\PY{l+s+s1}{\PYZsq{}}\PY{l+s+s1}{20050101}\PY{l+s+s1}{\PYZsq{}}\PY{p}{,} \PY{l+s+s1}{\PYZsq{}}\PY{l+s+s1}{20160101}\PY{l+s+s1}{\PYZsq{}}\PY{p}{,} \PY{n}{freq}\PY{o}{=}\PY{l+s+s1}{\PYZsq{}}\PY{l+s+s1}{MS}\PY{l+s+s1}{\PYZsq{}}\PY{p}{,} \PY{n}{closed}\PY{o}{=}\PY{l+s+s1}{\PYZsq{}}\PY{l+s+s1}{left}\PY{l+s+s1}{\PYZsq{}}\PY{p}{)}
            \PY{p}{)}
         \PY{p}{)}
\PY{n}{df\PYZus{}pet}\PY{o}{.}\PY{n}{head}\PY{p}{(}\PY{p}{)}
\end{Verbatim}
\end{tcolorbox}

            \begin{tcolorbox}[breakable, size=fbox, boxrule=.5pt, pad at break*=1mm, opacityfill=0]
\prompt{Out}{outcolor}{5}{\boxspacing}
\begin{Verbatim}[commandchars=\\\{\}]
                 PET
2005-01-01  2.988970
2005-02-01  3.145340
2005-03-01  3.213324
2005-04-01  3.474628
2005-05-01  3.597509
\end{Verbatim}
\end{tcolorbox}
        
    \begin{tcolorbox}[breakable, size=fbox, boxrule=1pt, pad at break*=1mm,colback=cellbackground, colframe=cellborder]
\prompt{In}{incolor}{6}{\boxspacing}
\begin{Verbatim}[commandchars=\\\{\}]
\PY{k+kn}{from} \PY{n+nn}{hidrokit}\PY{n+nn}{.}\PY{n+nn}{contrib}\PY{n+nn}{.}\PY{n+nn}{taruma} \PY{k+kn}{import} \PY{n}{hk88}

\PY{n}{df\PYZus{}sta} \PY{o}{=} \PY{n}{hk88}\PY{o}{.}\PY{n}{read\PYZus{}workbook}\PY{p}{(}\PY{n}{sta\PYZus{}path}\PY{p}{,} \PY{p}{[}\PY{l+s+s1}{\PYZsq{}}\PY{l+s+s1}{STA\PYZus{}B}\PY{l+s+s1}{\PYZsq{}}\PY{p}{]}\PY{p}{)}
\PY{n}{df\PYZus{}sta}\PY{o}{.}\PY{n}{head}\PY{p}{(}\PY{p}{)}
\end{Verbatim}
\end{tcolorbox}

            \begin{tcolorbox}[breakable, size=fbox, boxrule=.5pt, pad at break*=1mm, opacityfill=0]
\prompt{Out}{outcolor}{6}{\boxspacing}
\begin{Verbatim}[commandchars=\\\{\}]
           STA\_B
2001-01-01     0
2001-01-02     0
2001-01-03    45
2001-01-04     0
2001-01-05     5
\end{Verbatim}
\end{tcolorbox}
        
    \hypertarget{kode}{%
\section{KODE}\label{kode}}

    \begin{tcolorbox}[breakable, size=fbox, boxrule=1pt, pad at break*=1mm,colback=cellbackground, colframe=cellborder]
\prompt{In}{incolor}{0}{\boxspacing}
\begin{Verbatim}[commandchars=\\\{\}]
\PY{c+c1}{\PYZsh{} ref: https://stackoverflow.com/q/29612705/4886384}
\PY{k}{def} \PY{n+nf}{upsampling}\PY{p}{(}\PY{n}{df}\PY{p}{,} \PY{n}{freq}\PY{o}{=}\PY{l+s+s1}{\PYZsq{}}\PY{l+s+s1}{D}\PY{l+s+s1}{\PYZsq{}}\PY{p}{,} \PY{n}{fill\PYZus{}method}\PY{o}{=}\PY{l+s+s1}{\PYZsq{}}\PY{l+s+s1}{ffill}\PY{l+s+s1}{\PYZsq{}}\PY{p}{,} 
               \PY{n}{use\PYZus{}inter}\PY{o}{=}\PY{k+kc}{False}\PY{p}{,} \PY{n}{inter\PYZus{}method}\PY{o}{=}\PY{l+s+s1}{\PYZsq{}}\PY{l+s+s1}{linear}\PY{l+s+s1}{\PYZsq{}}\PY{p}{,} \PY{n}{inter\PYZus{}keys}\PY{o}{=}\PY{p}{\PYZob{}}\PY{p}{\PYZcb{}}\PY{p}{,} 
               \PY{n}{reindex}\PY{o}{=}\PY{k+kc}{False}\PY{p}{)}\PY{p}{:}
    \PY{n}{start} \PY{o}{=} \PY{n}{df}\PY{o}{.}\PY{n}{index}\PY{o}{.}\PY{n}{min}\PY{p}{(}\PY{p}{)} \PY{o}{\PYZhy{}} \PY{n}{pd}\PY{o}{.}\PY{n}{DateOffset}\PY{p}{(}\PY{n}{day}\PY{o}{=}\PY{l+m+mi}{1}\PY{p}{)}
    \PY{n}{end} \PY{o}{=} \PY{n}{df}\PY{o}{.}\PY{n}{index}\PY{o}{.}\PY{n}{max}\PY{p}{(}\PY{p}{)} \PY{o}{\PYZhy{}} \PY{n}{pd}\PY{o}{.}\PY{n}{DateOffset}\PY{p}{(}\PY{n}{day}\PY{o}{=}\PY{l+m+mi}{31}\PY{p}{)}
    \PY{n}{date} \PY{o}{=} \PY{n}{pd}\PY{o}{.}\PY{n}{date\PYZus{}range}\PY{p}{(}\PY{n}{start}\PY{p}{,} \PY{n}{end}\PY{p}{,} \PY{n}{freq}\PY{o}{=}\PY{n}{freq}\PY{p}{)}

    \PY{n}{newdf} \PY{o}{=} \PY{n}{df}\PY{o}{.}\PY{n}{reindex}\PY{p}{(}\PY{n}{date}\PY{p}{)}

    \PY{k}{if} \PY{n}{reindex}\PY{p}{:}
        \PY{k}{return} \PY{n}{newdf}

    \PY{k}{if} \PY{n}{use\PYZus{}inter}\PY{p}{:}
        \PY{k}{return} \PY{n}{newdf}\PY{o}{.}\PY{n}{interpolate}\PY{p}{(}\PY{n}{method}\PY{o}{=}\PY{n}{inter\PYZus{}method}\PY{p}{,} \PY{o}{*}\PY{o}{*}\PY{n}{inter\PYZus{}keys}\PY{p}{)}
    \PY{k}{else}\PY{p}{:}
        \PY{k}{return} \PY{n}{newdf}\PY{o}{.}\PY{n}{fillna}\PY{p}{(}\PY{n}{method}\PY{o}{=}\PY{n}{fill\PYZus{}method}\PY{p}{)}
\end{Verbatim}
\end{tcolorbox}

    \hypertarget{penerapan}{%
\section{PENERAPAN}\label{penerapan}}

Pada \emph{notebook} ini terdapat dua dataset yaitu \texttt{df\_sta}
(data hujan stasiun B) dan \texttt{df\_pet} (data evapotranspirasi). Dua
dataset ini memiliki perbedaan dari jumlah data dan observasinya, pada
\texttt{df\_sta} data dalam bentuk observasi harian, sedangkan pada
\texttt{df\_pet} data berupa observasi bulanan.

Dalam pemodelan F.J. Mock (\texttt{hidrokit.contrib.taruma.hk96}) data
masukan bisa memiliki sembarang periode (per 3 hari, 15 hari, dll).
Untuk data \texttt{df\_sta} bisa dilakukan rekap data menggunakan modul
\texttt{contrib.taruma.hk98}, sedangkan data \texttt{df\_pet} tidak bisa
menyesuaikan dengan periode tersebut karena data bukan data harian.

Fungsi \texttt{upsampling} menangani permasalahan tersebut dengan
mengubah data bulanan menjadi harian dengan pengisian berbagai metode
(interpolasi, atau isian dari awal bulan).

    \hypertarget{fungsi-upsampling}{%
\subsection{\texorpdfstring{Fungsi
\texttt{upsampling}}{Fungsi upsampling}}\label{fungsi-upsampling}}

Fungsi ini mengubah data bulanan/periode ke data harian dengan
menggunakan berbagai metode (isian atau interpolasi).

Argumen yang harus diberikan:

\begin{itemize}
\tightlist
\item
  \texttt{df}: dataset dalam objek \texttt{pandas.DataFrame} dengan
  index berupa objek \texttt{DateTimeIndex}/\texttt{Timestamp}.
\end{itemize}

Argumen opsional:

\begin{itemize}
\tightlist
\item
  \texttt{fill\_method=\textquotesingle{}ffill\textquotesingle{}}:
  mengisi data yang kosong dengan metode \emph{forward fill}. Metode
  merupakan isian valid untuk argumen \texttt{method} pada metode
  \texttt{pandas.DataFrame.fillna()}.
\item
  \texttt{freq=\textquotesingle{}D\textquotesingle{}}: mengubah dalam
  bentuk harian.
\item
  \texttt{use\_inter=None}: isi \texttt{True} jika ingin menggunakan
  \texttt{inter\_method}.
\item
  \texttt{inter\_method=\textquotesingle{}linear\textquotesingle{}}:
  metode interpolasi.
\item
  \texttt{inter\_keys=\{\}}: \emph{keywords} untuk metode interpolasi.
\item
  \texttt{reindex=False}: isi \texttt{True} jika hanya mendapatkan
  keluaran tanpa mengisi data yang kosong.
\end{itemize}

    \hypertarget{dataset-df_pet}{%
\subsubsection{\texorpdfstring{Dataset
\texttt{df\_pet}}{Dataset df\_pet}}\label{dataset-df_pet}}

Berikut informasi \texttt{df\_pet} sebelum dilakukan \emph{upsampling}.

    \begin{tcolorbox}[breakable, size=fbox, boxrule=1pt, pad at break*=1mm,colback=cellbackground, colframe=cellborder]
\prompt{In}{incolor}{8}{\boxspacing}
\begin{Verbatim}[commandchars=\\\{\}]
\PY{n}{df\PYZus{}pet}\PY{o}{.}\PY{n}{info}\PY{p}{(}\PY{p}{)}
\PY{n}{df\PYZus{}pet}\PY{o}{.}\PY{n}{head}\PY{p}{(}\PY{p}{)}
\end{Verbatim}
\end{tcolorbox}

    \begin{Verbatim}[commandchars=\\\{\}]
<class 'pandas.core.frame.DataFrame'>
DatetimeIndex: 132 entries, 2005-01-01 to 2015-12-01
Freq: MS
Data columns (total 1 columns):
PET    132 non-null float64
dtypes: float64(1)
memory usage: 2.1 KB
    \end{Verbatim}

            \begin{tcolorbox}[breakable, size=fbox, boxrule=.5pt, pad at break*=1mm, opacityfill=0]
\prompt{Out}{outcolor}{8}{\boxspacing}
\begin{Verbatim}[commandchars=\\\{\}]
                 PET
2005-01-01  2.988970
2005-02-01  3.145340
2005-03-01  3.213324
2005-04-01  3.474628
2005-05-01  3.597509
\end{Verbatim}
\end{tcolorbox}
        
    \begin{tcolorbox}[breakable, size=fbox, boxrule=1pt, pad at break*=1mm,colback=cellbackground, colframe=cellborder]
\prompt{In}{incolor}{9}{\boxspacing}
\begin{Verbatim}[commandchars=\\\{\}]
\PY{n}{df\PYZus{}pet}\PY{o}{.}\PY{n}{plot}\PY{p}{(}\PY{p}{)}\PY{p}{;}
\end{Verbatim}
\end{tcolorbox}

    \begin{center}
    \adjustimage{max size={0.9\linewidth}{0.9\paperheight}}{output_15_0.png}
    \end{center}
    { \hspace*{\fill} \\}
    
    Sebelum dilakukan \emph{upsampling} data hanya memiliki \(132\) data
observasi dengan frekuensi bulanan \texttt{MS}.

    \hypertarget{menggunakan-forward-fill-default}{%
\subsubsection{\texorpdfstring{Menggunakan \emph{forward fill}
(\emph{default})}{Menggunakan forward fill (default)}}\label{menggunakan-forward-fill-default}}

    \begin{tcolorbox}[breakable, size=fbox, boxrule=1pt, pad at break*=1mm,colback=cellbackground, colframe=cellborder]
\prompt{In}{incolor}{10}{\boxspacing}
\begin{Verbatim}[commandchars=\\\{\}]
\PY{n}{df\PYZus{}pet\PYZus{}ffill} \PY{o}{=} \PY{n}{upsampling}\PY{p}{(}\PY{n}{df\PYZus{}pet}\PY{p}{)}
\PY{n}{df\PYZus{}pet\PYZus{}ffill}\PY{o}{.}\PY{n}{info}\PY{p}{(}\PY{p}{)}
\PY{n}{df\PYZus{}pet\PYZus{}ffill}\PY{o}{.}\PY{n}{head}\PY{p}{(}\PY{p}{)}
\end{Verbatim}
\end{tcolorbox}

    \begin{Verbatim}[commandchars=\\\{\}]
<class 'pandas.core.frame.DataFrame'>
DatetimeIndex: 4017 entries, 2005-01-01 to 2015-12-31
Freq: D
Data columns (total 1 columns):
PET    4017 non-null float64
dtypes: float64(1)
memory usage: 62.8 KB
    \end{Verbatim}

            \begin{tcolorbox}[breakable, size=fbox, boxrule=.5pt, pad at break*=1mm, opacityfill=0]
\prompt{Out}{outcolor}{10}{\boxspacing}
\begin{Verbatim}[commandchars=\\\{\}]
                PET
2005-01-01  2.98897
2005-01-02  2.98897
2005-01-03  2.98897
2005-01-04  2.98897
2005-01-05  2.98897
\end{Verbatim}
\end{tcolorbox}
        
    \begin{tcolorbox}[breakable, size=fbox, boxrule=1pt, pad at break*=1mm,colback=cellbackground, colframe=cellborder]
\prompt{In}{incolor}{11}{\boxspacing}
\begin{Verbatim}[commandchars=\\\{\}]
\PY{n}{df\PYZus{}pet\PYZus{}ffill}\PY{o}{.}\PY{n}{plot}\PY{p}{(}\PY{p}{)}\PY{p}{;}
\end{Verbatim}
\end{tcolorbox}

    \begin{center}
    \adjustimage{max size={0.9\linewidth}{0.9\paperheight}}{output_19_0.png}
    \end{center}
    { \hspace*{\fill} \\}
    
    Dataframe \texttt{df\_pet} setelah dilakukan \emph{upsampling} menjadi
\(4017\) data observasi dengan frekuensi harian \texttt{D}.

    \hypertarget{menggunakan-interpolasi}{%
\subsubsection{Menggunakan interpolasi}\label{menggunakan-interpolasi}}

    \begin{tcolorbox}[breakable, size=fbox, boxrule=1pt, pad at break*=1mm,colback=cellbackground, colframe=cellborder]
\prompt{In}{incolor}{12}{\boxspacing}
\begin{Verbatim}[commandchars=\\\{\}]
\PY{n}{df\PYZus{}pet\PYZus{}inter} \PY{o}{=} \PY{n}{upsampling}\PY{p}{(}\PY{n}{df\PYZus{}pet}\PY{p}{,} \PY{n}{use\PYZus{}inter}\PY{o}{=}\PY{k+kc}{True}\PY{p}{,} \PY{n}{inter\PYZus{}method}\PY{o}{=}\PY{l+s+s1}{\PYZsq{}}\PY{l+s+s1}{linear}\PY{l+s+s1}{\PYZsq{}}\PY{p}{)}
\PY{n}{df\PYZus{}pet\PYZus{}inter}\PY{o}{.}\PY{n}{info}\PY{p}{(}\PY{p}{)}
\PY{n}{df\PYZus{}pet\PYZus{}inter}\PY{o}{.}\PY{n}{head}\PY{p}{(}\PY{p}{)}
\end{Verbatim}
\end{tcolorbox}

    \begin{Verbatim}[commandchars=\\\{\}]
<class 'pandas.core.frame.DataFrame'>
DatetimeIndex: 4017 entries, 2005-01-01 to 2015-12-31
Freq: D
Data columns (total 1 columns):
PET    4017 non-null float64
dtypes: float64(1)
memory usage: 62.8 KB
    \end{Verbatim}

            \begin{tcolorbox}[breakable, size=fbox, boxrule=.5pt, pad at break*=1mm, opacityfill=0]
\prompt{Out}{outcolor}{12}{\boxspacing}
\begin{Verbatim}[commandchars=\\\{\}]
                 PET
2005-01-01  2.988970
2005-01-02  2.994014
2005-01-03  2.999058
2005-01-04  3.004102
2005-01-05  3.009146
\end{Verbatim}
\end{tcolorbox}
        
    \begin{tcolorbox}[breakable, size=fbox, boxrule=1pt, pad at break*=1mm,colback=cellbackground, colframe=cellborder]
\prompt{In}{incolor}{13}{\boxspacing}
\begin{Verbatim}[commandchars=\\\{\}]
\PY{n}{df\PYZus{}pet\PYZus{}inter}\PY{o}{.}\PY{n}{plot}\PY{p}{(}\PY{p}{)}\PY{p}{;}
\end{Verbatim}
\end{tcolorbox}

    \begin{center}
    \adjustimage{max size={0.9\linewidth}{0.9\paperheight}}{output_23_0.png}
    \end{center}
    { \hspace*{\fill} \\}
    
    Perbedaan dengan \emph{forward fill} dapat dilihat dari grafiknya.

    \begin{tcolorbox}[breakable, size=fbox, boxrule=1pt, pad at break*=1mm,colback=cellbackground, colframe=cellborder]
\prompt{In}{incolor}{14}{\boxspacing}
\begin{Verbatim}[commandchars=\\\{\}]
\PY{n}{df\PYZus{}pet\PYZus{}quad} \PY{o}{=} \PY{n}{upsampling}\PY{p}{(}\PY{n}{df\PYZus{}pet}\PY{p}{,} \PY{n}{use\PYZus{}inter}\PY{o}{=}\PY{k+kc}{True}\PY{p}{,} \PY{n}{inter\PYZus{}method}\PY{o}{=}\PY{l+s+s1}{\PYZsq{}}\PY{l+s+s1}{quadratic}\PY{l+s+s1}{\PYZsq{}}\PY{p}{)}
\PY{n}{df\PYZus{}pet\PYZus{}quad}\PY{o}{.}\PY{n}{info}\PY{p}{(}\PY{p}{)}
\PY{n}{df\PYZus{}pet\PYZus{}quad}\PY{o}{.}\PY{n}{head}\PY{p}{(}\PY{p}{)}
\end{Verbatim}
\end{tcolorbox}

    \begin{Verbatim}[commandchars=\\\{\}]
<class 'pandas.core.frame.DataFrame'>
DatetimeIndex: 4017 entries, 2005-01-01 to 2015-12-31
Freq: D
Data columns (total 1 columns):
PET    3987 non-null float64
dtypes: float64(1)
memory usage: 62.8 KB
    \end{Verbatim}

            \begin{tcolorbox}[breakable, size=fbox, boxrule=.5pt, pad at break*=1mm, opacityfill=0]
\prompt{Out}{outcolor}{14}{\boxspacing}
\begin{Verbatim}[commandchars=\\\{\}]
                 PET
2005-01-01  2.988970
2005-01-02  2.996154
2005-01-03  3.003196
2005-01-04  3.010095
2005-01-05  3.016851
\end{Verbatim}
\end{tcolorbox}
        
    \begin{tcolorbox}[breakable, size=fbox, boxrule=1pt, pad at break*=1mm,colback=cellbackground, colframe=cellborder]
\prompt{In}{incolor}{15}{\boxspacing}
\begin{Verbatim}[commandchars=\\\{\}]
\PY{n}{df\PYZus{}pet\PYZus{}quad}\PY{o}{.}\PY{n}{plot}\PY{p}{(}\PY{p}{)}\PY{p}{;}
\end{Verbatim}
\end{tcolorbox}

    \begin{center}
    \adjustimage{max size={0.9\linewidth}{0.9\paperheight}}{output_26_0.png}
    \end{center}
    { \hspace*{\fill} \\}
    
    Jika menggunakan metode \texttt{quadratic}, grafik menjadi lebih halus.

    \hypertarget{melakukan-reindex-saja}{%
\subsubsection{\texorpdfstring{Melakukan \emph{reindex}
saja}{Melakukan reindex saja}}\label{melakukan-reindex-saja}}

Argumen ini disediakan agar memberi kebebasan jika tidak ingin
menggunakan metode interpolasi/isian.

    \begin{tcolorbox}[breakable, size=fbox, boxrule=1pt, pad at break*=1mm,colback=cellbackground, colframe=cellborder]
\prompt{In}{incolor}{16}{\boxspacing}
\begin{Verbatim}[commandchars=\\\{\}]
\PY{n}{df\PYZus{}pet\PYZus{}re} \PY{o}{=} \PY{n}{upsampling}\PY{p}{(}\PY{n}{df\PYZus{}pet}\PY{p}{,} \PY{n}{reindex}\PY{o}{=}\PY{k+kc}{True}\PY{p}{)}
\PY{n}{df\PYZus{}pet\PYZus{}re}\PY{o}{.}\PY{n}{info}\PY{p}{(}\PY{p}{)}
\PY{n}{df\PYZus{}pet\PYZus{}re}\PY{o}{.}\PY{n}{head}\PY{p}{(}\PY{p}{)}
\end{Verbatim}
\end{tcolorbox}

    \begin{Verbatim}[commandchars=\\\{\}]
<class 'pandas.core.frame.DataFrame'>
DatetimeIndex: 4017 entries, 2005-01-01 to 2015-12-31
Freq: D
Data columns (total 1 columns):
PET    132 non-null float64
dtypes: float64(1)
memory usage: 62.8 KB
    \end{Verbatim}

            \begin{tcolorbox}[breakable, size=fbox, boxrule=.5pt, pad at break*=1mm, opacityfill=0]
\prompt{Out}{outcolor}{16}{\boxspacing}
\begin{Verbatim}[commandchars=\\\{\}]
                PET
2005-01-01  2.98897
2005-01-02      NaN
2005-01-03      NaN
2005-01-04      NaN
2005-01-05      NaN
\end{Verbatim}
\end{tcolorbox}
        
    \hypertarget{menggabungkan-dengan-modul-taruma.hk98-rekap-data-harian}{%
\subsection{\texorpdfstring{Menggabungkan dengan modul
\texttt{taruma.hk98} (rekap data
harian)}{Menggabungkan dengan modul taruma.hk98 (rekap data harian)}}\label{menggabungkan-dengan-modul-taruma.hk98-rekap-data-harian}}

Modul ini bisa digabungkan dengan modul \texttt{taruma.hk98} yang
digunakan untuk merekap data harian.

Karena \texttt{df\_pet} sudah diubah menjadi data harian
(\texttt{df\_pet\_ffill}), maka bisa digabungkan dengan
\texttt{df\_sta}. Dataframe \texttt{dataset} dimulai dari tahun 2005
karena \texttt{df\_pet} hanya dimulai dari tahun 2005.

    \begin{tcolorbox}[breakable, size=fbox, boxrule=1pt, pad at break*=1mm,colback=cellbackground, colframe=cellborder]
\prompt{In}{incolor}{17}{\boxspacing}
\begin{Verbatim}[commandchars=\\\{\}]
\PY{n}{dataset} \PY{o}{=} \PY{n}{pd}\PY{o}{.}\PY{n}{concat}\PY{p}{(}\PY{p}{[}\PY{n}{df\PYZus{}sta}\PY{p}{,} \PY{n}{df\PYZus{}pet\PYZus{}ffill}\PY{p}{]}\PY{p}{,} \PY{n}{axis}\PY{o}{=}\PY{l+m+mi}{1}\PY{p}{)}\PY{o}{.}\PY{n}{loc}\PY{p}{[}\PY{l+s+s1}{\PYZsq{}}\PY{l+s+s1}{20050101}\PY{l+s+s1}{\PYZsq{}}\PY{p}{:}\PY{p}{]}
\PY{n}{dataset}
\end{Verbatim}
\end{tcolorbox}

            \begin{tcolorbox}[breakable, size=fbox, boxrule=.5pt, pad at break*=1mm, opacityfill=0]
\prompt{Out}{outcolor}{17}{\boxspacing}
\begin{Verbatim}[commandchars=\\\{\}]
           STA\_B       PET
2005-01-01    14  2.988970
2005-01-02     0  2.988970
2005-01-03     7  2.988970
2005-01-04    37  2.988970
2005-01-05     0  2.988970
{\ldots}          {\ldots}       {\ldots}
2015-12-27     6  3.133179
2015-12-28    32  3.133179
2015-12-29     0  3.133179
2015-12-30     0  3.133179
2015-12-31     0  3.133179

[4017 rows x 2 columns]
\end{Verbatim}
\end{tcolorbox}
        
    Setelah mempersiapkan \texttt{dataset}, dapat dilakukan rekap
menggunakan modul \texttt{taruma.hk98}. Pada data curah hujan, akan
dilakukan penjumlahan curah hujan, jumlah hari hujan, jumlah hari
periode. Sedangkan pada data PET, hanya merekap dengan merata-ratakan
nilainya setiap periode. Periode yang akan digunakan adalah per 11 hari
(3 periode).

    \begin{tcolorbox}[breakable, size=fbox, boxrule=1pt, pad at break*=1mm,colback=cellbackground, colframe=cellborder]
\prompt{In}{incolor}{18}{\boxspacing}
\begin{Verbatim}[commandchars=\\\{\}]
\PY{k+kn}{from} \PY{n+nn}{hidrokit}\PY{n+nn}{.}\PY{n+nn}{contrib}\PY{n+nn}{.}\PY{n+nn}{taruma} \PY{k+kn}{import} \PY{n}{hk98}

\PY{n}{NDAYS} \PY{o}{=} \PY{l+s+s1}{\PYZsq{}}\PY{l+s+s1}{11D}\PY{l+s+s1}{\PYZsq{}}
\PY{n}{n\PYZus{}rain} \PY{o}{=} \PY{k}{lambda} \PY{n}{x}\PY{p}{:} \PY{p}{(}\PY{n}{x}\PY{o}{\PYZgt{}}\PY{l+m+mi}{0}\PY{p}{)}\PY{o}{.}\PY{n}{sum}\PY{p}{(}\PY{p}{)}
\PY{n}{ufunc}\PY{p}{,} \PY{n}{ufunc\PYZus{}col} \PY{o}{=} \PY{p}{[}\PY{n}{np}\PY{o}{.}\PY{n}{sum}\PY{p}{,} \PY{n}{n\PYZus{}rain}\PY{p}{,} \PY{n+nb}{len}\PY{p}{]}\PY{p}{,} \PY{p}{[}\PY{l+s+s1}{\PYZsq{}}\PY{l+s+s1}{SUM}\PY{l+s+s1}{\PYZsq{}}\PY{p}{,} \PY{l+s+s1}{\PYZsq{}}\PY{l+s+s1}{NRAIN}\PY{l+s+s1}{\PYZsq{}}\PY{p}{,} \PY{l+s+s1}{\PYZsq{}}\PY{l+s+s1}{NDAYS}\PY{l+s+s1}{\PYZsq{}}\PY{p}{]}

\PY{n}{dataset\PYZus{}pet} \PY{o}{=} \PY{n}{hk98}\PY{o}{.}\PY{n}{summary\PYZus{}station}\PY{p}{(}\PY{n}{dataset}\PY{p}{,} \PY{l+s+s1}{\PYZsq{}}\PY{l+s+s1}{PET}\PY{l+s+s1}{\PYZsq{}}\PY{p}{,} \PY{n}{np}\PY{o}{.}\PY{n}{mean}\PY{p}{,} \PY{l+s+s1}{\PYZsq{}}\PY{l+s+s1}{PET}\PY{l+s+s1}{\PYZsq{}}\PY{p}{,} 
                                   \PY{n}{n\PYZus{}days}\PY{o}{=}\PY{n}{NDAYS}\PY{p}{)}
\PY{n}{dataset\PYZus{}prep} \PY{o}{=} \PY{n}{hk98}\PY{o}{.}\PY{n}{summary\PYZus{}station}\PY{p}{(}\PY{n}{dataset}\PY{p}{,} \PY{l+s+s1}{\PYZsq{}}\PY{l+s+s1}{STA\PYZus{}B}\PY{l+s+s1}{\PYZsq{}}\PY{p}{,} \PY{n}{ufunc}\PY{p}{,} \PY{n}{ufunc\PYZus{}col}\PY{p}{,} 
                                    \PY{n}{n\PYZus{}days}\PY{o}{=}\PY{n}{NDAYS}\PY{p}{)}

\PY{n}{data\PYZus{}input} \PY{o}{=} \PY{n}{pd}\PY{o}{.}\PY{n}{concat}\PY{p}{(}\PY{p}{[}\PY{n}{dataset\PYZus{}prep}\PY{p}{,} \PY{n}{dataset\PYZus{}pet}\PY{p}{]}\PY{p}{,} \PY{n}{axis}\PY{o}{=}\PY{l+m+mi}{1}\PY{p}{)}
\PY{n}{data\PYZus{}input}\PY{o}{.}\PY{n}{columns} \PY{o}{=} \PY{n}{data\PYZus{}input}\PY{o}{.}\PY{n}{columns}\PY{o}{.}\PY{n}{droplevel}\PY{p}{(}\PY{l+m+mi}{0}\PY{p}{)}
\PY{n}{data\PYZus{}input}\PY{o}{.}\PY{n}{head}\PY{p}{(}\PY{l+m+mi}{10}\PY{p}{)}
\end{Verbatim}
\end{tcolorbox}

            \begin{tcolorbox}[breakable, size=fbox, boxrule=.5pt, pad at break*=1mm, opacityfill=0]
\prompt{Out}{outcolor}{18}{\boxspacing}
\begin{Verbatim}[commandchars=\\\{\}]
              SUM  NRAIN  NDAYS       PET
2005-01-01  102.0    6.0   11.0  2.988970
2005-01-12    0.0    0.0   11.0  2.988970
2005-01-23    0.0    0.0    9.0  2.988970
2005-02-01    0.0    0.0   11.0  3.145340
2005-02-12    0.0    0.0   11.0  3.145340
2005-02-23    0.0    0.0    6.0  3.145340
2005-03-01    0.0    0.0   11.0  3.213324
2005-03-12    0.0    0.0   11.0  3.213324
2005-03-23    0.0    0.0    9.0  3.213324
2005-04-01    0.0    0.0   11.0  3.474628
\end{Verbatim}
\end{tcolorbox}
        
    \begin{tcolorbox}[breakable, size=fbox, boxrule=1pt, pad at break*=1mm,colback=cellbackground, colframe=cellborder]
\prompt{In}{incolor}{19}{\boxspacing}
\begin{Verbatim}[commandchars=\\\{\}]
\PY{n}{data\PYZus{}input}\PY{o}{.}\PY{n}{sample}\PY{p}{(}\PY{l+m+mi}{10}\PY{p}{)}
\end{Verbatim}
\end{tcolorbox}

            \begin{tcolorbox}[breakable, size=fbox, boxrule=.5pt, pad at break*=1mm, opacityfill=0]
\prompt{Out}{outcolor}{19}{\boxspacing}
\begin{Verbatim}[commandchars=\\\{\}]
              SUM  NRAIN  NDAYS       PET
2014-07-12   32.0    4.0   11.0  3.601199
2007-12-12  190.0    7.0   11.0  2.987743
2010-06-12   72.0    2.0   11.0  3.243436
2007-12-01   86.0    5.0   11.0  2.987743
2013-03-01  113.0    4.0   11.0  3.463320
2007-07-01    0.0    0.0   11.0  3.970709
2007-03-12  245.0    4.0   11.0  3.386749
2013-11-23   69.0    2.0    8.0  2.908408
2010-07-23    0.0    0.0    9.0  3.240201
2007-10-01    0.0    0.0   11.0  4.060168
\end{Verbatim}
\end{tcolorbox}
        
    \hypertarget{changelog}{%
\section{Changelog}\label{changelog}}

\begin{verbatim}
- 20191219 - 1.1.0 - Initial
\end{verbatim}

\hypertarget{copyright-2019-taruma-sakti-megariansyah}{%
\paragraph{\texorpdfstring{Copyright © 2019
\href{https://taruma.github.io}{Taruma Sakti
Megariansyah}}{Copyright © 2019 Taruma Sakti Megariansyah}}\label{copyright-2019-taruma-sakti-megariansyah}}

Source code in this notebook is licensed under a
\href{https://choosealicense.com/licenses/mit/}{MIT License}. Data in
this notebook is licensed under a
\href{https://creativecommons.org/licenses/by/4.0/}{Creative Common
Attribution 4.0 International}.


    % Add a bibliography block to the postdoc
    
    
    
\end{document}
